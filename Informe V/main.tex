\documentclass[twocolumn, 12pt]{article}

\usepackage[utf8]{inputenc}
\usepackage[english, spanish]{babel}
\usepackage{fullpage} % changes the margin
\usepackage{graphicx} 
\usepackage{amsmath}
\usepackage{enumitem} 
\usepackage{chngcntr}
\usepackage{setspace}
\usepackage{url}
\usepackage{csquotes}
\usepackage{float}
\usepackage{verbatim}
\usepackage{tabularx}
\usepackage{amsmath}

\counterwithin{figure}{section}
\renewcommand{\thesection}{\arabic{section}} 
\renewcommand{\thesubsection}{\thesection.\arabic{subsection}}
\renewcommand{\baselinestretch}{1.5}

\usepackage[style=apa, maxnames=6, minnames=3, backend=biber]{biblatex}
\DefineBibliographyStrings{english}{%chktex-file 1 chktex-file 6
    andothers = {\em et\addabbrvspace al\adddot}
}
\addbibresource{./Bibliography/bibliography.bib}

\usepackage{array}
\usepackage{enumitem}

\raggedbottom{}

\begin{document}

\begin{titlepage}
	\centering
	\includegraphics[width=0.3\textwidth]{Images/logo_utb.png}\par\vspace{1cm}
	{\scshape\LARGE Universidad Tecnológica de Bolívar \par}
	\vspace{1cm}

	{\scshape\Large FÍSICA ELÉCTRICA \par}
	\vspace{.2cm}

	% chktex-file 8
	{\scshape\Large H1 - C \par}
	\vspace{1cm}
	% chktex-file 8
	\slshape {\Large \bfseries{}Informe de Laboratorio No. V\\}
	\vspace{1cm}

	\slshape {\itshape{} Mauro González, T00067622 \\}
	\slshape {\itshape{} German De Armas Castaño, T00068765 \\}
	\slshape {\itshape{} Angel Vega Rodriguez, T00068186 \\}
	\slshape {\itshape{} Juan Jose Osorio Ariza, T00067316 \\}
	\slshape {\itshape{} Juan Eduardo barón, T00065901 \\}
	\vfill
	Revisado Por \\
	Gabriel Hoyos Gomez Casseres\\
	{\large \today\par}
\end{titlepage}

% ----------------------------------------------------------------------|>
\section{Introducción}

% \newpage

% ----------------------------------------------------------------------|>
\section{Objetivos}

\subsection{Objetivo general}

\subsection{Objetivo especifico}

% \newpage

% ----------------------------------------------------------------------|>
\section{Marco Teórico}

% -------------------------------------------------|>
\subsection*{Formula para calcular el campo magnético dentro de un solenoide}

% chktex-file 24
\begin{equation}
	B = {\frac{\mu_0 N I_b}{L}} \label{eq:FormulaCampoMagneticoSolenoide}
\end{equation}

\nocite{Garcia2016}

Donde: \hfill \break{} \textit{{\large $\mu_o$}}, es la permeabilidad del
espacio libre, \textit{{\large $N$}}, es la cantidad de vueltas del alambre,
\textit{{\large $I_b$}}, es la corriente suministrada, \textit{{\large $L$}},
la longitud del solenoide.

Dando como resultado (\textit{{\large $B$}}), siendo el campo magnético
del solenoide.

% -------------------------------------------------|>
\subsection*{Formula de fuerza magnética}

\begin{itemize}[label=$\triangleright$]
	\item {\large $Fm = B I_e d$}
	\item {\large $Fm = K I_e$}
\end{itemize}

% ----------------------------------------------------------------------|>
\section{Montaje Experimental}

% ----------------------------------------------------------------------|>
\section{Datos Experimentales}

% --------------------------------------------------|>
% ! Tabla de la toma de datos (1)

% chktex-file 44
\begin{tabularx}{0.9\linewidth}{|>{\centering\arraybackslash}X|>{\centering\arraybackslash}X|}
	\hline
	$Fm$ \textit{(mN)} & $I_e$ \textit{(A)} \\ \hline
	0.04               & 0.98               \\ \hline
	0.13               & 1.95               \\ \hline
	0.17               & 3.00               \\ \hline
	0.23               & 3.90               \\ \hline
	0.30               & 5.16               \\ \hline
\end{tabularx}

\subsection*{Constantes}

\begin{itemize}[label=$\triangleright$]
	\item {\large $N$}: $120$ \textit{(Vueltas)} $\Rightarrow$ {\large $1.20 \times 10^{2}$}
	\item {\large $d$}: $0.04$ \textit{(M)} $\Rightarrow$ {\large $4 \times 10^{-2}$}
	\item {\large $L$}: $0.4$ \textit{(M)} $\Rightarrow$ {\large $4 \times 10^{-1}$}
	\item {\large $\mu_0$}: $4 \pi \times 10^{-7}$ \textit{($T \cdot m/A$)} $\Rightarrow$ {\large $1.26 \times 10^{-6}$}
\end{itemize}

% ----------------------------------------------------------------------|>
\section{Análisis de datos}

% ----------------------------------------------------|>
\subsection*{Calcule el campo magnético en la bobina con la ecuación~(\ref{eq:FormulaCampoMagneticoSolenoide})}

% ! Tabla para el calculo de B Teorico

% chktex-file 44
\begin{tabularx}{0.9\linewidth}{|>{\centering\arraybackslash}X|>{\centering\arraybackslash}X|}
	\hline
	$I_e$ \textit{(A)} & $B$ Teorico \textit{(T)} \\ \hline
	0.98               & $3.69 \times 10^{-4}$    \\ \hline
	1.95               & $7.35 \times 10^{-4}$    \\ \hline
	3.00               & $1.13 \times 10^{-3}$    \\ \hline
	3.90               & $1.47 \times 10^{-3}$    \\ \hline
	5.16               & $1.95 \times 10^{-3}$    \\ \hline
\end{tabularx}

% ----------------------------------------------------------------------|>
\section{Conclusiones}

\newpage

% \bibliography{./Bibliography/bibliography.bib}
\printbibliography

\end{document}