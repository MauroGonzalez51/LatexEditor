\documentclass[letterpaper, 12pt]{report}

\usepackage[utf8]{inputenc}
\usepackage[english, spanish]{babel}
\usepackage{fullpage} % changes the margin
\usepackage{graphicx}
\usepackage{amsmath}
\usepackage{enumitem}
\usepackage{chngcntr}
\usepackage{setspace}
\usepackage{url}
\usepackage{csquotes}
\usepackage{float}
\usepackage{verbatim}
\usepackage{tabularx}
\usepackage{amsmath}

\counterwithin{figure}{section}
\renewcommand{\thesection}{\arabic{section}}
\renewcommand{\thesubsection}{\thesection.\arabic{subsection}}
\renewcommand{\baselinestretch}{1.5}

\usepackage[style=numeric, maxnames=6, minnames=3, backend=biber, parentracker=true, sorting=none]{biblatex}
\DefineBibliographyStrings{english}{%chktex-file 1 chktex-file 6
    andothers = {\em et\addabbrvspace al\adddot}
}
\addbibresource{./Bibliography/bibliography.bib}

\usepackage{array}
\usepackage{enumitem}

\usepackage{setspace}
\setlength{\parskip}{\baselineskip}

\begin{document}

\section*{Experimento 1}

\begin{itemize}
    \item ¿Por qué se desvía la aguja del galvanómetro cuando el imán entra o sale de la bobina?
    Esto se debe a que al introducir el imán en el interior del solenoide se crea una corriente eléctrica (corrientes inducidas) que alimenta a la bobina y produce un campo magnético en la bobina, y cuando el imán entra y sale se genera una variación en ese campo magnético, por esta razón la aguja del galvanómetro se mueve. (principio de oersted y Faraday) 
    
\end{itemize}



\newpage

\printbibliography

\end{document}