\documentclass[letterpaper, 12pt]{report}

\usepackage[utf8]{inputenc}
\usepackage[english, spanish]{babel}
\usepackage{fullpage} % changes the margin
\usepackage{graphicx}
\usepackage{amsmath}
\usepackage{enumitem}
\usepackage{chngcntr}
\usepackage{setspace}
\usepackage{url}
\usepackage{csquotes}
\usepackage{float}
\usepackage{verbatim}
\usepackage{tabularx}
\usepackage{amsmath}

% \usepackage{fancyhdr}

% \setlength{\headheight}{15pt}
% \pagestyle{fancy}

% \lhead{}
% \chead{}
% \rhead{\thepage}

% \lfoot{}
% \cfoot{}
% \rfoot{}

\counterwithin{figure}{section}
\renewcommand{\thesection}{\arabic{section}}
\renewcommand{\thesubsection}{\thesection.\arabic{subsection}}
\renewcommand{\baselinestretch}{1.5}

\usepackage[style=numeric, maxnames=6, minnames=3, backend=biber, parentracker=true, sorting=none]{biblatex}
\DefineBibliographyStrings{english}{%chktex-file 1 chktex-file 6
    andothers = {\em et\addabbrvspace al\adddot}
}
\addbibresource{./Bibliography/bibliography.bib}

\DeclareFieldFormat{labelnumberwidth}{[#1]\hfill}
\defbibenvironment{bibliography}
{\list
    {\printtext[labelnumberwidth]{%
            \printfield{prefixnumber}%
            \printfield{labelnumber}}}
    {\setlength{\labelwidth}{\labelnumberwidth}%
        \setlength{\leftmargin}{\labelwidth}%
        \setlength{\labelsep}{\biblabelsep}%
        \addtolength{\leftmargin}{\labelsep}%
        \setlength{\itemsep}{\bibitemsep}%
        \setlength{\parsep}{\bibparsep}}\renewcommand*{\makelabel}[1]{\hss##1}}
{\endlist} {\item}%

% \usepackage[style=apa]{biblatex}

\usepackage{array}
\usepackage{enumitem}

\usepackage{setspace}
\setlength{\parskip}{\baselineskip}

\begin{document}

\chapter*{Análisis de: ``La mujer del animal''}

\nocite{Pelicula}

``La mujer del animal'' (2016) es una película colombiana dirigida por Víctor Gaviria y
producida por Dani Goggel, que se centra en la vida de Amparo
\textit{(Natalia Polo)}, una joven que vive en un barrio pobre de Medellín
y que es víctima de la violencia y el abuso por parte de su marido, el
``Animal'' \textit{(Tito Alexander Gómez)}. Cabe mencionar que los
acontecimientos de la película suceden durante los años 70. Época en la que
Colombia vivió un período de violencia y conflictos sociales y políticos. 
En ese momento, el país estaba en medio de una guerra civil no declarada y 
las guerrillas marxistas se estaban consolidando en varias regiones del país. 

La película se basa en los estereotipos de género presentes
en la sociedad colombiana, donde los hombres son vistos
como violentos y dominantes, mientras que las mujeres son
vistas como sumisas y dependientes. Algunos de estos
estereotipos se mantienen por tradición, otros están
justificados a través de traumas o por sentimientos de
culpabilidad que están normalizados en la sociedad, que
harían a cualquiera dudar sobre las razones por las que se
permiten estos actos. El personaje del ``Animal'' es un
claro ejemplo de esta representación, ya que es retratado
como un hombre agresivo y violento.

Analizando el papel de la mujer en la película, se
evidencia una clara muestra de la opresión y la violencia
que sufren muchas mujeres en situaciones de pobreza y
vulnerabilidad. Amparo es un personaje que lucha por su
libertad y su autonomía, pero que se ve constantemente
sometida a la voluntad del ``Animal'', quien la trata como
un objeto y la obliga a prostituirse para ganar dinero.

Adicionalmente, enfatiza temáticas como la violencia, la
pobreza, la exclusión social y la corrupción. Estas
temáticas son relatadas como una ``realidad cotidiana'' en
los barrios pobres de Medellín, asimismo como en varias
zonas de Colombia, donde las personas viven en condiciones
precarias y son victimas de la delincuencia y la violencia.

En este sentido, presenta un contexto social y político que
es fundamental para entender la situación de los
personajes. La pobreza y la exclusión social son las
principales causas de la violencia y la delincuencia, y la
película muestra cómo las personas que viven en estas
condiciones se ven obligadas a tomar decisiones extremas
para sobrevivir. Y precisamente por esto, es que estas
problemáticas son naturalizadas y aceptadas por gran parte
de la población, llega un punto en que es tomado como
``normal''.

Asimismo, denuncia la corrupción y la falta de justicia en
Colombia. En la película, los personajes se enfrentan a un
sistema judicial que es incapaz de protegerlos, lo que los
obliga a buscar soluciones por su cuenta.

Además, también pone en evidencia cómo la violencia y la
opresión hacia las mujeres son perpetuadas por la falta de
educación y de oportunidades para las mujeres en Colombia.
Amparo, la protagonista de la película, es una mujer que no
ha tenido acceso a la educación ni a oportunidades
laborales, lo que la convierte en una presa fácil para la
violencia y el abuso.

La calidad cinematográfica de la película se destaca por su
realismo y crudeza al mostrar una realidad dura y difícil
de ver, pero necesaria para entender la situación de muchas
personas en Colombia. En este sentido, la película tiene
características que asemejan a un documental, como cuando
obligan a Amparo a prostituirse para ganar dinero o cuando
se escenifica el modus operandi de los secuestradores de
adolescentes. A pesar de esto, en esencia, sigue siendo un
drama psicológico que juega con las perspectivas y la
actuación de Natalia Polo es destacable, ya que logra
transmitir la angustia y el sufrimiento de su personaje de
manera muy convincente.

En conclusión, ``La mujer del animal'' es una película que
aborda problemáticas sociales y políticas importantes en
Colombia, como la violencia de género, la pobreza, la
exclusión social y la corrupción. La película muestra una
realidad cruda y dolorosa, pero que es necesaria para
entender la situación de muchas personas en Colombia. No
deja indiferente al espectador y que invita a reflexionar
sobre las problemáticas sociales y políticas que afectan a
Colombia. La película es una denuncia de la violencia y la
injusticia en Colombia, y un llamado a la reflexión y a la
acción para cambiar esta realidad.

\newpage

\printbibliography

\end{document}