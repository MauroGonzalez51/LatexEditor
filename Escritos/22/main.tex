\documentclass[letterpaper, 12pt]{report}

\usepackage[utf8]{inputenc}
\usepackage[english, spanish]{babel}

\usepackage{fullpage}
\usepackage{graphicx}
\usepackage{amsmath}
\usepackage{enumitem}
\usepackage{chngcntr}
\usepackage{setspace}
\usepackage{xurl}
\usepackage{csquotes}
\usepackage{float}
\usepackage{verbatim}
\usepackage{tabularx}
\usepackage{amsmath}
\usepackage{caption}
\usepackage{bm}
\usepackage{wrapfig}
\usepackage{siunitx}

\counterwithin{figure}{section}
\renewcommand{\thesection}{\arabic{section}}
\renewcommand{\thesubsection}{\thesection.\arabic{subsection}}
\renewcommand{\baselinestretch}{2}
\renewcommand{\thefigure}{\arabic{figure}}

\usepackage[style=apa, maxnames=6, minnames=3]{biblatex}
\DefineBibliographyStrings{english}{%chktex-file 1 chktex-file 6
      andothers = {\em et\addabbrvspace al\adddot}
}

\addbibresource{./Bibliography/bibliography.bib}

\DeclareCiteCommand{\textcite}
{\usebibmacro{prenote}}
{\usebibmacro{citeindex}%
      \printnames{labelname}%
      \setunit{\addcomma\addspace}%
      \printfield{year}%
      \setunit{\addcomma\addspace}%
      \printfield{title}}
{\multicitedelim}
{\usebibmacro{postnote}}

\DeclareCiteCommand{\cite}
{\usebibmacro{prenote}}
{\usebibmacro{citeindex}%
      (\printnames{labelname}%
      \setunit{\addcomma\addspace}%
      \printfield{year}%
      \setunit{\addcomma\addspace}%
      \printfield{title})}
{\multicitedelim}
{\usebibmacro{postnote}}

\usepackage{array}
\usepackage{enumitem}

\setlength{\parskip}{\baselineskip}

\newcommand{\bolditalic}[1]{\textbf{\textit{#1}}}

\renewcommand{\comment}[1]{{\small $\ll$#1$\gg$}}

% chktex-file 24

\begin{document}

\chapter*{Importancia de la Estadística en la Ingeniería de Sistemas}

\noindent\makebox[\linewidth]{\rule{\textwidth}{0.2pt}}

\begin{itemize}
      \item Mauro Gonzalez,~\textit{T00067622}
\end{itemize}

\noindent\makebox[\linewidth]{\rule{\textwidth}{0.2pt}}

La estadística descriptiva es una rama fundamental de la
estadística que se centra en la recopilación, ordenación,
representación y síntesis de información numérica con el
objetivo de poder transmitir de manera clara y eficaz sus
características fundamentales~\cite{Rodríguez_2023}. Sus
inicios se remontan a la antigüedad, siendo utilizada por
civilizaciones como la egipcia, la china y la griega,
principalmente en contextos relacionados con la
administración pública y el comercio. En la Edad Media, la
estadística continuó desarrollándose, pero fue en el siglo
XVII con el surgimiento de la teoría de probabilidades y el
método científico cuando adquirió una base más
formal~\cite{MonicaGhilardi}.

En una sociedad repleta de información, conocer de
estadística, se ha vuelto casi que una necesidad, la
evolución de la tasa de desempleo, el porcentaje de
descuento de los anuncios publicitarios, los medios
divulgando información sobre las intenciones de voto de los
ciudadanos, etc~\dots, es decir, la estadística se ha
convertido en una ciencia aplicada a lo cotidiano~\cite{Rodríguez_2023b}.

Con la contextualización ya en mente, ahora si se puede su
importancia en el campo en cuestión. Y es que en realidad
no están tan separadas el uno de la otra, me explico, la
Ingeniería de sistemas, o mas concretamente la informática,
es un medio en cuestión para la estadística.

Analizar volúmenes inmensos de datos cada dia seria un
trabajo complicado incluso para las empresas si no tuvieran
algún tipo de algoritmo que les permita automatizar todo
este proceso. ¿Como saben paginas como \textit{YouTube} que
videos recomendarle? ó ¿Que publicación recomendarle
mientras navega por \textit{Instagram}? Todo esto hace
parte de una recolección (recopilación) de sus gustos, para
luego en base a esto mostrar contenido que potencialmente
le podría gustar.

Claramente la estadística, no sirve para mostrarte
solamente que foto ver después, hay muchísimas aplicaciones
dentro de la ingeniería. Optimizar sistemas, impulsar la
innovación, o el análisis de datos en tiempo real son solo
algunas de las cosas que se pueden hacer. Aunque con todo
este tema, hay algo que no se puede dejar de lado, y es el
riesgo a la privacidad que todo esto supone, pero eso ya es
tema para después.

\nocite{Jeanluc_2023}
\nocite{Cutipa_2023}

\printbibliography

\end{document}