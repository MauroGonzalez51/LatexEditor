\documentclass[letterpaper, 12pt]{article}

\usepackage[utf8]{inputenc}
\usepackage[english, spanish]{babel}

\usepackage{newtxtext}

\usepackage{fullpage}
\usepackage{graphicx}
\usepackage{amsmath}
\usepackage{enumitem}
\usepackage{chngcntr}
\usepackage{setspace}
\usepackage{xurl}
\usepackage{csquotes}
\usepackage{float}
\usepackage{verbatim}
\usepackage{tabularx}
\usepackage{amsmath}
\usepackage{caption}
\usepackage{bm}
\usepackage{wrapfig}
\usepackage{siunitx}

\counterwithin{figure}{section}
\renewcommand{\thesection}{\arabic{section}}
\renewcommand{\thesubsection}{\thesection.\arabic{subsection}}
\renewcommand{\baselinestretch}{2}
\renewcommand{\thefigure}{\arabic{figure}}

\usepackage[style=apa, maxnames=6, minnames=3]{biblatex}
\DefineBibliographyStrings{english}{%chktex-file 1 chktex-file 6
      andothers = {\em et\addabbrvspace al\adddot}
}

\addbibresource{./Bibliography/bibliography.bib}

\usepackage{array}
\usepackage{enumitem}

\setlength{\parskip}{\baselineskip}

\newcommand{\bolditalic}[1]{\textbf{\textit{#1}}}

\DeclareSIUnit{\COP}{COP}
\newcommand{\cop}[1]{\$\SI{#1}{\COP}}

\DeclareSIUnit{\DOLLAR}{USD}
\newcommand{\dollar}[1]{\$\SI{#1}{\DOLLAR}}

\renewcommand{\comment}[1]{{\small $\ll$#1$\gg$}}

% chktex-file 24

\begin{document}

\section*{Análisis de caso de ética profesional}

\noindent\makebox[\linewidth]{\rule{\textwidth}{0.4pt}}

\begin{itemize}[label=$\triangleright$]
      \item Mauro González \textit{(T00067622)}
      \item German De Armas Castaño \textit{(T00068765)}
      \item Angel Vega Rodriguez \textit{(T00068186)}
      \item Juan Jose Osorio Ariza \textit{(T00067316)}
\end{itemize}

\noindent\makebox[\linewidth]{\rule{\textwidth}{0.4pt}}

\nocite{Batista_2022}
\nocite{Hoyos_2023}
\nocite{LilianaCarmen}
\nocite{vera2019revision}

% ! -------------------------------------------------------------------|>
% ! Introducción +  Contexto

El 27 de abril de 2017 parecía ser un día normal en la
bulliciosa ciudad de Cartagena de Indias. Sin embargo, en
cuestión de momentos, la vida de muchas familias cambiaría
para siempre debido a un incidente que según muchas
personas, ya se veía venir a lo lejos.

El incidente en cuestión fue el colapso del edificio
``Portal de Blas de Lezo II'', una construcción supervisada
por el grupo conocido como ``Clan Quiroz'', un clan
familiar originario del sur de Bolívar y dedicado a la
construcción, levantó al menos 36 edificios en zonas de
estrato medio en Cartagena de Indias, sin que mediaran
licencias o permisos. El desastre ocurrido en ``Portal de
Blas de Lezo II'' pronto revelaría una red de ilegalidades
asociadas con varias construcciones dirigidas por este
grupo, donde al menos 16 construcciones fueron levantadas
de forma precaria, desconociendo las normas de
sismorresistencia y con máximo ahorro en los materiales.

Este incidente, más allá de ser un desastre arquitectónico,
pone de manifiesto una serie de dilemas éticos que merecen
ser analizados. La pregunta que surge es: ¿Cómo pudo
suceder algo así? ¿Dónde estaban los controles y las
regulaciones que debían prevenir tal catástrofe? En el
siguiente análisis, intentaremos desentrañar las decisiones
éticas, o la falta de ellas, que llevaron a esta tragedia.
Examinaremos las responsabilidades de los constructores,
las autoridades locales y otros actores involucrados, y
reflexionaremos sobre las lecciones que podemos aprender
para prevenir que tales desastres no ocurran en el futuro.

% ! -------------------------------------------------------------------|>
% ! Justificar la caída del edificio = Falla estructural

Los fallos fueron diversos desde las columnas que no tenía
las unidades requeridas por la estructura hasta los tiempos
de construcción \comment{hay que tener en cuenta que fue
una obra que se construyó en menos de cinco meses cuando lo
normal es al menos un año}. El descaro del clan Quiroz fue
tanta que en la licencia falsa denota la construcción de
lotes de unos de 340 metros y el curador afirmó que la
normal parte vigente establece que deberían ser de 480.
Pretendían robar 140 metros de tierra de la forma más
irreal posible. Es innegable ver que todos los dilemas
éticos están conformados por un conflicto de intereses de
los constructores inescrupulosos, actuando sin miedo pues
saben que impunemente pueden adelantar sus obras violando
los requisitos de la ley sin ser sancionados o las obras
clandestinas demolidas el asunto es sencillo todo se
arregla con dinero dentro de los gastos de construcción se
incluye una suma destinada a pagar las picúas.

% ! -------------------------------------------------------------------|>
% ! Hablar desde la parte ética referente a los involucrados

A pesar de todo esto y la negligencia de los constructores,
no era necesario hacer un análisis estructural totalmente
detallado para darse cuenta de la situación. Este es el
caso de muchas personas externas, que de alguna u otra
forma estaban involucradas en la construcción del edificio.

Siguiendo esta linea, \textit{Libardo Castaño Castilla},
residente en la ciudad de Cartagena, desempeñaba su papel
como transportista de materiales. Desde su perspectiva, el
desplome era algo evidente, pero al considerarse como ajeno
al resultado, tenia limitada capacidad de acción.

\begin{quote}
      ``Todo el que sabía hacer su trabajo, o que al menos conocía
      sobre el tema, sabía que este edificio se iba a caer'' \\
      \textit{Libardo Castaño}
\end{quote}

Esta frase encapsula la preocupación y consciencia
compartida por quienes entendían la situación. Desde el
impacto en la dignidad humana, se evidencia cómo acciones
impulsadas por el interés propio, como en el caso de la
constructora, arrojan sombras de desconfianza y temor sobre
la comunidad. Esta situación resalta la importancia de
considerar y priorizar la seguridad y el bienestar de las
personas sobre cualquier interés individual o corporativo.

Al observar más allá de los escombros del edificio, emergen
las cicatrices invisibles pero profundas en el tejido
social. La negligencia en la construcción no solo
comprometió la estabilidad física del inmueble, sino que
también socavó la dignidad y seguridad de quienes lo
habitaban y de la comunidad en general. La confianza,
elemento fundamental en cualquier sociedad, se ve
fracturada cuando las acciones egoístas de unos pocos se
anteponen al bienestar colectivo. Este episodio oscuro deja
una lección resonante sobre la responsabilidad compartida
de salvaguardar la integridad de nuestras comunidades y
destaca la necesidad de construir sobre cimientos éticos
que prioricen la vida y el respeto mutuo.

En el contexto de la justicia, este incidente revela una
clara infracción a los principios fundamentales que
deberían regir la construcción y el desarrollo urbano. La
negligencia de los constructores al obviar señales
evidentes de peligro no solo constituye un acto de
irresponsabilidad individual, sino que también plantea
cuestionamientos sobre la integridad de las regulaciones y
controles existentes en el ámbito de la construcción.

La frase citada de Libardo Castaño, resuena como un llamado
a la responsabilidad compartida. Este episodio plantea
preguntas sobre la rendición de cuentas y la necesidad de
fortalecer los mecanismos que garanticen la seguridad y el
bienestar de la comunidad.

Este incidente destaca la importancia de un sistema de
justicia que no solo reaccione ante tragedias, sino que
también trabaje proactivamente para prevenir situaciones
peligrosas. Además, resalta la necesidad de considerar cómo
los actores individuales y las instituciones pueden ser
responsables ante la ley cuando su conducta pone en peligro
la vida y el bienestar de los demás.

\printbibliography

\end{document}