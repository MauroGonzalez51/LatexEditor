\documentclass[letterpaper, 12pt]{article}

\usepackage[utf8]{inputenc}
\usepackage[english, spanish]{babel}

\usepackage{newtxtext}

\usepackage{fullpage}
\usepackage{graphicx}
\usepackage{amsmath}
\usepackage{enumitem}
\usepackage{chngcntr}
\usepackage{setspace}
\usepackage{url}
\usepackage{csquotes}
\usepackage{float}
\usepackage{verbatim}
\usepackage{tabularx}
\usepackage{amsmath}
\usepackage{caption}
\usepackage{bm}
\usepackage{wrapfig}
\usepackage{siunitx}

\counterwithin{figure}{section}
\renewcommand{\thesection}{\arabic{section}}
\renewcommand{\thesubsection}{\thesection.\arabic{subsection}}
\renewcommand{\baselinestretch}{2}
\renewcommand{\thefigure}{\arabic{figure}}

\usepackage[style=numeric, maxnames=6, minnames=3, backend=biber, parentracker=true, sorting=none]{biblatex}
\DefineBibliographyStrings{english}{%chktex-file 1 chktex-file 6
      andothers = {\em et\addabbrvspace al\adddot}
}

\addbibresource{./Bibliography/bibliography.bib}

\usepackage{array}
\usepackage{enumitem}

\setlength{\parskip}{\baselineskip}

\newcommand{\bolditalic}[1]{\textbf{\textit{#1}}}

\DeclareSIUnit{\COP}{COP}
\newcommand{\cop}[1]{\$\SI{#1}{\COP}}

\DeclareSIUnit{\DOLLAR}{USD}
\newcommand{\dollar}[1]{\$\SI{#1}{\DOLLAR}}

\renewcommand{\comment}[1]{{\small $\ll$#1$\gg$}}

% chktex-file 24

\begin{document}

\section*{La lucha por la igualdad de género y diversidad en el sistema judicial colombiano}

\noindent\makebox[\linewidth]{\rule{\textwidth}{0.4pt}}

Valentina Daniela Del Rio Jimenez \textit{(T00081360)}

\noindent\makebox[\linewidth]{\rule{\textwidth}{0.4pt}}

Al explorar la narrativa de Penélope, ecos históricos
emergen en las páginas del tiempo, revelando las
persistentes barreras que las mujeres han enfrentado para
ingresar al ámbito público. A pesar de los avances en la
sociedad, los datos proporcionados por María Adelaida
Ceballos ponen al descubierto de manera cruda la realidad
contemporánea del sistema judicial en Colombia. La escasa
presencia de mujeres en roles de liderazgo, como evidencian
las estadísticas de la fiscalía general y diversas
instancias judiciales, destapa un desequilibrio
profundamente arraigado.

A pesar de los notables avances que ha experimentado la
sociedad en diversas áreas, el análisis crítico de las
cifras actuales revela que la justicia colombiana permanece
como un terreno donde las mujeres, lejos de haber superado
las barreras históricas, continúan siendo sombras de su
pasado. Este fenómeno evidencia la durabilidad de las
configuraciones o patrones existentes que limitan la plena
participación y representación femenina en el ámbito
judicial, evidenciando que el camino hacia la igualdad de
género en este sector está lejos de ser completo.

A lo largo de su historia, la fiscalía general de Colombia
ha tenido una escasez notable de liderazgo femenino, lo que
evidencia la falta de equidad de género en las posiciones
de poder. Durante el año 2019, María Paulina Riveros asumió
el cargo de fiscal general de manera interina, lo cual
resalta como un caso excepcional y efímero de liderazgo
femenino en esta institución. Sin embargo, este ejemplo
específico demuestra claramente que las mujeres siguen
siendo subrepresentadas en roles de liderazgo en el ámbito
judicial, ya que su tiempo en el cargo fue extremadamente
breve, apenas 24 horas. Fue nombrada vicefiscal general el
14 de mayo de 2019, pero renunció al día siguiente, el 15
de mayo del mismo año. En la actualidad, la fiscalía
general, una de las instituciones fundamentales del sistema
judicial, desde que se creó en 1991, la mayoría de los que
han estado a cargo de esta entidad han sido hombres, en
total 14 lideres masculinos han ocupado el puesto, lo que
deja claro que las mujeres han tenido muy poca
representación en la dirección de esta organización.

Siguiendo esta línea de análisis, nos percatamos de que
esta falta de representación no se limita únicamente al
género; se extiende hacia otras comunidades marginadas,
desde personas afrodescendientes e indígenas hasta aquellos
en situaciones de discapacidad. El panorama, descrito por
Daniel Gómez Mazo, pinta una imagen desalentadora de las
altas esferas judiciales, que se aleja considerablemente de
la diversidad que caracteriza a la población colombiana. Es
esencial enfrentar los desafíos estructurales existentes y
avanzar hacia un sistema de justicia verdaderamente
inclusivo y representativo para todos los ciudadanos. Esto
implica tomar medidas concretas, tales como, asegurar la
igualdad de acceso a la justicia y establecer políticas que
eliminen barreras económicas, permitiendo una participación
plena en el sistema judicial. Además, se intercede por la
adopción de decisiones inclusivas, participativas y
representativas, promoviendo la diversidad en la toma de
decisiones judiciales con la representación de diversos
géneros, etnias y grupos sociales. El acceso público a la
información es fundamental para asegurar la transparencia y
la rendición de cuentas de las instituciones. Para
lograrlo, es necesario contar con instituciones eficaces,
responsables y transparentes, que garanticen que la
información esté disponible de manera clara, completa y
actualizada para todas las personas, sin restricciones
injustificadas.

Cuando excluimos o limitamos la voz y la presencia de las
mujeres, perdemos la oportunidad de construir una justicia
que realmente refleje a todos, que sea inclusiva y sensible
a las necesidades y expectativas de la gente. Es esencial
reconocer la importancia de la diversidad para lograr una
sociedad más equitativa y justa. La referencia a
Schopenhauer destaca una perspectiva que considera que las
mujeres deberían llevar vidas más ``silenciosas, más
insignificantes y dulces'' en comparación con los hombres.
Es importante señalar que estas opiniones reflejan
estereotipos de género y roles tradicionales que han sido
desafiados y cuestionados a lo largo del tiempo. La lucha
que se presenta actualmente por la igualdad de género ha
buscado derribar estos estereotipos y reconocer el valor y
la contribución de las mujeres en todos los ámbitos de la
sociedad.

La idea de que las mujeres deberían llevar vidas más
serenas o restringidas en comparación con los hombres se
fundamenta en prejuicios fijados que han favorecido la
discriminación y la marginación de las mujeres en diversos
ámbitos, como la educación, la política y el ámbito
laboral. A lo largo de la historia, las mujeres han librado
una constante batalla contra estas expectativas,
esforzándose día a día por desafiar las restricciones
impuestas y demostrar su capacidad y valor en todos los
aspectos de la vida. Este enfrentamiento con estereotipos
de género es esencial para propiciar una transformación
sociocultural que permita reconocer y respetar la
diversidad de habilidades y contribuciones de las mujeres
en la sociedad.

Las estadísticas presentadas no son un fenómeno nuevo, sino
más bien un reflejo persistente de las barreras históricas
que las mujeres han enfrentado en su búsqueda de posiciones
de liderazgo en el sistema judicial. Estas cifras son un
eco doloroso de épocas pasadas, evidenciando que las
mujeres continúan enfrentando desafíos significativos para
acceder a roles de liderazgo en este ámbito.

La lucha por la igualdad de género y la diversidad en el
sistema judicial requiere medidas concretas, desde la
eliminación de barreras económicas hasta la promoción de la
diversidad en la toma de decisiones judiciales. Reconocer y
desafiar los estereotipos de género y los roles
tradicionales es esencial para propiciar una transformación
sociocultural que valore y respete las contribuciones de
todas las personas en la sociedad.

En última instancia, la reflexión sobre la representación
de género en el sistema judicial colombiano nos insiste a
considerar la importancia de la diversidad para la
construcción de una justicia verdaderamente inclusiva. Solo
a través de la eliminación de barreras y la promoción de la
diversidad en todos los niveles, podremos avanzar hacia un
sistema judicial que refleje y represente a toda la
sociedad de manera equitativa y justa.

\nocite{Colprensa_2016}
\nocite{FiscalíaGeneralNación_2022}
\nocite{Vacía_Vacía_2023}
\nocite{agenda}

\printbibliography

\end{document}