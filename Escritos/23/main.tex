\documentclass[conference]{IEEEtran}
\IEEEoverridecommandlockouts{}

\usepackage{cite}
\usepackage{amsmath,amssymb,amsfonts}
\usepackage{algorithmic}
\usepackage{graphicx}
\usepackage{textcomp}
\usepackage{xcolor}
\usepackage{tabularx}
\usepackage{float}

\def\BibTeX{{\rm B\kern-.05em{\sc i\kern-.025em b}\kern-.08em
    T\kern-.1667em\lower.7ex\hbox{E}\kern-.125emX}}
\begin{document}

\title{NVIDIA y su relación con el mercado Taiwanés\\}

\author{\IEEEauthorblockN{1\textsuperscript{st} Valentina Del Rio}
	\IEEEauthorblockA{\textit{Universidad Tecnologica de Bolivar} \\
		\textit{UTB}\\
		Cartagena, Colombia \\
		vdelrio@utb.edu.co}

	\and
	\IEEEauthorblockN{1\textsuperscript{st} Juliana Arias}
	\IEEEauthorblockA{\textit{Universidad Tecnologica de Bolivar} \\
		\textit{UTB}\\
		Cartagena, Colombia \\
		thornej@utb.edu.co}

	\and
	\IEEEauthorblockN{1\textsuperscript{st} Valentina Niño}
	\IEEEauthorblockA{\textit{Universidad Tecnologica de Bolivar} \\
		\textit{UTB}\\
		Cartagena, Colombia \\
		ninov@utb.edu.co}

	\and
	\IEEEauthorblockN{1\textsuperscript{st} Maria Grau}
	\IEEEauthorblockA{\textit{Universidad Tecnologica de Bolivar} \\
		\textit{UTB}\\
		Cartagena, Colombia \\
		graum@utb.edu.co}

	\and
	\IEEEauthorblockN{1\textsuperscript{st} Valeria Santamaria}
	\IEEEauthorblockA{\textit{Universidad Tecnologica de Bolivar} \\
		\textit{UTB}\\
		Cartagena, Colombia \\
		vsantamaria@utb.edu.co}
}

\maketitle

\bibliographystyle{IEEEtran}

% chktex-file 44 
% chktex-file 24

\begin{abstract}
	NVIDIA Corporation, founded in 1993 by Jen-Hsun Huang, Chris
	Malachowsky, and Curtis Priem in Santa Clara, California,
	is a leading technology company with a significant impact
	on the computing and artificial intelligence industries.
	Specializing in the design and development of graphics
	processing units (GPUs), NVIDIA has revolutionized graphics
	in personal computers, workstations, and servers, affecting
	video games, 3D designs, scientific simulations, and more. The
	company has also played a crucial role in advancing artificial
	intelligence, with its GPUs being integral to deep learning
	and large-scale data processing tasks. NVIDIA has developed
	libraries and tools that empower researchers and developers
	to harness the power of AI.\@{}

	This paper constitutes financial analysis comparing the
	progress of NVIDIA's business plan with the Taiwanese
	market. The analysis examines NVIDIA's growth, strategic
	initiatives, and markets performance, highlighting how
	these elements align with and diverge from trends and
	developments in Taiwan's technology sector.
\end{abstract}

\begin{IEEEkeywords}
	NVIDIA, financial analysis, Taiwanese market
\end{IEEEkeywords}

\nocite{Trazadores_cúbicos_2014}
\nocite{Chapra_Canale}

\section{Introducción}

NVIDIA Corporation, fundada en 1993 por Jen-Hsun Huang, Chris
Malachowsky y Curtis Priem en Santa Clara, California, es una
empresa líder en tecnología que ha dejado una huella
significativa en la industria de la informática y la
inteligencia artificial. Se especializa en el diseño y
desarrollo de unidades de procesamiento de gráficos (GPU),
así como en soluciones de inteligencia artificial.

NVIDIA es conocida por su innovación en las GPU.~Estas unidades
son esenciales para el procesamiento gráfico en computadoras
personales, estaciones de trabajo y servidores, revolucionando
los gráficos en videojuegos, diseño 3D, simulaciones científicas
y más.

La empresa también ha desempeñado un papel fundamental en
el avance de la inteligencia artificial. Sus GPU se utilizan
en tareas de aprendizaje profundo y procesamiento de datos
masivos; ha desarrollado bibliotecas y herramientas que
permiten a los investigadores y desarrolladores aprovechar
la potencia de la IA.\@{}

\section{Historia}

NVIDIA, una empresa líder en tecnología de procesamiento gráfico
ha tenido un gran impacto en la industria desde que fue fundada
en 1993 en Silicon Valley, California. En sus primeros años,
NVIDIA se enfocó en desarrollar soluciones gráficas para
computadoras personales, previendo la creciente necesidad de
gráficos avanzados en el mercado emergente de los videojuegos.

En 1999, NVIDIA marcó un hito significativo en la historia de la
industria con el lanzamiento de su GPU GeForce 256. NVIDIA se
estableció como líder en rendimiento gráfico gracias a la
introducción de esta revolucionaria GPU con capacidades sin
precedentes.

Una empresa líder en el mercado de tarjetas gráficas. Desde ese
momento, la empresa ha continuado innovando y ampliando su alcance,
incursionando en campos como la inteligencia artificial, el cómputo
de alto rendimiento y la conducción autónoma.

La introducción de la arquitectura CUDA en 2006 fue un momento
crucial en la historia de NVIDIA, ya que permitió a las GPUs
llevar a cabo tareas de computación general, abriendo nuevas posibilidades en campos como la ciencia, la investigación y el aprendizaje profundo.

NVIDIA ha lanzado innovaciones recientes, como la arquitectura
Turing en 2018, que trajo tecnologías avanzadas como el trazado
de rayos en tiempo real y el aprendizaje profundo acelerado por
GPU.~La serie GeForce RTX 30, lanzada en 2020,
estableció nuevos estándares de rendimiento gráfico para
juegos y aplicaciones de creación de contenido, consolidando
aún más el liderazgo de NVIDIA en la industria.

Su historia, tal como se presenta en su línea de tiempo
corporativa, abarca varias décadas de innovación y avances
en la industria de la tecnología, desde 1993 hasta el 2020
desde su fundación.

\begin{figure}[H]
	\begin{center}
		\includegraphics[width=\linewidth]{./Images/LineaTiempo.png}
		\caption{Linea de tiempo de NVIDIA}
	\end{center}
\end{figure}

\section{Objetivos y vision de la empresa}

La compañía NVIDIA Corporation es reconocida por ofrecer soluciones
en gráficos, computación y redes a nivel global. La oferta incluye
productos como las tarjetas gráficas GeForce, perfectas para
jugadores y usuarios de PC, y el servicio de transmisión de
juegos GeForce NOW.~Además, ofrecen soluciones para
profesionales, como las GPU Quadro/NVIDIA RTX, que se
utilizan en estaciones de trabajo para tareas gráficas
exigentes. También ofrecen una variedad de productos
para centros de datos y redes, como plataformas
informáticas y de redes, y se especializan en el área
de la conducción automatizada con su plataforma NVIDIA DRIVE.\@{}

Los videojuegos, la visualización profesional, los centros
de datos y la industria automotriz son algunos de los
principales sectores que se benefician de los productos de NVIDIA.~La
compañía vende sus productos a varios tipos de clientes,
como fabricantes de equipos originales, distribuidores y
proveedores de servicios en la nube.

\section{Marco Teorico}

\subsection*{Posicionamiento de NVIDIA en el mercado de tecnología y GPU}

\begin{itemize}
	\item Revisión de literatura sobre la evolución de NVIDIA
	      como líder en el mercado de procesamiento gráfico (GPU) y
	      tecnologías relacionadas.

	\item Análisis de las estrategias de marketing,
	      investigación y desarrollo que han contribuido al éxito de
	      NVIDIA en el mercado.

	\item Exploración de la relación entre la innovación
	      tecnológica, la calidad del producto y la percepción de la
	      marca en la posición competitiva de NVIDIA.\@{}
\end{itemize}

\subsection*{Mercados más y menos significativos para NVIDIA}

\begin{itemize}
	\item Identificación y análisis de los mercados verticales más
	      importantes para NVIDIA, como la inteligencia artificial, los
	      videojuegos, la computación en la nube, el automóvil autónomo
	      y la visualización profesional.

	\item Evaluación de oportunidades y desafíos en cada mercado,
	      incluyendo factores económicos, regulatorios y tecnológicos.

	\item Investigación de los mercados emergentes y nichos
	      de mercado que podrían ofrecer nuevas oportunidades de
	      crecimiento para NVIDIA, así como los mercados que pueden
	      estar disminuyendo en importancia.
\end{itemize}

\subsection*{Relación entre NVIDIA y el mercado Taiwanés}

\begin{itemize}
	\item Examen de la colaboración histórica entre NVIDIA y las
	      empresas taiwanesas en la fabricación de componentes
	      electrónicos, como chips de GPU y tarjetas gráficas.

	\item Análisis de la integración vertical y horizontal de la
	      cadena de suministro de NVIDIA en Taiwán, incluyendo
	      relaciones con fabricantes de semiconductores,
	      ensambladores de tarjetas gráficas y otros proveedores de
	      tecnología.

	\item Investigación de las políticas gubernamentales,
	      las condiciones económicas y las dinámicas empresariales en
	      Taiwán que pueden afectar la relación entre NVIDIA y el mercado
	      taiwanés.
\end{itemize}

\subsection*{Implicaciones para el futuro de NVIDIA}

\begin{itemize}
	\item Discusión de las oportunidades estratégicas y los riesgos
	      potenciales que enfrenta NVIDIA en el contexto de su posición
	      en el mercado y sus relaciones con Taiwán y otros mercados clave.

	\item Propuesta de recomendaciones para NVIDIA basadas en
	      las tendencias identificadas en la investigación, incluyendo
	      áreas de expansión, colaboración empresarial y gestión de riesgos.
\end{itemize}

\section{Posición en el mercado}

NVIDIA es líder en varios mercados clave, incluidos los videojuegos,
la inteligencia artificial, la computación de alto rendimiento y
la automoción. Su dominio en estos sectores le proporciona una
posición sólida en la industria de la tecnología y le permite
influir en la dirección futura de la misma

En el primer trimestre de 2024, Nvidia fue responsable del
11.02\% del rendimiento del índice de rendimiento total
del S\&P 500, seguido por Microsoft con un 3.69\%, Meta
con un 3.20\% y Amazon con un 2.83\%; lo que nos permite
decir que Nvidia ha sido el contribuyente más crítico a las
ganancias del mercado en 2024.

La posición económica de NVIDIA es bastante sólida y ha sido
así durante varios años y esto se refleja en sus estados financieros.

Ingresos y crecimiento: NVIDIA ha experimentado un crecimiento
constante en sus ingresos a lo largo de los años, impulsado por
su liderazgo en tecnologías de procesamiento gráfico (GPU) y
su expansión hacia áreas como inteligencia artificial, centros
de datos y automoción. Su capacidad para diversificar sus ingresos
ha contribuido a un crecimiento estable y significativo.

Rentabilidad: esta empresa ha mantenido márgenes de ganancia
saludables, lo que indica su eficiencia operativa y su capacidad
para generar beneficios a partir de sus operaciones comerciales.
Esto ha sido respaldado por su enfoque en productos de alto
rendimiento y soluciones tecnológicas innovadoras.

Balance financiero: NVIDIA tiene un balance financiero sólido,
con una posición de efectivo estable y una gestión prudente de
la deuda. Esto le proporciona flexibilidad financiera para
invertir en investigación y desarrollo, realizar adquisiciones
estratégicas y realizar recompras de acciones para generar valor
para los accionistas.

\begin{figure}[H]
	\begin{center}
		\includegraphics[width=\linewidth]{./Images/Anexo2.png}
		\caption{}
	\end{center}
\end{figure}

\subsubsection*{Mercados más significativos para NVIDIA}

\begin{itemize}
	\item Estados Unidos:
	      NVIDIA tiene su sede en Santa Clara, California,
	      Estados Unidos. Aquí se hace gran parte de su investigación,
	      desarrollo y operaciones.

	\item Taiwán:
	      Taiwán es un país importante para NVIDIA debido a su
	      papel en la fabricación de componentes electrónicos.
	      NVIDIA tiene una presencia significativa en Taiwán,
	      especialmente en el ámbito de las GPU y la tecnología.

	\item China:
	      China es un mercado crucial para NVIDIA debido a su
	      tamaño y crecimiento económico. La empresa tiene
	      operaciones y colaboraciones en China para atender a
	      la creciente demanda de tecnología.

	\item Japón:
	      Japón es un país líder en tecnología y un mercado
	      importante para NVIDIA.~La empresa trabaja con socios
	      japoneses en áreas como la inteligencia artificial,
	      la robótica y la computación de alto rendimiento.

	\item Alemania:
	      Alemania es un centro tecnológico en Europa y un
	      mercado estratégico para NVIDIA.~La empresa colabora
	      con instituciones académicas y empresas alemanas en
	      proyectos de investigación y desarrollo.

	\item Reino Unido:
	      NVIDIA tiene una presencia significativa en el Reino
	      Unido, especialmente en áreas como la inteligencia
	      artificial, la visualización profesional y la computación
	      de alto rendimiento.

	\item India:
	      India es un mercado emergente con un gran potencial
	      para NVIDIA.~La empresa tiene oficinas y colaboraciones
	      en India para atender a las necesidades tecnológicas del país.

	\item Francia:
	      Francia es otro país europeo importante para NVIDIA.~La
	      empresa trabaja con socios franceses en proyectos
	      de investigación, desarrollo y aplicaciones tecnológicas.

\end{itemize}

\subsection*{Mercados menos significativos para NVIDIA (no tiene lazos comerciales):}

\begin{itemize}
	\item Singapur:
	      Aunque Singapur es un país pequeño, ha desempeñado un
	      papel importante en el éxito de NVIDIA.~En el tercer trimestre de
	      ventas de 2023, Singapur ocupó el cuarto lugar en los rankings de ventas
	      de NVIDIA, contribuyendo al 15% de sus ingresos. Esto se debe en parte a su enfoque en el procesamiento de datos y la tecnología.

	\item Emiratos Árabes Unidos (EAU):
	      Aunque EAU no es un mercado central
	      para NVIDIA, no se menciona con frecuencia como un destino importante
	      para la empresa. Sin embargo, su enfoque en la inteligencia artificial y
	      la tecnología podría cambiar esto en el futuro.

	\item Holanda (Países Bajos):
	      Aunque Holanda es favorable para
	      emprendedores, no se destaca como un mercado crucial para NVIDIA.~Sin
	      embargo, su economía fuerte y políticas empresariales sólidas podrían
	      atraer más atención en el futuro.

	\item Finlandia:
	      A pesar de ser un centro tecnológico, Finlandia no se
	      menciona ampliamente como un mercado clave para NVIDIA.~Sin embargo,
	      su enfoque en la educación empresarial y la innovación podría
	      cambiar esta percepción.

	\item Canadá:
	      Aunque Canadá es un país tecnológicamente avanzado, no se
	      considera uno de los mercados más relevantes para NVIDIA.~Sin embargo, su
	      comunidad de investigación y desarrollo sigue siendo un activo importante.

\end{itemize}

\section{Innovación tecnológica}

NVIDIA es reconocida por su innovación en tecnología de GPU y su
capacidad para desarrollar soluciones avanzadas que impulsan el
progreso en varias industrias. Esta capacidad para innovar y
mantenerse a la vanguardia de la tecnología contribuye a su
fortaleza económica y competitiva.

\begin{itemize}
	\item GPU para videojuegos:
	      NVIDIA ha sido pionera en el desarrollo de GPUs de alto rendimiento
	      para la industria de los videojuegos. Sus tarjetas gráficas GeForce
	      son reconocidas por su potencia, rendimiento y calidad de gráficos,
	      lo que ha contribuido mucho a avanzar en los gráficos de videojuegos
	      y a la experiencia de juego inmersa.

	\item Computación de alto rendimiento (HPC):
	      NVIDIA ha desarrollado GPUs especializadas para aplicaciones de HPC,
	      permitiendo un procesamiento masivo de datos y un rendimiento
	      excepcional en aplicaciones científicas, de investigación y de
	      análisis de datos. Su arquitectura CUDA
	      (Compute Unified Device Architecture) ha sido fundamental en este
	      ámbito.

	\item Inteligencia Artificial (IA) y Aprendizaje Automático (ML):
	      NVIDIA ha ampliado su enfoque más allá de los gráficos para
	      convertirse en un jugador importante en el campo de la IA y
	      el ML.~Sus GPUs aceleradoras, como la serie NVIDIA Tesla y las
	      unidades de procesamiento tensorial (TPU), son ampliamente utilizadas
	      en aplicaciones de entrenamiento y ejecución de modelos de IA/ML.\@{}

	\item Centros de datos y computación en la nube:
	      NVIDIA ha desarrollado soluciones específicas para aplicaciones
	      en centros de datos y entornos de computación en la nube. Sus
	      GPUs Tesla y su plataforma de computación NVIDIA DGX son
	      utilizadas por empresas y proveedores de servicios en la nube para
	      acelerar el procesamiento de datos y el entrenamiento de modelos
	      de IA a escala.

	\item Automoción:
	      NVIDIA ha incursionado en el mercado automotriz con su plataforma
	      NVIDIA DRIVE, que ofrece soluciones avanzadas de procesamiento
	      de datos para vehículos autónomos y sistemas de asistencia al
	      conductor. Su tecnología se utiliza en sistemas de conducción
	      autónoma de nivel 2 a nivel 5, así como en sistemas de
	      infoentretenimiento y visualización avanzada.

	\item Realidad Virtual y Aumentada:
	      NVIDIA ha desarrollado tecnología para impulsar experiencias
	      inmersivas de realidad virtual (VR) y aumentada (AR). Sus GPUs
	      y software están presentes en dispositivos de VR/AR de alta gama,
	      lo que permite gráficos de alta calidad y un rendimiento fluido
	      en aplicaciones de entretenimiento, diseño y formación.

\end{itemize}

\section{Análisis de la relación entre NVIDIA y el mercado Taiwanés}

Taiwán desempeña un papel crucial en la cadena de suministro
global de NVIDIA, principalmente debido a su sólida
infraestructura en la industria de semiconductores. La isla
alberga algunas de las fábricas más avanzadas del mundo, como
las de Taiwán Semiconductor Manufacturing Company (TSMC), que
es un socio clave para NVIDIA.\@{}

\subsection*{Lazos históricos de NVIDIA}

La relación entre NVIDIA y Taiwán tiene raíces profundas, destacadas
por el hecho de que el fundador de NVIDIA, Jensen Huang, nació en
Taiwán. Este vínculo personal ha influenciado la conexión de la
empresa con la isla, estableciendo una sede asiática en 1995. La
temprana decisión de ubicarse en Taiwán subraya la importancia
estratégica de la isla en el crecimiento de NVIDIA.\@{}

\subsection*{Presencia Significativa en Taiwán}

\begin{itemize}
	\item Centro de I+D:\@{}
	      Taiwán alberga uno de los centros de investigación y
	      desarrollo más importantes de NVIDIA.~Este centro no solo
	      diseña y prueba chips y tecnologías de vanguardia, sino que
	      también desempeña un papel crucial en la innovación continua
	      de la empresa. El enfoque en I+D en Taiwán permite a NVIDIA
	      mantenerse constante en un mercado altamente competitivo.

	\item Producción:
	      La isla es un eje central en la cadena de suministro global
	      de NVIDIA.~Las fábricas en Taiwán ensamblan tarjetas gráficas
	      y otros componentes esenciales. La proximidad de estas
	      instalaciones a otros actores clave de la industria de
	      semiconductores en la región facilita una logística eficiente
	      y reduce tiempos de respuesta en la producción.

\end{itemize}

\subsection*{Sus diferentes colaboraciones con otros entes}

NVIDIA colabora estrechamente con universidades e institutos de
investigación en Taiwán. Estas colaboraciones son vitales para
impulsar la innovación en áreas emergentes como la inteligencia
artificial (IA) y la computación en la nube. Estas asociaciones no
solo fomentan la innovación, sino que también crean un ecosistema de
talento especializado en tecnologías avanzadas.

\subsection*{Impacto Económico}

\begin{itemize}
	\item Generación de Empleo:
	      NVIDIA es un empleador significativo en Taiwán,
	      proporcionando miles de empleos en sus instalaciones.
	      Esta presencia no solo crea oportunidades laborales, sino
	      que también ayuda a desarrollar una fuerza laboral altamente
	      cualificada en el ámbito tecnológico.

	\item Contribución al PIB:\@{}
	      Las operaciones de NVIDIA en Taiwán contribuyen
	      notablemente al producto interno bruto (PIB) de
	      la isla. La inversión en infraestructura, salarios y
	      colaboraciones locales tiene un efecto multiplicador en
	      la economía taiwanesa.

\end{itemize}

\subsection{Mas allá de lo que es el Hardware}

\begin{itemize}
	\item IA y Computación en la Nube:
	      NVIDIA está trabajando con empresas taiwanesas para
	      desarrollar soluciones avanzadas de IA y computación en
	      la nube. Estas colaboraciones están transformando diversas
	      industrias, como la atención médica, el transporte y la
	      manufactura, posicionando a Taiwán como un líder en
	      tecnología avanzada.

	\item Startups y Emprendimiento:
	      NVIDIA apoya activamente el ecosistema de startups en
	      Taiwán a través de programas de inversión y mentoría.
	      Este apoyo fomenta la innovación y el emprendimiento,
	      creando un entorno vibrante para nuevas empresas tecnológicas.

\end{itemize}

\subsection*{Retos y Oportunidades}

\begin{itemize}
	\item Competencia:
	      La competencia en el mercado de semiconductores es
	      intensa. NVIDIA enfrenta desafíos constantes de
	      competidores como AMD e Intel. Sin embargo, su fuerte
	      presencia en I+D y su capacidad de innovación continua
	      le otorgan una ventaja competitiva.

	\item Dependencia del Mercado Chino:
	      Las restricciones impuestas por Estados Unidos a las
	      exportaciones de tecnología a China han tenido un impacto
	      significativo en el negocio de NVIDIA en la región,
	      incluido Taiwán. Este desafío resalta la necesidad de
	      diversificar mercados y buscar oportunidades más allá de China.
\end{itemize}


\section{Ventaja comparativa y comercio}

Taiwán tiene una posición dominante en la industria de semiconductores, lo
que está propiamente vinculado al comercio y a la ventaja comparativa de
Taiwán en la cadena de suministro global de NVIDIA.~Taiwán no solo sobresale
como líder mundial en la fabricación de chips, sino que también ha creado un
ecosistema sólido para respaldar a las principales empresas tecnológicas del
mundo, incluyendo NVIDIA.\@{}

\subsection*{Exportaciones de Semiconductores}

Uno de los principales exportadores mundiales de semiconductores es Taiwán.
Sus ingresos comerciales dependen en gran medida de las exportaciones de
circuitos integrados (ICs). En el año 2020, alrededor del 60\% de las
exportaciones totales de Taiwán pertenecieron a productos electrónicos
y componentes.

\subsection*{Relaciones Comerciales con NVIDIA}

Las GPUs fabricadas en Taiwán son exportadas globalmente, y NVIDIA es uno de
los principales clientes de TSMC.~Esto abarca mercados importantes como
Estados Unidos, Europa y Asia, donde la demanda de componentes de alto
rendimiento para aplicaciones de inteligencia artificial, videojuegos y
centros de datos es considerable.

\section{Ventaja Comparativa}

\subsection*{Tecnología de Fabricación Avanzada}

\begin{itemize}
	\item TSMC y la Tecnología de Procesos:
	      La ventaja competitiva
	      significativa de NVIDIA se deriva de la capacidad de TSMC para
	      fabricar chips utilizando procesos avanzados de 7nm y 5nm. La
	      creación de GPUs más rápidas y eficientes es crucial para
	      aplicaciones de alto rendimiento, proceso que se logra mediante
	      estos procesos.

	\item Innovación Continua:
	      Las empresas taiwanesas aseguran
	      que mantengan su liderazgo tecnológico al seguir
	      invirtiendo en I+D. Por ejemplo, TSMC invierte anualmente
	      miles de millones de dólares en el desarrollo de nuevas
	      tecnologías de fabricación.

\end{itemize}

\subsection*{Eficiencia y Calidad}

\begin{itemize}
	\item Mano de Obra Calificada:
	      En ingeniería y manufactura de semiconductores, Taiwán tiene
	      una fuerza laboral altamente calificada. La calidad y
	      confiabilidad de los productos se ven influenciadas
	      significativamente por la experiencia acumulada en la producción
	      de componentes electrónicos de alta precisión.

	\item Infraestructura Logística: La entrega rápida y
	      eficiente de productos a los mercados globales se logra en
	      Taiwán gracias a su infraestructura logística bien desarrollada,
	      lo que reduce tanto los tiempos de respuesta como los costos
	      asociados.

\end{itemize}

\subsection*{Impacto Global}

\subsubsection*{Dependencia Global de Taiwán}

\begin{itemize}
	\item Cadena de Suministro Global:
	      El suministro de semiconductores avanzados por parte de
	      Taiwán es crucial para numerosas industrias tecnológicas a
	      nivel global. Esto abarca tanto a NVIDIA como a otros grandes
	      de la tecnología, como Apple, AMD y Qualcomm.

	\item Seguridad Económica: La importancia estratégica
	      de la isla en la economía mundial se destaca por la
	      dependencia global de los semiconductores fabricados en Taiwán.
	      Cualquier interrupción en su cadena de suministro puede causar
	      impactos significativos en varias industrias a nivel mundial.
\end{itemize}

\section{Expansión en Mercados Emergentes}

\begin{itemize}
	\item Estrategia:
	      NVIDIA ha descubierto y aprovechado las oportunidades en mercados
	      emergentes, donde la demanda de tecnologías avanzadas está
	      creciendo rápidamente. La estrategia incluye la expansión
	      geográfica y la penetración en nuevos segmentos de mercado.

	\item GeForce NOW y Cloud Gaming: Un servicio de juegos en
	      la nube llamado GeForce NOW ha sido lanzado por NVIDIA, el
	      cual brinda a los usuarios la oportunidad de jugar títulos de
	      alta gama sin tener que invertir en hardware costoso. NVIDIA
	      abre un nuevo mercado, alcanzando a usuarios que prefieren jugar
	      en la nube.

	\item Mercado de AI y Machine Learning: NVIDIA ha ampliado
	      su presencia en mercados emergentes como la inteligencia
	      artificial y el aprendizaje automático, proporcionando hardware
	      y software que simplifican el desarrollo de aplicaciones
	      avanzadas en estos campos.

\end{itemize}

\section{Conclusión}

El análisis de NVIDIA revela una empresa que ha sido líder en tecnología
de procesamiento gráfico desde su fundación en 1993. Su
historia está marcada por hitos como el lanzamiento de la GPU
GeForce 256 y la introducción de la arquitectura CUDA, que amplió
el alcance de las GPU hacia la computación general.

NVIDIA ha diversificado sus ingresos y ha mantenido una posición
financiera sólida, con un crecimiento constante en ingresos y
márgenes de ganancia saludables. Tiene una fuerte presencia global
y colabora estrechamente con socios en mercados clave como Estados
Unidos, Taiwán, China y Japón.

La relación entre NVIDIA y el mercado taiwanés destaca la importancia
de Taiwán en la cadena de suministro tecnológico global y el papel
clave que desempeña en la fabricación de componentes electrónicos.
NVIDIA está bien posicionada para liderar la industria tecnológica
gracias a su capacidad de innovación,
adaptación y crecimiento sostenido.


\bibliography{./Bibliography/bibliography.bib}

\end{document}
