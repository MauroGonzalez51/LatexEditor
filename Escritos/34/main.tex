\documentclass[letterpaper, 12pt]{article}

\usepackage[utf8]{inputenc}
\usepackage[english, spanish]{babel}
% \usepackage{newtxtext}
\usepackage{fullpage}
\usepackage{graphicx}
\usepackage{amsmath}
\usepackage{enumitem}
\usepackage{chngcntr}
\usepackage{setspace}
\usepackage{url}
\usepackage{csquotes}
\usepackage{float}
\usepackage{verbatim}
\usepackage{tabularx}
\usepackage{amsmath}
\usepackage{caption}
\usepackage{bm}
\usepackage{colortbl}
\usepackage{xcolor}
\usepackage{multicol}
\usepackage{wrapfig}
\usepackage{multirow}

% \usepackage{hyperref}

\counterwithin{figure}{section}
\renewcommand{\thesection}{\arabic{section}}
\renewcommand{\thesubsection}{\thesection.\arabic{subsection}}
\renewcommand{\baselinestretch}{1.5}

\usepackage[style=apa, maxnames=6, minnames=3, backend=biber, parentracker=true, sorting=none]{biblatex}
\DefineBibliographyStrings{english}{
      andothers = {\em et\addabbrvspace al\adddot}
}
\addbibresource{./Bibliography/bibliography.bib}

\usepackage{array}

\setlength{\parskip}{0pt}

\raggedbottom{}

\newcommand{\bolditalic}[1]{\textbf{\textit{#1}}}

\begin{document}

\begin{titlepage}
      \centering
      \includegraphics[width=0.3\textwidth]{Images/logo_utb.png}\par\vspace{1cm}
      {\scshape\LARGE Universidad Tecnológica de Bolívar \par}
      \vspace{1cm}

      {\scshape\Large Seminario de Investigación \par}
      \vspace{1cm}

      \slshape {\Large \bfseries{}Brechas laborales de los jóvenes en Cartagena: Un análisis de las causas y los impactos \\}
      \vspace{3cm}

      \slshape {\itshape{} Valentina Daniela Del Rio Jimenez \\}
      \slshape {\itshape{} Maria Celeste Espinoza Lopez \\}
      \slshape {\itshape{} Alanys Marin Frias \\}
      \slshape {\itshape{} Daymareth Amaya Cantillo \\}
      
      \vfill
      Revisado Por \\
      Katty Johanna Gomez Acevedo \\
      {\large \today\par}
\end{titlepage}

\nocite{*}

% chktex-file 1 
% chktex-file 6
% chktex-file 8
% chktex-file 24
% chktex-file 44

\newpage

\tableofcontents
\newpage

\section{Formulación de la idea de investigación}

El desempleo de los jóvenes en la Cartagena-Colombia se viene dando 
desde hace años debido a problemas estructurales que afectan directamente
al desarrollo económico de la ciudad. Tomando a la parte de la población 
con grupo etario entre los 15 y 28 años se evidencian las disparidades de 
la inserción laboral, dado por factores como la inexperiencia, las diferencias entre la oferta y demanda en el mercado laboral, etc. 
Adicionalmente, este problema perpetúa los ciclos de pobreza en Cartagena. 

Por tanto, la idea inicial de esta investigación es analizar las brechas
laborales que perjudican directamente a los jóvenes cartageneros, al 
comprender las causas del contexto y sus efectos económicos. Con el fin de 
identificar estrategias que potencialmente ayuden a mitigar estas 
desigualdades y fomenten la inclusión de los jóvenes en el mercado laboral. 

\section{Planteamiento del problema}

Cartagena es una de las ciudades más desiguales del país socialmente 
hablando, y esto se ve reflejado en su mercado laboral. Los jóvenes se 
enfrentan con el desafío de tasas de desocupación significativamente más
altas en comparación con el resto de la población, que van aumentando 
incluso cada vez más paulatinamente.  

En el año 2014, la tasa de desempleo juvenil en Cartagena se sitúa 
en $17,2\%$. En Cartagena en promedio para el 2014, había 754 mil 
personas en edad de trabajar, de los cuales 417 mil estaban ocupadas 
($55\%$); a su vez la tasa de informalidad en la ciudad para este periodo 
era de $54,2\%$, bajó $1.3$ puntos porcentuales; es decir que en 
Cartagena solo el $29.4\%$ de la población en edad de trabajar se 
desempeñó en un empleo formal, lo que equivale a 222 mil personas 
aproximadamente, según datos del DANE.

Según la Gran Encuesta Integrada de Hogares (corresponde al promedio anual)
del DANE la tasa de desempleo 2014 de jóvenes entre los 14 y 28 años
Cartagena cuenta con (17,2\%).

2022-2023 la tasa de desempleo en Cartagena solo bajó $0,3$ puntos 
porcentuales (p.p.), siendo una reducción sustancialmente menor a la del 
resto de las 13 áreas (13 áreas sin Cartagena), donde la disminución fue de
$1,3$ p.p. A pesar de lo anterior, la tasa de desempleo de Cartagena fue de
$11,1\%$ en el primer trimestre del año, ubicándose por debajo de la del
resto de las 13 áreas ($12,2\%$). 

% TODO: Image1

La disminución en la cantidad de ocupados del último año afectó a los
informales, (los formales si tuvieron un aumento en el empleo) y fue más 
notoria entre las mujeres y entre los jóvenes. Lo anterior se da en contravía
de lo sucedido en el promedio de las otras áreas metropolitanas, donde se 
observa un aumento en los ocupados, jalonado por las mujeres y donde los 
jóvenes también muestran recuperación.

% TODO: Image2

% TODO: Image3

En Cartagena, la tasa de desempleo de las personas entre 15 y 28 años subió
$1,3$ p.p. entre el primer trimestre de 2022 y el primero de 2023, siendo uno 
de los tres incrementos más altos entre las 13 áreas metropolitanas. Con 
ello, en el primer trimestre 2023, la tasa de desempleo joven en Cartagena 
fue de $20,5\%$, ubicándose por encima de la tasa promedio de las otras 
áreas, para este grupo de edad ($18,9\%$) y muy por encima de la tasa 
calculada para los mayores de 29 años en esta misma ciudad ($8,4\%$), 
situación que es común en todas las ciudades, ya que los jóvenes son en 
general la franja estaría más afectada por el desempleo, mostrando tasas de 
desempleo persistente y sensiblemente más altas que las de las otras franjas.
En septiembre del 2023 la tasa de desempleo en Cartagena tuvo un leve 
crecimiento y se situó en $10,5\%$, $0,1\%$ más alta que en igual mes de 2022
$(10,4\%)$, pero comparada con igual indicador nacional, que fue del $9,3\%$,
la diferencia es de $1,2\%$.

El dato más preocupante es que entre los jóvenes (15 a 28 años) de esta 
capital, la tasa de desempleo creció $1,4\%$, al pasar de $17,3\%$ en 
septiembre del año pasado a $18,7\%$ en igual mes de este año, según reporte
del mercado laboral revelado recientemente por el DANE. 

% TODO: Image4

% TODO: Image5

En Cartagena la Tasa General de Participación a septiembre fue del $66,4\%$ 
y la tasa de ocupación del $59,5\%$. Las personas ocupadas en la ciudad son
430 mil y las desocupadas 50 mil. 

2023-2024: En el último año, (entre el primer trimestre de 2023 y el primero
de 2024) la tasa de desempleo de Cartagena subió $2,6$ puntos porcentuales,
para ubicarse en $13,7\%$. Este aumento, que fue el 2do más fuerte entre las
áreas metropolitanas, se produjo más por un incremento en la fuerza laboral
que por una pérdida fuerte de empleos, ya que la misma fue de solo un $1$\%.
Pese a que la caída no fue grande, la misma se suma a las anteriores y con 
ello Cartagena se aleja de alcanzar sus niveles de empleo precrisis (primer
trimestre de 2020), mientras que las otras áreas ya lo hicieron. La 
disminución en la cantidad de ocupados del último año afectó a los informales 
y se concentró en los hombres y en la población joven. Por su parte, los 
formales, las mujeres y los mayores de 29 años presentaron aumentos en sus 
niveles de empleo. Por otro lado, al comparar con los niveles precrisis, las 
mujeres y los jóvenes son los que muestran más rezago en la recuperación.

En Cartagena, la tasa de desempleo de las personas entre 15 y 28 años fue de
$26,5\%$ en el primer trimestre 2024, siendo la segunda tasa más alta entre
los jóvenes a nivel metropolitano, después de la de Cúcuta. Esta tasa, se 
ubica muy por encima de la tasa calculada para los mayores de 29 años en esta
misma ciudad ($10,2\%$), situación que es común en todas las ciudades, ya que
los jóvenes son en general la franja etaria más afectada por el 
desempleo, mostrando tasas de desempleo persistente y sensiblemente más altas 
que las de las otras franjas. (Gráfico 8) Además del alto nivel, se tiene que 
en el último año (entre el primer trimestre de 2023 y el primero de 2024), la 
tasa de desempleo de las personas jóvenes de Cartagena subió 6,0 p.p., siendo 
el incremento más alto entre las 13 áreas metropolitanas. Como referencia, se 
tiene que la tasa de los jóvenes en el resto de las 13 áreas subió en $1,0$ 
p.p. (para quedar en $19,9\%$) y que la tasa de los cartageneros mayores 
de 29 años también aumentó, pero solo en $1,8$ p.p.

El desempleo entre los jóvenes es un problema grave que impacta el desarrollo 
económico y social de numerosas ciudades. Existen diversos factores 
interrelacionados que explican su origen, tales como el crecimiento de la 
población, la escasez de oportunidades laborales y las dificultades en la educación y formación profesional.

Uno de los principales factores del desempleo juvenil es el rápido incremento
de la población joven. Cuando una ciudad experimenta un crecimiento 
significativo en poco tiempo, el mercado laboral enfrenta presión, ya 
que no puede integrar a toda la nueva fuerza laboral. Esto se agrava 
cuando los sectores productivos carecen del impulso necesario para 
generar suficientes empleos en relación con el aumento de la población activa.

Otro aspecto importante es la desconexión entre el sistema educativo y las demandas 
reales del mercado laboral. Muchos jóvenes no pueden encontrar empleo porque su 
formación y habilidades no se alinean con lo que buscan las empresas. Esto puede 
ocurrir porque los programas formativos no están actualizados o porque no se han 
implementado suficientes iniciativas para conectar a los jóvenes con experiencias 
laborales desde una edad temprana. Como resultado, la falta de experiencia y 
habilidades específicas se convierte en una barrera para acceder al mercado laboral formal. 

Al cuestionarse las principales causas de las brechas laborales de los jóvenes en 
Cartagena, algunos de los factores que causan el desempleo juvenil son: 

\begin{enumerate}
      \item La falta de experiencia laboral previa que suelen exigir las empresas, sin contar a aquellos quienes buscan su primer empleo.
      
      \item Las diferentes habilidades que ofertan los jóvenes desde el aspecto profesional hasta el personal, no van alineadas con las demandas del mercado laboral.
      
      \item La mayoría de los jóvenes trabajan en condiciones informales. 
\end{enumerate}

Lo cual demuestra que la situación no solo tiene un gran impacto en los jóvenes, sino 
que también podría tener repercusiones negativas para distintos ámbitos de la ciudad. 

\subsection{Objetivo general}

Comprender las diferentes y principales variables socioeconómicas que impactan las 
brechas laborales para los jóvenes cartageneros.

\subsection{Objetivos especificos}

\begin{itemize}
      \item Identificar las barreras que estructuralmente desafían la inserción 
      laboral de los jóvenes. 
      
      \item Interpretar las fluctuaciones en el comportamiento de la tasa de 
      desempleo en los jóvenes de Cartagena durante el periodo de los últimos diez años. 

      \item Reconocer la relación que tienen la academia local y las políticas 
      públicas con el empleo juvenil. 

      \item Brindar potencialmente estrategias efectivas para reducir las brechas 
      laborales, y por el contrario incentivar la creación de empleo para los jóvenes.  
\end{itemize}

\section{Justificación del tema}

Conocer los antecedentes y las implicaciones es relevante, teniendo en cuenta 
que los jóvenes son una parte representativa de la población activa y la 
exclusión laboral de los mismos solamente agrava más el crecimiento 
económico de la ciudad, ralentizándolo y perpetuando los ciclos de pobreza 
intergeneracionales. 

En contextos de pobreza, la situación se agrava aún más. Las familias con menos 
recursos enfrentan dificultades para costear estudios superiores o cursos de 
especialización, lo que limita las oportunidades de los jóvenes para acceder 
a trabajos mejor remunerados. Muchas veces, la necesidad económica lleva a 
los jóvenes a aceptar empleos informales o de baja calidad, perpetuando 
un ciclo de precariedad y limitando su desarrollo profesional a largo plazo.

Las repercusiones del desempleo juvenil no solo afectan a quienes 
lo sufren, sino también a su entorno social y económico. La falta de 
empleo genera frustración y desmotivación, lo que puede derivar en problemas 
de salud mental, debilitar las relaciones familiares y disminuir la 
participación comunitaria. Desde una perspectiva económica, una alta 
tasa de desempleo juvenil implica menos capacidad de consumo y 
productividad, lo que impacta negativamente el crecimiento de la región.

Los efectos de este desempleo juvenil son diversas y preocupantes:

\begin{itemize}
      \item Aumento de la informalidad y el subempleo, frente a la escasez de 
      oportunidades formales, muchos jóvenes optan por trabajos informales 
      o subempleos, lo que perpetúa la precariedad laboral.

      \item Exposición a la vulnerabilidad social, la falta de empleo 
      estable pone a los jóvenes en situaciones de riesgo, como la 
      delincuencia o la violencia doméstica.

      \item Efecto en la economía local, la inexistencia de participación 
      de los jóvenes en el mercado laboral limita el crecimiento 
      económico y la innovación en la ciudad.
\end{itemize}

Por otro lado, en el último trienio los sectores que más empleo han generado en las 
13 áreas analizadas son las actividades profesionales y administrativas las cuales 
generaron 290.000 trabajos, dando como resultado un $31\%$ de crecimiento, la 
administración pública, educación y salud un aumento de 193.000, $14,8\%$ de 
empleo y el comercio 180.000, $9,1\%$. En contraste, la industria, la 
construcción y las actividades inmobiliarias aún no alcanzan niveles 
prepandemia, con una deuda de 141.000 puestos de trabajo, según los estudios del DANE. 

% TODO: Table1

Por lo cual, se deben diseñar e implementar políticas públicas y programas que no 
únicamente mejoren la empleabilidad de los jóvenes, sino que también fortalezca 
la economía de Cartagena.  

De manera que, esta investigación beneficiaria a los jóvenes, tanto como 
a las entes públicos y privados con intereses en la inserción laboral. Además, 
quienes busquen generar conciencia sobre la importancia de planes de 
acción eficientes y equitativos como actores clave para sobrellevar 
esta problemática.

\subsection{Viabilidad del estudio}

\begin{itemize}
      \item Recursos disponibles:
            \begin{itemize}
                  \item Tiempo disponible para realizar trabajo y compartir 
                  información bibliográfica o literatura relevante. 

                  \item Base de datos de estadísticas oficiales como Departamento 
                  Administrativo Nacional de Estadística (DANE) y su Gran Encuesta 
                  Integrada de Hogares (GEIH). 

                  \item Uso de software como Excel para procesar e interpretar
                  datos cuantitativos. 
            \end{itemize}

      \item Accesibilidad a datos:
            La información necesaria para llevar a cabo esta investigación puede 
            recolectarse de fuentes como informes o fichas técnicas oficiales. Así 
            como también, investigaciones anteriormente realizadas con respecto 
            al tema a abordar.  
      
      \item Limitaciones:
            \begin{itemize}
                  \item Información limitada con respecto el problema teniendo en 
                  cuenta el sector empresarial y la calidad de compartirla públicamente. 

                  \item Sesgos sobre los estratos socioeconómicos de los 
                  jóvenes cartageneros. 
            \end{itemize}
\end{itemize}

\newpage
\printbibliography{}

\end{document}