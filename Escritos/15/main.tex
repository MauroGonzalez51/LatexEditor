\documentclass[letterpaper, 12pt]{article}

\usepackage[utf8]{inputenc}
\usepackage[english, spanish]{babel}
\usepackage{fullpage}
\usepackage{graphicx}
\usepackage{amsmath}
\usepackage{enumitem}
\usepackage{chngcntr}
\usepackage{setspace}
\usepackage{url}
\usepackage{csquotes}
\usepackage{float}
\usepackage{verbatim}
\usepackage{tabularx}
\usepackage{amsmath}
\usepackage{caption}
\usepackage{bm}
\usepackage{wrapfig}
\usepackage{siunitx}

\counterwithin{figure}{section}
\renewcommand{\thesection}{\arabic{section}}
\renewcommand{\thesubsection}{\thesection.\arabic{subsection}}
\renewcommand{\baselinestretch}{2}

\usepackage[style=numeric, maxnames=6, minnames=3, backend=biber, parentracker=true, sorting=none]{biblatex}
\DefineBibliographyStrings{english}{%chktex-file 1 chktex-file 6
    andothers = {\em et\addabbrvspace al\adddot}
}

\addbibresource{./Bibliography/bibliography.bib}

\usepackage{array}
\usepackage{enumitem}

\setlength{\parskip}{\baselineskip}

\newcommand{\bolditalic}[1]{\textbf{\textit{#1}}}

\DeclareSIUnit{\COP}{COP}
\newcommand{\cop}[1]{\$\SI{#1}{\COP}}

\DeclareSIUnit{\DOLLAR}{USD}
\newcommand{\dollar}[1]{\$\SI{#1}{\DOLLAR}}

\begin{document}

\section*{Análisis: Seguridad en Cartagena de Indias}

\noindent\makebox[\linewidth]{\rule{\textwidth}{0.4pt}}

\begin{itemize}[label=$\diamond$]
    \item Mauro González~\textit{(T00067622)}
    \item Valentina Del Rio~\textit{(T00081360)}
    \item Germán De Armas~\textit{(T00068765)}
\end{itemize}

\noindent\makebox[\linewidth]{\rule{\textwidth}{0.4pt}}

\nocite{fontalvo_leyes2019}
\nocite{correa_ramirez2021}
\nocite{alvis_arrieta2011}
\nocite{castillo_avila2013}
\nocite{villa_moncaris2020}
\nocite{gomez_bustamante2019}
\nocite{montoya_ruiz2013}

% ! Introducción

% * Dar una pequeña descripción de la temática
% * ademas de establecer de que se va a hablar
% 

La seguridad en cualquier sociedad es un derecho
fundamental que abarca diversas dimensiones de la vida
cotidiana y que impacta directamente en el bienestar y la
calidad de vida de sus habitantes.

En el contexto de Cartagena de Indias, una ciudad histórica
y turística ubicada en la costa caribeña de Colombia, la
cuestión de la seguridad se presenta como un desafío
multifacético que aborta tanto las dimensiones
tradicionales de la seguridad ciudadana como aquellas que
afectan a los grupos mas vulnerables de la población.

Este trabajo se propone analizar la situación de la
seguridad en Cartagena desde una perspectiva
interdisciplinaria que abarca aspectos cuantitativos y
cualitativos. Se exploraran las políticas publicas de
seguridad ciudadana, condiciones de trabajo en general,
desafíos que enfrenta la ciudad desde hace mas de una
década, seguridad de las mujeres en el contexto urbano. A
través de este análisis, buscamos comprender la complejidad
de la seguridad en Cartagena y contribuir a la discusión de
políticas publicas que promuevan un entorno más seguro y
equitativo para todos los residentes de la ciudad.

% ! ---------------------------------------------------------------------|>

% * --------------------------------------------------------------|>
% * Seguridad Económica 

Cartagena presenta serios problemas de seguridad desde
tiempo atrás, que hasta la fecha, muchos de estos siguen
sin ser resueltos, o incluso, pueden ser mas graves de lo
que fueron por ejemplo, hace 10 años.~\textit{Seguridad
    Económica}, definida como la capacidad de las personas, los
hogares, o las comunidades de satisfacer sus necesidades
básicas de manera sostenible y con
dignidad~\cite{DeLaCruzRoja_2015}, en ese sentido, en
Cartagena, el cual el ingreso per cápita durante el año
2020 fue de tan solo \cop{355.004}~\cite{DANE_2021},
representando la linea de pobreza monetaria. Esto se
traduce en \dollar{85} aproximadamente, siendo esto en la
mayoría de casos, mucho menos que suficiente para ``poder
llegar a fin de mes''.

% * --------------------------------------------------------------|>
% * Seguridad Social 

\textsuperscript{1} En una ciudad tan grande y agitada como
lo es Cartagena de Indias, la seguridad ciudadana puede definirse como una
necesidad social. Este concepto se refiere a las exigencias
específicas de la población vinculadas con la delincuencia
y las situaciones de vulnerabilidad y riesgo para sus
personas y bienes, las cuales estarían estrechamente
asociadas a la policía pública, que tiene la función de
resolver, o al menos minimizar, los efectos negativos de
dichas amenazas.

\footnotetext[1]{\cite{fontalvo_leyes2019}. Cita que respalda los
    párrafos siguientes}

En la actualidad la seguridad ciudadana se ve resumida a
soluciones a corto plazo planteadas por los gobiernos
locales que se sienten sectorizadas e insuficientes, para
ser más específicos, los ciudadanos sienten que están
siendo categorizados y excluidos con las políticas
públicas.

Estas políticas implementadas en un territorio determinado
deben tener un papel real en los efectos que desean buscar;
deben verse reflejadas en resultados altamente efectivos de
tal forma que cumplan con el fin último de su acepción como
lo es la garantía y protección de uno o más derechos y la
satisfacción de las necesidades en los habitantes. Sin
embargo, resulta compleja esta garantía si se tiene en
cuenta que las necesidades de los ciudadanos cartageneros
en materia de seguridad se han visto afectadas en gran
manera por factores alternos.

% * --------------------------------------------------------------|>
% * Seguridad Social 

\newpage

\printbibliography

\end{document}