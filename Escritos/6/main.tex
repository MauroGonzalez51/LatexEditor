\documentclass[letterpaper, 12pt]{report}

\usepackage[utf8]{inputenc}
\usepackage[english, spanish]{babel}
\usepackage{fullpage} % changes the margin
\usepackage{graphicx}
\usepackage{amsmath}
\usepackage{enumitem}
\usepackage{chngcntr}
\usepackage{setspace}
\usepackage{url}
\usepackage{csquotes}
\usepackage{float}
\usepackage{verbatim}
\usepackage{tabularx}
\usepackage{amsmath}

\counterwithin{figure}{section}
\renewcommand{\thesection}{\arabic{section}}
\renewcommand{\thesubsection}{\thesection.\arabic{subsection}}
\renewcommand{\baselinestretch}{1.5}

\usepackage[style=numeric, maxnames=6, minnames=3, backend=biber, parentracker=true, sorting=none]{biblatex}
\DefineBibliographyStrings{english}{%chktex-file 1 chktex-file 6
    andothers = {\em et\addabbrvspace al\adddot}
}
\addbibresource{./Bibliography/bibliography.bib}

\DeclareFieldFormat{labelnumberwidth}{[#1]\hfill}
\defbibenvironment{bibliography}
{\list
    {\printtext[labelnumberwidth]{%
            \printfield{prefixnumber}%
            \printfield{labelnumber}}}
    {\setlength{\labelwidth}{\labelnumberwidth}%
        \setlength{\leftmargin}{\labelwidth}%
        \setlength{\labelsep}{\biblabelsep}%
        \addtolength{\leftmargin}{\labelsep}%
        \setlength{\itemsep}{\bibitemsep}%
        \setlength{\parsep}{\bibparsep}}\renewcommand*{\makelabel}[1]{\hss##1}}
{\endlist} {\item}%

% \usepackage[style=apa]{biblatex}

\usepackage{array}
\usepackage{enumitem}

\usepackage{setspace}
\setlength{\parskip}{\baselineskip}

\begin{document}

\chapter*{Análisis de: ``Las flores de la guerra''}

\noindent\makebox[\linewidth]{\rule{\textwidth}{0.4pt}}

Por:
\begin{itemize}[label=$\triangleright$]
    \item Mauro Gonzalez T00067622
    \item German de Armas T00068765
\end{itemize}

\noindent\makebox[\linewidth]{\rule{\textwidth}{0.4pt}}

\nocite{Pelicula}

``Las flores de la guerra'' es una película de origen chino estrenada el 16 de
diciembre de 2011, es un drama bélico histórico, dirigida por \textit{Zhang Yimou}.
La película está ambientada en 1937, en la segunda guerra mundial, más
específicamente en la invasión de Japón hacia China en el puerto de
\textit{Nankín} en búsqueda de extender el imperio \textit{Nipón} en todo el mundo.

La trama sigue a un grupo de prostitutas que se refugian en
una iglesia católica después de que la ciudad de Nanjing es
invadida por el ejército Japonés. El personaje principal,
John Miller (interpretado por \textit{Christian Bale}), un
americano que trabaja como enterrador en la iglesia,
intenta proteger a las mujeres y los niños refugiados de
los soldados japoneses.

La película cuenta con una magnífica ambientación que logra
transportar al espectador a los parajes desérticos
consecuencias de una devastadora guerra y permite
visualizar los terribles acontecimientos consecuencia de la
invasión, los vestuarios también acompañan a esta dinámica
de ubicarte en dicha época, además a la película no se
abstiene de mostrar escenas desgarradoras y violentas que
muestran la cruda realidad de una línea de asalto. Lo
anterior se combina con las actuaciones de la película,
donde cada uno hace un gran trabajo, mostrando su
sufrimiento, alegría, momentos dramáticos sin mencionar que
transmite innumerables sensaciones de indignación, ternura,
tristeza y suspenso.

Aunque la película es ampliamente aclamada, se puede
señalar como punto débil algunos efectos visuales que han
envejecido y no se mantienen a la altura de la producción
de gran presupuesto que es. En particular, la edición de la
película presenta algunos cortes abruptos que pueden
interrumpir la atmósfera creada y distraer al espectador de
la trama.

La película presenta escenas sumamente impactantes, como la
muerte de las dos niñas y las dos mujeres del burdel,
demuestra la existencia que en esa y en muchas guerras y
batallas hay hombres terribles que aprovecharán de
cualquier oportunidad para deshumanizar a todos los que se
encuentren en el bando contrario, siendo principalmente el
foco de la violencia las mujeres como objetivo predilecto,
aunque también se destacan personajes tanto masculinos como
femeninos que, independientemente de su rango, profesión o
nacionalidad, demuestran una ética y humanidad
excepcionales al ayudar a los demás, como John, las
prostitutas y el teniente japonés.

Una frase muy conocida menciona: “lo único que hace falta
para que triunfe el mal es que los buenos no hagan nada”.
Dicha frase resume toda la película pues muestra la
capacidad de las personas para unirse y luchar juntas en
tiempos de crisis, independientemente de su origen o
antecedentes.

Al principio, John, interpretado por Christian Bale, puede
parecer un patán insoportable y un ebrio sin vergüenza.
Pero más adelante se revela que ha perdido todo, desde su
hogar hasta su familia, y por eso ahoga sus penas en
alcohol. A pesar de esto, siempre intenta ayudar a los
refugiados en la iglesia y establece buenas relaciones con
George, las niñas y las damas del burdel, haciendo todo lo
que está en su mano. No obstante, la película evita el
estereotipo del héroe americano que siempre derrota a los
malos y salva el día, ya que John no logra salvar a todas
las niñas del ataque y, en un gesto de superioridad, matan
a una de ellas frente a él. A medida que avanza la trama,
se explica que su comportamiento protector surge de su
culpa por no haber podido proteger a su propia hija.

La guerra puede sacar lo peor del ser humano, pero también
lo mejor y esta película lo demuestra con las prostitutas
del burdel que al inicio parecían solo mujeres terribles,
frívolas y egoístas que solo verían por su bienestar, que
solo quieren irse del lugar y sobrevivir, pero terminan
arriesgando sus vidas en un acto altruista para salvar a
las niñas de un terrible destino para que ellas logren
vivir una mejor vida comparada a la suya.

Desafortunadamente el destino de ellas y de George es
incierto y queda a especulación del espectador saber si
lograron salir o no. Esto puede ser del gusto o disgusto de
cada uno, en lo personal considero que, si bien ayuda a la
intriga y al suspenso, es verdad que se siente
anticlimático y crea incertidumbre sobre el paradero de las
13 y más si la líder se logrará o no encontrar con John.

Las flores de la guerra es una gran película con una gran
carga emocional, que no solo muestra las consecuencias de
la guerra, sino que se enfoca en el drama personal
demostrando que hasta las personas más comunes pueden ser
héroes y que muchas veces la persona que menos te lo
esperas puede convertirse en uno. La película es hermosa y
dramática, es una montaña rusa de emociones que te recuerda
por que la guerra es como el infierno en la tierra.

\newpage

\printbibliography

\end{document}