\documentclass[letterpaper, 12pt]{report}

\usepackage[utf8]{inputenc}
\usepackage[english, spanish]{babel}

\usepackage{fullpage}
\usepackage{graphicx}
\usepackage{amsmath}
\usepackage{enumitem}
\usepackage{chngcntr}
\usepackage{setspace}
\usepackage{xurl}
\usepackage{csquotes}
\usepackage{float}
\usepackage{verbatim}
\usepackage{tabularx}
\usepackage{amsmath}
\usepackage{caption}
\usepackage{bm}
\usepackage{wrapfig}
\usepackage{siunitx}

\counterwithin{figure}{section}
\renewcommand{\thesection}{\arabic{section}}
\renewcommand{\thesubsection}{\thesection.\arabic{subsection}}
\renewcommand{\baselinestretch}{2}
\renewcommand{\thefigure}{\arabic{figure}}

\usepackage[style=apa, maxnames=6, minnames=3]{biblatex}
\DefineBibliographyStrings{english}{%chktex-file 1 chktex-file 6
      andothers = {\em et\addabbrvspace al\adddot}
}

\addbibresource{./Bibliography/bibliography.bib}

\DeclareCiteCommand{\textcite}
{\usebibmacro{prenote}}
{\usebibmacro{citeindex}%
      \printnames{labelname}%
      \setunit{\addcomma\addspace}%
      \printfield{year}%
      \setunit{\addcomma\addspace}%
      \printfield{title}}
{\multicitedelim}
{\usebibmacro{postnote}}

\DeclareCiteCommand{\cite}
{\usebibmacro{prenote}}
{\usebibmacro{citeindex}%
      (\printnames{labelname}%
      \setunit{\addcomma\addspace}%
      \printfield{year}%
      \setunit{\addcomma\addspace}%
      \printfield{title})}
{\multicitedelim}
{\usebibmacro{postnote}}

\usepackage{array}
\usepackage{enumitem}

\setlength{\parskip}{\baselineskip}

\newcommand{\bolditalic}[1]{\textbf{\textit{#1}}}

\renewcommand{\comment}[1]{{\small $\ll$#1$\gg$}}

% chktex-file 24

\begin{document}

\chapter*{Actividad II}

\noindent\makebox[\linewidth]{\rule{\textwidth}{0.2pt}}

\begin{itemize}[label=$\triangleright$]
      \item Valentina Daniela Del Rio Jiménez
      \item Alanys Marin Frias
\end{itemize}

\noindent\makebox[\linewidth]{\rule{\textwidth}{0.2pt}}

\nocite{*}

En este trabajo se ha escogido la revista científica CEA del ITM (Instituto
Tecnológico Metropolitano) por su afinidad con nuestro programa académico
(Finanzas y Negocios Internacionales). Puesto que, se enfoca en difundir el
conocimiento con alcance en temáticas como la gestión empresarial y
organizacional, la gestión económica y financiera, etc.

De esta revista se han escogido dos artículos. El primero, con enfoque
cualitativo y titulado Explorando el vínculo entre capacidades financieras y
aprendizaje organizacional en pymes: un análisis del estado del arte. Y el
segundo titulado Análisis de activos financieros en Colombia: cobertura de
posiciones con bitcoin, con un enfoque cuantitativo.

\section*{Explorando el vínculo entre capacidades financieras y aprendizaje organizacional en pymes: un análisis del estado de arte}

Este artículo cuenta con un enfoque cualitativo, porque su metodología se
fundamenta por una revisión sistémica y bibliométrica de la literatura
relevante existente, interpretando teorías y no centrándose en medir
numéricamente.

Las ideas de investigación deducibles del artículo son:

\begin{itemize}
      \item Explorar la relación entre las capacidades financieras y el aprendizaje
            organizacional en pymes (Pequeñas Y Medianas Empresas), deduciendo que hay una
            estrecha conexión para el éxito empresarial.

      \item Resaltar que la integración de las áreas de capacidades financieras y
            aprendizaje organizacional en pymes se alinean de manera efectiva y vital con
            la innovación empresarial y los ODS (Objetivos de Desarrollo Sostenible).

      \item Identificar la relación de las capacidades financieras y el aprendizaje
            organizacional en pymes como necesaria para la gestión estratégica de
            decisiones, garantizando la ventaja competitiva y el crecimiento en un mercado
            económico dinámico.

      \item Adoptar las nuevas tecnologías son cruciales en la relación de CF (capacidades
            financieras) y AO (aprendizaje organizacional, para mejorar la competitividad
            de las pymes en la era digital).
\end{itemize}

\section*{Análisis de activos financieros en Colombia: cobertura de posiciones con bitcoin}

Este artículo tiene un enfoque cuantitativo, debido a que está basado en
análisis de serie de tiempo en su metodología, y los modelos de correlación
condicional dinámica (DCC) y GARCH\@{}.

Las ideas de investigación que se pueden deducir del artículo son:

\begin{itemize}
      \item Identificar si existen activos que puedan servir como cobertura para las
            inversiones en bitcoin.

      \item Indagar si el alcance del microcrédito puede contarse como una herramienta que
            rompa las barreras de pobreza al permitir inversiones en actividades
            productivas (sector primario).

      \item Estudiar si los servicios financieros adaptados a poblaciones de bajos ingresos
            pueden mejorar su calidad de vida.

      \item Focalizar el microcrédito para empoderar a grupos vulnerables, especialmente
            mujeres, al proporcionarles recursos para generar ingresos y tomar decisiones
            económicas.

\end{itemize}

\printbibliography

\end{document}