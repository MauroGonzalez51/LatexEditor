\documentclass[letterpaper, 12pt]{report}

\usepackage[utf8]{inputenc}
\usepackage[english, spanish]{babel}
\usepackage{fullpage} % changes the margin
\usepackage{graphicx}
\usepackage{amsmath}
\usepackage{enumitem}
\usepackage{chngcntr}
\usepackage{setspace}
\usepackage{url}
\usepackage{csquotes}
\usepackage{float}
\usepackage{verbatim}
\usepackage{tabularx}
\usepackage{amsmath}

\counterwithin{figure}{section}
\renewcommand{\thesection}{\arabic{section}}
\renewcommand{\thesubsection}{\thesection.\arabic{subsection}}
\renewcommand{\baselinestretch}{1.5}

\usepackage[style=numeric, maxnames=6, minnames=3, backend=biber, parentracker=true, sorting=none]{biblatex}
\DefineBibliographyStrings{english}{%chktex-file 1 chktex-file 6
    andothers = {\em et\addabbrvspace al\adddot}
}
\addbibresource{./Bibliography/bibliography.bib}

\usepackage{array}
\usepackage{enumitem}

\usepackage{setspace}
\setlength{\parskip}{\baselineskip}

\begin{document}

\chapter*{Análisis de: ``Talentos ocultos''}

\noindent\makebox[\linewidth]{\rule{\textwidth}{0.4pt}}

Por: Mauro Gonzalez T00067622

\noindent\makebox[\linewidth]{\rule{\textwidth}{0.4pt}}

\nocite{Pelicula}

% Sinopsis
% Descripción de personajes principales/colectivos
% Ventajas/Privilegios => personajes principales
% Desventajas => personajes principales
% Contextualizar la película, ¿que pasaba?
% ¿Porque se invisibiliza históricamente a las mujeres?
% ¿Cual es el legado?
% Explicar los elementos de interseccionalidad
% Comentarios de la película

La película, ``Talentos ocultos'' (2016) dirigida por
\textit{Theodore Melf}, y basada en el libro ``Hidden
figures: The American dream and the untold story of the
black women mathematicians who helped win the space race''
de \textit{Margot Lee Shetterly}. Cabe mencionar que el
filme esta basado en hechos reales.

Cuenta la historia de tres mujeres afroamericanas
brillantes, \textit{Katherine Johnson} (Taraji P. Henson),
\textit{Dorothy Vaughan} (Octavia Spencer) y \textit{Mary
    Jackson} (Janelle Monáe) quienes trabajaron en la Nasa en
la década de 1960 como matemáticas y científicas, que
jugaron un papel fundamental en la carrera espacial de
Estados Unidos.

\textit{Katherine Johnson}, era una mujer cabeza de hogar, madre de tres
hijas y viuda (no mencionan cuando muere su esposo). Ella hace parte de un
equipo de matemáticas de la Nasa, ``las computadoras humanas''. Junto a
\textit{Dorothy}, \textit{Mary} y muchas mujeres mas, son
discriminadas/marginadas en el trabajo debido a su raza y género, a pesar de
sus incansables esfuerzos para contribuir al desarrollo del proyecto.

A pesar de esto, físicamente, las mujeres iban acorde a los
ideales o lo que se daba por ``bello'' para la época
(considerando que esta misma percepción social va cambiando
con los años). Por lo que en la película, se siente como si
ellos solo fueran discriminados por el hecho de ser
mujeres, o por la raza a la que pertenecen.

Luego, la película se ``fragmenta'' en el sentido en que
cada mujer empieza a luchar por lo que quiere lograr.

\textit{Katherine} es promovida a un nuevo puesto, en el que ahora se encarga
de estudiar los cálculos de la trayectoria, etc \dots \@. Aquí ella se
encuentra con un nuevo ambiente de trabajo, todo un equipo de
ingenieros y/o matemáticos que desde el primer dia la empezaron a
discriminar.~\textit{Paul Stafford} (Jim Parsons), quien se infiere
que es ``la mano derecha'' del jefe del equipo \textit{Al Harrison}
(Kevin Costner), discrimina constantemente a \textit{Katherine} desvirtuando
sus conocimientos, ``mis números son perfectos, no hace falta que los revises'',
fueron muchas veces en las que \textit{Paul} le dice esto a \textit{Katherine}.

\textit{Paul} se nota como alguien que rechaza instintivamente a las mujeres,
en el sentido en que ``funciona a la antigua'', el hombre trabaja y la mujer
se queda en la casa. Y es algo que a él le cuesta mucho aceptar,
\textit{Katherine} esta a su nivel pero desafortunadamente es algo de lo que
se da cuenta ya pasado bastante en la película. Precisamente aquí sucede
algo que personalmente me llamó bastante la atención, \textit{Paul} tiene
un desarrollo como persona muy grande, al principio nos lo muestran como
alguien sinceramente despreciable, poco a poco él va aceptando a
\textit{Katherine} dentro del equipo, haciendo que la orden de
\textit{John Glenn} (Glen Powell) a \textit{Katherine} de revisar los datos
de la trayectoria sea la gota que colmo el vaso. Cabe mencionar un gesto que
significó bastante en el desarrollo de este personaje, en una pequeña
escena se alcanza a ver como le entrega una taza de café a \textit{Katherine}
en su taza preferida, un pequeño gesto que significo mucho.

Ademas de \textit{Paul}, \textit{Katherine} también tiene
que lidiar con \textit{Al Harrison} y aquí si no hay mucho
de lo que hablar.~\textit{Harrison} tiene un modo de pensar
que seria algo como ``si sabes hacer algo, entonces lo
haces'', basado en una meritocracia en la que si había que
hacer algo, no importa si la persona que lo sabia hacer era
hombre o mujer, su raza o cualquier otra cosa, quedabas
como el encargado de hacer eso. En ese sentido, la relación
de estos dos personajes,\textit{Harrison} y
\textit{Katherine} va creciendo muchísimo a lo largo de
toda la película (hablando en términos de confianza), poco
a poco \textit{Harrison} se va dando mas cuenta de lo capaz
que es \textit{Katherine}.

En un principio se puede pensar que \textit{Harrison} esta
siendo condescendiente con \textit{Katherine}, pero no, mas
tarde en la película se da cuenta de que ella es
fundamental para el equipo. Con la llegada de \textit{IBM}
se pensó que las ``calculadoras humanas'' no serian
necesarias. Todo lo contrario \textit{Katherine} fue quien
reviso los cálculos de esta maquina (como en la situación
con \textit{John Glenn}), situación que también influyo en
la relación con \textit{Paul}.

\textit{Mary}, por otro lado se enfrenta a una situación un poco diferente,
ella quiere ser ingeniera pero no puede, aunque siendo mas específicos, no le
permiten serlo. La escuela a la que ella quiere y necesita ir, solo pueden
asistir hombres blancos por lo que en un principio ella se vio
derrotada/decepcionada debido a que no podia cumplir lo que quería.

En contraste con la historia de \textit{Katherine},
\textit{Mary} se enfrentó a un obstáculo diferente. A pesar
de su deseo de convertirse en ingeniera, la escuela que
ella necesitaba y quería asistir solo permitía la entrada
de hombres blancos. Sin embargo, sus compañeros de trabajo
la alentaron a presentar cargos ante un juez para luchar
por sus derechos. Hecho que alentó a \textit{Mary} a luchar
por lo que ella quería, concluyendo con ella haciendo parte
de la escuela, pero tomando clases por las noches.

A pesar de ser un personaje ``colectivo'', \textit{Dorothy}
también enfrenta discriminación y lucha por su derecho a un
puesto justo y merecido. A menudo se le niega la
oportunidad de avanzar en su carrera debido a su género y
raza. Pero su dedicación y habilidad no pasan
desapercibidas por sus colegas masculinos, quienes
finalmente reconocen su valía y la apoyan en su lucha por
la igualdad en el lugar de trabajo.

El personaje de \textit{Dorothy} refleja la lucha colectiva
de las mujeres afroamericanas en ese momento histórico. A
pesar de las barreras y la discriminación que enfrentaron,
estas mujeres perseveraron y lucharon por sus derechos. La
historia de Dorothy muestra que, aunque puede haber
obstáculos en el camino hacia la igualdad, la determinación
y el trabajo duro pueden superar incluso los prejuicios más
arraigados.

Alrededor de 60 años mas tarde, el legado de
\textit{Katherine Johnson}, \textit{Mary Jackson} y
\textit{Dorothy Vaughan} es incalculable. A través de su
arduo trabajo y perseverancia, lograron abrir nuevas
posibilidades para las mujeres y las personas de color en
la industria aeroespacial y en campos de la ciencia, la
tecnología, la ingeniería y las matemáticas.

Además de sus contribuciones específicas, su legado también
incluye la inspiración que han brindado a una nueva
generación de mujeres y minorías para perseguir carreras en
STEM, así como la lucha continua por la igualdad y la
justicia social.

La historia de estas mujeres ha demostrado que el talento y
la inteligencia no están limitados a género, raza o
cualquier otra categoría social, y que todos merecemos las
mismas oportunidades para desarrollar nuestras habilidades
y hacer una contribución significativa al mundo. Su legado
continuará inspirando a muchas personas en todo el mundo
durante muchas generaciones.

\newpage

\printbibliography

\end{document}