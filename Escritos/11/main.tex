\documentclass[letterpaper, 12pt]{article}

\usepackage[utf8]{inputenc}
\usepackage[english, spanish]{babel}
\usepackage{fullpage}
\usepackage{graphicx}
\usepackage{amsmath}
\usepackage{enumitem}
\usepackage{chngcntr}
\usepackage{setspace}
\usepackage{url}
\usepackage{csquotes}
\usepackage{float}
\usepackage{verbatim}
\usepackage{tabularx}
\usepackage{amsmath}
\usepackage{caption}
\usepackage{bm}
% \usepackage{hyperref}

\counterwithin{figure}{section}
\renewcommand{\thesection}{\arabic{section}}
\renewcommand{\thesubsection}{\thesection.\arabic{subsection}}
\renewcommand{\baselinestretch}{1.5}

\usepackage[style=apa, maxnames=6, minnames=3, backend=biber]{biblatex}
\DefineBibliographyStrings{english}{%chktex-file 1 chktex-file 6
	andothers = {\em et\addabbrvspace al\adddot}
}
\addbibresource{./Bibliography/bibliography.bib}

\usepackage{array}
\usepackage{enumitem}

\setlength{\parskip}{20pt}

\newcommand{\bolditalic}[1]{\textbf{\textit{#1}}}

\begin{document}

\begin{titlepage}
	\centering
	\includegraphics[width=0.3\textwidth]{Images/logo_utb.png}\par\vspace{1cm}
	{\scshape\LARGE Universidad Tecnológica de Bolívar \par}
	\vspace{1cm}

	{\scshape\Large Constitución \par}
	\vspace{.2cm}

	% chktex-file 8
	% {\scshape\Large H1 - C \par}
	\vspace{3cm}
	% chktex-file 8
	\slshape {\Large \bfseries{} Reseña   \\}
	\vspace{2cm}

	\slshape {\itshape{} Mauro González, T00067622 \\}
	\slshape {\itshape{} German De Armas Castaño, T00068765 \\}
	\slshape {\itshape{} Valentina Daniela del Rio Jimenez, T00081360 \\}
	% \slshape {\itshape{} Angel Vega Rodriguez, T00068186 \\}
	% \slshape {\itshape{} Juan Jose Osorio Ariza, T00067316 \\}
	% \slshape {\itshape{} Juan Eduardo barón, T00065901 \\}
	\vfill
	Revisado Por \\
	Gabriel Hoyos Gomez Casseres\\
	{\large \today\par}
\end{titlepage}

% ! 1) Elegir un elemento * CRISIS ECONÓMICA
% ! 2) Hacer la reseña, el contexto de esa wea
% ! 3) Diagrama

// * INTRODUCCIÓN

La década de los años 80 en Colombia se caracterizó por una
serie de eventos significativos que dejaron una profunda
huella en la historia del país. Las calles de la nación se
vieron ensombrecidas por la creciente presión ejercida por
los grupos armados insurgentes, como las Fuerzas Armadas
Revolucionarias de Colombia (FARC) y el Ejército de
Liberación Nacional (ELN), que buscaban cambios radicales
en la estructura política y económica.

Estos grupos llevaron a cabo ataques guerrilleros y
secuestros que generaron un clima de inseguridad en muchas
regiones del país. Además, Colombia acumuló una
considerable deuda externa, principalmente debido a la
caída de los precios del café, uno de los principales
productos de exportación del país, así como desequilibrios
fiscales causados por la falta de ingresos para cubrir los
gastos públicos y una inflación que alcanzó niveles
extremadamente altos, erosionando el poder adquisitivo de
los colombianos.

% * -----------------------------------------------------------------------|>

% + DESARROLLO

En 1977, los presidentes de Panamá, Omar Torrijos, y
Estados Unidos, Jimmy Carter, acordaron los Tratados
Torrijos-Carter, que se dividieron en dos documentos: el
Tratado sobre la Operación y Neutralidad Permanente del
Canal de Panamá y el Tratado del Canal de Panamá. Estos
tratados se firmaron en 1977 y entraron en vigor en 1979,
lo que llevó a que el Canal de Panamá pasara a ser más
controlado por Panamá, esto significo la oficial perdida de
uno de los recursos económicos más preciados del territorio
colombiano.

La totalidad de la población colombiana se encontraba
profundamente agotada por la ineptitud del Estado. Este
agotamiento se debía a la percepción de que todas las
decisiones eran tomadas en nombre de un ente todopoderoso,
lo que permitía que solo unos pocos privilegiados
disfrutaran de una calidad de vida óptima. A partir del año
1990, se empezaron a tomar medidas para abordar esta
creciente preocupación, motivada en gran parte por la
amenaza inminente que representaba a la presidencia de
César Gaviria. Estas medidas incluyeron la liberalización
de las importaciones, la desaceleración del ritmo de
devaluación de la moneda, la reducción de aranceles y, lo
que es aún más relevante, la independencia del Banco de la
República con un enfoque claro en la lucha contra la
inflación.

A pesar de estos esfuerzos, persistía la incertidumbre en
cuanto a si estas acciones serían suficientes para evitar
un retorno a la situación anterior después de las próximas
elecciones. El 11 de marzo de 1990, durante las elecciones
locales en Colombia, y como resultado del impulso generado
por el movimiento estudiantil, surgió lo que se conocería
como ``La séptima papeleta''. Esta papeleta adicional se
añadió a los seis tarjetones tradicionales de esa votación,
que abarcaban el Senado, la Cámara de Representantes, la
Alcaldía, la Asamblea, el Consejo, y la consulta interna
del Partido Liberal para definir al candidato presidencial.

La séptima papeleta representaba los anhelos frustrados del
pueblo colombiano en busca de reformar la Constitución de
1886 mediante un mandato popular. Aunque ni las autoridades
electorales ni el movimiento estudiantil pudieron
determinar cuántos ciudadanos votaron a favor de esta
alternativa constitucional el 11 de marzo, su impacto fue
innegable. Como respuesta a este movimiento, se organizó un
plebiscito constitucional en las elecciones presidenciales
del 27 de mayo. Ese día, ya de manera formal y vinculante,
$5.236.863$ colombianos votaron a favor y $230.080$ en
contra de una Asamblea Constituyente. Eso dio origen a la
Constitución del 4 de julio de 1991, una de las más
innovadoras y respetadas de América Latina; la base del
andamiaje institucional actual de Colombia.

% + -----------------------------------------------------------------------|>

% ! CONCLUSION 

la situación económica adversa en Colombia durante la
década de 1980 creó un caldo de cultivo propicio para la
aparición de la séptima papeleta, ya que reflejaba el deseo
de la población de impulsar reformas significativas en la
Constitución para abordar los desafíos económicos y
sociales de la época.Esta iniciativa representó un llamado
a la acción ante una crisis económica que había afectado
profundamente a la sociedad colombiana.

La séptima papeleta se convirtió en un medio para canalizar
la frustración y las demandas de una población que
percibían a la constitución de 1886 ineficiente e
inadecuada para abordar los desafíos económicos y sociales
del momento. La actuación del Movimiento Estudiantil de la
Séptima Papeleta, en efecto, fue subversiva y
revolucionaria desde un principio porque eran jóvenes, que
no guardaron silencio frente a los actos de violencia,
armándose de valor para exigirle al Gobierno el
cumplimiento de su deber, criticando principalmente de
forma directa la falta de transparencia del Congreso de la
República.

Al trazar una alternativa para reformar la constitución
incentivaron a todo el pueblo colombiano a reaccionar. La
presión política ejercida por el movimiento estudiantil y
otros grupos sociales llevó al sistema político actual y a
las autoridades a reconocer la vital importancia de atender
las inquietudes económicas y sociales de la población
colombiana. Esto demostró que el pueblo no era pasivo ni
indiferente ante los conflictos y problemas del país, sino
que estaba dispuesto a hacer oír su voz.

\nocite{Héctor}
\nocite{crisispolitica} \nocite{Carrero}

\printbibliography

\end{document}