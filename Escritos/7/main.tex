\documentclass[letterpaper, 12pt]{article}

\usepackage[utf8]{inputenc}
\usepackage[english, spanish]{babel}
\usepackage{fullpage} % changes the margin
\usepackage{graphicx}
\usepackage{amsmath}
\usepackage{enumitem}
\usepackage{chngcntr}
\usepackage{setspace}
\usepackage{url}
\usepackage{csquotes}
\usepackage{float}
\usepackage{verbatim}
\usepackage{tabularx}
\usepackage{amsmath}

\counterwithin{figure}{section}
\renewcommand{\thesection}{\arabic{section}}
\renewcommand{\thesubsection}{\thesection.\arabic{subsection}}
\renewcommand{\baselinestretch}{1.5}

\usepackage[maxbibnames=3,style=authoryear,backend=biber]{biblatex}
\DeclareNameAlias{sortname}{family-given}
\DeclareNameAlias{default}{family-given}
\renewcommand*{\multinamedelim}{\addcomma\space}
\renewcommand*{\finalnamedelim}{\addspace\&\space}
\DefineBibliographyStrings{english}{%chktex-file 1 chktex-file 6
    andothers = {et\addspace al\adddot}
}

\DeclareFieldFormat{cite}{\mkbibemph{#1}}
\renewbibmacro*{cite}{%
    \printtext[bibhyperref]{%
        \printfield{title}%
        \setunit{\addcomma\space}%
        \printdate}}

\addbibresource{./Bibliography/bibliography.bib}

\usepackage{array}
\usepackage{enumitem}
\usepackage{breqn}

\raggedbottom{}

\usepackage{array}
\usepackage{enumitem}

\usepackage{setspace}
\setlength{\parskip}{\baselineskip}

\begin{document}

\section*{Ensayo: Educación sexual en Cartagena}

\noindent\makebox[\linewidth]{\rule{\textwidth}{0.4pt}}

Por: Mauro Gonzalez T00067622

\noindent\makebox[\linewidth]{\rule{\textwidth}{0.4pt}}

\bigskip

% -------------------------------------------------------------------------------------------------------------------|>

La deficiente educación sexual en Cartagena es el resultado
de la falta de compromiso tanto del sistema educativo como
de los padres~\footnote{Núcleo familiar en general} en la
formación integral de los adolescentes y niños. Ambos
actores tienen la responsabilidad de brindar información y
herramientas adecuadas para una planificación de vida en
temas de sexualidad, lo que ayudaría a frenar los altos
índices de embarazo en adolescentes y promover una vida
sexual saludable y responsable. Cabe mencionar que no se
trata de blancos y negros, sino realmente una escala de
grises, en el que influyen muchas más variables externas.

Y esto no se trata de ``querer echarle la culpa a alguien''
por no hacer lo que ``se debería hacer''. No, ese no es el
punto del ensayo. La idea es describir los causantes
--principales-- del estado actual de la educación sexual en
Cartagena.~\footnote{Durante el desarrollo del ensayo se
    utilizará mucho la frase ``los hijos'' ya que por motivos
    prácticos es muy útil para referirse tanto a hijos como
    hijas en general.}

% Como ya fue mencionado anteriormente, la temática a tratar
% tiene bastante matices por lo que se irán estudiando a
% medida que avanza el escrito.

Uno de los principales actores en la problemática, es todo
este tema de `los padres'', ellos son los que dedican --o
al menos deberían-- su tiempo a enseñarle cosas a su hijo,
y en cierto modo cae en elección de los padres decidir que
aprenden sus hijos o no.

En ese sentido, de los padres depende la educación sexual
del hijo a temprana edad. Pero, ¿que pasa si los padres ven
la temática como tabú?, ó no la consideran ``necesaria''
para el crecimiento. Lamentablemente esta es la realidad de
muchas familias no solo en Cartagena, sino que en todo el
mundo. Ya sea por la vision que los padres tienen del tema,
la falta de confianza, problemas de comunicación en el
hogar, o incluso que los mismos hijos no quieren hablar con
sus padres, etc \dots. Son unas de las posibles causas por
las que no se les da el acompañamiento necesario.

\begin{displayquote}
    Sin lugar a duda, los factores del contexto sociocultural ocupan una relevante posición en
    torno al fenómeno del embarazo adolescente, Pantelides (2004), (como se citó en García y Barragán y
    Espinoza Romo, 2012) menciona que los factores sociales que influyen en tal fenómeno abarcan
    desde un nivel macrosocial (valores, normas, creencias sociales, estructura y roles socioeconómicos,
    distinciones étnicas, políticas públicas dirigidas a la salud reproductiva adolescente, etc.) hasta
    factores más próximos al individuo (como actitudes, creencias, zona de residencia, la estructura
    familiar y de los grupos de socialización compuestos por los docentes y sus grupos de pares,
    existencia de servicios de salud sexual y reproductiva, disponibilidad de recursos anticonceptivos,
    etc.).

    \textcite{Garcia2017}

\end{displayquote}

A medida que va a creciendo y se convierte en adolescente,
para sorpresa, empieza a obviar del tema, es decir, ``ya yo
se eso, no me lo tienes que decir'', luego cuando los
padres toman el coraje para intentar hablar con los hijos
del tema, estos son rechazados forzosamente. Y en cierto
modo puede que estén en lo correcto, siendo actualmente el
año 2022, es muy fácil acceder a cualquier información
alrededor del mundo. Debido a esto no seria extraño pensar
que puede que ya tenga conocimientos de la temática, pero
¿y si no?

\begin{displayquote}
    Otro de los factores familiares que influyen en el embarazo del adolescente es la educación
    sexual y reproductiva, definida por Espejo et al. (28) como “la información
    sobre conceptos teóricos, herramientas que ayuden a desarrollar las actitudes, la comunicación
    y valores en los adolescentes para la toma de decisiones sobre su sexualidad”. Al respecto, en
    el estudio de Sánchez-Valencia et al. (29), en Colombia, determinaron
    que el no recibir educación por parte de la madre representa 2,6 veces más
    la probabilidad de un embarazo precoz. Asimismo, Menacho-Barrera (18), en
    el Perú, encontró que la educación sexual y reproductiva inadecuada por
    parte de los padres tiene un $p = 0,042$ de significancia en asociación
    con el embarazo adolescente.

    \textcite{CubaSancho2022}
\end{displayquote}

% -------------------------------------------------------------------------------------------------------------------|>

La educación en general es definida como, un proceso de
formación practica y metodológica que se le da a una
persona en vias de desarrollo y conocimiento. A su vez se
le suministran herramientas y saberes esenciales para
ponerlos en practica en la vida
cotidiana.~\nocite{sánchez_2022}

Hay algo fundamental en esta definición, son ``saberes para
ponerlos en practica''. Partiendo de esto las escuelas en
general no tienen ninguna justificación para no impartir la
temática.

Por otro lado, muchos países Europeos han implementado la
educación sexual de manera obligatoria en las escuelas,
dando como ejemplo países como Bélgica, Dinamarca, España,
Estonia, Finlandia, etc
\dots.~\nocite{FundacionSexpol}~(\cite{PhotoPaísesEuropa}).
Cae en evidencia que esto esto es algo relativamente
reciente pero que poco a poco mas países han accedido a
implementarla de manera obligatoria.

\nocite{bancoMundial}

Por ejemplo, tomando como foco a España, país en el que
según estudios, se registran 6 embarazos adolescentes en
mujeres entre 16 y 19 años por cada 1000 habitantes. Una
cifra una cifra que es minima en comparación a la situación
en Colombia, que asciende a 60 embarazos por cada 1000
habitantes.

Siguiendo con la idea, analizando los cifras de embarazo en
adolescentes, es difícil no relacionar los datos que España
presenta, junto con las políticas de educación sexual en el
país. Este como muchos otros países en los que se ve una
tendencia a disminuir los embarazos en adolescentes debido
a las políticas que se implementan, aunque, esta conclusion
no puede generalizar, hay países en los que funciona,
asimismo hay otros en los que no, por lo que se trataría
como una evidencia, no como un hecho.

% -------------------------------------------------------------------------------------------------------------------|>

\nocite{MinEducacion}
\nocite{CartagenaComoVamos__1}
\nocite{CifrasCartagena}

Por otro lado, anteriormente se menciono que hay factores
externos pero que influyen directa o indirectamente a la
temática. Cartagena presenta serios problemas de educación
en general, aunque actualmente hay en progreso muchas
proyectos encaminados a enmendar esto, la realidad es que
queda mucho camino por recorrer. Entre el top en Colombia
de ciudades con mayor indice de pobreza, Cartagena ocupa el
4to lugar, presentando un 27\% de la población.

En muchos casos, el acceso a la educación en sectores de
Cartagena, es considerado como un privilegio en el cual, no
es de extrañar que jóvenes en toda la ciudad crezcan sin la
totalidad de los saberes básicos, mucho menos en temas de
sexualidad. En los barrios menos favorecidos, el concepto
``educación sexual'' o``sexualidad'' son prácticamente
desconocidos, asimismo, estos mismos barrios son los que
presentan los indices de embarazo adolescente mas altos en
la ciudad.

No es solo un tema de falta de educación sexual, radica en
un problema mas general, la educación en Cartagena debe de
ser mas completa, abarcar mas mas temática, tener una mejor
calidad, para asi concretizar a las personas
(principalmente adolescentes) sobre la importancia de
conocer la sexualidad.

% -------------------------------------------------------------------------------------------------------------------|>

En conclusión, considero que la deficiente educación sexual
en Cartagena es un problema que afecta no solo a los
adolescentes y jóvenes, sino a la sociedad en su conjunto.
A través de este ensayo he argumentado que tanto la familia
como el sistema educativo tienen una responsabilidad
compartida en la formación integral de los adolescentes y
niños en temas de sexualidad. Además, he señalado que
existen factores externos que influyen en esta
problemática, como la falta de acceso a información
adecuada y la influencia de estereotipos culturales
negativos sobre la sexualidad.

Es necesario que se promueva una educación sexual integral
que fomente una vida sexual saludable y responsable en la
juventud, lo que permitirá reducir los altos índices de
embarazo en adolescentes y mejorar la calidad de vida de la
sociedad en su conjunto.

\newpage

\printbibliography

\end{document}