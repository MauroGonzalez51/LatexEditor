\documentclass[letterpaper, 12pt]{report}

\usepackage[utf8]{inputenc}
\usepackage[english, spanish]{babel}
\usepackage{fullpage} % changes the margin
\usepackage{graphicx}
\usepackage{amsmath}
\usepackage{enumitem}
\usepackage{chngcntr}
\usepackage{setspace}
\usepackage{url}
\usepackage{csquotes}
\usepackage{float}
\usepackage{verbatim}
\usepackage{tabularx}
\usepackage{amsmath}

\counterwithin{figure}{section}
\renewcommand{\thesection}{\arabic{section}}
\renewcommand{\thesubsection}{\thesection.\arabic{subsection}}
\renewcommand{\baselinestretch}{1.5}

\usepackage[style=numeric, maxnames=6, minnames=3, backend=biber, parentracker=true, sorting=none]{biblatex}
\DefineBibliographyStrings{english}{%chktex-file 1 chktex-file 6
    andothers = {\em et\addabbrvspace al\adddot}
}
\addbibresource{./Bibliography/bibliography.bib}

% \DeclareFieldFormat{labelnumberwidth}{[#1]\hfill}
% \defbibenvironment{bibliography}
% {\list
%     {\printtext[labelnumberwidth]{%
%             \printfield{prefixnumber}%
%             \printfield{labelnumber}}}
%     {\setlength{\labelwidth}{\labelnumberwidth}%
%         \setlength{\leftmargin}{\labelwidth}%
%         \setlength{\labelsep}{\biblabelsep}%
%         \addtolength{\leftmargin}{\labelsep}%
%         \setlength{\itemsep}{\bibitemsep}%
%         \setlength{\parsep}{\bibparsep}}\renewcommand*{\makelabel}[1]{\hss##1}}
% {\endlist} {\item}%

% \usepackage[style=apa]{biblatex}

\usepackage{array}
\usepackage{enumitem}

\usepackage{setspace}
\setlength{\parskip}{\baselineskip}

\begin{document}

\chapter*{Ensayo: Educación sexual en Cartagena}

\noindent\makebox[\linewidth]{\rule{\textwidth}{0.4pt}}

Por: Mauro Gonzalez T00067622

\noindent\makebox[\linewidth]{\rule{\textwidth}{0.4pt}}

NOTA:\@ Párrafos al azar, luego los acomodare, notas con
``//''

\bigskip

{\small // Tesis, lo mas probable es que la deje en un párrafo sola, y la inserte
    al principio del escrito.}

La deficiente educación sexual en Cartagena es el resultado
de la falta de compromiso tanto del sistema educativo como
de los padres en la formación integral de los adolescentes
y niños. Ambos actores tienen la responsabilidad de brindar
información y herramientas adecuadas para una planificación
de vida en temas de sexualidad, lo que ayudaría a frenar
los altos índices de embarazo en adolescentes y promover
una vida sexual saludable y responsable. Cabe mencionar que
no se trata de blancos y negros, sino realmente una escala
de grises, en el que influyen muchas mas variables
externas.

Y esto no se trata de ``querer echarle la culpa a alguien''
por no hacer lo que ``se debería hacer''. No, ese no es el
punto del ensayo. La idea es describir los causantes
--principales-- del estado actual de la educación sexual en
Cartagena.

Como ya fue mencionado anteriormente, la temática a tratar
tiene bastante matices por lo que se irán estudiando a
medida que avanza el escrito.

Uno de los principales actores en la problemática, es todo
este tema de `los padres'', ellos son los que dedican --o
al menos deberían-- su tiempo a enseñarle cosas a su hijo,
y en cierto modo cae en elección de los padres decidir que
aprenden sus hijos o no.

    {\small // Aquí, quisiera introducir la cita o al menos hablar de como es visto
        el tema de educación sexual por parte de los padres, y de como incluso puede
        llegar a ser tratado como tabú}

    {\small // Párrafo explorando la temática desde el lado de los padres}

En ese sentido, de los padres depende la educación sexual
del hijo a temprana edad. Pero, ¿que pasa si los padres ven
la temática como tabú?, ó no la consideran ``necesaria''
para el crecimiento. Lamentablemente esta es la realidad de
muchas familias no solo en Cartagena, sino que en todo el
mundo. Ya sea por la vision que los padres tienen del tema,
la falta de confianza, problemas de comunicación en el
hogar, o incluso que los mismos hijos no quieren hablar con
sus padres, etc \dots. Son unas de las posibles causas por
las que no se les da el acompañamiento necesario.

    {\small // Párrafo explorando la temática desde el lado de los hijos}

A medida que va a creciendo y se convierte en adolescente,
para sorpresa, empieza a obviar del tema, es decir, ``ya yo
se eso, no me lo tienes que decir'', luego cuando los
padres toman el coraje para intentar hablar con los hijos
del tema, estos son rechazados forzosamente. Y en cierto
modo puede que estén en lo correcto, siendo actualmente el
año 2022, es muy fácil acceder a cualquier información
alrededor del mundo. Debido a esto no seria extraño pensar
que puede que ya tenga conocimientos de la temática, pero
¿y si no?

{\small // Intentando dar una vuelta completa, ahora empezando a hablar
de como se desarrolla en los colegios}

Por otro lado, no solo es hablar de las situaciones dentro
del hogar, hay que pensar también en las escuelas, ¿como es
tratado la temática en este contexto?.

Siendo asi, la realidad es que es bastante decepcionante
--para no decir que es nula--. La educación sexual en los
colegios no existe y en caso de lo contrario, otra vez, no
es tomado en serio, es visto como una materia de relleno
que tampoco tiene constancia para hablar de la temática.

Se supone que la idea de entrar a un colegio y pasar varios
años ahí, es para formarse de una manera básica en todas
las areas del conocimiento. Pero ¿en donde queda el conocer
la sexualidad?, conocer el cuerpo humano, conocer los impulsos que tenemos 
como especie, comportamientos, actitudes, etc \dots\@, que se supone
que deberían ser enseñadas en el colegio.

\newpage

\printbibliography

\end{document}