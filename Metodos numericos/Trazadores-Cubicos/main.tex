\documentclass[conference]{IEEEtran}
\IEEEoverridecommandlockouts{}

\usepackage{cite}
\usepackage{amsmath,amssymb,amsfonts}
\usepackage{algorithmic}
\usepackage{graphicx}
\usepackage{textcomp}
\usepackage{xcolor}
\usepackage{tabularx}
\usepackage{float}

\def\BibTeX{{\rm B\kern-.05em{\sc i\kern-.025em b}\kern-.08em
    T\kern-.1667em\lower.7ex\hbox{E}\kern-.125emX}}
\begin{document}

\title{Trazadores Cúbicos\\}

\author{\IEEEauthorblockN{1\textsuperscript{st} Mauro Alonso Gonzalez Figueroa}
	\IEEEauthorblockA{\textit{Universidad Tecnologica de Bolivar} \\
		\textit{UTB}\\
		Cartagena, Colombia \\
		maugonzalez@utb.edu.co}}

\maketitle

\bibliographystyle{IEEEtran}

% chktex-file 44 
% chktex-file 24

\begin{abstract} This paper focuses on Cubic Splines, a numerical method
	widely used for interpolating functions. Cubic splines are piecewise
	polynomials that are used for interpolation of data points. The method
	involves constructing a series of cubic polynomials that pass through
	specified points and have continuous first and second derivatives. This
	paper explores the theoretical foundation of cubic splines, including
	their construction, properties, and applications. Additionally, it
	discusses practical considerations such as the choice of boundary
	conditions and the use of natural, clamped, or periodic splines. The goal
	is to provide a comprehensive understanding of cubic splines and their
	utility in numerical analysis and scientific computing.
\end{abstract}

\begin{IEEEkeywords}
	Cubic Splines, Interpolation, Polynomials
\end{IEEEkeywords}

\nocite{Trazadores_cúbicos_2014}

\section{Introducción}

La interpolación numérica, un pilar fundamental en disciplinas científicas y
técnicas, se emplea para estimar valores intermedios entre datos discretos
conocidos.

Entre los métodos de interpolación más destacados se encuentra
el método de trazadores cúbicos. Estos trazadores proveen una forma eficiente
y precisa de aproximar una función entre puntos discretos mediante polinomios
cúbicos.

Su utilidad abarca diversas aplicaciones, desde el ajuste de curvas
en el análisis de datos hasta la simulación numérica en ingeniería y ciencias
naturales. Los trazadores cúbicos encuentran aplicación en una amplia gama de
campos, incluyendo ingeniería, ciencias de la computación, matemáticas
aplicadas y ciencias sociales, destacándose por su versatilidad y eficacia
en la interpolación y el análisis de datos.

En este documento, exploraremos la teoría, el uso y las aplicaciones
prácticas del método de trazadores cúbicos, desde su fundamentación
matemática hasta su implementación en diversas áreas del conocimiento.

\section{Teoría del método}

Un trazador cúbico $S$ es una función a trozos que interpola a $f$ en los
$n + 1$ puntos $(x_0, y_0), (x_1, y_1), (x_2, y_2), \dots, (x_n, y_n)$
(con $a = x_0 < x_1 < \cdots < x_n = b$). S es definida de la siguiente manera,

\subsection{Trazadores Lineales}

Dos puntos distintos cualquiera determinan un segmento de recta, de esta
manera los trazadores de primer grado para un grupo de datos ordenados
pueden definirse como un grupo de funciones lineales

\begin{align*}
	f(x)   & = f(x_{0}) + m_{0}(x - x_{0})             & x_{0} \leq x \leq x_{1}     \\
	f(x)   & = f(x_{1}) + m_{1}(x - x_{1}),            & x_{1} \leq x \leq x_{2}     \\
	\vdots &                                           &                             \\
	f(x)   & = f(x_{n - 1}) + m_{n - 1}(x - x_{n - 1}) & x_{n - 1} \leq x \leq x_{n}
\end{align*}

Donde $m$ es la pendiente de la recta,

\begin{align*}
	m = \frac{f(x_{i + 1}) - f(x_{i})}{x_{i + 1} - x_{i}}
\end{align*}

Es crucial destacar que los trazadores de primer grado no presentan
suavidad. Es decir, en los puntos donde convergen dos trazadores
\textit{(conocidos como nodos)}, la pendiente experimenta un cambio brusco.
Formalmente, la primera derivada de la función es discontinua en estos
puntos. Esta limitación se supera mediante el empleo de trazadores
polinomiales de grado superior, que garantizan suavidad en los nodos al
igualar las derivadas en esos puntos.

\subsection{Trazadores Cuadráticos}

Para garantizar la continuidad de las derivadas de orden m en los nodos,
es necesario emplear un trazador de al menos un grado de m + 1. En el
caso específico de los trazadores cuadráticos, se busca obtener un
polinomio de segundo grado para cada intervalo entre los datos. De manera
general, el polinomio en cada intervalo se representa como:

\begin{align}
	f_{i}(x) = a_{i}{x}^{2} + b_{i}x + c_{i}
	\label{eq:cuadratic_equation}
\end{align}

Los trazadores cuadráticos, requieren $3n$ ecuaciones o condiciones
para evaluar las incógnitas, estas son:

\begin{enumerate}
	\item \textit{Los valores de la función deben ser los mismos en los puntos
		      donde se unen dos polinomios consecutivos}

	      Esta condición se representa como,

	      \begin{align}
		      a_{i - 1}{x_{i - 1}}^{2} + b_{i - 1}x_{i - 1} + c_{i - 1} = f(x_{i - 1})
		      \label{eq:18.29}
	      \end{align}

	      \begin{align}
		      a_{i}{x_{i - 1}}^{2} + b_{i}x_{i - 1} + c_{i} = f(x_{i - 1})
		      \label{eq:18.30}
	      \end{align}

	      Para $i = 2$, cada una proporciona $n - 1$, en total $2n - 2$ condiciones

	\item \textit{La primera y la última función deben pasar a través
		      de los puntos extremos.}

	      \begin{align}
		      a_{1}{x_{0}}^{2} + b_{1}x_{0} + c_{1} = f(x_{0})
		      \label{eq:18.31}
	      \end{align}

	      \begin{align}
		      a_{n}{x_{n}}^{2} + b_{n}x_{n} + c_{n} = f(x_{n})
		      \label{eq:18.32}
	      \end{align}

	      En total tenemos, $2n - 2 + 2 = 2n$ condiciones

	\item \textit{Las primeras derivadas en los nodos interiores deben
		      ser iguales}

	      La derivada de la ecuación~(\ref{eq:cuadratic_equation}) es,

	      \begin{align*}
		      f'(x) = 2ax + b
	      \end{align*}

	      Por lo tanto, la condición se representa como,

	      \begin{align}
		      2a_{i - 1}x_{i - 1} + b_{i - 1} = 2a_{i}x_{i - 1} + b_{i}
		      \label{eq:18.33}
	      \end{align}

	      Para $i = 2$, esto representa otras $n - 1$ condiciones, con
	      un total de $2n + n - 1 = 3n - 1$

	\item \textit{Supongamos que en el primer punto la segunda derivada es cero.}
	      Matemáticamente, esto se expresa como la condición de que la segunda
	      derivada de la ecuación 18.28, que es 2ai, sea igual a cero en este punto.
	      Esta condición implica que los dos primeros puntos se conectarán
	      con una línea recta.

	      \begin{align}
		      a_{1} = 0
		      \label{eq:18.34}
	      \end{align}

	      Por ultimo, representando una ultima condición, resultando
	      en $3n$
\end{enumerate}

\subsection*{Ejemplo practico}

Suponga los datos,

\begin{table}[H]
	\begin{tabularx}{\linewidth}{|>{\centering\arraybackslash}X|>{\centering\arraybackslash}X|}
		\hline
		\textbf{X} & \textbf{Y} \\ \hline
		$2,0$      & $4,0$      \\ \hline
		$4,5$      & $3,2$      \\ \hline
		$7,8$      & $5,3$      \\ \hline
		$12,0$     & $2,4$      \\ \hline
	\end{tabularx}
	\label{tab:ejemplo_tabularx}~\caption{Tabla de datos}
\end{table}

En este problema, se tienen 4 datos y $n = 3$ intervalos, por lo tanto
se deben determinar $3(3) = 9$ incógnitas

Las ecuaciones~(\ref{eq:18.29}) y~(\ref{eq:18.30}) nos dan $2(3) - 2 = 4$
condiciones

\begin{align*}
	20,25a_{1} + 4,5b_{1} + c_{1} = 3,2 \\
	20,25a_{2} + 4,5b_{2} + c_{2} = 3,2 \\
	60,84a_{2} + 7b_{2} + c_{2} = 5,3   \\
	60,84a_{3} + 7b_{3} + c_{3} = 5,3
\end{align*}

Evaluando a la primera y la última función con los valores inicial y
final~(\ref{eq:18.31})

\begin{align*}
	4a_{1} + 2b_{1} + c_{1} = 4
\end{align*}

Y la ecuación~(\ref{eq:18.32})

\begin{align*}
	144a_{3} + 12b_{3} + c_{3} = 2,4
\end{align*}

La continuidad de las derivadas crea adicionalmente $3 - 1 = 2$ condiciones~(\ref{eq:18.33})

\begin{align*}
	9a_{1} + b_{1} = 9a_{2} + b_{2} \\
	15,6a_{2} + b_{2} = 15,6a_{3} + b_{3}
\end{align*}

Por ultimo, la condición~(\ref{eq:18.34}) determina que $a_{1} = 0$, con esto
el problema se resume en la resolución de un sistema $8 \times 8$.

Resolviendo el sistema obtenemos los valores

\begin{align*}
	a_{1} & = 0             & b_{1} & = -0,32        & c_{1} & = 4,64         \\
	a_{2} & \approx  0,16   & b_{2} & \approx -1,76  & c_{2} & \approx 7,88   \\
	a_{3} & \approx 0,00649 & b_{3} & \approx -0,661 & c_{3} & \approx 9,6325 \\
\end{align*}

\bibliography{./Bibliography/bibliography.bib}

\end{document}
