\documentclass[conference]{IEEEtran}
\IEEEoverridecommandlockouts{}

\usepackage{cite}
\usepackage{amsmath,amssymb,amsfonts}
\usepackage{algorithmic}
\usepackage{graphicx}
\usepackage{textcomp}
\usepackage{xcolor}
\def\BibTeX{{\rm B\kern-.05em{\sc i\kern-.025em b}\kern-.08em
    T\kern-.1667em\lower.7ex\hbox{E}\kern-.125emX}}
\begin{document}

\title{Trazadores Cúbicos\\}

\author{\IEEEauthorblockN{1\textsuperscript{st} Mauro Alonso Gonzalez Figueroa}
	\IEEEauthorblockA{\textit{Universidad Tecnologica de Bolivar} \\
		\textit{UTB}\\
		Cartagena, Colombia \\
		maugonzalez@utb.edu.co}}

\maketitle

\bibliographystyle{IEEEtran}

\begin{abstract} This paper focuses on Cubic Splines, a numerical method
	widely used for interpolating functions. Cubic splines are piecewise
	polynomials that are used for interpolation of data points. The method
	involves constructing a series of cubic polynomials that pass through
	specified points and have continuous first and second derivatives. This
	paper explores the theoretical foundation of cubic splines, including
	their construction, properties, and applications. Additionally, it
	discusses practical considerations such as the choice of boundary
	conditions and the use of natural, clamped, or periodic splines. The goal
	is to provide a comprehensive understanding of cubic splines and their
	utility in numerical analysis and scientific computing.
\end{abstract}

\begin{IEEEkeywords}
	Cubic Splines, Interpolation, Polynomials
\end{IEEEkeywords}

\nocite{Trazadores_cúbicos_2014}

\section{Introducción}

La interpolación numérica, un pilar fundamental en disciplinas científicas y
técnicas, se emplea para estimar valores intermedios entre datos discretos
conocidos.

Entre los métodos de interpolación más destacados se encuentra
el método de trazadores cúbicos. Estos trazadores proveen una forma eficiente
y precisa de aproximar una función entre puntos discretos mediante polinomios
cúbicos.

Su utilidad abarca diversas aplicaciones, desde el ajuste de curvas
en el análisis de datos hasta la simulación numérica en ingeniería y ciencias
naturales. Los trazadores cúbicos encuentran aplicación en una amplia gama de
campos, incluyendo ingeniería, ciencias de la computación, matemáticas
aplicadas y ciencias sociales, destacándose por su versatilidad y eficacia
en la interpolación y el análisis de datos.

En este documento, exploraremos la teoría, el uso y las aplicaciones 
prácticas del método de trazadores cúbicos, desde su fundamentación 
matemática hasta su implementación en diversas áreas del conocimiento.

\section{Teoría del método}

Un trazador cúbico $S$ es una función a trozos que interpola a $f$ en los
$n + 1$ puntos $(x_0, y_0), (x_1, y_1), (x_2, y_2), \dots, (x_n, y_n)$
(con $a = x_0 < x_1 < \cdots < x_n = b$). S es definida de la siguiente manera,

\[
	S(x) =
	\begin{cases}
		S_{0}(x)     & \text{si } x \in [x_{0}, x_{1}]     \\
		S_{1}(x)     & \text{si } x \in [x_{1}, x_{2}]     \\
		\vdots       & \vdots                              \\
		S_{n - 1}(x) & \text{si } x \in [x_{n - 1}, x_{n}]
	\end{cases}
\]

\bibliography{./Bibliography/bibliography.bib}

\end{document}
