\documentclass[conference]{IEEEtran}
\IEEEoverridecommandlockouts{}

\usepackage{cite}
\usepackage{amsmath,amssymb,amsfonts}
\usepackage{algorithmic}
\usepackage{graphicx}
\usepackage{textcomp}
\usepackage{xcolor}
\usepackage{tabularx}
\usepackage{float}

\def\BibTeX{{\rm B\kern-.05em{\sc i\kern-.025em b}\kern-.08em
    T\kern-.1667em\lower.7ex\hbox{E}\kern-.125emX}}
\begin{document}

\title{Método Simplex\\}

\author{\IEEEauthorblockN{1\textsuperscript{st} Mauro Alonso Gonzalez Figueroa}
\IEEEauthorblockA{\textit{Universiddad Tecnologica de Bolivar} \\
    \textit{UBT}\\
    Cartagena, Colombia \\
    maugonzalez@utb.edu.co}
\and
\IEEEauthorblockN{2\textsuperscript{nd} German De Armas Castaño}
\IEEEauthorblockA{\textit{Universidad Tecnologica de Bolivar} \\
    \textit{UTB}\\
    Cartagena, Colombia \\
    gdearmas@utb.edu.co}
\and
\IEEEauthorblockN{3\textsuperscript{rd} Angel Vega Rodriguez}
\IEEEauthorblockA{\textit{Universidad Tecnologica de Bolivar} \\
    \textit{UTB}\\
    Cartagena, Colombia \\
    anvega@utb.edu.co}
}

\maketitle

\bibliographystyle{IEEETran}

\begin{abstract}
In the realm of numerical analysis, the Gauss-Seidel method stands as an 
iterative technique employed to resolve systems of linear equations. It 
represents a refined version of the Jacobi method, often exhibiting 
superior convergence properties. Within the scope of this paper, we delve 
into the application of the Gauss-Seidel method to address a system of 
linear equations arising from a financial planning conundrum. Our 
endeavor demonstrates the efficacy and efficiency of the Gauss-Seidel 
method in tackling this type of system.
\end{abstract}

\begin{IEEEkeywords}
Gauss-Seidel method, linear equations, financial planning
\end{IEEEkeywords}

\nocite{source_code}

\section{Introduction}
En el ámbito de la gestión empresarial, la Investigación de Operaciones 
\textit{(IO)} se erige como una herramienta fundamental para la optimización de 
procesos y la toma de decisiones estratégicas. Esta disciplina, basada 
en el método científico y el análisis matemático, permite a las 
organizaciones abordar problemas complejos de manera sistemática y 
eficiente, conduciéndolas hacia el logro de sus objetivos.

En esencia, la IO se caracteriza por la aplicación de un enfoque 
interdisciplinario que integra conocimientos de diversas áreas, como 
las matemáticas, la estadística, la ingeniería y la economía. Este 
enfoque colaborativo permite a los equipos de trabajo abordar problemas 
complejos desde múltiples perspectivas, generando soluciones innovadoras 
y efectivas.

\section{Gauss-Seidel}

Dentro del amplio arsenal de técnicas empleadas en la IO, el método 
Gauss-Seidel se destaca como un algoritmo iterativo de gran utilidad para 
la resolución de sistemas de ecuaciones lineales. Este método, ampliamente 
reconocido por su simplicidad y eficiencia, se basa en la actualización 
progresiva de las variables del sistema en función de las aproximaciones 
más recientes de las demás. A través de un proceso iterativo, las 
estimaciones se van refinando hasta alcanzar un punto de convergencia, donde 
las soluciones satisfacen los criterios establecidos.

En el contexto de la programación lineal, el método Gauss-Seidel cobra 
especial relevancia como herramienta complementaria al método simplex. Si 
bien el método simplex se considera el enfoque tradicional para resolver 
este tipo de problemas, el método Gauss-Seidel ofrece una alternativa viable 
para abordar problemas de gran escala o aquellos con características 
específicas que dificultan la aplicación del simplex.

\bibliography{./Bibliography/bibliography.bib}

\end{document}
