\documentclass[conference]{IEEEtran}
\IEEEoverridecommandlockouts{}

\usepackage{cite}
\usepackage{amsmath,amssymb,amsfonts}
\usepackage{algorithmic}
\usepackage{graphicx}
\usepackage{textcomp}
\usepackage{xcolor}
\def\BibTeX{{\rm B\kern-.05em{\sc i\kern-.025em b}\kern-.08em
    T\kern-.1667em\lower.7ex\hbox{E}\kern-.125emX}}
\begin{document}

\title{Convergencia en el método de Newton-Raphson\\}

\author{\IEEEauthorblockN{1\textsuperscript{st} Mauro Alonso Gonzalez Figueroa}
	\IEEEauthorblockA{\textit{Universidad Tecnologica de Bolivar} \\
		\textit{UTB}\\
		Cartagena, Colombia \\
		maugonzalez@utb.edu.co}}

\maketitle

\bibliographystyle{IEEEtran}

\begin{abstract} Este documento se centra en el método de Newton-Raphson, un
	algoritmo ampliamente utilizado para encontrar aproximaciones numéricas
	de las raíces de una función real. Aunque este método es conocido por su
	convergencia cuadrática, existen ciertas condiciones en las que esta
	convergencia no se mantiene. En particular, la convergencia cuadrática
	puede fallar cuando la derivada de la función en la raíz es cero o cuando
	el valor inicial está demasiado cerca de un extremo\cite{convergence}. Este
	trabajo explora
	estas situaciones problemáticas y propone soluciones para garantizar
	la convergencia cuadrática del método de Newton-Raphson. Se discuten
	estrategias como la elección de un punto de partida adecuado y la
	modificación del algoritmo para manejar funciones con derivadas nulas
	en sus raíces. El objetivo es proporcionar una comprensión más profunda
	del comportamiento del método de Newton-Raphson y mejorar su eficacia
	en la práctica.
\end{abstract}

\section{Introducción}

El método de Newton-Raphson, nombrado así por Sir Isaac Newton y Joseph
Raphson, es un algoritmo para encontrar aproximaciones numéricas a las
raíces (o ceros) de una función de valor real. Es un método iterativo
que comienza con una suposición inicial y luego utiliza la derivada de
la función para aproximar la raíz\cite{metodos_numericos}.

Aunque el método de Newton-Raphson es conocido por su eficiencia y
convergencia cuadrática, no está exento de limitaciones. En particular,
el método puede fallar en converger para ciertos tipos de funciones o
suposiciones iniciales. Por ejemplo, si la derivada de la función en la
raíz es cero, o si la suposición inicial está demasiado cerca de un extremo,
el método puede no converger cuadráticamente.

Este trabajo tiene como objetivo explorar estos escenarios en detalle,
entender por qué el método de Newton-Raphson falla en converger en estos
casos, y proponer soluciones para asegurar la convergencia cuadrática del
método. Al mejorar nuestra comprensión del método de Newton-Raphson, podemos
mejorar su efectividad en la práctica y ampliar su aplicabilidad en diversos
campos de la ciencia y la ingeniería.

\nocite{bolivar2005metodo}

\nocite{canterometodo}

\section{Teoría del método}

El método de Newton-Raphson posibilita la determinación de una raíz de una
ecuación no lineal, siempre y cuando se inicie con una estimación
inicial adecuada. El procedimiento iterativo de Newton se deriva a partir
del desarrollo de Taylor de la función en torno a la estimación inicial.

De forma generalizada, se obtiene la aproximación:

\begin{equation}
	x_{n + 1} = x_0 - \frac{f(x)}{f'()}
\end{equation}

\subsection{Condiciones de convergencia}

Las condiciones de convergencia para el método de Newton-Raphson se pueden
resumir de la siguiente manera:

\begin{itemize}
	\item Existencia de la Raíz: Se requiere que dentro de un intervalo
	      de trabajo dado [a, b], la condición $f(a) * f(b) < 0$ sea satisfecha.

	\item Unicidad de la Raíz:
	      En el intervalo [a, b], la derivada de $f(x)$ no debe ser igual a cero.

	\item Concavidad: La gráfica de la función $f(x)$ en el intervalo [a, b]
	      debe ser cóncava, ya sea hacia arriba o hacia abajo. Esto se verifica
	      asegurando que $f''(x) \leqslant 0$ o $f''(x) \geqslant 0$ para todo x en [a, b].

	\item Intersección de la Tangente a $f(x)$ dentro de [a, b]:
	      Es crucial garantizar que la tangente a la curva en el extremo del
	      intervalo [a, b], donde $f'(x)$ es mínima, intersecte el eje $x$
	      dentro de [a, b]. Esta condición asegura que la sucesión de
	      valores de $x_i$ permanezca dentro de [a, b].

	      \begin{equation*}
		      \frac{\left\lvert f(x) \right\rvert }{\left\lvert f'(x) \right\rvert} \leqslant (b - a)
	      \end{equation*}
\end{itemize}

\section{Problemas del método}

Como fue mencionado anteriormente, el método de Newton-Raphson, presenta
algunas dificultades en ciertos casos.

\begin{itemize}
	\item Punto de inflexión [$F''(x) = 0$] en la vecindad de una raíz:
	      La presencia de un punto de inflexión, donde la segunda derivada,
	      $F''(x)$, se anula, en la cercanía de una raíz puede influir en
	      el comportamiento del método de Newton-Raphson.

	\item Tendencia del método a oscilar alrededor de un mínimo o un
	      máximo local:
	      El método puede exhibir oscilaciones alrededor de mínimos o
	      máximos locales, lo cual puede afectar su convergencia,
	      especialmente si la función presenta características extremas
	      en la vecindad de la raíz.

	\item Valor inicial cercano a una raíz salta a una posición varias
	      raíces más lejos:
	      Existe la posibilidad de que un valor inicial cercano a una raíz
	      se desplace a una posición considerablemente distante, lo cual
	      se atribuye a pendientes cercanas a cero, generando una tendencia
	      a alejarse del área de interés.

	\item Pendiente cero [$f'(x) = 0$] causa una división entre cero
	      en la fórmula de Newton-Raphson:
	      Cuando la derivada primera de la función se anula [$f'(x) = 0$], se
	      produce una división entre cero en la fórmula de Newton-Raphson,
	      provocando que la solución se desplace horizontalmente y no
	      intersecte el eje x. Este fenómeno puede resultar en una
	      divergencia del método.
\end{itemize}

\section{Soluciones a los problemas del método}

Una vez identificados los problemas asociados al método de Newton-Raphson, es
imperativo considerar posibles soluciones para mitigar estas dificultades.

\begin{itemize}
	\item Punto de inflexión [$F''(x) = 0$] en la vecindad de una raíz:
	      \begin{itemize}
		      \item Emplear técnicas de análisis gráfico para identificar
		            la presencia de puntos de inflexión en la vecindad de las
		            raíces antes de aplicar el método.

		      \item Considerar variantes del método, como el método de la
		            secante, que pueden ser menos susceptibles a condiciones de
		            inflexión
	      \end{itemize}

	\item Tendencia del método a oscilar alrededor de un mínimo o un máximo
	      local:
	      \begin{itemize}
		      \item Ajustar el tamaño del paso en cada iteración para evitar
		            oscilaciones excesivas.

		      \item Utilizar información adicional sobre la función, \\ como
		            la segunda derivada, para adaptar la convergencia del método
		            en presencia de extremos locales.
	      \end{itemize}

	\item Valor inicial cercano a una raíz salta a una posición varias
	      raíces más lejos:
	      \begin{itemize}
		      \item Refinar la estimación inicial mediante métodos de búsqueda de
		            raíces previos para obtener una aproximación más precisa.

		      \item Implementar técnicas de control de convergencia, como la
		            verificación de la distancia entre iteraciones sucesivas.
	      \end{itemize}

	\item Pendiente cero [$f'(x) = 0$] causa una división entre cero en la
	      fórmula de Newton-Raphson:
	      \begin{itemize}
		      \item Modificar la formulación del método para evitar la división
		            por cero, por ejemplo, mediante la aplicación de técnicas de
		            regularización.
		      \item Explorar métodos alternativos, como el método de la
		            secante, que no dependen de la derivada primera.

	      \end{itemize}
\end{itemize}

\section{Raíces multiples}

La presencia de raíces múltiples es otro desafío que puede afectar la
convergencia del método de Newton-Raphson. Las raíces múltiples se
caracterizan por tener derivadas iguales a cero en el punto de la raíz, lo que
conduce a problemas en la fórmula de Newton-Raphson que involucra la
división por la derivada primera. Aquí hay algunas estrategias para
abordar este problema:

\begin{itemize}
	\item Método de la Secante: En lugar de depender de la derivada primera,
	      el método de la secante utiliza una aproximación numérica de la derivada
	      basada en dos puntos consecutivos. Este método es menos sensible a las
	      raíces múltiples y puede ser una alternativa efectiva.

	\item Multiplicidad Conocida:
	      Si se conoce la multiplicidad de la raíz (es decir, cuántas veces
	      se repite una raíz), se pueden ajustar las fórmulas de Newton-Raphson
	      para manejar directamente esta multiplicidad. La fórmula
	      modificada se obtiene dividiendo la función y su derivada
	      por la multiplicidad.

	\item Métodos Iterativos Específicos:
	      Existen métodos iterativos específicos diseñados para tratar
	      raíces múltiples, como el método de Bairstow para polinomios.
	      Estos métodos pueden ser más robustos en situaciones donde las
	      raíces múltiples son frecuentes.

	\item Algoritmos de Optimización:
	      En casos más complejos, considerar algoritmos de optimización no
	      basados en derivadas, como el algoritmo de Levenberg-Marquardt, que
	      se utiliza comúnmente para problemas de ajuste de curvas y puede
	      manejar raíces múltiples.

	\item Inicialización Cuidadosa:
	      La elección de una estimación inicial cercana a la raíz múltiple
	      correcta puede mejorar la convergencia. Métodos adicionales, como
	      la búsqueda de raíces previa, pueden ayudar a obtener estimaciones
	      iniciales más precisas.

\end{itemize}

\section{Conclusiones}

En conclusión, el método de Newton-Raphson, aunque potente y eficaz para 
la aproximación de raíces de ecuaciones no lineales, presenta ciertas 
limitaciones y desafíos que deben ser abordados con precaución. La 
identificación de puntos de inflexión, la tendencia a oscilar alrededor 
de mínimos o máximos locales, la posible divergencia de valores iniciales 
y la presencia de raíces múltiples son aspectos críticos que pueden afectar 
la convergencia del método.

\bibliography{./Bibliography/bibliography.bib}

\end{document}
