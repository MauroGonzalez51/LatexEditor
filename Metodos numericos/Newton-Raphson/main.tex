\documentclass[conference]{IEEEtran}
\IEEEoverridecommandlockouts{}

\usepackage{cite}
\usepackage{amsmath,amssymb,amsfonts}
\usepackage{algorithmic}
\usepackage{graphicx}
\usepackage{textcomp}
\usepackage{xcolor}
\def\BibTeX{{\rm B\kern-.05em{\sc i\kern-.025em b}\kern-.08em
    T\kern-.1667em\lower.7ex\hbox{E}\kern-.125emX}}
\begin{document}

\title{Convergencia en el método de Newton-Raphson\\}

\author{\IEEEauthorblockN{1\textsuperscript{st} Mauro Alonso Gonzalez Figueroa}
	\IEEEauthorblockA{\textit{Universidad Tecnologica de Bolivar} \\
		\textit{UTB}\\
		Cartagena, Colombia \\
		maugonzalez@utb.edu.co}}

\maketitle

\bibliographystyle{IEEEtran}

\begin{abstract} Este documento se centra en el método de Newton-Raphson, un
	algoritmo ampliamente utilizado para encontrar aproximaciones numéricas
	de las raíces de una función real. Aunque este método es conocido por su
	convergencia cuadrática, existen ciertas condiciones en las que esta
	convergencia no se mantiene. En particular, la convergencia cuadrática
	puede fallar cuando la derivada de la función en la raíz es cero o cuando
	el valor inicial está demasiado cerca de un extremo\cite{convergence}. Este
	trabajo explora
	estas situaciones problemáticas y propone soluciones para garantizar
	la convergencia cuadrática del método de Newton-Raphson. Se discuten
	estrategias como la elección de un punto de partida adecuado y la
	modificación del algoritmo para manejar funciones con derivadas nulas
	en sus raíces. El objetivo es proporcionar una comprensión más profunda
	del comportamiento del método de Newton-Raphson y mejorar su eficacia
	en la práctica.
\end{abstract}

\section{Introducción}

El método de Newton-Raphson, nombrado así por Sir Isaac Newton y Joseph
Raphson, es un algoritmo para encontrar aproximaciones numéricas a las
raíces (o ceros) de una función de valor real. Es un método iterativo
que comienza con una suposición inicial y luego utiliza la derivada de
la función para aproximar la raíz\cite{metodos_numericos}.

Aunque el método de Newton-Raphson es conocido por su eficiencia y
convergencia cuadrática, no está exento de limitaciones. En particular,
el método puede fallar en converger para ciertos tipos de funciones o
suposiciones iniciales. Por ejemplo, si la derivada de la función en la
raíz es cero, o si la suposición inicial está demasiado cerca de un extremo,
el método puede no converger cuadráticamente.

Este trabajo tiene como objetivo explorar estos escenarios en detalle,
entender por qué el método de Newton-Raphson falla en converger en estos
casos, y proponer soluciones para asegurar la convergencia cuadrática del
método. Al mejorar nuestra comprensión del método de Newton-Raphson, podemos
mejorar su efectividad en la práctica y ampliar su aplicabilidad en diversos
campos de la ciencia y la ingeniería.

\nocite{bolivar2005metodo}

\section{Desarrollo de Contenidos}

\bibliography{./Bibliography/bibliography.bib}

\end{document}
