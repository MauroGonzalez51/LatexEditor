\documentclass[letterpaper, 12pt]{report}
\usepackage[utf8]{inputenc}
\usepackage[english, spanish]{babel}
\usepackage{fullpage} % changes the margin
\usepackage{graphicx} 
\usepackage{enumitem} 
\usepackage{chngcntr}
\counterwithin{figure}{section}
\renewcommand{\thesection}{\arabic{section}} 
\renewcommand{\thesubsection}{\thesection.\arabic{subsection}}
\renewcommand{\baselinestretch}{1.5}
\usepackage{float}
\bibliographystyle{apalike}


\begin{document}

% Hoja de presentación ----------------------------------------------------------------------|>

% \begin{titlepage}
% 	\centering
% 	\includegraphics[width=0.3\textwidth]{./Images/logo_utb.png}\par\vspace{1cm}
% 	{\scshape\LARGE Universidad Tecnológica de Bolívar \\}

% 	\vspace{3cm}
% 	{\scshape\Large Taller de escritura académica \\}

% 	\vspace{2.5cm}
% 	\slshape {\itshape{} Mauro González, T00067622 \\}

% 	{\large \today\par}
% \end{titlepage}

% Contenido ---------------------------------------------------------------------------------|>

\chapter*{Importancia de la coherencia en los textos escritos}

La coherencia por definición es una propiedad semántica y pragmática del texto
que establece relaciones entre cada oración con las demás que conforman una
secuencia y a su vez con los conceptos que aparecen en cada oración, es decir,
nos permite adherir ideas para que juntas cobren un sentido global.

\vspace{.5cm}

Por lo anterior, su importancia reside en que es necesaria para poder
comunicarnos bien y mas importante aún, dar a entender lo que queremos
expresar (ya sea oral o de manera escrita).

\vspace{.5cm}

Asimismo, también es importante para la escritura académica, ya que la anterior
depende de un flujo de ideas suave y lógico, para así no
fatigar al lector, y a su vez mejorar la legibilidad de nuestro trabajo.
~\cite{ImportanciaCoherencia}

\newpage

\bibliography{./Bibliography/bibliography.bib}

\end{document}