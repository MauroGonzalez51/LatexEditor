%& --shell-escape
\documentclass[letterpaper, 12pt]{report}
\usepackage[utf8]{inputenc}
\usepackage[english, spanish]{babel}
\usepackage{fullpage} % changes the margin
\usepackage{graphicx} 
\usepackage{amsmath}
\usepackage{enumitem} 
\usepackage{chngcntr}
\usepackage{setspace}
\usepackage{url}
\counterwithin{figure}{section}
\renewcommand{\thesection}{\arabic{section}} 
\renewcommand{\thesubsection}{\thesection.\arabic{subsection}}
\renewcommand{\baselinestretch}{1.5}
\usepackage{float}
\usepackage{verbatim}
\usepackage{apacite}
\bibliographystyle{apacite}
\setlength\belowcaptionskip{10pt}
\linespread{1.5}

\begin{document}

\chapter*{ETS y embarazo a temprana edad en CTG}

% 
% ! Tópico: Educación sexual en Cartagena 
% 
% 

En Cartagena de Indias, durante el año 2022 se presentaron un total de 2649
embarazos en madres adolescentes, de los cuales 91 corresponden a madres de 10
a 14 años~\cite{CifrasCartagena}. Lo cual es curioso, ya que esto supone una
supuesta disminución en los casos tomando como referencia pasados años.

\bigskip

Pero el punto no es exponer cuantos embarazos adolescentes suceden en Cartagena
cada año, la idea es analizar la causa detrás de esta problemática. La educación
sexual es definida como un proceso de enseñanza y aprendizaje de alta calidad
acerca de temas relacionados con la sexualidad y la salud reproductiva, ademas
ayuda a que las personas obtengan las herramientas necesarias para manejar su
relación con ellas mismas, sus parejas, comunidades, y con su propia salud
sexual~\cite{EducacionSexualDefinicion}.

\bigskip

A pesar de ser un tema muy relevante para la comunidad, muchas veces tiende a ser
menospreciado por los padres de familia, o incluso el mismo sistema educativo,
en el que muchos jóvenes adolescentes se gradúan sin siquiera conocer la
temática. Tomando en cuenta a Cartagena de Indias, en el año 2022 se ha
evidenciado un aumento del 27\% en los casos de VIH en comparación al año
2021, siendo los estrados 1 y 2, los que presentan un mayor numero de reportes
~\cite{CasosVIH}. % .....

\bigskip

% ! Tesis .... ('Tesis que sugieren')
La educación sexual supone una necesidad que ayudaría a frenar los indices de
embarazo en adolescentes, y a su vez ayudar a la población a tener una mejor
planificación de vida en temas de sexualidad.

\newpage

\bibliography{./Bibliography/bibliography.bib}

\end{document}