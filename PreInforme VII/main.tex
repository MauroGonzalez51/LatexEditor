\documentclass[twocolumn, 12pt]{article}

\usepackage[utf8]{inputenc}
\usepackage[english, spanish]{babel}
\usepackage{fullpage} % changes the margin
\usepackage{graphicx}
\usepackage{amsmath}
\usepackage{enumitem}
\usepackage{chngcntr}
\usepackage{setspace}
\usepackage{url}
\usepackage{csquotes}
\usepackage{float}
\usepackage{verbatim}
\usepackage{tabularx}
\usepackage{amsmath}

\counterwithin{figure}{section}
\renewcommand{\thesection}{\arabic{section}}
\renewcommand{\thesubsection}{\thesection.\arabic{subsection}}
\renewcommand{\baselinestretch}{1.5}

\usepackage[style=apa, maxnames=6, minnames=3, backend=biber]{biblatex}
\DefineBibliographyStrings{english}{%chktex-file 1 chktex-file 6
	andothers = {\em et\addabbrvspace al\adddot}
}
\addbibresource{./Bibliography/bibliography.bib}

\usepackage{array}
\usepackage{enumitem}
\usepackage{breqn}

\raggedbottom{}

\begin{document}

% 									|		PRACTICA 6                 |
%  									|		GUÍA No. 8                 |

\begin{titlepage}
	\centering
	\includegraphics[width=0.3\textwidth]{Images/logo_utb.png}\par\vspace{1cm}
	{\scshape\LARGE Universidad Tecnológica de Bolívar \par}
	\vspace{1cm}

	{\scshape\Large FÍSICA ELÉCTRICA \par}
	\vspace{.2cm}

	% chktex-file 8
	{\scshape\Large H1 - C \par}
	\vspace{1cm}
	% chktex-file 8 chktex-file 13
	\slshape {\Large \bfseries{} LAB 7 - CIRCUITOS DE CORRIENTE DIRECTA. LEYES DE KIRCHHOFF  \\}
	\slshape {\small \bfseries{} Guía de laboratorio No. 5}
	\vspace{1cm}

	\slshape {\itshape{} Mauro González, T00067622 \\}
	\slshape {\itshape{} German De Armas Castaño, T00068765 \\}
	\slshape {\itshape{} Angel Vega Rodriguez, T00068186 \\}
	\slshape {\itshape{} Juan Jose Osorio Ariza, T00067316 \\}
	\slshape {\itshape{} Juan Eduardo barón, T00065901 \\}
	\vfill
	Revisado Por \\
	Gabriel Hoyos Gomez Casseres\\
	{\large \today\par}
\end{titlepage}

% ----------------------------------------------------------------------|>
\section{Introducción}

Los circuitos de corriente directa (DC) son aquellos en los
que la corriente eléctrica fluye en una dirección constante
y continua. Esto significa que la polaridad de la fuente de
energía ya sea una batería o un generador, no cambia con el
tiempo. En un circuito de corriente directa, los electrones
fluyen desde el polo negativo de la fuente de energía hacia
el polo positivo a través de los conductores del circuito.

Los circuitos de corriente directa son relativamente
simples de diseñar y construir, ya que la corriente fluye
en una sola dirección y no hay preocupaciones sobre la fase
de la corriente. Además, los circuitos de corriente directa
son eficientes en la transmisión de energía, lo que los
hace útiles en aplicaciones que requieren una alta
eficiencia energética.

Ahora bien, las leyes de Kirchhoff \textit{(ley de nodos)}
son un conjunto de reglas fundamentales en la teoría de
circuitos eléctricos que describen cómo se comporta la
corriente eléctrica en un circuito cerrado a partir de los
siguientes postulados:

\begin{enumerate}
	\item En cualquier punto de un circuito cerrado, la suma de las
	      corrientes que entran en ese punto debe ser igual a la suma
	      de las corrientes que salen de ese punto. En otras
	      palabras, la corriente que entra en un nodo o punto de
	      conexión debe ser igual a la corriente que sale de ese
	      nodo.

	\item Ley de voltajes de Kirchhoff (también conocida como la ley
	      de mallas): Esta ley establece que, en cualquier circuito
	      cerrado, la suma algebraica de las caídas de voltaje en
	      todas las resistencias alrededor de cualquier trayectoria
	      cerrada (o malla) debe ser igual a la suma algebraica de
	      las fuerzas electromotrices (fuentes de voltaje) alrededor
	      de la misma trayectoria cerrada.
\end{enumerate}

Durante la experiencia se presentará una explicación
detallada de las leyes de Kirchhoff, la electricidad y el
análisis de circuitos, todo acompañado de sus respectivas
ecuaciones y definiciones.

% ----------------------------------------------------------------------|>
\section{Objetivos}

% -----------------------------------|>
\subsection*{Objetivo General}

\begin{itemize}[label=$\triangleright$]
	\item Identificar el funcionamiento de los circuitos y comprobar
	      las diferentes leyes que pueden intervenir en estos (Leyes
	      de Kirchhoff, Ley de malla, ley de nodos).
\end{itemize}

% -----------------------------------|>
\subsection*{Objetivos específicos}

\begin{itemize}[label=$\triangleright$]
	\item Comprobar en qué circunstancias se cumple la ley de malla.
	\item Precisar los voltajes correspondientes para evitar
	      recalentamientos y/o fundición de los resistores.
	\item Verificar los factores influyentes en la ley de nodos.
\end{itemize}

% ----------------------------------------------------------------------|>
\section{Preparación de la practica}

% --------------------------------------------------|>
\subsection*{¿Cuáles son las Leyes de Kirchhoff?}

\nocite{KhanAcademy}

Las leyes de Kirchhoff del voltaje y la corriente están en
el corazón del análisis de circuitos. Con estas dos leyes,
más las ecuaciones para cada componente individual
\textit{(resistor, capacitor, inductor)}, tenemos el
conjunto de herramientas básicas que necesitamos para
comenzar a analizar circuitos.

% --------------------|>
\subsubsection*{La ley de corriente de Kirchhoff}

La ley de la corriente de Kirchhoff dice que la suma de
todas las corrientes que fluyen hacia un nodo es igual a la
suma de las corrientes que salen del nodo. Se puede
escribir como:

{\Large
\begin{equation}
	\sum i_{adentro} = \sum i_{afuera}
\end{equation}

}

% --------------------|>
\subsubsection*{Ley de voltaje de Kirchhoff}

La suma de los voltajes alrededor de una malla es igual a
cero. Podemos escribir la ley de voltaje de Kirchhoff como:

{\Large
\begin{equation}
	\sum v_n = 0
\end{equation}
}

donde \textit{n} es el número de voltajes de los
componentes en la malla. También puedes enunciar la ley de
voltaje de Kirchhoff de otra manera: alrededor de una
malla, la suma de subidas de voltaje es igual a la suma de
bajadas de voltaje.

	{\Large
		\begin{equation}
			\sum v_{subida} = \sum v_{bajada}
		\end{equation}
	}

La ley de voltaje de Kirchhoff presenta algunas propiedades
que son de gran utilidad en el análisis de circuitos
eléctricos.

\begin{itemize}
	\item En primer lugar, es posible trazar una trayectoria cerrada
	      en el circuito eléctrico que inicie en cualquier nodo y
	      regrese al mismo, lo que se conoce como una malla.
	\item Al caminar alrededor de esta malla y sumar los voltajes de
	      los componentes que se encuentran en ella, se obtiene como
	      resultado cero. Cabe destacar que la dirección en la que se
	      recorre la malla no altera la validez de la ley de voltaje
	      de Kirchhoff.
	\item Si el circuito eléctrico cuenta con múltiples mallas, cada
	      una de ellas puede ser tratada por separado aplicando la
	      ley de voltaje de Kirchhoff de manera individual.
\end{itemize}

% --------------------------------------------------|>
\subsection*{Resuelva el punto 9 del análisis (solución del circuito a estudiar para las corrientes).
	Indique el procedimiento donde utiliza las leyes de Kirchhoff.}

Para aplicar las leyes de Kirchhoff y obtener una expresión
para calcular las corrientes en el circuito, es necesario
seguir los siguientes pasos:

\begin{enumerate}
	\item Identificar los nodos del circuito y etiquetarlos con un
	      número.
	\item Identificar las mallas del circuito y etiquetarlas con una
	      letra.
	\item Aplicar la ley de nodos de Kirchhoff a cada nodo del
	      circuito, estableciendo que la suma de las corrientes que
	      entran en un nodo es igual a la suma de las corrientes que
	      salen del mismo.
	\item Aplicar la ley de voltajes de Kirchhoff a cada malla del
	      circuito, estableciendo que la suma de las caídas de
	      voltaje en una malla es igual a la suma de las fuentes de
	      voltaje en la misma malla.
	\item Resolver el sistema de ecuaciones obtenido a partir de las
	      leyes de Kirchhoff para encontrar las corrientes en el
	      circuito en términos de las resistencias y la fuerza
	      electromotriz (fem).
\end{enumerate}

Una vez que se tiene el sistema de ecuaciones, se pueden
utilizar técnicas de álgebra lineal para resolver las
ecuaciones y encontrar las corrientes en el circuito. La
expresión final para calcular las corrientes dependerá de
la configuración específica del circuito y las fuentes de
energía utilizadas.

% ----------------------------------------------------------------------|>
\section{Resumen del procedimiento}

El primer paso del experimento es armar el montaje que se
muestra en la figura 1 de la guía con la ayuda del
simulador. El montaje consta de 2 baterías, 4 resistencias
y los cables que unen todo el circuito. Luego conectamos un
medidor de amperímetros entre las secciones del circuito.
Tener en cuenta el voltaje suministrado para no fundir las
resistencias. En este caso presencial armamos el mismo
montaje con la ayuda de una placa protoboard. Después
anotar los valores de la resistencia en la tabla número 1.
Seguido medir la diferencia de potencia en las secciones
que nos indica la tabla número 2 y se registran los datos,
del mismo modo medimos las corrientes que nos indica la
tabla número 3 y registramos los datos. Por último,
cambiamos los polos de las baterías, y volvemos a medir los
datos de la tabla número 3.

\newpage

\printbibliography

\end{document}