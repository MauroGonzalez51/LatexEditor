\documentclass[twocolumn, 12pt]{article}

\usepackage[utf8]{inputenc}
\usepackage[english, spanish]{babel}
\usepackage{fullpage}
\usepackage{graphicx}
\usepackage{amsmath}
\usepackage{enumitem}
\usepackage{chngcntr}
\usepackage{setspace}
\usepackage{url}
\usepackage{csquotes}
\usepackage{float}
\usepackage{verbatim}
\usepackage{tabularx}
\usepackage{amsmath}
\usepackage{caption}

\counterwithin{figure}{section}
\renewcommand{\thesection}{\arabic{section}}
\renewcommand{\thesubsection}{\thesection.\arabic{subsection}}
\renewcommand{\baselinestretch}{1.5}

\usepackage[style=apa, maxnames=6, minnames=3, backend=biber]{biblatex}
\DefineBibliographyStrings{english}{%chktex-file 1 chktex-file 6
    andothers = {\em et\addabbrvspace al\adddot}
}
\addbibresource{./Bibliography/bibliography.bib}

\usepackage{array}
\usepackage{enumitem}

\setlength{\parskip}{0pt}

\raggedbottom{}

\begin{document}

\begin{titlepage}
    \centering
    \includegraphics[width=0.3\textwidth]{Images/logo_utb.png}\par\vspace{1cm}
    {\scshape\LARGE Universidad Tecnológica de Bolívar \par}
    \vspace{1cm}

    {\scshape\Large FÍSICA ELÉCTRICA \par}
    \vspace{.2cm}

    % chktex-file 8
    {\scshape\Large M2 - C \par}
    \vspace{1cm}
    % chktex-file 8
    \slshape {\Large \bfseries{}Informe de Laboratorio No. VII\\}
    \vspace{1cm}

    \slshape {\itshape{} Mauro González, T00067622 \\}
    \slshape {\itshape{} German De Armas Castaño, T00068765 \\}
    \slshape {\itshape{} Angel Vega Rodriguez, T00068186 \\}
    \slshape {\itshape{} Juan Jose Osorio Ariza, T00067316 \\}
    \slshape {\itshape{} Juan Eduardo barón, T00065901 \\}
    \vfill
    Revisado Por \\
    Gabriel Hoyos Gomez Casseres\\
    {\large \today\par}
\end{titlepage}

% ----------------------------------------------------------------------|>
\section{Introducción}

% ----------------------------------------------------------------------|>
\section{Objetivos}

\subsection{Objetivo general}

\subsection{Objetivos específicos}

% ----------------------------------------------------------------------|>
\section{Marco Teórico}

% -------------------------------------------------------|>
\subsection{Leyes de Kirchhoff} \nocite{KhanAcademy__Kirchhoff}

% --------------------------------|>
\subsubsection{Ley de Corriente}

La ley de corriente de Kirchhoff establece que la suma
algebraica de todas las corrientes que convergen en un nodo
es igual a la suma algebraica de las corrientes que
divergen del nodo. Matemáticamente, se puede expresar de la
siguiente manera:

{\large
\begin{equation}
    \sum i_{adentro} = \sum i_{afuera}
\end{equation}
}

% --------------------------------|>
\subsubsection{Ley de Voltaje}

La suma de los voltajes alrededor de una malla es igual a
cero, lo cual se puede expresar mediante la ley de voltaje
de Kirchhoff de la siguiente manera:

En su forma general, considerando \textit{n} como el número
de voltajes de los componentes en la malla, se tiene:

{\large
\begin{equation}
    \sum v_n = 0
\end{equation}
}

Asimismo, la ley de voltaje de Kirchhoff se puede enunciar
de la siguiente manera: alrededor de una malla, la suma de
las subidas de voltaje es igual a la suma de las bajadas de
voltaje. Esto se expresa como:

{\large
\begin{equation}
    \sum v_{\text{subida}} = \sum v_{\text{bajada}}
\end{equation}
}

% ----------------------------------------------------------------------|>
\section{Montaje Experimental}

% ----------------------------------------------------------------------|>
\section{Datos Experimentales}

\begin{table}[H]
    \captionsetup{justification=centering}
    \centering

    % chktex-file 44
    \begin{tabularx}{0.9\linewidth}{|>{\centering\arraybackslash}X|>{\centering\arraybackslash}X|>{\centering\arraybackslash}X|>{\centering\arraybackslash}X|}
        \multicolumn{4}{c}{Valor de resistencias $(K\Omega)$} \\ \hline

        $R1$ & $R2$   & $R3$   & $R4$                         \\ \hline
        $10$ & $0.47$ & $0.33$ & $2.2$                        \\ \hline
    \end{tabularx}

    \caption{Resistencia}

    % chktex-file 24
    \label{tab:datosExperimentales__Resistencias}
\end{table}

\vspace{.5cm}

\begin{table}[H]
    \captionsetup{justification=centering}
    \centering

    \begin{tabularx}{0.9\linewidth}{|>{\centering\arraybackslash}X|>{\centering\arraybackslash}X|>{\centering\arraybackslash}X|>{\centering\arraybackslash}X|>{\centering\arraybackslash}X|}
        \multicolumn{5}{c}{Valor de corrientes $(mA)$} \\ \hline

        $I1$  & $I2$  & $I3$  & $I4$  & $I5$           \\ \hline
        $9.9$ & $3.6$ & $6.1$ & $0.5$ & $5.3$          \\ \hline
    \end{tabularx}

    \caption{Corriente}

    % chktex-file 24
    \label{tab:datosExperimentales__Corriente}
\end{table}

\vspace{.5cm}

\begin{table}[H]
    \captionsetup{justification=centering}
    \centering

    \begin{tabularx}{0.9\linewidth}{|>{\centering\arraybackslash}X|>{\centering\arraybackslash}X|>{\centering\arraybackslash}X|>{\centering\arraybackslash}X|}

        \multicolumn{4}{c}{Diferencias de potencial $(V)$}        \\ \hline
        $E1 \linebreak = B_{ab}$ & $V_{bc}$ & $V_{cd}$ & $V_{ef}$ \\ \hline
        $11.92$                  & $-10.06$ & $-1.86$  & $-1.8$   \\ \hline

        $E2 \linebreak = V_{gh}$ & $V_{hf}$ & $V_{ce}$ & \dots    \\ \hline
        $-0.45$                  & $-1.4$   & $0$      & \dots    \\ \hline
    \end{tabularx}

    \caption{Voltaje}

    % chktex-file 24
    \label{tab:datosExperimentales__DIFFPotencial}
\end{table}

\vspace{.5cm}

\begin{table}[H]
    \captionsetup{justification=centering}
    \centering

    \begin{tabularx}{0.9\linewidth}{|>{\centering\arraybackslash}X|>{\centering\arraybackslash}X|>{\centering\arraybackslash}X|>{\centering\arraybackslash}X|>{\centering\arraybackslash}X|}
        \multicolumn{5}{c}{Valor de corrientes $(mA)$} \\ \hline

        $I1$   & $I2$   & $I3$  & $I4$  & $I5$         \\ \hline
        $9.98$ & $3.49$ & $5.9$ & $0.5$ & $5.6$        \\ \hline

    \end{tabularx}
    \caption{Corriente \textit{(FEM)}Invertida}

    % chktex-file 24
    \label{tab:datosExperimentales__Corriente-FEMInvertida}
\end{table}

% ----------------------------------------------------------------------|>
\section{Análisis de datos}

% ----------------------------------------------------------------------|>
\section{Conclusiones}

\printbibliography

\end{document}