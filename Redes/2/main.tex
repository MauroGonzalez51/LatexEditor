\documentclass[letterpaper, 12pt]{article}

\usepackage[utf8]{inputenc}
\usepackage[english, spanish]{babel}

\usepackage{newtxtext}

\usepackage{fullpage}
\usepackage{graphicx}
\usepackage{amsmath}
\usepackage{enumitem}
\usepackage{chngcntr}
\usepackage{setspace}
\usepackage{xurl}
\usepackage{csquotes}
\usepackage{float}
\usepackage{verbatim}
\usepackage{tabularx}
\usepackage{amsmath}
\usepackage{caption}
\usepackage{bm}
\usepackage{wrapfig}
\usepackage{siunitx}

\counterwithin{figure}{section}
\renewcommand{\thesection}{\arabic{section}}
\renewcommand{\thesubsection}{\thesection.\arabic{subsection}}
\renewcommand{\baselinestretch}{1.5}
\renewcommand{\thefigure}{\arabic{figure}}

\usepackage[style=apa, maxnames=6, minnames=3]{biblatex}
\DefineBibliographyStrings{english}{%chktex-file 1 chktex-file 6
      andothers = {\em et\addabbrvspace al\adddot}
}

\addbibresource{./Bibliography/bibliography.bib}

\usepackage{array}
\usepackage{enumitem}

\setlength{\parskip}{\baselineskip}

\newcommand{\bolditalic}[1]{\textbf{\textit{#1}}}

\DeclareSIUnit{\COP}{COP}
\newcommand{\cop}[1]{\$\SI{#1}{\COP}}

\DeclareSIUnit{\DOLLAR}{USD}
\newcommand{\dollar}[1]{\$\SI{#1}{\DOLLAR}}

\renewcommand{\comment}[1]{{\small $\ll$#1$\gg$}}

% chktex-file 24

\begin{document}

\section*{Who invented the internet?}

\begin{itemize}[label=$\triangleright$]
    \item Mauro Gonzalez Figueroa \textit{(T00067622)}
\end{itemize}

\nocite{*}

Hay una diferencia entre Internet y la World Wide Web. La World Wide Web es lo
que facilita compartir información utilizando todas las computadoras
interconectadas en Internet.

Conectar tantos mainframes llevó a la pregunta: ¿qué se necesita hacer para
facilitar las comunicaciones entre ellos?

Científicos de todo el mundo intentaron responder a esta pregunta, llegando a
algunos conceptos clave:

\begin{itemize}
    \item Conmutación de paquetes: El concepto de ``cortar'' archivos en paquetes más
          pequeños cuando se envían y reensamblarlos al recibirlos. Este método mejora la
          eficiencia y la fiabilidad de la transmisión de datos.
    \item TCP/IP:@{}Había diferentes infraestructuras de red, pero no podían comunicarse
          entre sí. Los protocolos TCP/IP resolvieron este problema creando el lenguaje
          básico de comunicación para Internet, permitiendo que diversas redes se
          interconecten y compartan datos sin problemas.
    \item World Wide Web: Una interfaz creada con la introducción de \textit{HTTP}
          (Hypertext Transfer Protocol), \textit{HTML} (Hypertext Markup Language) y
          \textit{URL} (Uniform Resource Locator), que hizo posibles los navegadores web
          y revolucionó la forma en que se accede y comparte la información.
\end{itemize}

La invención de Internet no se puede atribuir a una sola persona; hubo muchos
elementos fundamentales diferentes que colectivamente construyeron lo que ahora
conocemos como ``Internet''.

\printbibliography

\end{document}
