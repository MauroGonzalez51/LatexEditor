\documentclass[letterpaper, 12pt]{article}

\usepackage[utf8]{inputenc}
\usepackage[english, spanish]{babel}

\usepackage{newtxtext}

\usepackage{fullpage}
\usepackage{graphicx}
\usepackage{amsmath}
\usepackage{enumitem}
\usepackage{chngcntr}
\usepackage{setspace}
\usepackage{xurl}
\usepackage{csquotes}
\usepackage{float}
\usepackage{verbatim}
\usepackage{tabularx}
\usepackage{amsmath}
\usepackage{caption}
\usepackage{bm}
\usepackage{wrapfig}
\usepackage{siunitx}

\counterwithin{figure}{section}
\renewcommand{\thesection}{\arabic{section}}
\renewcommand{\thesubsection}{\thesection.\arabic{subsection}}
\renewcommand{\baselinestretch}{1.5}
\renewcommand{\thefigure}{\arabic{figure}}

\usepackage[style=apa, maxnames=6, minnames=3]{biblatex}
\DefineBibliographyStrings{english}{%chktex-file 1 chktex-file 6
      andothers = {\em et\addabbrvspace al\adddot}
}

\addbibresource{./Bibliography/bibliography.bib}

\usepackage{array}
\usepackage{enumitem}

\setlength{\parskip}{\baselineskip}

\newcommand{\bolditalic}[1]{\textbf{\textit{#1}}}

\DeclareSIUnit{\COP}{COP}
\newcommand{\cop}[1]{\$\SI{#1}{\COP}}

\DeclareSIUnit{\DOLLAR}{USD}
\newcommand{\dollar}[1]{\$\SI{#1}{\DOLLAR}}

\renewcommand{\comment}[1]{{\small $\ll$#1$\gg$}}

% chktex-file 24

\begin{document}

\section*{Who invented the internet?}

\begin{itemize}[label=$\triangleright$]
      \item Mauro Gonzalez Figueroa \textit{(T00067622)}
\end{itemize}

\nocite{*}

There's a difference between the Internet and the World Wide Web. The World
Wide Web is what makes it easy to share information using all the
interconnected computers on the Internet.

Connecting so many mainframes led to the question: what do you need to do to
make communications between them easier?

Scientists around the world tried to answer this question, coming up with some
key concepts:

\begin{itemize}
      \item Packet switching: The concept of ``cutting'' files into smaller packets when sent
            and reassembling them upon receiving. This method improves the efficiency and
            reliability of data transmission.
      \item TCP/IP:\@{}There were different network infrastructures, but they weren't able
            to communicate with each other. The TCP/IP protocols solved this problem by
            creating the basic communication language for the Internet, allowing diverse
            networks to interconnect and share data seamlessly.
      \item World Wide Web: An interface created with the introduction of \textit{HTTP}
            (Hypertext Transfer Protocol), \textit{HTML} (Hypertext Markup Language), and
            \textit{URL} (Uniform Resource Locator), making web browsers possible and
            revolutionizing how information is accessed and shared.
\end{itemize}

The invention of the Internet can't be attributed to just one person; there
were many different foundational elements that collectively built what we now
know as ``the Internet''.

\printbibliography

\end{document}