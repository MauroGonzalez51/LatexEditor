\documentclass[letterpaper, 12pt]{article}

\usepackage[utf8]{inputenc}
\usepackage[english, spanish]{babel}

\usepackage{newtxtext}

\usepackage{fullpage}
\usepackage{graphicx}
\usepackage{amsmath}
\usepackage{enumitem}
\usepackage{chngcntr}
\usepackage{setspace}
\usepackage{xurl}
\usepackage{csquotes}
\usepackage{float}
\usepackage{verbatim}
\usepackage{tabularx}
\usepackage{amsmath}
\usepackage{caption}
\usepackage{bm}
\usepackage{wrapfig}
\usepackage{siunitx}

\counterwithin{figure}{section}
\renewcommand{\thesection}{\arabic{section}}
\renewcommand{\thesubsection}{\thesection.\arabic{subsection}}
\renewcommand{\baselinestretch}{1.5}
\renewcommand{\thefigure}{\arabic{figure}}

\usepackage[style=apa, maxnames=6, minnames=3]{biblatex}
\DefineBibliographyStrings{english}{%chktex-file 1 chktex-file 6
      andothers = {\em et\addabbrvspace al\adddot}
}

\addbibresource{./Bibliography/bibliography.bib}

\usepackage{array}
\usepackage{enumitem}

\setlength{\parskip}{\baselineskip}

\newcommand{\bolditalic}[1]{\textbf{\textit{#1}}}

\DeclareSIUnit{\COP}{COP}
\newcommand{\cop}[1]{\$\SI{#1}{\COP}}

\DeclareSIUnit{\DOLLAR}{USD}
\newcommand{\dollar}[1]{\$\SI{#1}{\DOLLAR}}

\renewcommand{\comment}[1]{{\small $\ll$#1$\gg$}}

% chktex-file 24

\begin{document}

\section*{Historia del Internet}

\begin{itemize}[label=$\triangleright$]
    \item Mauro Gonzalez Figueroa \textit{(T00067622)}
\end{itemize}

\nocite{*}

Todo comenzó en \textit{1957}, cuando las computadoras solo eran capaces de
hacer una cosa a la vez (esto se llamaba procesamiento por lotes), por lo que
los ingenieros tuvieron que colocarlas en salas separadas. Todo se hacía
mediante conexiones de cable, incluso la programación.

Más tarde, comenzó a surgir un concepto, \textit{Time-Sharing}, que es la
capacidad de ``compartir'' la potencia de procesamiento de una computadora
entre múltiples usuarios.

Durante la Guerra Fría, se crearon algunas redes para evitar la pérdida
repetitiva de datos y garantizar que la comunicación permaneciera robusta.

\begin{itemize}
    \item ARPANET:@{} Creada para manejar la comunicación a través de la red sin exponer
          la computadora principal. La idea era usar el \textit{IMP} (Interface Message
          Processor) conectado a un mainframe. Esta tecnología también servía como una
          interfaz que al mismo tiempo estaba conectada a otros IMPs en una red
          \textit{(IMP Subnet)}.
\end{itemize}

A medida que las redes continuaron desarrollándose, quedó claro que era
necesario un sistema descentralizado para mantener la integridad y
disponibilidad de la información. Esta descentralización tenía como objetivo
prevenir puntos únicos de falla y asegurar que la red pudiera continuar
funcionando incluso si partes de ella eran comprometidas o destruidas. Esto
llevó al desarrollo de redes robustas e interconectadas donde los datos podían
viajar por múltiples caminos para llegar a su destino, mejorando tanto la
seguridad como la fiabilidad.

El enfoque descentralizado también facilitó el crecimiento de internet, ya que
permitió una infraestructura más resiliente y escalable. Esta innovación allanó
el camino para la vasta red global de la que dependemos hoy en día, donde la
información se intercambia y procesa continuamente a través de una multitud de
sistemas interconectados.

\printbibliography

\end{document}
