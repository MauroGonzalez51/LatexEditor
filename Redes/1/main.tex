\documentclass[letterpaper, 12pt]{article}

\usepackage[utf8]{inputenc}
\usepackage[english, spanish]{babel}

\usepackage{newtxtext}

\usepackage{fullpage}
\usepackage{graphicx}
\usepackage{amsmath}
\usepackage{enumitem}
\usepackage{chngcntr}
\usepackage{setspace}
\usepackage{xurl}
\usepackage{csquotes}
\usepackage{float}
\usepackage{verbatim}
\usepackage{tabularx}
\usepackage{amsmath}
\usepackage{caption}
\usepackage{bm}
\usepackage{wrapfig}
\usepackage{siunitx}

\counterwithin{figure}{section}
\renewcommand{\thesection}{\arabic{section}}
\renewcommand{\thesubsection}{\thesection.\arabic{subsection}}
\renewcommand{\baselinestretch}{1.5}
\renewcommand{\thefigure}{\arabic{figure}}

\usepackage[style=apa, maxnames=6, minnames=3]{biblatex}
\DefineBibliographyStrings{english}{%chktex-file 1 chktex-file 6
      andothers = {\em et\addabbrvspace al\adddot}
}

\addbibresource{./Bibliography/bibliography.bib}

\usepackage{array}
\usepackage{enumitem}

\setlength{\parskip}{\baselineskip}

\newcommand{\bolditalic}[1]{\textbf{\textit{#1}}}

\DeclareSIUnit{\COP}{COP}
\newcommand{\cop}[1]{\$\SI{#1}{\COP}}

\DeclareSIUnit{\DOLLAR}{USD}
\newcommand{\dollar}[1]{\$\SI{#1}{\DOLLAR}}

\renewcommand{\comment}[1]{{\small $\ll$#1$\gg$}}

% chktex-file 24

\begin{document}

\section*{History of the Internet}


\begin{itemize}[label=$\triangleright$]
      \item Mauro Gonzalez Figueroa \textit{(T00067622)}
\end{itemize}

\nocite{*}

It all begins in \textit{1957}, when computers were only capable of doing one
thing at a time (this was called batch processing), so the engineers had to put
them in separate rooms. Everything was done via cable connections, even
programming.

Later on, a concept began to arise, \textit{Time-Sharing}, which is the ability
to ``share'' the processing power of one computer among multiple users.

During the Cold War, some networks were created to avoid repetitive data loss
and ensure communication remained robust.

\begin{itemize}
      \item ARPANET:\@{} Created to handle communication through the network without
            exposing the main computer. The idea was to use the \textit{IMP} (Interface
            Message Processor) connected to a mainframe. This technology also served as an
            interface that at the same time was connected to other IMPs in a network
            \textit{(IMP Subnet)}.
\end{itemize}

As networks continued to develop, it became clear that a decentralized system
was necessary to maintain the integrity and availability of information. This
decentralization aimed to prevent single points of failure and ensure that the
network could continue to function even if parts of it were compromised or
destroyed. This led to the development of robust, interconnected networks where
data could travel multiple paths to reach its destination, enhancing both
security and reliability.

The decentralized approach also facilitated the growth of the internet, as it
allowed for a more resilient and scalable infrastructure. This innovation paved
the way for the vast, global network we rely on today, where information is
continuously exchanged and processed across a multitude of interconnected
systems.

\printbibliography

\end{document}