\documentclass[twocolumn, 12pt]{article}

\usepackage[utf8]{inputenc}
\usepackage[english, spanish]{babel}
\usepackage{fullpage}
\usepackage{graphicx}
\usepackage{amsmath}
\usepackage{enumitem}
\usepackage{chngcntr}
\usepackage{setspace}
\usepackage{url}
\usepackage{csquotes}
\usepackage{float}
\usepackage{verbatim}
\usepackage{tabularx}
\usepackage{amsmath}
\usepackage{caption}
\usepackage{bm}
% \usepackage{hyperref}

\counterwithin{figure}{section}
\renewcommand{\thesection}{\arabic{section}}
\renewcommand{\thesubsection}{\thesection.\arabic{subsection}}
\renewcommand{\baselinestretch}{1.5}

\usepackage[style=numeric, maxnames=6, minnames=3, backend=biber, parentracker=true, sorting=none]{biblatex}
\DefineBibliographyStrings{english}{%chktex-file 1 chktex-file 6
      andothers = {\em et\addabbrvspace al\adddot}
}

\addbibresource{./Bibliography/bibliography.bib}

\usepackage{array}
\usepackage{enumitem}

\setlength{\parskip}{0pt}

\newcommand{\bolditalic}[1]{\textbf{\textit{#1}}}

\newcommand{\Celsius}[0]{°$\mathcal{C}$}
\newcommand{\Kelvin}[0]{°$\mathcal{K}$}
\newcommand{\Fahrenheit}[0]{°$\mathcal{F}$}

% chktex-file 44

\begin{document}

\begin{titlepage}
      \centering
      \includegraphics[width=0.3\textwidth]{Images/logo_utb.png}\par\vspace{1cm}
      {\scshape\LARGE Universidad Tecnológica de Bolívar \par}
      \vspace{1cm}

      {\scshape\Large FÍSICA CALOR Y ONDAS \par}
      \vspace{.2cm}

      % chktex-file 8
      {\scshape\Large Grupo 1 \par}
      \vspace{2cm}
      % chktex-file 8
      \slshape {\Large \bfseries{}LAB 6 - LEY DE STEFAN-BOLTZMANN PARA LA RADIACIÓN\\}
      \slshape {\small \bfseries{} Guía de laboratorio No. 6}
      \vspace{4cm}

      \slshape {\itshape{} Mauro González, T00067622 \\}
      % \slshape {\itshape{} German De Armas Castaño, T00068765 \\}
      % \slshape {\itshape{} Angel Vega Rodriguez, T00068186 \\}
      % \slshape {\itshape{} Juan Jose Osorio Ariza, T00067316 \\}
      % \slshape {\itshape{} Jorge Alberto Rueda Salgado, T00068722 \\}
      \vfill
      Revisado Por \\
      Duban Andres Paternina Verona\\
      {\large \today\par}
\end{titlepage}

% ! ----------------------------------------------------------------------|>
\section{Introducción}

La ley de Stefan-Boltzmann para la radiación es uno de los
pilares fundamentales de la física que describe cómo los
objetos emiten radiación térmica en función de su
temperatura. En esta experiencia, exploraremos este
principio físico crucial al calibrar una termopila para la
medición de la temperatura absoluta. La termopila es un
dispositivo que aprovecha el efecto termoeléctrico para
cuantificar la radiación térmica de un cuerpo negro, que es
un objeto idealizado que absorbe y emite radiación de
manera perfecta. Esta práctica nos permitirá comprender
mejor cómo los cuerpos emiten energía en forma de radiación
térmica y cómo esta radiación se relaciona con la
temperatura.

% ! ----------------------------------------------------------------------|>
\section{Objetivos}

\subsection{Objetivo general}

El objetivo principal de esta experiencia es calibrar una
termopila para medir la temperatura absoluta de un objeto
utilizando la ley de Stefan-Boltzmann. A través de esta
calibración, podremos cuantificar de manera precisa la
radiación térmica emitida por el objeto y relacionarla con
su temperatura.

\subsection{Objetivos específicos}

\begin{itemize}[label=$\triangleright$]
      \item Comprender el concepto de cuerpo negro y su importancia en
            la radiación térmica.

      \item Familiarizarse con el funcionamiento de una termopila y el
            principio físico que la rige.

      \item Definir las cantidades físicas clave relacionadas con la
            radiación, como la potencia y la intensidad.

      \item Aplicar la ley de Stefan-Boltzmann para relacionar la
            potencia radiante con la temperatura de un cuerpo negro.

      \item Registrar y analizar los datos experimentales para calibrar
            la termopila y determinar la temperatura absoluta del
            objeto.
\end{itemize}

% ! ----------------------------------------------------------------------|>
\section{Preparación de la practica}

% * ----------------------------------------------------------------|>
\subsection{El cuerpo negro~\cite{El_cuerpo_negro}}

El término radiación se refiere a la emisión continua de
energía desde la superficie de cualquier cuerpo, esta
energía se denomina radiante y es transportada por las
ondas electromagnéticas que viajan en el vacío a la
velocidad de $3 \times 10^{8} \frac{m}{s}$. Las ondas de
radio, las radiaciones infrarrojas, la luz visible, la luz
ultravioleta, los rayos X y los rayos gamma, constituyen
las distintas regiones del espectro electromagnético.

La superficie de un cuerpo negro es un caso límite, en el
que toda la energía incidente desde el exterior es
absorbida y toda la energía incidente desde el interior es
emitida.

No existe en la naturaleza un cuerpo negro, incluso el
negro de humo refleja el 1\% de la energía incidente.

Sin embargo, un cuerpo negro se puede sustituir con gran
aproximación por una cavidad con una pequeña abertura. La
energía radiante incidente a través de la abertura, es
absorbida por las paredes en múltiples reflexiones y
solamente una mínima proporción escapa (se refleja) a
través de la abertura.

% * ----------------------------------------------------------------|>
\subsection{En este experimento ¿qué equipo se usará como aproximación a un cuerpo
      negro?}

En la experiencia 6, se utilizará el ``Black body
accessory'' como una aproximación a un cuerpo negro. La ley
de Stefan-Boltzmann se aplica a la radiación de un cuerpo
negro ideal, que absorbe y emite toda la radiación que
incide sobre él. Aunque ningún objeto en la vida real es un
cuerpo negro perfecto, algunos accesorios o dispositivos se
diseñan para que sean lo más cercanos posible a un cuerpo
negro y sigan aproximadamente esta ley para la radiación.

% * ----------------------------------------------------------------|>
\subsection{¿Qué es una termopila y qué principio físico rige su comportamiento?}

Las termopilas convierten la energía térmica en energía
eléctrica. Las termopilas utilizan varios termopares
conectados en serie o en paralelo. Las termopilas se
utilizan para la detección de temperatura sin contacto. La
función de una termopila es transferir la radiación de
calor emitida por el objeto a una salida de tensión. La
salida está en el rango de decenas o cientos de
milivoltios~\cite{Termopilas}.

Las termopilas funcionan según el principio físico del
efecto Seebeck, que es un fenómeno termoeléctrico en el
cual se genera una diferencia de voltaje eléctrico (una
fuerza electromotriz o fem) en un circuito eléctrico cuando
se establece una diferencia de temperatura a lo largo de
sus extremos. Este efecto es responsable de la generación
de electricidad a partir de diferencias de temperatura y es
la base de funcionamiento de las termopilas.

% * ----------------------------------------------------------------|>
\subsection{Las siguientes cantidades físicas sirven para cuantificar la radiación \dots}

\begin{itemize}[label=$\triangleright$]
      \item Potencia: La potencia radiante (P) es la cantidad de
            energía radiante emitida o recibida por un objeto o sistema
            en una unidad de tiempo determinada. En el contexto de la
            radiación, esta energía radiante puede manifestarse en
            forma de radiación térmica, como el calor infrarrojo, y es
            fundamental para comprender cómo los objetos emiten y
            absorben energía térmica a través de la radiación
            electromagnética~\cite{Potencia}.

      \item Intensidad: La intensidad en el ámbito de la física se
            refiere a la potencia transferida por unidad de área, donde
            el área se encuentra en un plano perpendicular a la
            dirección de propagación de la energía. En el Sistema
            Internacional de Unidades (SI), la intensidad se mide en
            vatios por metro cuadrado ($\frac{W}{m^2}$). Esta medida se
            utiliza principalmente en el estudio de las ondas, como la
            radiación electromagnética, para cuantificar la cantidad de
            energía transmitida por unidad de área~\cite{Intensidad}.
\end{itemize}

% * ----------------------------------------------------------------|>
\subsection{Que dice la ley de Stefan-Boltzmann para la radiación de un cuerpo negro~\cite{Stefan-Boltzmann}}

La energía radiada por un radiador de cuerpo negro por
segundo, por unidad de superficie, es proporcional a la
cuarta potencia de la temperatura absoluta y está dada por

\begin{equation*}
      \begin{gathered}
            \frac{P}{A} = \sigma T^{4} \frac{j}{m^{2}s} \\
            \sigma = 5.6703 \times 10^{-8}~W/m^{2}K^{4}
      \end{gathered}
\end{equation*}

Para objetos calientes distintos de los radiadores ideales,
la ley se expresa en la forma

\begin{equation*}
      \frac{P}{A} = e \sigma T^{4}
\end{equation*}

donde e es la emisividad del objeto (e = 1 para el radiador
ideal). Si el objeto caliente está radiando energía hacia
su entorno mas frío a un temperatura $T_c$, la tasa de
pérdida de radiación neta, toma la forma

\begin{equation*}
      P = e \sigma A (T^{4} - T_{c}^{4})
\end{equation*}

La fórmula de Stefan-Boltzmann, también, está relacionada
con la densidad de energía en la radiación hacia un volumen
de espacio determinado.

% ! ----------------------------------------------------------------------|>
\section{Resumen del procedimiento}

En esta experiencia, utilizaremos un cilindro de latón
bruñido como una aproximación a un cuerpo negro. El
cilindro se calentará a diferentes temperaturas en un horno
eléctrico, y se medirá la radiación térmica emitida por
este cuerpo negro utilizando una termopila de Moll. Para
calibrar la termopila, registraremos la temperatura
ambiente y la diferencia de potencial generada por la
termopila debido a la radiación del cuerpo negro. Luego,
aplicaremos la ley de Stefan-Boltzmann para determinar la
temperatura absoluta del cilindro de latón y explorar la
relación entre la temperatura y la radiación térmica. Esta
práctica nos permitirá comprender mejor cómo se cuantifica
la radiación térmica y cómo se aplica la ley de
Stefan-Boltzmann en situaciones experimentales.

\printbibliography

\end{document}