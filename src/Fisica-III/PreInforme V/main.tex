\documentclass[twocolumn, 12pt]{article}

\usepackage[utf8]{inputenc}
\usepackage[english, spanish]{babel}
\usepackage{fullpage}
\usepackage{graphicx}
\usepackage{amsmath}
\usepackage{enumitem}
\usepackage{chngcntr}
\usepackage{setspace}
\usepackage{url}
\usepackage{csquotes}
\usepackage{float}
\usepackage{verbatim}
\usepackage{tabularx}
\usepackage{amsmath}
\usepackage{caption}
\usepackage{bm}
% \usepackage{hyperref}

\counterwithin{figure}{section}
\renewcommand{\thesection}{\arabic{section}}
\renewcommand{\thesubsection}{\thesection.\arabic{subsection}}
\renewcommand{\baselinestretch}{1.5}

\usepackage[style=numeric, maxnames=6, minnames=3, backend=biber, parentracker=true, sorting=none]{biblatex}
\DefineBibliographyStrings{english}{%chktex-file 1 chktex-file 6
    andothers = {\em et\addabbrvspace al\adddot}
}

\addbibresource{./Bibliography/bibliography.bib}

\usepackage{array}
\usepackage{enumitem}

\setlength{\parskip}{0pt}

\newcommand{\bolditalic}[1]{\textbf{\textit{#1}}}

\newcommand{\Celsius}[0]{°$\mathcal{C}$}
\newcommand{\Kelvin}[0]{°$\mathcal{K}$}
\newcommand{\Fahrenheit}[0]{°$\mathcal{F}$}

% chktex-file 44

\begin{document}

\begin{titlepage}
    \centering
    \includegraphics[width=0.3\textwidth]{Images/logo_utb.png}\par\vspace{1cm}
    {\scshape\LARGE Universidad Tecnológica de Bolívar \par}
    \vspace{1cm}

    {\scshape\Large FÍSICA CALOR Y ONDAS \par}
    \vspace{.2cm}

    % chktex-file 8
    {\scshape\Large Grupo 1 \par}
    \vspace{1cm}
    % chktex-file 8
    \slshape {\Large \bfseries{}LAB 5 - CALOR ESPECÍFICO DE LOS SÓLIDOS\\}
    \slshape {\small \bfseries{} Guía de laboratorio No. 5}
    \vspace{4cm}

    \slshape {\itshape{} Mauro González, T00067622 \\}
    % \slshape {\itshape{} German De Armas Castaño, T00068765 \\}
    % \slshape {\itshape{} Angel Vega Rodriguez, T00068186 \\}
    % \slshape {\itshape{} Juan Jose Osorio Ariza, T00067316 \\}
    % \slshape {\itshape{} Jorge Alberto Rueda Salgado, T00068722 \\}
    \vfill
    Revisado Por \\
    Duban Andres Paternina Verona\\
    {\large \today\par}
\end{titlepage}

% ! ----------------------------------------------------------------------|>
\section{Introducción}

El estudio de las propiedades térmicas de los sólidos es
fundamental para comprender su comportamiento ante cambios
de temperatura y su capacidad para almacenar y transferir
calor. En esta experiencia, nos enfocaremos en determinar
el calor específico de diferentes sólidos mediante un
experimento de transferencia de calor. Para lograr esto,
aplicaremos principios fundamentales de la termodinámica y
utilizaremos un calorímetro para medir con precisión los
cambios de temperatura en el sistema.

% ! ----------------------------------------------------------------------|>
\section{Objetivos}

\subsection{Objetivo general}

El objetivo principal de esta experiencia es determinar el
calor específico de varios sólidos. El calor específico es
una propiedad fundamental que describe la cantidad de calor
necesaria para elevar la temperatura de una unidad de masa
de un material en una unidad de temperatura. Este parámetro
es crucial para comprender cómo los sólidos responden a la
transferencia de calor y cómo almacenan energía térmica.

\subsection{Objetivos específicos}

\begin{itemize}[label=$\triangleright$]
    \item Comprender los conceptos de temperatura, calor y calor
          específico de una sustancia.

    \item Familiarizarse con las escalas de temperatura y su
          importancia en las mediciones térmicas.

    \item Investigar los principios de la ley cero de la
          termodinámica y la primera ley de la termodinámica.

    \item Determinar la masa equivalente del calorímetro utilizado en
          el experimento.

    \item Calcular el calor específico de diferentes sólidos
          utilizando un calorímetro adiabático.

    \item Comparar los valores experimentales obtenidos con los
          valores teóricos conocidos para validar los resultados.
\end{itemize}

% ! ----------------------------------------------------------------------|>
\section{Preparación de la practica}

% * ----------------------------------------------------------------|>
\subsection{¿Que es temperatura?~\cite{Significados_2022_temperatura}}

La temperatura es una magnitud física que indica la energía
interna de un cuerpo, de un objeto o del medio ambiente en
general, medida por un termómetro.

Dicha energía interna se expresa en términos de calor y
frío, siendo el primero asociado con una temperatura más
alta, mientras que el frío se asocia con una temperatura
más baja.

Las unidades de medida de temperatura son los grados
Celsius (\Celsius), los grados Fahrenheit (\Fahrenheit) y
los grados Kelvin (\Kelvin). El cero absoluto (0 \Kelvin)
corresponde a $-273,15$ \Celsius.

% * ----------------------------------------------------------------|>
\subsection{¿Qué es calor?~\cite{Fernandes_2022_calor}}

El calor es el proceso de transferencia de energía que
fluye entre un sistema y su ambiente a causa de la
diferencia de temperatura entre ellos.

La transferencia de energía o calor se establece entre un
cuerpo y el ambiente que le rodea cuando existe una
diferencia de temperatura. Esta transferencia termina
cuando se llega al equilibrio térmico, esto es, cuando la
temperatura entre las partes es la misma.

% * ----------------------------------------------------------------|>
\subsection{¿Qué es el calor especifico de una sustancia?~\cite{Fernandes_2022_calor}~\cite{Significados_2016_calorEspecifico}}

Como calor específico se conoce la magnitud física que
expresa la cantidad de calor que requiere una sustancia por
unidad de masa para que su temperatura aumente en una
unidad, temperatura que es medida, por lo general, en
grados Celsius.

Como tal, el calor específico es una propiedad intensiva de
la materia, pues su valor es representativo de cada
sustancia o materia, cada una de las cuales, a su vez,
presenta valores diferentes de acuerdo con el estado en que
se encuentre (líquido, sólido o gaseoso).

La fórmula de calor específico es $c = C / m$, donde $c$
representa el calor específico la sustancia, $C$ la
capacidad térmica y m su masa. De modo que para obtener el
calor específico es necesario dividir la capacidad térmica
entre la masa.

    {\small
        \begin{equation*}
            \begin{gathered}
                CalorEspecifico (c) = \frac{CapacidadCalorica}{masa} \\
                CalorEspecifico (c) = \frac{c}{m}
            \end{gathered}
        \end{equation*}

    }

% * ----------------------------------------------------------------|>
\subsection{Las escalas de temperaturas.~\cite{Escalasdetemperatura}}

Las tres escalas de temperatura más comunes son: Celsius,
Fahrenheit y Kelvin. Una escala de temperatura puede ser
creada identificando dos temperaturas fácilmente
reproducibles.

La escala Celsius (\Celsius) toma en cuenta el valor 0 °
para el punto de fusión del agua, mientras que el punto de
ebullición del agua corresponde a 100°. En el caso de la
escala Fahrenheit (\Fahrenheit), la más utilizada en
Estados Unidos por ejemplo, el punto de fusión del agua
está a los 32\Fahrenheit y el de ebullición a los
212\Fahrenheit

\subsubsection{Conversion entre escalas}

\begin{table}[H]
    \begin{tabularx}{.9\linewidth}{|>{\centering\arraybackslash}X|>{\centering\arraybackslash}X|}
        \hline
        Convertir                          & Ecuación                                             \\\hline
        \Celsius~$\rightarrow$ \Fahrenheit & $\frac{9}{5} \cdot T($\Celsius$) + 32$               \\\hline
        \Fahrenheit~$\leftarrow$ \Celsius  & $\frac{5}{9} \cdot T($\Fahrenheit$) - 32$            \\\hline
        \Celsius~$\rightarrow$ \Kelvin     & \Celsius~$+~273,15$                                  \\\hline
        \Kelvin~$\leftarrow$ \Celsius      & \Kelvin~$-~273,15$                                   \\\hline
        \Fahrenheit~$\rightarrow$ \Kelvin  & $(\frac{5}{9} \cdot T($\Fahrenheit$) - 32) + 273,15$ \\\hline
        \Kelvin~$\rightarrow$\Fahrenheit   & $(\frac{9}{5} \cdot T($\Kelvin$) - 273,15) + 32$     \\\hline
    \end{tabularx}
\end{table}

% * ----------------------------------------------------------------|>
\subsection{Investigue sobre la ley cero de la termodinámica~\cite{Fernández_0}}

Se dice que dos cuerpos están en equilibrio térmico cuando,
al ponerse en contacto, sus variables de estado no cambian.
En torno a esta simple idea se establece la ley cero.

La ley cero de la termodinámica establece que, cuando dos
cuerpos están en equilibrio térmico con un tercero, estos
están a su vez en equilibrio térmico entre sí.

% * ----------------------------------------------------------------|>
\subsection{Investigue sobre la primera ley de la termodinámica~\cite{Fernández_1}}

La primera ley de la termodinámica relaciona el trabajo y
el calor transferido intercambiado en un sistema a través
de una nueva variable termodinámica, la energía interna.
Dicha energía ni se crea ni se destruye, sólo se
transforma.

\subsubsection{Energia interna~\cite{Fernández_1}}

La energía interna de un sistema es una caracterización
macroscópica de la energía microscópica de todas las
partículas que lo componen. Un sistema está formado por
gran cantidad de partículas en movimiento. Cada una de
ellas posee:

\begin{itemize}[label=$\triangleright$]
    \item Energía cinética, por el hecho de encontrarse a una
          determinada velocidad.

    \item Energía potencial gravitatoria, por el hecho de encontrarse
          en determinadas posiciones unas respecto de otras.

    \item Energía potencial elástica, por el hecho vibrar en el
          interior del sistema.
\end{itemize}

\subsubsection{Primera ley de la termodinamica~\cite{Fernández_1}}

La primera ley de la termodinámica establece una relación
entre la energía interna del sistema y la energía que
intercambia con el entorno en forma de calor o trabajo.

La primera ley de la termodinámica determina que la energía
interna de un sistema aumenta cuando se le transfiere calor
o se realiza un trabajo sobre él. Su expresión depende del
criterio de signos para sistemas termodinámicos elegido:

\begin{itemize}
    \item Criterio \textbf{IUPAC}: Se considera positivo aquello que
          aumenta la energía interna del sistema, o lo que es lo
          mismo, el trabajo recibido o el calor absorbido.

          \begin{equation*}
              \Delta U = Q + W
          \end{equation*}

    \item Criterio \textbf{Tradicional}: Se considera positivo el
          calor absorbido y el trabajo que realiza el sistema sobre
          el entorno.

          \begin{equation*}
              \Delta U = Q - W
          \end{equation*}
\end{itemize}

% ! ----------------------------------------------------------------------|>
\section{Resumen del procedimiento}

En esta experiencia, primero determinaremos la masa
equivalente del calorímetro (mk) utilizando agua y un
bloque sólido caliente. Luego, procederemos a medir el
calor específico de varios sólidos. Para esto, calentaremos
los sólidos y los colocaremos en el calorímetro junto con
agua, registrando las temperaturas iniciales y finales para
calcular el calor específico de cada sólido. Además,
compararemos los resultados experimentales con los valores
teóricos conocidos para validar nuestras mediciones y
comprender cómo diferentes materiales almacenan y
transfieren calor.

\printbibliography

\end{document}