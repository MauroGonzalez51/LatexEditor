\documentclass[twocolumn, 12pt]{article}

\usepackage[utf8]{inputenc}
\usepackage[english, spanish]{babel}
\usepackage{fullpage}
\usepackage{graphicx}
\usepackage{amsmath}
\usepackage{enumitem}
\usepackage{chngcntr}
\usepackage{setspace}
\usepackage{url}
\usepackage{csquotes}
\usepackage{float}
\usepackage{verbatim}
\usepackage{tabularx}
\usepackage{amsmath}
\usepackage{caption}
\usepackage{bm}
% \usepackage{hyperref}

\counterwithin{figure}{section}
\renewcommand{\thesection}{\arabic{section}}
\renewcommand{\thesubsection}{\thesection.\arabic{subsection}}
\renewcommand{\baselinestretch}{1.5}

\usepackage[style=numeric, maxnames=6, minnames=3, backend=biber, parentracker=true, sorting=none]{biblatex}
\DefineBibliographyStrings{english}{%chktex-file 1 chktex-file 6
      andothers = {\em et\addabbrvspace al\adddot}
}

\addbibresource{./Bibliography/bibliography.bib}

\usepackage{array}
\usepackage{enumitem}

\setlength{\parskip}{0pt}

\newcommand{\bolditalic}[1]{\textbf{\textit{#1}}}

\newcommand{\Celsius}[0]{°$\mathcal{C}$}
\newcommand{\Kelvin}[0]{°$\mathcal{K}$}
\newcommand{\Fahrenheit}[0]{°$\mathcal{F}$}

% chktex-file 44

\begin{document}

\begin{titlepage}
      \centering
      \includegraphics[width=0.3\textwidth]{Images/logo_utb.png}\par\vspace{1cm}
      {\scshape\LARGE Universidad Tecnológica de Bolívar \par}
      \vspace{1cm}

      {\scshape\Large FÍSICA CALOR Y ONDAS \par}
      \vspace{.2cm}

      % chktex-file 8
      {\scshape\Large Grupo 1 \par}
      \vspace{2cm}
      % chktex-file 8
      \slshape {\Large \bfseries{}LAB 7 - EFECTO FOTOELÉCTRICO\\}
      \slshape {\small \bfseries{} Guía de laboratorio No. 7}
      \vspace{4cm}

      \slshape {\itshape{} Mauro González, T00067622 \\}
      % \slshape {\itshape{} German De Armas Castaño, T00068765 \\}
      % \slshape {\itshape{} Angel Vega Rodriguez, T00068186 \\}
      % \slshape {\itshape{} Juan Jose Osorio Ariza, T00067316 \\}
      % \slshape {\itshape{} Jorge Alberto Rueda Salgado, T00068722 \\}
      \vfill
      Revisado Por \\
      Duban Andres Paternina Verona\\
      {\large \today\par}
\end{titlepage}

% ! ----------------------------------------------------------------------|>
\section{Introducción}

El efecto fotoeléctrico, un fenómeno fundamental en la
física, desempeña un papel crucial en la comprensión de la
dualidad onda-partícula de la luz. Este experimento tiene
como objetivo explorar este efecto y verificar la ecuación
de Einstein para el mismo.

El efecto fotoeléctrico implica la emisión de electrones
por un material cuando es iluminado con luz de cierta
longitud de onda, y su energía cinética depende de la
frecuencia de la luz incidente. Albert Einstein propuso la
idea de que la luz consiste en partículas llamadas fotones,
cada uno con energía proporcional a su frecuencia, lo que
proporciona una explicación cuántica de este efecto. En
esta práctica, aplicaremos esta teoría para calcular la
constante de Planck y la función de trabajo de la celda
fotoeléctrica.

% ! ----------------------------------------------------------------------|>
\section{Objetivos}

\subsection{Objetivo general}

Verificar experimentalmente la ecuación de Einstein para el
efecto fotoeléctrico y determinar la constante de Planck
($h$) y la función de trabajo ($W_0$) de la celda
fotoeléctrica.

\subsection{Objetivos específicos}

\begin{itemize}[label=$\triangleright$]
      \item Establecer la relación entre la tensión límite ($V_0$) y la
            frecuencia de la luz incidente en la celda fotoeléctrica.

      \item Construir una gráfica que relacione V0 y la frecuencia para
            verificar la ecuación de Einstein.

      \item Utilizar técnicas de regresión para encontrar una función
            analítica que se ajuste a los datos experimentales.
\end{itemize}

% ! ----------------------------------------------------------------------|>
\section{Preparación de la practica}

% + --------------------------------------------------------------|>
\subsection*{¿En qué consiste el efecto fotoeléctrico?~\cite{cite_3}}

Cuando la luz brilla en un metal, los electrones pueden ser
expulsados de la superficie del metal en un fenómeno
conocido como el efecto fotoeléctrico. También, a este
proceso suele llamársele fotoemisión, y a los electrones
que son expulsados del metal, fotoelectrones. En términos
de su comportamiento y sus propiedades, los fotoelectrones
no son diferentes de otros electrones. El prefijo foto
simplemente nos indica que los electrones han sido
expulsados de la superficie de un metal por la luz
incidente.

% + --------------------------------------------------------------|>
\subsection*{¿Por qué el efecto fotoeléctrico evidencia la naturaleza corpuscular de la radiación
      electromagnética?}

El efecto fotoeléctrico es una prueba de la naturaleza
corpuscular de la luz~\cite{cite_1}. Este fenómeno fue
descubierto por Heinrich Hertz en 1887 y estudiado
sistemáticamente por Philipp Lenard, quienes encontraron
que la teoría ondulatoria de la luz no podía explicar este
fenómeno~\cite{cite_2}.

La teoría corpuscular de la luz fue propuesta por Isaac
Newton en 1704, quien sugirió que la luz está formada por
partículas diminutas que se desplazan a una velocidad
enorme y en línea recta~\cite{cite_2}. Sin embargo, esta
teoría no pudo explicar todos los fenómenos asociados con
la luz.

Albert Einstein, en 1905, rescató el modelo corpuscular
para explicar el efecto fotoeléctrico. Según Einstein, la
luz está formada por partículas llamadas fotones cuya
energía viene dada por la frecuencia de la
radiación~\cite{cite_2}. Observó que la emisión de
fotoelectrones era independiente de la intensidad de la luz
y estaba asociada, para un mismo metal, a cierto tipo de
radiaciones~\cite{cite_2}. Además, cuanto mayor fuera la
frecuencia de la radiación, mayor parecía la energía de los
electrones emitidos~\cite{cite_2}.

Einstein postuló que los fotones de una radiación debían
tener un valor mínimo de energía para que, al chocar con
los electrones de la superficie del metal, fuesen capaces
de transferirles la energía necesaria para hacerlos
abandonar el metal~\cite{cite_2}. A este valor de energía
lo llamó energía umbral~\cite{cite_2}.

Estos hallazgos contradecían las predicciones basadas en el
modelo ondulatorio de la luz, que sugerían que el aumento
de la amplitud de la luz incrementaría la energía cinética
de los fotoelectrones emitidos, mientras que el aumento de
la frecuencia incrementaría la corriente
medida~\cite{cite_3}. En cambio, los experimentos mostraron
que el aumento en la frecuencia incrementaba la energía
cinética de los fotoelectrones, mientras que el aumento en
la amplitud de la luz incrementaba la
corriente~\cite{cite_3}.

Por lo tanto, el efecto fotoeléctrico proporciona evidencia
sólida para apoyar el modelo corpuscular de la luz y
sugiere que la luz tiene una naturaleza dual: se comporta
tanto como onda como partícula.

% + --------------------------------------------------------------|>
\subsection*{Explique por qué la energía cinética de los fotoelectrones menos ligados se puede
      calcular con la ecuación (3)}

La ecuación $K_{\max}= e V_0$, se utiliza para calcular la
energía cinética máxima de los fotoelectrones menos ligados
en un experimento de efecto fotoeléctrico. Aquí, $K_{\max}$
es la energía cinética máxima de los fotoelectrones, $e$ es
la carga del electrón y $V_0$ es la tensión de frenado1.

En el efecto fotoeléctrico, cada fotoelectrón es arrancado
por un fotón y abandona el átomo con una energía cinética
que depende de la frecuencia de la luz incidente. Los
electrones menos ligados son aquellos que requieren menos
energía para ser liberados del metal.

La tensión de frenado ($V_0$) es la tensión mínima
necesaria para detener los electrones más energéticos
(menos ligados) y evitar que lleguen al ánodo1. Por lo
tanto, la energía cinética máxima de los fotoelectrones
menos ligados se puede calcular como el producto de la
carga del electrón y la tensión de frenado1.

% + --------------------------------------------------------------|>
\subsection*{Describa el procedimiento a utilizar para calcular un valor aproximado de la constante
      de Planck.}

\begin{enumerate}
      \item Montaje Óptico:
            \begin{itemize}[label=$\triangleright$]
                  \item Conectar la bobina de reactancia a la red eléctrica.
                  \item Montar y encender la lámpara de mercurio.
                  \item Asegurar la celda fotoeléctrica hacia la lámpara de
                        mercurio.
                  \item Alinear y enfocar la luz en la zona sensible de la celda.
            \end{itemize}

      \item Montaje Eléctrico:
            \begin{itemize}[label=$\triangleright$]
                  \item Conectar un electrómetro amplificador y un multímetro.
                  \item Conectar la fuente de alimentación de 12V.
            \end{itemize}

      \item Medición:
            \begin{itemize}[label=$\triangleright$]
                  \item Usar un filtro para seleccionar una longitud de onda.
                  \item Descargar el capacitor y registrar V0.
                  \item Repetir para diferentes longitudes de onda y ajustar la
                        intensidad luminosa.
            \end{itemize}

      \item Análisis de Datos:
            \begin{itemize}[label=$\triangleright$]
                  \item Crear una gráfica de V0 en función de la frecuencia ($v$).
                  \item Realizar un análisis de regresión para obtener la ecuación
                        lineal.
                  \item Calcular la constante de Planck ($h$), la función de
                        trabajo ($W_0$) y la frecuencia de corte.
            \end{itemize}
\end{enumerate}

% ! ----------------------------------------------------------------------|>
\section{Resumen del procedimiento}

En este experimento, haremos uso de una celda fotoeléctrica
y una lámpara de mercurio de alta presión para investigar
el efecto fotoeléctrico. Primero, ajustaremos el montaje
óptico de manera que la luz incida en la celda
fotoeléctrica de manera precisa. Luego, mediremos la
tensión límite ($V_0$) en el capacitor que acumula la
energía de los electrones liberados. Variaremos la longitud
de onda de la luz incidente utilizando filtros de
interferencia y controlaremos la intensidad de la luz con
el diafragma. Al analizar la relación entre V0 y la
frecuencia de la luz, obtendremos información que nos
permitirá calcular la constante de Planck y la función de
trabajo de la celda fotoeléctrica, lo que respaldará la
teoría cuántica propuesta por Einstein.

\printbibliography

\end{document}