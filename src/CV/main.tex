%!TEX TS-program = xelatex
%!TEX encoding = UTF-8 Unicode
% Awesome CV LaTeX Template for CV/Resume
%
% This template has been downloaded from:
% https://github.com/posquit0/Awesome-CV
%
% Author:
% Claud D. Park <posquit0.bj@gmail.com>
% http://www.posquit0.com
%
% Template license:
% CC BY-SA 4.0 (https://creativecommons.org/licenses/by-sa/4.0/)
%

%-------------------------------------------------------------------------------
% CONFIGURATIONS
%-------------------------------------------------------------------------------
% A4 paper size by default, use 'letterpaper' for US letter
\documentclass[11pt, a4paper]{awesome-cv}

% Configure page margins with geometry
\geometry{left=1.4cm, top=.8cm, right=1.4cm, bottom=1.8cm, footskip=.5cm}

% Specify the location of the included fonts
\fontdir[fonts/]

% Color for highlights
% Awesome Colors: awesome-emerald, awesome-skyblue, awesome-red, awesome-pink, awesome-orange
%                 awesome-nephritis, awesome-concrete, awesome-darknight
\colorlet{awesome}{awesome-skyblue}
% Uncomment if you would like to specify your own color
% \definecolor{awesome}{HTML}{850DE6}

% Colors for text
% Uncomment if you would like to specify your own color
% \definecolor{darktext}{HTML}{414141}
% \definecolor{text}{HTML}{333333}
% \definecolor{graytext}{HTML}{5D5D5D}
% \definecolor{lighttext}{HTML}{999999}

% Set false if you don't want to highlight section with awesome color
\setbool{acvSectionColorHighlight}{true}

% If you would like to change the social information separator from a pipe (|) to something else
\renewcommand{\acvHeaderSocialSep}{\quad\textbar\quad}

%-------------------------------------------------------------------------------
% PACKAGES
%-------------------------------------------------------------------------------

% chktex-file 1

\usepackage{hyperref}
\usepackage{hyperxmp}
\usepackage[english]{babel}
\usepackage{iflang}

\hypersetup{
    pdftitle={Mauro Gonzalez $-$ CV},
    pdfauthor={Mauro Gonzalez},
    pdfsubject={Curriculum Vitae},
    pdfkeywords={CV, resume, data science, LaTeX},
    pdfcontactemail={mauroalonso.g.f2004@gmail.com}
}

\begin{filecontents*}{resume.json}
{
    "basics": {
        "name": "Mauro Gonzalez",
        "email": "mauroalonso.g.f2004@gmail.com",
        "phone": "+57 311 4060465"
    }
}
\end{filecontents*}

%-------------------------------------------------------------------------------
%	PERSONAL INFORMATION
%	Comment any of the lines below if they are not required
%-------------------------------------------------------------------------------
% Available options: circle|rectangle,edge/noedge,left/right
\photo[circle,edge,left]{./images/profile.jpg}
\name{Mauro}{Gonzalez}
\position{System's Engineer{\enskip\cdotp\enskip}Fullstack Developer}
\address{Cartagena, Colombia}

\mobile{(+57) 3114060365}
\email{mauroalonso.g.f2004@gmail.com}
% \homepage{www.posquit0.com}
\github{MauroGonzalez51}
\linkedin{mauro-gonzalez-figueroa}
% \gitlab{gitlab-id}
% \stackoverflow{SO-id}{SO-name}
% \twitter{@twit}
% \skype{skype-id}
% \reddit{reddit-id}
% \medium{madium-id}
% \googlescholar{googlescholar-id}{name-to-display}
%% \firstname and \lastname will be used
% \googlescholar{googlescholar-id}{}
\extrainfo{ORCID:~\href{https://orcid.org/0009-0005-9072-2860}{0009-0005-9072-2860}}

% \quote{``Be the change that you want to see in the world."}


%-------------------------------------------------------------------------------
\begin{document}

% Print the header with above personal informations
% Give optional argument to change alignment(C: center, L: left, R: right)
\makecvheader{}

% Print the footer with 3 arguments(<left>, <center>, <right>)
% Leave any of these blank if they are not needed
\makecvfooter
  {\today}
  {Mauro Gonzalez~~~·~~~Curriculum Vitae}
  {\thepage}


%-------------------------------------------------------------------------------
%	CV/RESUME CONTENT
%	Each section is imported separately, open each file in turn to modify content
%-------------------------------------------------------------------------------
\IfLanguageName{spanish}{
  \cvsection{Educación}
  \begin{cventries}
    \cventry
      {Educación Secundaria}
      {Institución Educativa La Concepción - Sede Turbaco}
      {Turbaco, Bolívar, Colombia}
      {2016 - 2021}
      {\begin{cvitems}\item {Finalizó la educación secundaria con énfasis en formación académica.}\end{cvitems}}
    
    \cventry
      {Ingeniería de Sistemas y Computación (En curso)}
      {Universidad Tecnológica de Bolívar}
      {Cartagena, Colombia}
      {2022 - Actualidad}
      {\begin{cvitems}\item {Actualmente matriculado, con 8 semestres completados.}\end{cvitems}}
  \end{cventries}
}{}

\IfLanguageName{english}{
  \cvsection{Education}
  \begin{cventries}
    \cventry
      {Secondary Education}
      {Institución Educativa La Concepción - Sede Turbaco}
      {Turbaco, Bolívar, Colombia}
      {2016 - 2021}
      {\begin{cvitems}\item {Completed secondary education with emphasis on academic formation.}\end{cvitems}}
    
    \cventry
      {B.S. in Systems and Computer Engineering (In Progress)}
      {Universidad Tecnológica de Bolívar}
      {Cartagena, Colombia}
      {2022 - Present}
      {\begin{cvitems}\item {Currently enrolled, 8 semesters completed.}\end{cvitems}}
  \end{cventries}
}{}

% chktex-file 1

\en{
  \cvsection{Skills}
  \begin{cvskills}
    \cvskill
      {DevOps \& Tools}
      {Docker, Git, GitHub, Supabase, Postman, Arch Linux, VSCode, Neovim}

    \cvskill
      {Back-end Development}
      {Node.js, Express, Django, FastAPI, ActixWeb, GraphQL, Prisma, Drizzle, Postgres, SQLite}

    \cvskill
      {Front-end Development}
      {HTML5, CSS, TailwindCSS, JavaScript, TypeScript, React, Vue, NextJS, NuxtJS}

    \cvskill
      {Programming Languages}
      {Python, Java, Rust, C++, LaTeX}

    \cvskill
      {Languages}
      {Spanish (Native), English (Fluent)}
  \end{cvskills}
}

\es{
  \cvsection{Habilidades}
  \begin{cvskills}
    \cvskill
      {DevOps y Herramientas}
      {Docker, Git, GitHub, Supabase, Postman, Arch Linux, VSCode, Neovim}

    \cvskill
      {Desarrollo Back-end}
      {Node.js, Express, Django, FastAPI, ActixWeb, GraphQL, Prisma, Drizzle, Postgres, SQLite}

    \cvskill
      {Desarrollo Front-end}
      {HTML5, CSS, TailwindCSS, JavaScript, TypeScript, React, Vue, NextJS, NuxtJS}

    \cvskill
      {Lenguajes de Programación}
      {Python, Java, Rust, C++, LaTeX}

    \cvskill
      {Idiomas}
      {Español (Nativo), Inglés (Fluido)}
  \end{cvskills}
}


%-------------------------------------------------------------------------------
%	SECTION TITLE
%-------------------------------------------------------------------------------
\cvsection{Experience}


%-------------------------------------------------------------------------------
%	CONTENT
%-------------------------------------------------------------------------------
\begin{cventries}

%---------------------------------------------------------
\cventry
    {Software Developer} % Job title
    {Arsud} % Organization
    {Cartagena, Colombia} % Location
    {2025} % Date(s)
    {
      \begin{cvitems} % Description(s) of tasks/responsibilities
        \item {Collaborated with the development team to design and implement a PyP (Payments and Procedures) module for a telemedicine application.}
        \item {Contributed to the integration of the module into the existing system, ensuring compatibility and reliability.}
        \item {Applied best practices in back-end development and teamwork to deliver a functional solution within project timelines.}
        \item {Gained practical experience in healthcare-related software development, focusing on usability and secure data handling.}
      \end{cvitems}
    }


\end{cventries}

% \input{cv/extracurricular.tex}
%-------------------------------------------------------------------------------
%	SECTION TITLE
%-------------------------------------------------------------------------------
\cvsection{Honors \& Certifications}

%-------------------------------------------------------------------------------
%	CONTENT
%-------------------------------------------------------------------------------
\begin{cvhonors}

%---------------------------------------------------------
\cvhonor
    {Graduate} % Award
    {AWS Academy Graduate - Cloud Foundations} % Event
    {Amazon Web Services (AWS)} % Location
    {Nov. 2025} % Date(s)

%---------------------------------------------------------
\cvhonor
    {Participant} % Award
    {SAS Hackathon} % Event
    {SAS Institute} % Location
    {2024, 2025} % Date(s)

%---------------------------------------------------------
\cvhonor
    {Participant} % Award
    {NASA Space Apps Challenge} % Event
    {Global Hackathon} % Location
    {2024, 2025} % Date(s)

%---------------------------------------------------------
\end{cvhonors}

% \input{cv/presentation.tex}
% chktex-file 1

\en{
  \cvsection{Writing}
  
  \begin{cventries}
    \cventry
        {Co-Author} % Role
        {Geomagnetic disturbances and grid vulnerability: Correlating storm intensity with power system failures} % Title
        {PLOS One} % Location (Journal)
        {Jul. 2025} % Date(s)
        {
          \begin{cvitems} % Description(s)
            \item {Published peer-reviewed journal article analyzing the correlation between geomagnetic storm intensity and power system failures.}
            \item {DOI:~10.1371/journal.pone.0327716}
            \item {Contributors: Mauro González Figueroa; Rafael Duarte Coelho dos Santos; Daniel David Herrera Acevedo; David Sierra Porta}
          \end{cvitems}
        }
  \end{cventries}
}

\es{
  \cvsection{Publicaciones}

  \begin{cventries}
    \cventry
      {Coautor}
      {Perturbaciones geomagnéticas y vulnerabilidad de la red: Correlación entre la intensidad de tormentas y fallas en sistemas eléctricos}
      {PLOS One}
      {Jul. 2025}
      {
        \begin{cvitems}
          \item {Artículo publicado y revisado por pares que analiza la correlación entre la intensidad de tormentas geomagnéticas y fallas en sistemas eléctricos.}
          \item {DOI:~10.1371/journal.pone.0327716}
          \item {Colaboradores: Mauro González Figueroa; Rafael Duarte Coelho dos Santos; Daniel David Herrera Acevedo; David Sierra Porta}
        \end{cvitems}
      }
  \end{cventries}
}
% \input{cv/committees.tex}


%-------------------------------------------------------------------------------
\end{document}
