\documentclass[letterpaper, 12pt]{report}

\usepackage[utf8]{inputenc}
\usepackage[english, spanish]{babel}
\usepackage{fullpage}
\usepackage{graphicx}
\usepackage{amsmath}
\usepackage{enumitem}
\usepackage{chngcntr}
\usepackage{setspace}
\usepackage{url}
\usepackage{csquotes}
\usepackage{float}
\usepackage{verbatim}
\usepackage{tabularx}
\usepackage{amsmath}
\usepackage{caption}
\usepackage{bm}
\usepackage{xurl}
% \usepackage{hyperref}

\counterwithin{figure}{section}
\renewcommand{\thesection}{\arabic{section}}
\renewcommand{\thesubsection}{\thesection.\arabic{subsection}}
\renewcommand{\baselinestretch}{1.5}
\renewcommand{\thefigure}{\arabic{figure}}

\usepackage[style=apa, maxnames=6, minnames=3, backend=biber, parentracker=true, sorting=none]{biblatex}
\DefineBibliographyStrings{english}{%chktex-file 1 chktex-file 6
	andothers = {\em et\addabbrvspace al\adddot}
}
\addbibresource{./Bibliography/bibliography.bib}

\usepackage{array}
\usepackage{enumitem}

\setlength{\parskip}{20pt}

\newcommand{\bolditalic}[1]{\textbf{\textit{#1}}}

% chktex-file 24

\begin{document}

\chapter*{Ética y Responsabilidad Social en la Implementación de TIC}

\noindent\makebox[\linewidth]{\rule{\textwidth}{0.4pt}}

\begin{itemize}[label=$\triangleright$]
	\item Mauro Alonso Gonzalez Figueroa | \textit{T00067622}
\end{itemize}

\noindent\makebox[\linewidth]{\rule{\textwidth}{0.4pt}}

\subsection*{Resumen}

Este ensayo explora los dilemas éticos y las consideraciones de responsabilidad
social asociadas con la implementación y el uso de tecnologías de la
información en las empresas. Se examina la importancia de adoptar prácticas
transparentes y responsables en la gestión de datos personales, así como el
impacto de las TIC en la sostenibilidad social y ambiental. Además, se destaca
la relevancia de la ciberseguridad en la protección de la información
empresarial y cómo esta se integra en la responsabilidad social empresarial
(RSE). Finalmente, se analizan los riesgos cibernéticos comunes y las
estrategias para mitigarlos, subrayando la necesidad de una aproximación ética
y responsable en el uso de TIC\@{}

\bolditalic{Palabras claves}: \textit{Ética, Responsabilidad Social Empresarial (RSE), Tecnologías de la información (TIC), Ciberseguridad, Privacidad de datos, sostenibilidad, Riesgos cibernéticos}

\subsection*{Abstract}

This essay explores the ethical dilemmas and social responsibility
considerations associated with the implementation and use of information
technology in companies. It examines the importance of adopting transparent and
responsible practices in managing personal data, as well as the impact of ICT
on social and environmental sustainability. Additionally, it highlights the
relevance of cybersecurity in protecting business information and how it
integrates into corporate social responsibility (CSR). Finally, it analyzes
common cyber risks and strategies to mitigate them, emphasizing the need for an
ethical and responsible approach to ICT usage.

\bolditalic{Keywords}: \textit{Ethics, Corporate Social Responsibility (CSR), \\ Information Technology (ICT), Cybersecurity, Data Privacy, Sustainability, Cyber Risks}

\noindent\makebox[\linewidth]{\rule{\textwidth}{0.4pt}}

\subsection*{Introducción}

En la era digital, las Tecnologías de la Información y Comunicación (TIC) han
transformado profundamente el panorama empresarial, ofreciendo oportunidades
sin precedentes para la innovación, eficiencia y expansión global. Sin embargo,
esta rápida adopción de las TIC también ha traído consigo una serie de dilemas
éticos y consideraciones de responsabilidad social que las empresas deben
enfrentar. La implementación y uso de tecnologías avanzadas no solo afectan la
operativa interna y la competitividad de las organizaciones, sino que también
tienen un impacto significativo en la privacidad de los datos, la equidad en el
acceso a la información y la sostenibilidad ambiental.

La ética en el uso de TIC se refiere a la adopción de prácticas que respeten la
privacidad y la seguridad de la información personal de los individuos,
evitando el uso indebido de datos y asegurando la transparencia en las
operaciones digitales. Por otro lado, la responsabilidad social empresarial
(RSE) implica que las empresas no solo se enfoquen en la maximización de
beneficios, sino que también tomen en cuenta su impacto en la sociedad y el
medio ambiente, actuando de manera que promuevan el bienestar común y la
sostenibilidad.

Este ensayo explora cómo las empresas pueden integrar la ética y la
responsabilidad social en la implementación de TIC, abordando los desafíos y
oportunidades que surgen. Se analizarán aspectos clave como la gestión
responsable de datos, la importancia de la ciberseguridad, y las estrategias
para mitigar los riesgos cibernéticos. Al adoptar un enfoque ético y
responsable, las empresas no solo cumplen con sus obligaciones legales y
morales, sino que también fortalecen su reputación y contribuyen a un entorno
digital más seguro y equitativo.

\subsection*{Seguridad de los datos}

En la era digital actual, la protección de datos se ha convertido en un aspecto
fundamental para las empresas que dependen de su presencia en la web para
generar ingresos. La confianza del usuario es un factor crítico en este
contexto; los consumidores no solo buscan productos que cumplan con sus
expectativas, sino que también desean asegurar que sus datos personales estén
protegidos.

A menudo, las empresas invierten grandes sumas de dinero en soluciones de
seguridad que, aunque pueden ser costosas, prometen una alta protección contra
posibles vulnerabilidades. Esta inversión en tecnología no solo asegura el
funcionamiento eficaz de los productos, sino que también fomenta la confianza
del cliente. Sin embargo, la cuestión ética aquí radica en el nivel de
responsabilidad que las empresas tienen al implementar y mantener estas
soluciones tecnológicas.

Un caso relevante para ilustrar este tema es el de CrowdStrike, una empresa
estadounidense especializada en ciberseguridad que ofrece protección a nivel
del Kernel del sistema operativo. El Kernel es una parte crítica del sistema
operativo, y cualquier error en su seguridad puede resultar en consecuencias
desastrosas. CrowdStrike se ha ganado una sólida reputación por su capacidad
para ofrecer una seguridad robusta y confiable.

Sin embargo, el caso de CrowdStrike también pone de relieve las complejidades
de la confianza en la ciberseguridad. A pesar de sus altos estándares de
protección, la empresa ha enfrentado incidentes de seguridad que han puesto a
prueba la integridad de sus soluciones. Uno de los incidentes destacados
\textit{(el cual afecto alrededor de $8.5M$ de sistemas alrededor del mundo)}
implicó una vulnerabilidad que, a pesar de ser detectada y mitigada, reveló
cómo una brecha en la seguridad puede afectar la confianza depositada en una
empresa de ciberseguridad.

Este evento subraya una importante dimensión ética: la responsabilidad de las
empresas no termina en la implementación de tecnologías de seguridad. También
tienen un deber continuo de garantizar que sus soluciones sean efectivas y que
sus prácticas de seguridad se mantengan actualizadas para proteger los datos de
los usuarios. Las empresas que utilizan servicios de ciberseguridad, como
aquellas que confían en CrowdStrike, también comparten una responsabilidad
social en la protección de sus propios datos y en la implementación de medidas
adicionales para garantizar la seguridad.

A pesar de una falla en las soluciones de CrowdStrike, las empresas compradoras
deben asumir su parte en la responsabilidad social. Esto incluye la
implementación de prácticas de seguridad adecuadas, la capacitación de su
personal en ciberseguridad y la adopción de medidas preventivas para proteger
los datos de los usuarios. La colaboración entre proveedores de servicios de
seguridad y sus clientes es esencial para mitigar los riesgos y mantener la
confianza en la tecnología.

\subsection*{Riesgos en la implementación de nuevas TIC}

La implementación de nuevas Tecnologías de la Información y Comunicación (TIC)
presenta varios riesgos significativos que pueden afectar la seguridad de la
información y la operativa de las empresas. Estos riesgos incluyen una variedad
de amenazas cibernéticas, así como desafíos asociados con la transición a
nuevas tecnologías.

Entre las amenazas más comunes en la red se encuentran ransomware, phishing,
malware y spyware. El ransomware cifra los datos de una víctima y exige un
rescate para su liberación, mientras que el phishing engaña a los usuarios para
obtener información confidencial mediante correos electrónicos fraudulentos. El
malware puede dañar o robar datos, y el spyware puede espiar y recopilar
información sin el consentimiento del usuario.

La transición hacia nuevas TIC también conlleva desafíos adicionales. Integrar
nuevas tecnologías puede requerir una reestructuración de los sistemas
existentes y una actualización de los procesos operativos. Es crucial gestionar
estos riesgos de manera efectiva mediante la realización de auditorías de
seguridad, la implementación de medidas de protección adecuadas y la
capacitación continua del personal.

Las empresas deben tener un plan de respuesta a incidentes para manejar
cualquier brecha de seguridad y minimizar el impacto de los ataques.
Implementar políticas de seguridad rigurosas y mantenerse al día con las
mejores prácticas de ciberseguridad es esencial para protegerse durante el
proceso de transición, del mismo modo, garantizando la seguridad de sus
usuarios y previniendo al máximo cualquier inconveniente durante el proceso.

\nocite{*}

\printbibliography

\end{document}