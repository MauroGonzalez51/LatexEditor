\appendix

\section{Código Fuente}\label{source_code}
\begin{lstlisting}[language=python, inputencoding=utf8]
# -*- coding: utf-8 -*-
# %%
# type: ignore

import enum
from typing import Any, Dict, List, Optional, Tuple

from z3.z3 import (
    And,
    BoolSort,
    BoolVal,
    CheckSatResult,
    Const,
    DeclareSort,
    ExprRef,
    ForAll,
    Function,
    Implies,
    IntSort,
    ModelRef,
    Not,
    Solver,
    sat,
)

# %%


class ThreatLevel(enum.Enum):
    LOW = 1
    MEDIUM = 2
    HIGH = 3
    CRITICAL = 4


class AssetType(enum.Enum):
    SERVER = 1
    NETWORK_DEVICE = 2
    IOT_DEVICE = 3
    WORKSTATION = 4
    CLOUD_SERVICE = 5


class AttackType(enum.Enum):
    MALWARE = 1
    PHISHING = 2
    DDoS = 3
    DATA_BREACH = 4
    RANSOMWARE = 5
    INSIDER_THREAT = 6


# %%

ThreatSort = DeclareSort("Threat")
AssetSort = DeclareSort("Asset")
AttackSort = DeclareSort("Attack")
UserSort = DeclareSort("User")
IncidentSort = DeclareSort("Incident")

# %%
# functions and predicates for the ontology

affects = Function("affects", AttackSort, AssetSort, BoolSort())
has_vulnerability = Function("has_vulnerability", AssetSort, BoolSort())
threat_level = Function("threat_level", AttackSort, IntSort())
asset_criticality = Function("asset_criticality", AssetSort, IntSort())
requires_immediate_response = Function(
    "requires_immediate_response", IncidentSort, BoolSort()
)
user_has_access = Function("user_has_access", UserSort, AssetSort, BoolSort())
is_compromised = Function("is_compromised", AssetSort, BoolSort())

# %%
# PEAS metrics

response_time = Function("response_time", IncidentSort, IntSort())
containment_effectiveness = Function(
    "containment_effectiveness", IncidentSort, IntSort()
)
false_positive = Function("false_positive", IncidentSort, BoolSort())

# %%
solver = Solver()

# %%
threat = Const("threat", AttackSort)
asset = Const("asset", AssetSort)
user = Const("user", UserSort)
incident = Const("incident", IncidentSort)

# %%
# type: ignore

solver.add(
    ForAll(
        [threat, asset, incident],
        Implies(
            And(
                has_vulnerability(asset),
                affects(threat, asset),
                threat_level(threat) >= 3,
            ),
            requires_immediate_response(incident),
        ),
    )
)

solver.add(
    ForAll(
        [asset, incident],
        Implies(
            And(is_compromised(asset), asset_criticality(asset) >= 4),
            requires_immediate_response(incident),
        ),
    )
)

solver.add(
    ForAll(
        [user, asset],
        Implies(
            And(Not(user_has_access(user, asset)), is_compromised(asset)),
            BoolVal(True),
        ),
    )
)

solver.add(
    ForAll(
        [incident],
        Implies(
            requires_immediate_response(incident), response_time(incident) <= 15
        ),
    )
)

# %%
# type: ignore


class CybersecuritySensors:
    def __init__(self, solver: Solver) -> None:
        self.solver: Solver = solver

    def network_anomaly_sensor(
        self, traffic_data: Dict[str, Any]
    ) -> Optional[ExprRef]:
        anomaly_detected: bool = traffic_data.get("unusual_patterns", False)

        if anomaly_detected:
            new_threat: ExprRef = Const(
                f"network_threat_{id(traffic_data)}", AttackSort
            )
            self.solver.add(threat_level(new_threat) >= 2)
            return new_threat

        return None

    def file_integrity_sensor(
        self, file_changes: Dict[str, Any]
    ) -> Optional[ExprRef]:
        """Monitor de integridad de archivos"""
        if file_changes.get("unauthorized_changes", False):
            affected_asset: ExprRef = Const(
                f"asset_{file_changes['asset_id']}", AssetSort
            )
            self.solver.add(is_compromised(affected_asset))
            return affected_asset

        return None

    def user_behavior_sensor(
        self, user_activity: Dict[str, Any]
    ) -> Optional[Tuple[ExprRef, ExprRef]]:
        """Sensor UEBA (User and Entity Behavior Analytics)"""
        if user_activity.get("anomalous_behavior", False):
            suspicious_user: ExprRef = Const(
                f"user_{user_activity['user_id']}", UserSort
            )
            target_asset: ExprRef = Const(
                f"asset_{user_activity['target_asset']}", AssetSort
            )
            self.solver.add(Not(user_has_access(suspicious_user, target_asset)))
            return (suspicious_user, target_asset)

        return None


# %%
# type: ignore


class CyberSecurityActuators:
    def __init__(self, solver: Solver) -> None:
        self.solver: Solver = solver

    def block_malicious_traffic(self, threat_source: str) -> bool:
        return True

    def isolate_compromised_device(self, asset_id: str) -> bool:
        """Aislamiento de dispositivos comprometidos"""
        print(f"Aislando dispositivo comprometido: {asset_id}")
        isolated_asset: ExprRef = Const(f"asset_{asset_id}", AssetSort)
        self.solver.add(Not(is_compromised(isolated_asset)))
        return True

    def generate_incident_report(
        self, incident_data: Dict[str, Any]
    ) -> Dict[str, Any]:
        """Generar reporte detallado del incidente"""
        analysis: Dict[str, Any] = incident_data.get("analysis", {})

        report: Dict[str, Any] = {
            "timestamp": incident_data.get("timestamp"),
            "severity": incident_data.get("severity"),
            "affected_assets": incident_data.get("assets", []),
            "recommended_actions": analysis.get("recommendations", []),
            "immediate_response_triggered": analysis.get(
                "immediate_response", False
            ),
            "actions_taken": incident_data.get("actions_taken", []),
            "z3_reasoning_used": analysis.get("z3_model_used", False),
        }

        return report

    def _get_recommendations(self, incident_data: Dict[str, Any]) -> List[str]:
        return ["Aplicar parches", "Monitorear actividad", "Revisar logs"]


# %%
# type: ignore


class CyberSecurityReasoningEngine:
    def __init__(
        self,
        solver: Solver,
        sensors: CybersecuritySensors,
        actuators: CyberSecurityActuators,
    ) -> None:
        self.solver: Solver = solver
        self.sensors: CybersecuritySensors = sensors
        self.actuators: CyberSecurityActuators = actuators

    def analyze_threat(self, threat_data: Dict[str, Any]) -> Dict[str, Any]:
        local_solver: Solver = Solver()

        for assertion in self.solver.assertions():
            local_solver.add(assertion)

        current_threat: ExprRef = Const(f"threat_{id(threat_data)}", AttackSort)
        current_asset: ExprRef = Const(f"asset_{id(threat_data)}", AssetSort)
        current_incident: ExprRef = Const(
            f"incident_{id(threat_data)}", IncidentSort
        )

        # Agregar hechos basados en los datos del incidente
        self._add_threat_facts(
            local_solver,
            current_threat,
            current_asset,
            current_incident,
            threat_data,
        )

        # Verificar si el modelo es satisfacible        check_result: CheckSatResult = local_solver.check()
        if check_result == sat:
            model: ModelRef = local_solver.model()

            # Evaluar usando el modelo Z3
            threat_severity: ThreatLevel = self._evaluate_threat_severity_z3(
                current_threat, model, threat_data
            )
            response_needed: bool = self._requires_immediate_response_z3(
                current_incident, model, threat_data
            )
            recommendations: List[str] = self._generate_recommendations_z3(
                model, threat_data, current_threat, current_asset
            )

            return {
                "severity": threat_severity,
                "immediate_response": response_needed,
                "recommendations": recommendations,
                "z3_model_used": True,
            }
        else:
            return {"error": "Inconsistencia en la ontologia detectada"}

    def _add_threat_facts(
        self,
        local_solver: Solver,
        threat: ExprRef,
        asset: ExprRef,
        incident: ExprRef,
        threat_data: Dict[str, Any],
    ) -> None:
        if threat_data.get("affects_critical_assets", False):
            local_solver.add(threat_level(threat) == 4)  # CRITICAL
        elif threat_data.get("widespread_impact", False):
            local_solver.add(threat_level(threat) == 3)  # HIGH
        elif threat_data.get("network_anomaly", False):
            local_solver.add(threat_level(threat) == 2)  # MEDIUM
        else:
            local_solver.add(threat_level(threat) == 1)  # LOW

        if threat_data.get("affects_critical_assets", False):
            local_solver.add(asset_criticality(asset) == 5)  # Maxima criticidad
        else:
            local_solver.add(asset_criticality(asset) == 2)  # Criticidad normal

        # Establecer relaciones
        local_solver.add(affects(threat, asset))

        if threat_data.get("compromised_assets", []):
            local_solver.add(is_compromised(asset))

        if threat_data.get("vulnerability_detected", False):
            local_solver.add(has_vulnerability(asset))

    def _evaluate_threat_severity_z3(
        self, threat: ExprRef, model: ModelRef, threat_data: Dict[str, Any]
    ) -> ThreatLevel:
        """Evaluar severidad usando el modelo Z3"""
        try:
            # Intentar evaluar el nivel de amenaza desde el modelo
            threat_level_val: Optional[ExprRef] = model.evaluate(
                threat_level(threat), model_completion=True
            )

            if threat_level_val is not None:
                level: int = threat_level_val.as_long()
                if level >= 4:
                    return ThreatLevel.CRITICAL
                elif level >= 3:
                    return ThreatLevel.HIGH
                elif level >= 2:
                    return ThreatLevel.MEDIUM
                else:
                    return ThreatLevel.LOW
        except Exception as e:
            print(f"Error evaluando nivel de amenaza: {e}")

        if threat_data.get("affects_critical_assets", False):
            return ThreatLevel.CRITICAL
        elif threat_data.get("widespread_impact", False):
            return ThreatLevel.HIGH
        else:
            return ThreatLevel.MEDIUM

    def _requires_immediate_response_z3(
        self, incident: ExprRef, model: ModelRef, threat_data: Dict[str, Any]
    ) -> bool:
        """Determinar respuesta inmediata usando Z3"""
        try:
            immediate_response: Optional[ExprRef] = model.evaluate(
                requires_immediate_response(incident), model_completion=True
            )

            if immediate_response is not None:
                return bool(immediate_response)
        except Exception as e:
            print(f"Error evaluando respuesta inmediata: {e}")

        # Fallback
        return threat_data.get(
            "affects_critical_assets", False
        ) or threat_data.get("active_exploitation", False)

    def _generate_recommendations_z3(
        self,
        model: ModelRef,
        threat_data: Dict[str, Any],
        threat: ExprRef,
        asset: ExprRef,
    ) -> List[str]:
        recommendations: List[str] = []

        try:
            # Evaluar diferentes aspectos del modelo
            is_asset_compromised: Optional[ExprRef] = model.evaluate(
                is_compromised(asset), model_completion=True
            )
            threat_level_val: Optional[ExprRef] = model.evaluate(
                threat_level(threat), model_completion=True
            )
            asset_critical: Optional[ExprRef] = model.evaluate(
                asset_criticality(asset), model_completion=True
            )

            # Recomendaciones basadas en el estado del activo
            if is_asset_compromised and bool(is_asset_compromised):
                recommendations.append(
                    "CRITICO: Aislar activo comprometido inmediatamente"
                )
                recommendations.append(
                    "Realizar analisis forense del activo afectado"
                )

            # Recomendaciones basadas en nivel de amenaza
            if threat_level_val and threat_level_val.as_long() >= 3:
                recommendations.append(
                    "Activar protocolo de respuesta de emergencia"
                )
                recommendations.append("Notificar al equipo directivo")

            # Recomendaciones basadas en criticidad del activo
            if asset_critical and asset_critical.as_long() >= 4:
                recommendations.append("Implementar monitoreo continuo 24/7")
                recommendations.append(
                    "Realizar backup inmediato de datos criticos"
                )

        except Exception as e:
            print(f"Error evaluando modelo Z3: {e}")

        # Recomendaciones adicionales basadas en tipo de amenaza
        if threat_data.get("network_anomaly", False):
            recommendations.append("Analizar trafico de red en tiempo real")
            recommendations.append(
                "Implementar reglas de firewall restrictivas"
            )

        if threat_data.get("user_anomaly", False):
            recommendations.append(
                "Revisar credenciales y permisos de usuario"
            )
            recommendations.append("Forzar cambio de contrase\u00f1as")

        if threat_data.get("malware_detected", False):
            recommendations.append(
                "Ejecutar escaneo completo de antimalware"
            )
            recommendations.append(
                "Limpiar y desinfectar sistemas afectados"
            )

        if not recommendations:
            recommendations = [
                "Documentar incidente en el sistema SIEM",
                "Incrementar nivel de monitoreo",
                "Revisar logs de seguridad",
            ]

        return recommendations

    def respond_to_incident(
        self, incident_data: Dict[str, Any]
    ) -> Dict[str, Any]:
        print(
            f"Analizando incidente: {incident_data.get('timestamp', 'N/A')}"
        )

        analysis: Dict[str, Any] = self.analyze_threat(incident_data)

        if "error" in analysis:
            return analysis

        severity: ThreatLevel = analysis["severity"]
        print(f"[SEVERIDAD] Severidad detectada: {severity.name}")
        print(
            f"[ALERTA] Respuesta inmediata requerida: {analysis['immediate_response']}"
        )

        # Ejecutar actuadores si es necesario
        actions_taken: List[str] = []
        if analysis.get("immediate_response", False):
            if incident_data.get("network_threat"):
                source: str = incident_data.get("source", "unknown")
                if self.actuators.block_malicious_traffic(source):
                    actions_taken.append(f"Bloqueado trafico desde {source}")

            compromised_assets: List[str] = incident_data.get(
                "compromised_assets", []
            )
            if compromised_assets:
                for asset in compromised_assets:
                    if self.actuators.isolate_compromised_device(asset):
                        actions_taken.append(f"Aislado dispositivo {asset}")

        # Generar reporte detallado
        report: Dict[str, Any] = self.actuators.generate_incident_report(
            {
                "timestamp": incident_data.get("timestamp"),
                "severity": analysis.get("severity"),
                "assets": incident_data.get("compromised_assets", []),
                "analysis": analysis,
                "actions_taken": actions_taken,
            }
        )

        report["actions_taken"] = actions_taken
        report["z3_reasoning"] = analysis.get("z3_model_used", False)

        return report


# %%
sensors = CybersecuritySensors(solver=solver)
actuators = CyberSecurityActuators(solver=solver)
reasoning_engine = CyberSecurityReasoningEngine(
    solver=solver, sensors=sensors, actuators=actuators
)
\end{lstlisting}