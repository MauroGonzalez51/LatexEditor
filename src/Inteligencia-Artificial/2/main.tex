\documentclass[letterpaper, 12pt]{article}

% ===== Idioma y codificación =====
\usepackage[utf8]{inputenc}
\usepackage[spanish]{babel}

% ===== Diseño y formato general =====
\usepackage{fullpage}
\usepackage{setspace}
\usepackage{lmodern}
\usepackage{microtype}
\usepackage{graphicx}
\usepackage{caption}
\usepackage{float}
\usepackage{wrapfig}
\usepackage{multicol}
\usepackage{tabularx}
\usepackage{array}
\usepackage{colortbl}
\usepackage{xcolor}
\usepackage{multirow}
\usepackage{verbatim}
\usepackage{chngcntr}
\usepackage{tocloft}
\usepackage{bm}
\usepackage{enumitem}

% ===== Matemáticas =====
\usepackage{amsmath}

% ===== Hipervínculos =====
\usepackage{hyperref}
\hypersetup{
    colorlinks=true,
    linkcolor=blue!70!black,
    urlcolor=cyan!70!black,
    citecolor=magenta!70!black
}

% ===== Código fuente =====
\usepackage{listings}
\definecolor{codegreen}{rgb}{0,0.6,0}
\definecolor{codegray}{rgb}{0.5,0.5,0.5}
\definecolor{codepurple}{rgb}{0.58,0,0.82}
\definecolor{backcolour}{rgb}{0.95,0.95,0.92}
\lstdefinestyle{mystyle}{
    backgroundcolor=\color{backcolour},   
    commentstyle=\color{codegreen},
    keywordstyle=\color{magenta},
    numberstyle=\tiny\color{codegray},
    stringstyle=\color{codepurple},
    basicstyle=\ttfamily\footnotesize,
    breakatwhitespace=false,         
    breaklines=true,                 
    captionpos=b,                    
    keepspaces=true,                 
    numbers=left,                    
    numbersep=5pt,                  
    showspaces=false,                
    showstringspaces=false,
    showtabs=false,                  
    tabsize=2
}
\lstset{style=mystyle, frame=single, framerule=0.5pt, rulecolor=\color{gray!60}}

% ===== Citas y bibliografía =====
\usepackage{csquotes}
\usepackage[style=apa, maxnames=2, minnames=1, backend=biber, parentracker=true, sorting=none]{biblatex}
\DefineBibliographyStrings{english}{
      andothers = {\em et\addabbrvspace{} al\adddot{}\/}
}
\addbibresource{./Bibliography/bibliography.bib}

% ===== Personalización de secciones y TOC =====
\renewcommand{\thesection}{\arabic{section}}
\renewcommand{\thesubsection}{\thesection.\arabic{subsection}}
\renewcommand{\baselinestretch}{1.5}
\renewcommand{\cftsecfont}{\bfseries}
\renewcommand{\cftsecpagefont}{\bfseries}

% ===== Otros ajustes =====
\setlength{\parskip}{0pt}
\raggedbottom{}
\counterwithin{figure}{section}
\newcommand{\bolditalic}[1]{\textbf{\textit{#1}}}

% ===== chktex =====
% chktex-file 44
% chktex-file 24
% chktex-file 8

\begin{document}

\begin{titlepage}
    \centering
    \includegraphics[width=0.3\textwidth]{../../../images/logo_utb.png}
    \par\vspace{.5cm}

    {\scshape\LARGE Universidad Tecnológica de Bolívar \par}
    \vspace{1cm}

    {\scshape\Large Inteligencia Artificial \par}
    \vspace{1.5cm}

    \slshape {\Large \bfseries{}Agente Inteligente de Ciberseguridad\\}
    \vspace{2.5cm}

    \slshape {\itshape{} Mauro Alonso González Figueroa, T00067622 \\}
    \slshape {\itshape{} Juan Jose Jiménez Guardo, T00068278 \\}

    \vfill
    Revisado Por \\
    Edwin Alexander Puertas Del Castillo\\
    {\large \today\par}
\end{titlepage}

\tableofcontents{}
\newpage

% --------------------------------------------------------------------------------------
\section{Introduccion}

\subsection{Caso de Estudio}
La defensa cibernética es un conjunto de estrategias, tecnologías y procesos
diseñados para proteger los sistemas informáticos, redes y datos contra
amenazas cibernéticas, como ataques de malware, piratería informática, robo de
datos y otros riesgos de seguridad. Estas medidas de defensa buscan prevenir,
detectar, responder y recuperarse de ataques cibernéticos con el fin de
garantizar la confidencialidad, integridad y disponibilidad de la información y
los recursos digitales.

\subsection{Resumen Ejecutivo}
Este documento aborda el diseño e implementación de una ontología orientada a 
la defensa cibernética y su posterior aplicación en un agente inteligente. 
Primero se revisan los fundamentos teóricos de las ontologías: qué son, cómo se 
construyen actualmente y cuáles son las metodologías más utilizadas en su 
desarrollo. Luego, se analiza cómo estas ontologías pueden integrarse en 
sistemas inteligentes para mejorar la capacidad de detección, respuesta y 
resiliencia frente a amenazas cibernéticas. 

El propósito central es proponer una ontología que sirva de base para dotar a 
un agente inteligente de un marco de conocimiento estructurado, que le permita 
tomar decisiones informadas en escenarios de ciberseguridad.
\section{Modulo \textit{PEAS}}

El diseño del agente inteligente se describe formalmente mediante el modelo
\textbf{PEAS} (Performance, Environment, Actuators, Sensors), el cual permite
definir de manera estructurada los elementos fundamentales de su operación.

\subsection{Performance (Medición del Desempeño)}
Las métricas utilizadas para evaluar el desempeño del agente incluyen:
\begin{itemize}
    \item Tiempo Medio de Respuesta (MTTR).
    \item Tiempo Medio Entre Fallos (MTBF).
    \item Efectividad de contención ante incidentes.
    \item Porcentaje de ataques prevenidos.
    \item Número de falsos positivos.
    \item Estimación del impacto económico evitado.
\end{itemize}

\subsection{Environment (Entorno)}
El agente opera en:
\begin{itemize}
    \item Infraestructura de red corporativa.
    \item Servidores críticos y servicios en la nube.
    \item Dispositivos IoT y BYOD conectados a la red.
    \item Factores humanos y organizacionales, incluyendo políticas de seguridad, niveles
          de autorización y comportamiento del usuario.
\end{itemize}

\subsection{Actuators (Actuadores)}
Los actuadores del sistema incluyen:
\begin{itemize}
    \item Bloqueo automático de tráfico malicioso.
    \item Aislamiento de dispositivos comprometidos.
    \item Módulo SOAR para respuesta orquestada (bloqueo, escalamiento, notificación).
    \item Generación de reportes ejecutivos y dashboards en tiempo real.
    \item Aplicación de parches y restauración de servicios.
\end{itemize}

\subsection{Sensors (Sensores)}
El agente integra sensores especializados para la detección de amenazas:
\begin{itemize}
    \item Sensores de red para tráfico y anomalías.
    \item Monitores de integridad en sistemas y archivos.
    \item Antivirus y detección basada en firmas.
    \item Sensores basados en inteligencia de amenazas externa.
    \item Sensores de comportamiento de usuario (UEBA).
\end{itemize}
\section{Ontología}

\subsection{Definicion}

En IA y Web Semántica, una ontología es un modelo que define, con precisión y de forma formal, qué conceptos existen en un dominio, cómo se relacionan y qué reglas los gobiernan.  

\begin{quote}
    “An ontology is an explicit specification of a conceptualization”.\\
    \textit{“Una ontología es una especificación explícita de una conceptualización” }\\
    \hfill\cite{Gruber2009Ontology}
\end{quote}

\subsection{¿De qué está hecha una ontología?}

En la práctica, una ontología proporciona un vocabulario controlado junto con axiomas que otorgan significado computable a dicho vocabulario. Sus componentes típicos son:

\begin{itemize}
    \item \textbf{Clases o conceptos:} Representan categorías o conjuntos de objetos dentro del dominio.
    \item \textbf{Propiedades o relaciones:} Describen cómo se vinculan los conceptos entre sí.
    \item \textbf{Individuos o instancias:} Son los elementos concretos que pertenecen a las clases.
    \item \textbf{Restricciones (axiomas):} Reglas que limitan o definen el comportamiento y las relaciones entre los elementos anteriores.
\end{itemize}

\subsection{Lenguajes y estándares que la hacen operativa}

Las ontologías se formalizan principalmente con OWL (Web Ontology Language) sobre el modelo de grafo de RDF.~OWL aporta semántica formal para que los sistemas puedan inferir y verificar conocimiento, no solo almacenarlo; RDF define el modelo de tripletas (sujeto--predicado--objeto) que hace interoperable el intercambio de datos~\parencite{W3C2012OWL2Overview}.

Para validar calidad de datos frente a una ontología o esquema RDF, la recomendación \textit{W3C SHACL} permite expresar “shapes” (condiciones) y chequear conformidad de grafos; existen borradores recientes que amplían su alcance~\parencite{W3C2017SHACL}.

\subsection{Usos clave}

\begin{enumerate}
    \item Hablar el mismo lenguaje (interoperatividad semántica).
    Sirven para alinear significados entre equipos y sistemas distintos. Cuando varias aplicaciones comparten la misma ontología, se reduce la ambigüedad y se facilita el intercambio de información con sentido. Esta es, de hecho, una de las motivaciones fundacionales para “desarrollar una ontología”. 

    \item Integrar datos heterogéneos.
    RDF/OWL permiten fusionar fuentes dispares (bases relacionales, logs, APIs, documentos) aun si los esquemas difieren, porque los recursos y relaciones se identifican con URIs y se normalizan como grafos de tripletas~\parencite{W3C2014RDF}.

    \item Razonar e inferir conocimiento implícito.
    Al tener semántica formal, se pueden derivar hechos que no estaban explícitos (clasificación automática, detección de inconsistencias, implicaciones lógicas). OWL 2 fue diseñado justamente para habilitar interpretabilidad y razonamiento sobre contenido~\parencite{BaaderDLHB02}.

    \item Búsqueda y recuperación semántica.
    Con una ontología, las consultas van más allá de palabras clave: explotan jerarquías y relaciones (por ejemplo, recuperar todos los incidentes que afectan activos críticos con vulnerabilidades). Este uso está expuesto en guías prácticas de desarrollo ontológico. 

    \item Validación y control de calidad de los datos.
    Mediante SHACL, las organizaciones definen reglas de calidad (cardinalidades, rangos, dependencias) y pueden auditar automáticamente si sus grafos cumplen políticas y estándares. Esto impacta directamente en la gobernanza de datos~\parencite{Knublauch2017SHACL}.

    \item Reuso y extensibilidad del conocimiento.
    Las ontologías promueven reutilizar modelos consolidados (tiempo, unidades, ubicaciones) y extenderlos para nuevos proyectos, acelerando desarrollos y mejorando la consistencia entre equipos. 
\end{enumerate}

\subsection{Ontología en la Inteligencia Artificial}

En el contexto de la Inteligencia Artificial (IA), las ontologías se definen
como modelos formales que representan de manera estructurada el conocimiento de
un dominio específico. Estas permiten a los sistemas inteligentes interpretar,
razonar y actuar sobre datos de forma más precisa y
contextualizada.

Las ontologías facilitan la organización semántica de la información,
permitiendo que los algoritmos de IA realicen inferencias lógicas, identifiquen
patrones relevantes y generen respuestas adaptadas a las necesidades del
usuario. Al establecer conceptos, relaciones y reglas dentro de un marco
compartido, las ontologías mejoran la interoperabilidad entre sistemas,
optimizan la recuperación de información y permiten simular procesos cognitivos
humanos.

En aplicaciones prácticas, como buscadores semánticos, asistentes virtuales o
sistemas expertos, las ontologías permiten que la IA comprenda el significado
detrás de los datos, lo que se traduce en una interacción más eficiente,
personalizada y autónoma con el entorno
digital~\parencite{Martin2020OntologiasIA}.

\section{Construcción de la Ontología}

\subsection{Metodología de Desarrollo}

Para la construcción de la ontología de ciberseguridad, se adoptó un enfoque híbrido que combina la formalización lógica tradicional con herramientas modernas de razonamiento automatizado. En lugar de utilizar los lenguajes estándar como OWL (Web Ontology Language) o RDF (Resource Description Framework), se optó por implementar la ontología utilizando Z3 SMT Solver (Satisfiability Module Theories), lo que permite realizar razonamiento lógico de primer orden con verificación formal de consistencia.

Esta decisión metodológica se fundamenta en las siguientes ventajas:

\begin{itemize}
    \item Expresividad lógica: \textit{Z3} permite definir reglas complejas usando cuantificadores universales y existenciales
    
    \item Verificación automática: Capacidad de verificar satisfacibilidad y consistencia del modelo en tiempo real
    
    \item Integración nativa: Implementación directa en Python sin necesidad de parsers externos 
    
    \item Rendimiento: Optimizaciones específicas para problemas de satisfacibilidad booleana 
\end{itemize}

\subsection{Arquitectura de la Ontología}

La ontología se estructura siguiendo el modelo PEAS (Performance, Environment, Actuators, Sensors) para agentes inteligentes, proporcionando un framework completo para la respuesta automatizada a incidentes de ciberseguridad.

\subsubsection{Definición de Tipos Básicos \textit{(Sorts)}}

La ontología define cinco tipos fundamentales utilizando la función \verb|DeclareSort| de \textit{Z3}:

\begin{lstlisting}[language=python, caption={Definición de Tipos Básicos en Z3}, label={lst:sorts}]
ThreatSort = DeclareSort("Threat")      # Amenazas de seguridad
AssetSort = DeclareSort("Asset")        # Activos organizacionales
AttackSort = DeclareSort("Attack")      # Tipos de ataques especificos
UserSort = DeclareSort("User")          # Usuarios del sistema
IncidentSort = DeclareSort("Incident")  # Incidentes de seguridad
\end{lstlisting}

Estos sorts actúan como los tipos fundamentales sobre los cuales se construyen todas las relaciones y predicados de la ontología.

\subsubsection{Predicados y Funciones Ontológicas}

La ontología define un conjunto de predicados y funciones que capturan las relaciones semánticas del dominio de ciberseguridad:

\noindent Predicados de Relación:

\begin{itemize}
    \item \verb|affects(Attack, Asset, Bool)|: Define qué ataques afectan a qué activos
    \item \verb|has_vulnerability(Asset, Bool)|: Indica si un activo tiene vulnerabilidades conocidas
    \item \verb|is_compromised(Asset, Bool)|: Especifica el estado de compromiso de un activo
    \item \verb|user_has_access(User, Asset, Bool)|: Define permisos de acceso legítimos
\end{itemize}

\noindent Funciones de Valoración:

\begin{itemize}
    \item \verb|threat_level(Attack, Int)|: Asigna un nivel numérico de amenaza (1--4)
    \item \verb|asset_criticality(Asset, Int)|: Define la criticidad del activo (1--5)
    \item \verb|requires_immediate_response(Incident, Bool)|: Determina urgencia de respuesta
\end{itemize}

\subsubsection{Métricas PEAS}

Para operacionalizar el modelo PEAS, se definen métricas específicas que permiten evaluar el rendimiento del agente:

\begin{lstlisting}[language=python, caption={Definición de Métricas PEAS en Z3}, label={lst:peas_metrics}]
response_time = Function("response_time", IncidentSort, IntSort())
containment_effectiveness = Function("containment_effectiveness", IncidentSort, IntSort())
false_positive = Function("false_positive", IncidentSort, BoolSort())
\end{lstlisting}

\subsection{Reglas de Inferencia}

El núcleo de la ontología consiste en un conjunto de reglas lógicas que definen el comportamiento del sistema de respuesta automática. Estas reglas utilizan cuantificadores universales \verb|(ForAll)| e implicaciones lógicas \verb|(Implies)| para capturar el conocimiento experto del dominio.

\subsubsection{Regla de Respuesta Crítica}

\begin{lstlisting}[language=python, caption={Regla de Respuesta Crítica en Z3}, label={lst:critical_response}]
ForAll([threat, asset, incident],
    Implies(
        And(has_vulnerability(asset),
            affects(threat, asset),
            threat_level(threat) >= 3),
        requires_immediate_response(incident)
    )
)
\end{lstlisting}

Esta regla establece que cualquier amenaza de nivel alto ($\geq$3) que afecte a un activo vulnerable requiere respuesta inmediata.

\subsubsection{Regla de Protección de Activos Críticos}

\begin{lstlisting}[language=python, caption={Regla de Protección de Activos Críticos en Z3}, label={lst:protect_critical_assets}]
ForAll([asset, incident],
    Implies(
        And(is_compromised(asset), 
            asset_criticality(asset) >= 4),
        requires_immediate_response(incident)
    )
)
\end{lstlisting}

Especifica que el compromiso de activos de alta criticidad ($\geq$4) siempre requiere respuesta inmediata, independientemente del tipo de amenaza.

\subsubsection{Regla de Tiempo de Respuesta}

\begin{lstlisting}[language=python, caption={Regla de Tiempo de Respuesta en Z3}, label={lst:response_time}]
ForAll([incident],
    Implies(
        requires_immediate_response(incident),
        response_time(incident) <= 15
    )
)
\end{lstlisting}

Establece el constraint temporal de que los incidentes críticos deben ser atendidos en un máximo de 15 minutos.

\subsection{Implementación del Modelo PEAS}

\subsubsection{Sensores \textit{(Sensors)}}

Se implementaron tres tipos de sensores especializados que alimentan datos a la ontología:

\noindent Sensor de Anomalías de Red \verb|(CybersecuritySensors.network_anomaly_sensor)|:

\begin{itemize}
    \item Detecta patrones de tráfico inusuales
    \item Genera automáticamente entidades de amenaza en Z3
    \item Asigna niveles de amenaza basados en la severidad detectada
\end{itemize}

\noindent Sensor de Integridad de Archivos \verb|(CybersecuritySensors.file_integrity_sensor)|:

\begin{itemize}
    \item Monitorea cambios no autorizados en archivos críticos
    \item Marca activos como comprometidos en la base de conocimiento
    \item Vincula cambios de archivos con activos específicos
\end{itemize}

\noindent Sensor de Comportamiento de Usuario \verb|(CybersecuritySensors.user_behavior_sensor)|:

\begin{itemize}
    \item Implementa análisis UEBA (User and Entity Behavior Analytics)
    \item Detecta accesos anómalos mediante análisis de patrones
    \item Actualiza predicados de acceso en tiempo real
\end{itemize}

\subsubsection{Actuadores \textit{(Actuators)}}

Los actuadores implementan las respuestas automáticas del sistema:

\noindent Bloqueo de Tráfico Malicioso \verb|(CyberSecurityActuators.block_malicious_traffic)|:

\begin{itemize}
    \item Implementa medidas de contención de red
    \item Registra acciones tomadas para auditoría
    \item Actualiza el estado del sistema en la ontología
\end{itemize}

\noindent Aislamiento de Dispositivos \verb|(CyberSecurityActuators.isolate_compromised_device)|:

\begin{itemize}
    \item Ejecuta protocolos de cuarentena automatizada
    \item Modifica predicados de compromiso en Z3
    \item Mantiene trazabilidad de dispositivos aislados 
\end{itemize}

\noindent Generación de Reportes \verb|(CyberSecurityActuators.generate_incident_report)|:

\begin{itemize}
    \item Produce documentación estructurada de incidentes
    \item Integra análisis de Z3 con métricas operacionales
    \item Facilita el cumplimiento regulatorio y auditorías 
\end{itemize}

\subsubsection{Motor de Razonamiento \textit{(Reasoning Engine)}}

El componente central \verb|(CyberSecurityReasoningEngine)| integra todos los elementos de la ontología:

\noindent Análisis de Amenazas:

\begin{itemize}
    \item Crea instancias locales del solver Z3 para cada incidente
    \item Aplica hechos específicos basados en datos de sensores
    \item Evalúa reglas de inferencia para determinar respuestas 
\end{itemize}


\noindent Evaluación de Severidad:

\begin{itemize}
    \item Utiliza el modelo Z3 para calcular niveles de amenaza
    \item Implementa lógica de fallback para casos edge
    \item Mapea valores numéricos a categorías semánticas 
\end{itemize}

\subsubsection{Ventajas del Enfoque Adoptado}

La construcción de la ontología mediante Z3 SMT Solver proporciona ventajas significativas sobre enfoques tradicionales:

\begin{itemize}
    \item Verificación Formal: Garantías matemáticas de consistencia
    \item Razonamiento Eficiente: Optimizaciones específicas para satisfacibilidad
    \item Integración Directa: Sin overhead de translation entre lenguajes
    \item Expresividad: Capacidad de manejar lógica de primer orden compleja
    \item Extensibilidad: Fácil adición de nuevas reglas y predicados
\end{itemize}


% --------------------------------------------------------------------------------------

\section*{Conclusiones}
\addcontentsline{toc}{section}{Conclusiones}
El desarrollo e implementación de un agente inteligente de ciberseguridad permitió comprender la importancia de modelar formalmente el conocimiento mediante ontologías, así como la utilidad de herramientas de razonamiento lógico como Z3 para la verificación y análisis automático de relaciones y restricciones en el dominio.
A lo largo del trabajo, se evidenció que la representación estructurada y formal de conceptos facilita la detección de amenazas, la toma de decisiones automatizada y la adaptabilidad del sistema ante nuevos escenarios.

Si bien la ontología no se encuentra en un formato estándar como OWL o TTL, la aproximación lógica utilizada demostró ser efectiva para el razonamiento y la validación de reglas complejas. Este enfoque resalta la flexibilidad de la inteligencia artificial para abordar problemas de seguridad desde distintas perspectivas técnicas.

Como posibles mejoras futuras, se sugiere la integración de la ontología en plataformas interoperables, la incorporación de aprendizaje automático para enriquecer la detección de amenazas y la expansión del modelo a otros dominios de la ciberseguridad. En conclusión, el uso de agentes inteligentes y ontologías formales representa una estrategia robusta y escalable para enfrentar los desafíos actuales en la protección de sistemas digitales.

\section*{Recomendaciones}
\addcontentsline{toc}{section}{Recomendaciones}
A partir del desarrollo realizado, se proponen las siguientes recomendaciones y líneas de trabajo futuro:

\begin{itemize}
    \item \textbf{Interoperabilidad semántica:} Traducir la ontología desarrollada en Z3 a formatos estándar como OWL o TTL, permitiendo su integración con plataformas de la Web Semántica y herramientas como Protégé.
    \item \textbf{Integración de aprendizaje automático:} Incorporar técnicas de machine learning para enriquecer la detección de amenazas y permitir que el agente evolucione y se adapte a nuevos patrones de ataque de manera autónoma.
    \item \textbf{Validación en entornos reales:} Probar el agente en escenarios de ciberseguridad reales o simulados, evaluando su desempeño, robustez y capacidad de respuesta ante amenazas complejas.
    \item \textbf{Expansión del dominio:} Ampliar la ontología y el agente para cubrir otros ámbitos de la ciberseguridad, como la protección de infraestructuras críticas, IoT o sistemas industriales.
    \item \textbf{Interfaz de usuario y visualización:} Desarrollar interfaces gráficas que permitan a los usuarios interactuar con el agente, visualizar el estado del sistema y comprender las decisiones tomadas por el modelo.
    \item \textbf{Colaboración multiagente:} Explorar la posibilidad de implementar sistemas multiagente, donde varios agentes colaboren o compitan para mejorar la defensa cibernética de manera distribuida.
    \item \textbf{Consideraciones éticas y legales:} Analizar los aspectos éticos y regulatorios asociados al uso de agentes inteligentes en ciberseguridad, especialmente en lo referente a privacidad y toma de decisiones automatizada.
\end{itemize}


\newpage
\section*{Referencias}
\addcontentsline{toc}{section}{Referencias}
\renewcommand{\refname}{}
\renewcommand{\bibname}{}
\vspace{-1.5cm}
\printbibliography{}

\newpage
\section*{Nota sobre la representacion de la ontologia}
\addcontentsline{toc}{section}{Nota sobre la representacion de la ontologia}

\textbf{Disclaimer:}
La ontología presentada en este trabajo fue implementada utilizando el solucionador lógico Z3, lo que permite realizar razonamiento formal y verificación automática sobre los conceptos y relaciones definidos. Sin embargo, no se generó un archivo en formato OWL (.owl) o Turtle (.ttl), ya que el enfoque adoptado priorizó la expresividad lógica y la capacidad de inferencia computacional sobre la interoperabilidad con herramientas de la Web Semántica.
En caso de requerirse una representación en OWL o TTL, sería necesario traducir manualmente los conceptos, relaciones y restricciones aquí modelados a dichos formatos utilizando un editor de ontologías.

\newpage
\appendix

\section{Código Fuente}\label{source_code}
\begin{lstlisting}[language=python, inputencoding=utf8]
# -*- coding: utf-8 -*-
# %%
# type: ignore

import enum
from typing import Any, Dict, List, Optional, Tuple

from z3.z3 import (
    And,
    BoolSort,
    BoolVal,
    CheckSatResult,
    Const,
    DeclareSort,
    ExprRef,
    ForAll,
    Function,
    Implies,
    IntSort,
    ModelRef,
    Not,
    Solver,
    sat,
)

# %%


class ThreatLevel(enum.Enum):
    LOW = 1
    MEDIUM = 2
    HIGH = 3
    CRITICAL = 4


class AssetType(enum.Enum):
    SERVER = 1
    NETWORK_DEVICE = 2
    IOT_DEVICE = 3
    WORKSTATION = 4
    CLOUD_SERVICE = 5


class AttackType(enum.Enum):
    MALWARE = 1
    PHISHING = 2
    DDoS = 3
    DATA_BREACH = 4
    RANSOMWARE = 5
    INSIDER_THREAT = 6


# %%

ThreatSort = DeclareSort("Threat")
AssetSort = DeclareSort("Asset")
AttackSort = DeclareSort("Attack")
UserSort = DeclareSort("User")
IncidentSort = DeclareSort("Incident")

# %%
# functions and predicates for the ontology

affects = Function("affects", AttackSort, AssetSort, BoolSort())
has_vulnerability = Function("has_vulnerability", AssetSort, BoolSort())
threat_level = Function("threat_level", AttackSort, IntSort())
asset_criticality = Function("asset_criticality", AssetSort, IntSort())
requires_immediate_response = Function(
    "requires_immediate_response", IncidentSort, BoolSort()
)
user_has_access = Function("user_has_access", UserSort, AssetSort, BoolSort())
is_compromised = Function("is_compromised", AssetSort, BoolSort())

# %%
# PEAS metrics

response_time = Function("response_time", IncidentSort, IntSort())
containment_effectiveness = Function(
    "containment_effectiveness", IncidentSort, IntSort()
)
false_positive = Function("false_positive", IncidentSort, BoolSort())

# %%
solver = Solver()

# %%
threat = Const("threat", AttackSort)
asset = Const("asset", AssetSort)
user = Const("user", UserSort)
incident = Const("incident", IncidentSort)

# %%
# type: ignore

solver.add(
    ForAll(
        [threat, asset, incident],
        Implies(
            And(
                has_vulnerability(asset),
                affects(threat, asset),
                threat_level(threat) >= 3,
            ),
            requires_immediate_response(incident),
        ),
    )
)

solver.add(
    ForAll(
        [asset, incident],
        Implies(
            And(is_compromised(asset), asset_criticality(asset) >= 4),
            requires_immediate_response(incident),
        ),
    )
)

solver.add(
    ForAll(
        [user, asset],
        Implies(
            And(Not(user_has_access(user, asset)), is_compromised(asset)),
            BoolVal(True),
        ),
    )
)

solver.add(
    ForAll(
        [incident],
        Implies(
            requires_immediate_response(incident), response_time(incident) <= 15
        ),
    )
)

# %%
# type: ignore


class CybersecuritySensors:
    def __init__(self, solver: Solver) -> None:
        self.solver: Solver = solver

    def network_anomaly_sensor(
        self, traffic_data: Dict[str, Any]
    ) -> Optional[ExprRef]:
        anomaly_detected: bool = traffic_data.get("unusual_patterns", False)

        if anomaly_detected:
            new_threat: ExprRef = Const(
                f"network_threat_{id(traffic_data)}", AttackSort
            )
            self.solver.add(threat_level(new_threat) >= 2)
            return new_threat

        return None

    def file_integrity_sensor(
        self, file_changes: Dict[str, Any]
    ) -> Optional[ExprRef]:
        """Monitor de integridad de archivos"""
        if file_changes.get("unauthorized_changes", False):
            affected_asset: ExprRef = Const(
                f"asset_{file_changes['asset_id']}", AssetSort
            )
            self.solver.add(is_compromised(affected_asset))
            return affected_asset

        return None

    def user_behavior_sensor(
        self, user_activity: Dict[str, Any]
    ) -> Optional[Tuple[ExprRef, ExprRef]]:
        """Sensor UEBA (User and Entity Behavior Analytics)"""
        if user_activity.get("anomalous_behavior", False):
            suspicious_user: ExprRef = Const(
                f"user_{user_activity['user_id']}", UserSort
            )
            target_asset: ExprRef = Const(
                f"asset_{user_activity['target_asset']}", AssetSort
            )
            self.solver.add(Not(user_has_access(suspicious_user, target_asset)))
            return (suspicious_user, target_asset)

        return None


# %%
# type: ignore


class CyberSecurityActuators:
    def __init__(self, solver: Solver) -> None:
        self.solver: Solver = solver

    def block_malicious_traffic(self, threat_source: str) -> bool:
        return True

    def isolate_compromised_device(self, asset_id: str) -> bool:
        """Aislamiento de dispositivos comprometidos"""
        print(f"Aislando dispositivo comprometido: {asset_id}")
        isolated_asset: ExprRef = Const(f"asset_{asset_id}", AssetSort)
        self.solver.add(Not(is_compromised(isolated_asset)))
        return True

    def generate_incident_report(
        self, incident_data: Dict[str, Any]
    ) -> Dict[str, Any]:
        """Generar reporte detallado del incidente"""
        analysis: Dict[str, Any] = incident_data.get("analysis", {})

        report: Dict[str, Any] = {
            "timestamp": incident_data.get("timestamp"),
            "severity": incident_data.get("severity"),
            "affected_assets": incident_data.get("assets", []),
            "recommended_actions": analysis.get("recommendations", []),
            "immediate_response_triggered": analysis.get(
                "immediate_response", False
            ),
            "actions_taken": incident_data.get("actions_taken", []),
            "z3_reasoning_used": analysis.get("z3_model_used", False),
        }

        return report

    def _get_recommendations(self, incident_data: Dict[str, Any]) -> List[str]:
        return ["Aplicar parches", "Monitorear actividad", "Revisar logs"]


# %%
# type: ignore


class CyberSecurityReasoningEngine:
    def __init__(
        self,
        solver: Solver,
        sensors: CybersecuritySensors,
        actuators: CyberSecurityActuators,
    ) -> None:
        self.solver: Solver = solver
        self.sensors: CybersecuritySensors = sensors
        self.actuators: CyberSecurityActuators = actuators

    def analyze_threat(self, threat_data: Dict[str, Any]) -> Dict[str, Any]:
        local_solver: Solver = Solver()

        for assertion in self.solver.assertions():
            local_solver.add(assertion)

        current_threat: ExprRef = Const(f"threat_{id(threat_data)}", AttackSort)
        current_asset: ExprRef = Const(f"asset_{id(threat_data)}", AssetSort)
        current_incident: ExprRef = Const(
            f"incident_{id(threat_data)}", IncidentSort
        )

        # Agregar hechos basados en los datos del incidente
        self._add_threat_facts(
            local_solver,
            current_threat,
            current_asset,
            current_incident,
            threat_data,
        )

        # Verificar si el modelo es satisfacible        check_result: CheckSatResult = local_solver.check()
        if check_result == sat:
            model: ModelRef = local_solver.model()

            # Evaluar usando el modelo Z3
            threat_severity: ThreatLevel = self._evaluate_threat_severity_z3(
                current_threat, model, threat_data
            )
            response_needed: bool = self._requires_immediate_response_z3(
                current_incident, model, threat_data
            )
            recommendations: List[str] = self._generate_recommendations_z3(
                model, threat_data, current_threat, current_asset
            )

            return {
                "severity": threat_severity,
                "immediate_response": response_needed,
                "recommendations": recommendations,
                "z3_model_used": True,
            }
        else:
            return {"error": "Inconsistencia en la ontologia detectada"}

    def _add_threat_facts(
        self,
        local_solver: Solver,
        threat: ExprRef,
        asset: ExprRef,
        incident: ExprRef,
        threat_data: Dict[str, Any],
    ) -> None:
        if threat_data.get("affects_critical_assets", False):
            local_solver.add(threat_level(threat) == 4)  # CRITICAL
        elif threat_data.get("widespread_impact", False):
            local_solver.add(threat_level(threat) == 3)  # HIGH
        elif threat_data.get("network_anomaly", False):
            local_solver.add(threat_level(threat) == 2)  # MEDIUM
        else:
            local_solver.add(threat_level(threat) == 1)  # LOW

        if threat_data.get("affects_critical_assets", False):
            local_solver.add(asset_criticality(asset) == 5)  # Maxima criticidad
        else:
            local_solver.add(asset_criticality(asset) == 2)  # Criticidad normal

        # Establecer relaciones
        local_solver.add(affects(threat, asset))

        if threat_data.get("compromised_assets", []):
            local_solver.add(is_compromised(asset))

        if threat_data.get("vulnerability_detected", False):
            local_solver.add(has_vulnerability(asset))

    def _evaluate_threat_severity_z3(
        self, threat: ExprRef, model: ModelRef, threat_data: Dict[str, Any]
    ) -> ThreatLevel:
        """Evaluar severidad usando el modelo Z3"""
        try:
            # Intentar evaluar el nivel de amenaza desde el modelo
            threat_level_val: Optional[ExprRef] = model.evaluate(
                threat_level(threat), model_completion=True
            )

            if threat_level_val is not None:
                level: int = threat_level_val.as_long()
                if level >= 4:
                    return ThreatLevel.CRITICAL
                elif level >= 3:
                    return ThreatLevel.HIGH
                elif level >= 2:
                    return ThreatLevel.MEDIUM
                else:
                    return ThreatLevel.LOW
        except Exception as e:
            print(f"Error evaluando nivel de amenaza: {e}")

        if threat_data.get("affects_critical_assets", False):
            return ThreatLevel.CRITICAL
        elif threat_data.get("widespread_impact", False):
            return ThreatLevel.HIGH
        else:
            return ThreatLevel.MEDIUM

    def _requires_immediate_response_z3(
        self, incident: ExprRef, model: ModelRef, threat_data: Dict[str, Any]
    ) -> bool:
        """Determinar respuesta inmediata usando Z3"""
        try:
            immediate_response: Optional[ExprRef] = model.evaluate(
                requires_immediate_response(incident), model_completion=True
            )

            if immediate_response is not None:
                return bool(immediate_response)
        except Exception as e:
            print(f"Error evaluando respuesta inmediata: {e}")

        # Fallback
        return threat_data.get(
            "affects_critical_assets", False
        ) or threat_data.get("active_exploitation", False)

    def _generate_recommendations_z3(
        self,
        model: ModelRef,
        threat_data: Dict[str, Any],
        threat: ExprRef,
        asset: ExprRef,
    ) -> List[str]:
        recommendations: List[str] = []

        try:
            # Evaluar diferentes aspectos del modelo
            is_asset_compromised: Optional[ExprRef] = model.evaluate(
                is_compromised(asset), model_completion=True
            )
            threat_level_val: Optional[ExprRef] = model.evaluate(
                threat_level(threat), model_completion=True
            )
            asset_critical: Optional[ExprRef] = model.evaluate(
                asset_criticality(asset), model_completion=True
            )

            # Recomendaciones basadas en el estado del activo
            if is_asset_compromised and bool(is_asset_compromised):
                recommendations.append(
                    "CRITICO: Aislar activo comprometido inmediatamente"
                )
                recommendations.append(
                    "Realizar analisis forense del activo afectado"
                )

            # Recomendaciones basadas en nivel de amenaza
            if threat_level_val and threat_level_val.as_long() >= 3:
                recommendations.append(
                    "Activar protocolo de respuesta de emergencia"
                )
                recommendations.append("Notificar al equipo directivo")

            # Recomendaciones basadas en criticidad del activo
            if asset_critical and asset_critical.as_long() >= 4:
                recommendations.append("Implementar monitoreo continuo 24/7")
                recommendations.append(
                    "Realizar backup inmediato de datos criticos"
                )

        except Exception as e:
            print(f"Error evaluando modelo Z3: {e}")

        # Recomendaciones adicionales basadas en tipo de amenaza
        if threat_data.get("network_anomaly", False):
            recommendations.append("Analizar trafico de red en tiempo real")
            recommendations.append(
                "Implementar reglas de firewall restrictivas"
            )

        if threat_data.get("user_anomaly", False):
            recommendations.append(
                "Revisar credenciales y permisos de usuario"
            )
            recommendations.append("Forzar cambio de contrase\u00f1as")

        if threat_data.get("malware_detected", False):
            recommendations.append(
                "Ejecutar escaneo completo de antimalware"
            )
            recommendations.append(
                "Limpiar y desinfectar sistemas afectados"
            )

        if not recommendations:
            recommendations = [
                "Documentar incidente en el sistema SIEM",
                "Incrementar nivel de monitoreo",
                "Revisar logs de seguridad",
            ]

        return recommendations

    def respond_to_incident(
        self, incident_data: Dict[str, Any]
    ) -> Dict[str, Any]:
        print(
            f"Analizando incidente: {incident_data.get('timestamp', 'N/A')}"
        )

        analysis: Dict[str, Any] = self.analyze_threat(incident_data)

        if "error" in analysis:
            return analysis

        severity: ThreatLevel = analysis["severity"]
        print(f"[SEVERIDAD] Severidad detectada: {severity.name}")
        print(
            f"[ALERTA] Respuesta inmediata requerida: {analysis['immediate_response']}"
        )

        # Ejecutar actuadores si es necesario
        actions_taken: List[str] = []
        if analysis.get("immediate_response", False):
            if incident_data.get("network_threat"):
                source: str = incident_data.get("source", "unknown")
                if self.actuators.block_malicious_traffic(source):
                    actions_taken.append(f"Bloqueado trafico desde {source}")

            compromised_assets: List[str] = incident_data.get(
                "compromised_assets", []
            )
            if compromised_assets:
                for asset in compromised_assets:
                    if self.actuators.isolate_compromised_device(asset):
                        actions_taken.append(f"Aislado dispositivo {asset}")

        # Generar reporte detallado
        report: Dict[str, Any] = self.actuators.generate_incident_report(
            {
                "timestamp": incident_data.get("timestamp"),
                "severity": analysis.get("severity"),
                "assets": incident_data.get("compromised_assets", []),
                "analysis": analysis,
                "actions_taken": actions_taken,
            }
        )

        report["actions_taken"] = actions_taken
        report["z3_reasoning"] = analysis.get("z3_model_used", False)

        return report


# %%
sensors = CybersecuritySensors(solver=solver)
actuators = CyberSecurityActuators(solver=solver)
reasoning_engine = CyberSecurityReasoningEngine(
    solver=solver, sensors=sensors, actuators=actuators
)
\end{lstlisting}

\end{document}