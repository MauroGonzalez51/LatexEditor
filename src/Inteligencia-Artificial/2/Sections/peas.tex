\section{Modulo \textit{PEAS}}

El diseño del agente inteligente se describe formalmente mediante el modelo
\textbf{PEAS} (Performance, Environment, Actuators, Sensors), el cual permite
definir de manera estructurada los elementos fundamentales de su operación.

\subsection{Performance (Medición del Desempeño)}
Las métricas utilizadas para evaluar el desempeño del agente incluyen:
\begin{itemize}
    \item Tiempo Medio de Respuesta (MTTR).
    \item Tiempo Medio Entre Fallos (MTBF).
    \item Efectividad de contención ante incidentes.
    \item Porcentaje de ataques prevenidos.
    \item Número de falsos positivos.
    \item Estimación del impacto económico evitado.
\end{itemize}

\subsection{Environment (Entorno)}
El agente opera en:
\begin{itemize}
    \item Infraestructura de red corporativa.
    \item Servidores críticos y servicios en la nube.
    \item Dispositivos IoT y BYOD conectados a la red.
    \item Factores humanos y organizacionales, incluyendo políticas de seguridad, niveles
          de autorización y comportamiento del usuario.
\end{itemize}

\subsection{Actuators (Actuadores)}
Los actuadores del sistema incluyen:
\begin{itemize}
    \item Bloqueo automático de tráfico malicioso.
    \item Aislamiento de dispositivos comprometidos.
    \item Módulo SOAR para respuesta orquestada (bloqueo, escalamiento, notificación).
    \item Generación de reportes ejecutivos y dashboards en tiempo real.
    \item Aplicación de parches y restauración de servicios.
\end{itemize}

\subsection{Sensors (Sensores)}
El agente integra sensores especializados para la detección de amenazas:
\begin{itemize}
    \item Sensores de red para tráfico y anomalías.
    \item Monitores de integridad en sistemas y archivos.
    \item Antivirus y detección basada en firmas.
    \item Sensores basados en inteligencia de amenazas externa.
    \item Sensores de comportamiento de usuario (UEBA).
\end{itemize}