\section{Introduccion}

\subsection{Caso de Estudio}
La defensa cibernética es un conjunto de estrategias, tecnologías y procesos
diseñados para proteger los sistemas informáticos, redes y datos contra
amenazas cibernéticas, como ataques de malware, piratería informática, robo de
datos y otros riesgos de seguridad. Estas medidas de defensa buscan prevenir,
detectar, responder y recuperarse de ataques cibernéticos con el fin de
garantizar la confidencialidad, integridad y disponibilidad de la información y
los recursos digitales.

\subsection{Resumen Ejecutivo}
Este documento aborda el diseño e implementación de una ontología orientada a 
la defensa cibernética y su posterior aplicación en un agente inteligente. 
Primero se revisan los fundamentos teóricos de las ontologías: qué son, cómo se 
construyen actualmente y cuáles son las metodologías más utilizadas en su 
desarrollo. Luego, se analiza cómo estas ontologías pueden integrarse en 
sistemas inteligentes para mejorar la capacidad de detección, respuesta y 
resiliencia frente a amenazas cibernéticas. 

El propósito central es proponer una ontología que sirva de base para dotar a 
un agente inteligente de un marco de conocimiento estructurado, que le permita 
tomar decisiones informadas en escenarios de ciberseguridad.