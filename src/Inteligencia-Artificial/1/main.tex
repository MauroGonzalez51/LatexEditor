\documentclass[letterpaper, 12pt]{report}

\usepackage[utf8]{inputenc}
\usepackage[english, spanish]{babel}

\usepackage{fullpage}
\usepackage{setspace}
\setlength{\parskip}{\baselineskip}
\renewcommand{\baselinestretch}{1.5}
\raggedbottom{}

\usepackage{graphicx}
\usepackage{caption}
\usepackage{wrapfig}
\usepackage{float}
\usepackage{tabularx}
\usepackage{array}
\usepackage{siunitx}

\usepackage{amsmath}
\usepackage{bm}

\usepackage{enumitem}
\usepackage{chngcntr}
\usepackage{verbatim}
\usepackage{csquotes}

\usepackage{xurl}
\usepackage{hyperref}
\usepackage{booktabs}

% ===== Bibliografía =====
\usepackage[style=apa, maxnames=6, minnames=3]{biblatex}
\DefineBibliographyStrings{english}{
    andothers = {\em et\addabbrvspace{} al\adddot{}\/}
}
\addbibresource{./Bibliography/bibliography.bib}

\counterwithin{figure}{section}
\renewcommand{\thesection}{\arabic{section}}
\renewcommand{\thesubsection}{\thesection.\arabic{subsection}}
\renewcommand{\thefigure}{\arabic{figure}}

\newcommand{\bolditalic}[1]{\textbf{\textit{#1}}}
\renewcommand{\comment}[1]{{\small $\ll$#1$\gg$}}
\hypersetup{
    colorlinks=true,
    linkcolor=blue,
    urlcolor=blue,
    citecolor=blue,
    pdfborder={0 0 1}
}

% chktex-file 24

\begin{document}

\begin{titlepage}
      \centering
      \includegraphics[width=0.3\textwidth]{../../../images/logo_utb.png}\par\vspace{0.8cm}
      {\scshape\LARGE Universidad Tecnológica de Bolívar \par}
      \vspace{1cm}

      {\scshape\Large Inteligencia Artificial \par}
      \vspace{1cm}

      {\Large \bfseries Agente Inteligente\\}
      {\large Agente de Defensa Cibernetica}
      \vspace{2cm}

      {\itshape{} Mauro Alonso Gonzalez Figueroa, T00067622\/ \\}
      {\itshape{} Juan Jose Jimenez Guardo, T00068278\/ \\}
      \vfill
      Revisado Por\\
      Edwin Alexander Puertas Del Castillo\\
      {\large \today\par}
\end{titlepage}
\nocite{*}

\tableofcontents{}
\newpage

\section{Resumen Ejecutivo}

\begin{quote}
      \textbf{Caso de Estudio}: La defensa cibernética es un conjunto de estrategias, tecnologías y procesos
      diseñados para proteger los sistemas informáticos, redes y datos contra
      amenazas cibernéticas, como ataques de malware, piratería informática, robo de
      datos y otros riesgos de seguridad. Estas medidas de defensa buscan prevenir,
      detectar, responder y recuperarse de ataques cibernéticos con el fin de
      garantizar la confidencialidad, integridad y disponibilidad de la información y
      los recursos digitales.
\end{quote}

El agente inteligente de defensa cibernética está diseñado para proteger
infraestructuras digitales mediante la detección, prevención y respuesta
automatizada ante amenazas cibernéticas. Su arquitectura integra sensores
avanzados, actuadores de respuesta y métricas de desempeño que garantizan la
resiliencia y seguridad de los sistemas corporativos. El diseño incorpora
recomendaciones técnicas para mejorar la capacidad de reacción, la cobertura
ante nuevas amenazas y la alineación con estándares modernos de ciberseguridad.

\section{Introducción}

El objetivo de este informe es presentar el diseño y ajuste de un agente
inteligente enfocado en defensa cibernética, detallando su estructura bajo el
enfoque PEAS.~La importancia de diseñar agentes inteligentes radica en su
capacidad para automatizar la protección de sistemas críticos, reducir riesgos
y mejorar la eficiencia en la gestión de incidentes de seguridad. El alcance de
la actividad incluye la definición técnica del agente, sus componentes y la
justificación de los ajustes realizados según las recomendaciones
proporcionadas.

\section{Descripción del Agente Asignado}

\subsection{Tipo de agente}
El agente propuesto es de tipo basado en modelo, capaz de razonar sobre el
estado actual y anticipar acciones futuras para maximizar la seguridad.

\subsection{Dominio o entorno}
Opera en infraestructuras de red corporativa, servidores críticos y servicios
en la nube, considerando también el entorno humano y dispositivos IoT.

\subsection{Tareas principales}
Las tareas incluyen la detección de amenazas, bloqueo automático de tráfico
malicioso, aislamiento de dispositivos comprometidos, auditorías de seguridad,
aplicación de actualizaciones y generación de alertas.

\section{Ajustes del Diseño Basados en Recomendaciones Previas}

\subsection{Recomendaciones consideradas}
Las siguientes recomendaciones técnicas fueron incorporadas al diseño del
agente inteligente de defensa cibernética:

\begin{itemize}
      \item Incorporar métricas avanzadas de desempeño, como tiempo medio de respuesta \\
            (MTTR), tiempo medio entre fallos (MTBF), efectividad de contención y
            estimación del impacto económico evitado.
      \item Ampliar el entorno operativo considerando factores humanos y organizacionales,
            políticas internas, niveles de autorización, dispositivos BYOD y sistemas IoT,
            así como el comportamiento del usuario.
      \item Integrar sistemas de respuesta orquestada (SOAR) para automatizar el bloqueo,
            documentación, escalamiento y notificación de incidentes, junto con módulos de
            gestión de registros forenses y dashboards ejecutivos en tiempo real.
      \item Incorporar sensores basados en inteligencia de amenazas externa, diferenciando
            entre detección por firmas y por anomalías comportamentales, e incluyendo
            sensores de comportamiento del usuario (UEBA) para identificar amenazas
            internas.
\end{itemize}

\subsection{Cambios aplicados al diseño original}

El diseño del agente fue ajustado para incorporar todas las recomendaciones
técnicas recibidas, logrando una solución más robusta y alineada con las
mejores prácticas actuales en ciberseguridad. Los principales cambios
realizados son:

\begin{itemize}
      \item \textbf{Performance Measuring:} Se añadieron métricas avanzadas como el tiempo medio de respuesta (MTTR), tiempo medio
            entre fallos (MTBF), efectividad de contención y estimación del impacto económico evitado, además de los indicadores
            originales (porcentaje de ataques prevenidos, cantidad de ataques maliciosos detectados y falsos positivos).
      \item \textbf{Environment:} El entorno del agente se amplió para incluir factores humanos y organizacionales, políticas
            internas de seguridad, niveles de autorización, dispositivos BYOD y sistemas IoT, así como el comportamiento del usuario, además
            de la infraestructura de red, servidores críticos y servicios en la nube.
      \item \textbf{Actuators:} Se integró un sistema de respuesta orquestada (SOAR) para automatizar el bloqueo, documentación,
            escalamiento y notificación de incidentes. Se añadió un módulo de gestión de registros forenses y dashboards ejecutivos en
            tiempo real, complementando los sistemas de bloqueo, escaneo y monitoreo automáticos ya presentes.
      \item \textbf{Sensors:} Se incorporaron sensores basados en inteligencia de amenazas externa, diferenciando entre
            detección por firmas y por anomalías comportamentales, e incluyendo sensores de comportamiento del usuario (UEBA), además
            de los antivirus, monitores de integridad y sensores de red originales.
\end{itemize}

\subsection{Justificación técnica de los ajustes}

La aplicación de estos cambios responde a la necesidad de fortalecer la
capacidad operativa y adaptativa del agente frente a amenazas cada vez más
sofisticadas. Las nuevas métricas permiten una evaluación más precisa de la
eficiencia y resiliencia del sistema, facilitando la toma de decisiones y la
mejora continua. La ampliación del entorno operativo y la integración de
factores humanos e IoT aseguran una cobertura más completa y realista,
considerando los riesgos actuales en infraestructuras corporativas.

La incorporación de sistemas SOAR y gestión forense automatiza procesos
críticos, reduce el tiempo de respuesta y mejora la trazabilidad de incidentes,
lo que es esencial para auditorías y cumplimiento normativo. Finalmente, el uso
de sensores avanzados y diferenciados permite detectar tanto amenazas externas
como internas, aumentando la capacidad de prevención y respuesta ante ataques
complejos y comportamientos anómalos.

Estos ajustes posicionan al agente como una solución integral y moderna, capaz
de adaptarse a los desafíos dinámicos del entorno digital y alineada con los
estándares internacionales de ciberseguridad.

\section{Definición del Espacio de Estados}

\subsection{Descripción de las variables que definen cada estado}
El espacio de estados del agente inteligente de defensa cibernética se define
por las siguientes variables principales:

\begin{itemize}
      \item \textbf{Estado de operación}: Indica si el sistema está en monitoreo normal, detección de anomalía, respuesta automática, análisis/escalamiento, recuperación o mejora continua.
      \item \textbf{Nivel de amenaza}: Valor que representa la severidad de la actividad detectada (normal, sospechosa, maliciosa).
      \item \textbf{Estado de dispositivos}: Identifica si los dispositivos están operativos, comprometidos, aislados o en proceso de recuperación.
      \item \textbf{Alertas y reportes}: Registro de alertas generadas, reportes forenses y acciones ejecutadas.
      \item \textbf{Métricas de desempeño}: Incluye tiempo medio de respuesta (MTTR), tiempo medio entre fallos (MTBF), efectividad de contención y falsos positivos.
      \item \textbf{Actualización de conocimiento}: Estado de las firmas, IOCs y reglas heurísticas del sistema.
\end{itemize}

\subsection{Representación de los estados}
A continuación se presenta una tabla que resume los estados principales y sus
variables asociadas:

\begin{table}[H]
      \small
      \centering
      \begin{tabularx}{\textwidth}{>{\raggedright\arraybackslash}X
                  >{\raggedright\arraybackslash}X
                  >{\raggedright\arraybackslash}X
                  >{\raggedright\arraybackslash}X
                  >{\raggedright\arraybackslash}X}
            \toprule{}
            Estado                  & Nivel de amenaza & Dispositivo              & Acción                & Métricas          \\
            \midrule{}
            Monitoreo Normal        & Baja             & Operativo                & Supervisión           & MTBF              \\
            Detección de Anomalía   & Media            & Operativo / Comprometido & Análisis              & Falsos positivos  \\
            Respuesta Automática    & Alta             & Comprometido / Aislado   & Bloqueo, Aislamiento  & MTTR, Contención  \\
            Análisis y Escalamiento & Alta             & Comprometido             & Reporte, Escalamiento & Evidencia forense \\
            Recuperación            & Variable         & Restaurando              & Parches, Backups      & MTTR              \\
            Mejora Continua         & Baja             & Operativo                & Actualización         & Aprendizaje       \\
            \bottomrule{}
      \end{tabularx}
      \caption{Estados principales y variables asociadas del agente}
\end{table}

\subsection{Transiciones entre estados}

Las transiciones entre estados se producen en función de los eventos detectados
y las acciones del agente. El siguiente diagrama textual describe el flujo
principal:

\begin{verbatim}
Monitoreo Normal
    → [Anomalía detectada] → Detección de Anomalía

Detección de Anomalía
    → [Falso positivo] → Monitoreo Normal
    → [Confirmado ataque] → Respuesta Automática

Respuesta Automática
    → [Ataque controlado] → Recuperación
    → [Ataque complejo/no contenido] → Análisis y Escalamiento

Análisis y Escalamiento
    → [Caso resuelto] → Recuperación

Recuperación
    → [Servicios restaurados] → Mejora Continua

Mejora Continua
    → [Actualización completada] → Monitoreo Normal
\end{verbatim}

\section{Objetos del Agente}

\begin{itemize}
      \item \textbf{Sensor de red}: Analiza tráfico y detecta anomalías.
      \item \textbf{Actuador de bloqueo}: Ejecuta acciones de aislamiento y bloqueo.
      \item \textbf{Monitor de integridad}: Supervisa cambios no autorizados.
      \item \textbf{Módulo SOAR}: Orquesta respuestas y genera informes.
      \item \textbf{Gestor de registros forenses}: Prepara evidencia para auditorías.
\end{itemize}

\section{Predicados del Agente}

\begin{itemize}
      \item \texttt{Amenaza (x) $\rightarrow$ Bloquear (x)}
      \item \texttt{Comprometido (y) $\rightarrow$ Aislar (y)}
      \item \texttt{Alerta (z) $\rightarrow$ Notificar (z)}
      \item \texttt{UsuarioInusual (u) $\rightarrow$ Analizar (u)}
\end{itemize}

Si \texttt{Amenaza (TráficoMalicioso)} entonces \texttt{Bloquear
      (TráficoMalicioso)}.

\section{Conclusiones}
El proceso de diseño permitió fortalecer la capacidad operativa del agente,
integrando recomendaciones expertas y ampliando su cobertura ante amenazas.
Futuras mejoras pueden incluir el uso de aprendizaje automático para detección
proactiva y mayor integración con sistemas externos. El enfoque de agentes
inteligentes en IA resulta esencial para la protección dinámica y automatizada
de infraestructuras críticas.

\printbibliography{}

\end{document}