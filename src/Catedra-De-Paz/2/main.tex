\documentclass[letterpaper, 12pt]{article}

\usepackage[utf8]{inputenc}
\usepackage[english, spanish]{babel}

\usepackage{newtxtext}

\usepackage{fullpage}
\usepackage{graphicx}
\usepackage{amsmath}
\usepackage{enumitem}
\usepackage{chngcntr}
\usepackage{setspace}
\usepackage{xurl}
\usepackage{csquotes}
\usepackage{float}
\usepackage{verbatim}
\usepackage{tabularx}
\usepackage{amsmath}
\usepackage{caption}
\usepackage{bm}
\usepackage{wrapfig}
\usepackage{siunitx}

\counterwithin{figure}{section}
\renewcommand{\thesection}{\arabic{section}}
\renewcommand{\thesubsection}{\thesection.\arabic{subsection}}
\renewcommand{\baselinestretch}{2}
\renewcommand{\thefigure}{\arabic{figure}}

\usepackage[style=apa, maxnames=6, minnames=3]{biblatex}
\DefineBibliographyStrings{english}{%chktex-file 1 chktex-file 6
      andothers = {\em et\addabbrvspace al\adddot}
}

\addbibresource{./Bibliography/bibliography.bib}

\usepackage{array}
\usepackage{enumitem}

\setlength{\parskip}{\baselineskip}

\newcommand{\bolditalic}[1]{\textbf{\textit{#1}}}

\DeclareSIUnit{\COP}{COP}
\newcommand{\cop}[1]{\$\SI{#1}{\COP}}

\DeclareSIUnit{\DOLLAR}{USD}
\newcommand{\dollar}[1]{\$\SI{#1}{\DOLLAR}}

\renewcommand{\comment}[1]{{\small $\ll$#1$\gg$}}

% chktex-file 24

\begin{document}

\section*{Resiliencia en tiempos de conflicto}

\vspace{-1cm}

\noindent Los mecanismos de resistencia de Mayerlis Angarita en los Montes de Maria.

\noindent\makebox[\linewidth]{\rule{\textwidth}{0.4pt}}

\begin{itemize}[label=$\triangleright$]
      \item Abner Gabriel Orozco Ballestas \textit{(T00065222)}
      \item Mauro Alonso Gonzalez Figueroa \textit{(T00067622)}
      \item Angelis Dayanis Ahumedo Salas \textit{(T00069889)}
      \item Sharon Karina Benitez Amaya \textit{(T00066278)}
\end{itemize}

\subsection*{Antecedentes}

Mayerlis Angarita es una destacada activista y defensora de los derechos
humanos que ha dedicado su vida a trabajar por las víctimas del conflicto
armado en Colombia, especialmente en la región de los Montes de María. Su
activismo surge de su experiencia personal como víctima directa de la
violencia, cuando en 2001 fue desplazada forzosamente por paramilitares, lo que
la llevó a convertirse en una voz crucial para otras mujeres afectadas.

Uno de los aspectos más sobresalientes de su labor es la creación de la
organización Narrar para Vivir, un colectivo conformado por más de 800 mujeres
que, a través de la narración y el apoyo mutuo, han reconstruido sus historias
y fortalecido su resistencia frente a la violencia. Esta organización ha sido
un espacio de sanación y empoderamiento para mujeres que han sufrido
desplazamiento, violencia sexual, y otros abusos durante el conflicto armado en
Colombia. A través de su trabajo, Mayerlis ha logrado visibilizar la lucha de
las mujeres en esta región y ha contribuido a la construcción de la memoria
histórica, reconociendo el papel clave que ellas juegan en la búsqueda de la
paz.

Además, Angarita ha tenido un papel protagónico en los diálogos de paz de La
Habana entre el gobierno colombiano y las FARC, donde representó a las víctimas
y trabajó para garantizar que sus voces fueran escuchadas en el proceso de paz.
A pesar de los avances, Mayerlis ha enfrentado múltiples amenazas contra su
vida debido a su activismo, lo que refleja la peligrosidad de defender los
derechos humanos en Colombia, especialmente para mujeres activistas.

Su lucha incansable ha sido reconocida a nivel nacional e internacional. En
2018, fue galardonada con el Premio Nacional a la Defensa de los Derechos
Humanos y, en 2021, recibió el Premio Internacional Mujeres de Coraje, otorgado
por el Departamento de Estado de los Estados Unidos.

\subsection*{Marco Teorico}

\begin{itemize}[label=$\bullet$]
      \item \textbf{Resiliencia Comunitaria}: \\ La resiliencia comunitaria se refiere a la
            capacidad de una comunidad para adaptarse, resistir y recuperarse de
            adversidades y situaciones traumáticas, como lo son los conflictos armados.
            Según estudios sobre este concepto, la resiliencia va más allá de la
            resistencia individual y se construye colectivamente, involucrando la
            solidaridad, la cooperación y la construcción de redes de apoyo social. En el
            caso de Mayerlis Angarita, su iniciativa de formar la organización Narrar para
            Vivir, que agrupa a más de 800 mujeres, es un ejemplo concreto de cómo las
            comunidades víctimas del conflicto en los Montes de María han construido
            resiliencia a través de la narración de sus experiencias y el apoyo mutuo. Esta
            organización permite que las mujeres no solo sanen colectivamente, sino que
            también contribuyan a la restauración del tejido social que fue desintegrado
            por la violencia.

            Autores como Ungar (2008) señalan que la resiliencia en contextos de violencia
            implica la capacidad de transformar el sufrimiento en acción, lo que está
            claramente reflejado en el liderazgo de Angarita. Además, las iniciativas
            comunitarias que ella ha liderado subrayan la importancia de empoderar a las
            víctimas para que tomen un rol activo en la reconstrucción de sus vidas y
            comunidades.
      \item \textbf{Memoria Histórica como Herramienta de Resistencia}: \\
            La memoria histórica juega un papel crucial en los procesos de reconstrucción
            social en contextos de postconflicto. Según Jelin (2002), la memoria es un acto
            de resistencia porque desafía las narrativas oficiales que suelen invisibilizar
            o distorsionar las experiencias de las víctimas. Para Mayerlis Angarita, el
            rescate de la memoria histórica ha sido fundamental en su activismo. A través
            de su trabajo, ha impulsado la reconstrucción de la memoria de las mujeres
            víctimas del conflicto, utilizando el testimonio y la narración como
            herramientas para visibilizar sus experiencias y construir una identidad
            colectiva. Esta iniciativa de documentar las historias de vida de las mujeres
            de los Montes de María ha permitido que las víctimas transformen su dolor en un
            recurso para la justicia y la verdad.

            La memoria histórica también ha sido utilizada como un mecanismo para promover
            la reconciliación y la paz en la región. Las prácticas de Mayerlis, que
            combinan el empoderamiento comunitario y el rescate de la memoria, coinciden
            con los postulados de autores como Elizabeth Jelin y Paul Ricoeur, quienes
            sostienen que la memoria no solo es un acto de recordación del pasado, sino
            también una forma de acción política que puede influir en el futuro de una
            comunidad. En este contexto, las acciones de Angarita no solo han permitido
            sanar a nivel individual, sino que también han contribuido a la construcción de
            una narrativa colectiva de resistencia y superación.

            El trabajo de Mayerlis Angarita se inserta dentro de un marco teórico que
            reconoce la importancia de la acción comunitaria y la memoria en la
            transformación de sociedades post-conflicto. Su liderazgo no solo ha permitido
            empoderar a las mujeres víctimas, sino que ha generado modelos replicables en
            otras regiones afectadas por la violencia, ofreciendo lecciones clave sobre la
            construcción de paz en contextos de adversidad extrema. Su lucha es un
            testimonio vivo de cómo las estrategias de resistencia, cuando están basadas en
            la solidaridad, el reconocimiento del pasado y el trabajo colectivo, pueden ser
            una vía efectiva para superar el trauma y construir un futuro más justo y
            pacífico.
\end{itemize}

\subsection*{Marco normativo}

El marco normativo que se aplicará en este trabajo se fundamenta en diversas
leyes y regulaciones que buscan proteger y reparar a las víctimas del conflicto
armado en Colombia, así como en los principios de derechos humanos y el Derecho
Internacional Humanitario. A continuación, se detallan los componentes clave de
este marco:

\begin{enumerate}
      \item Ley de Víctimas y Restitución de Tierras (Ley 1448 de 2011)
            \begin{itemize}[label=$\bullet$]
                  \item Objetivo: Proporcionar atención, asistencia y reparación integral a las
                        víctimas del conflicto armado.
                  \item Derechos Reconocidos: Derecho a la verdad, justicia, reparación y no
                        repetición.
                  \item Mecanismos: Establecimiento de un Fondo para la Reparación de las Víctimas y
                        creación de comisiones para la implementación de la ley.

            \end{itemize}

      \item Ley 975 de 2005 (Ley de Justicia y Paz)
            \begin{itemize}
                  \item Propósito: Facilitar la desmovilización de grupos armados y establecer un marco
                        para la justicia transicional.
                  \item Relevancia: Asegura que las voces de las víctimas sean escuchadas en los
                        procesos judiciales relacionados con excombatientes.

            \end{itemize}

      \item Marco Jurídico para la Paz
            \begin{itemize}
                  \item Aprobación: Reformado en 2012 para facilitar diálogos con grupos armados como
                        las FARC\@{}.
                  \item Importancia: Garantiza la inclusión de las víctimas en los procesos de paz,
                        promoviendo su participación activa.
            \end{itemize}

      \item Ley 2078 de 2021
            \begin{itemize}
                  \item Prórroga: Extiende la vigencia de la Ley 1448, reafirmando el compromiso del
                        Estado con las víctimas.
                  \item Enfoque Integral: Refuerza los mecanismos existentes para garantizar el acceso
                        a derechos.
            \end{itemize}

      \item Ley 387 de 1997
            \begin{itemize}
                  \item Antecedente Clave: Establece medidas para prevenir el desplazamiento forzado y
                        atender a sus víctimas.
                  \item Impacto: Sienta las bases para el desarrollo de políticas públicas orientadas a
                        la atención humanitaria.
            \end{itemize}

      \item Principios del Derecho Internacional Humanitario (DIH)
            \begin{itemize}
                  \item Protección a Civiles: Establece normas que protegen a las personas no
                        combatientes durante los conflictos armados.
                  \item Responsabilidad del Estado: Obliga al Estado colombiano a garantizar la
                        protección y reparación a las víctimas.
            \end{itemize}
\end{enumerate}

Este marco normativo analizamos el trabajo de Mayerlis Angarita y su impacto en
la vida de las mujeres afectadas por el conflicto armado en Colombia. Integrar
y promover un enfoque integral que reconozca la importancia de la memoria
histórica y la resiliencia comunitaria en los proceso de construcción nacional
y reconstrucción social.

\subsection*{Avance del producto}

% chktex-file 44
% chktex-file 12
\begin{table}[H]
      \begin{tabularx}{.9\linewidth}{|>{\centering\arraybackslash}X|>{\centering\arraybackslash}X|>{\centering\arraybackslash}X|}
            \hline
            Abner Orozco               &                                                                                                                                                 & Cátedra para la paz                                                                       \\ \hline
            Mauro Gonzalez             &                                                                                                                                                 &                                                                                           \\\hline
            Angelis Ahumedo            &                                                                                                                                                 &                                                                                           \\\hline
            Sharon Benitez             &                                                                                                                                                 &                                                                                           \\\hline
            \multicolumn{3}{|c|}{\textbf{Intro} 0:00 $-$ 0:15}                                                                                                                                                                                                                       \\\hline
            ABNER OROZCO 0:15 $-$ 0:40 & Bienvenida al podcast. Nombre del Podcast y presentación a equipo de trabajo.                                                                   & con sonidos relacionados a nuestra cultura, un poco de música autóctona.                  \\\hline

            \multicolumn{3}{|c|}{\textbf{Cortina}}                                                                                                                                                                                                                                   \\\hline
            ABNER OROZCO 0:40 $-$ 1:20 & Realiza una introducción a lo temática y se hace introducción a la invitada de este capítulo, Mayerlis Angarita, para iniciar una conversación. & Es acompañado de silencios y música instrumental, que acompañen los relatos y respuestas. \\\hline

      \end{tabularx}
\end{table}

\end{document}