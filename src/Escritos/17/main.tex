\documentclass[letterpaper, 12pt]{article}

\usepackage[utf8]{inputenc}
\usepackage[english, spanish]{babel}
\usepackage{fullpage}
\usepackage{graphicx}
\usepackage{amsmath}
\usepackage{enumitem}
\usepackage{chngcntr}
\usepackage{setspace}
\usepackage{url}
\usepackage{csquotes}
\usepackage{float}
\usepackage{verbatim}
\usepackage{tabularx}
\usepackage{amsmath}
\usepackage{caption}
\usepackage{bm}
\usepackage{wrapfig}
\usepackage{siunitx}

\counterwithin{figure}{section}
\renewcommand{\thesection}{\arabic{section}}
\renewcommand{\thesubsection}{\thesection.\arabic{subsection}}
\renewcommand{\baselinestretch}{2}
\renewcommand{\thefigure}{\arabic{figure}}

\usepackage[style=numeric, maxnames=6, minnames=3, backend=biber, parentracker=true, sorting=none]{biblatex}
\DefineBibliographyStrings{english}{%chktex-file 1 chktex-file 6
    andothers = {\em et\addabbrvspace al\adddot}
}

\addbibresource{./Bibliography/bibliography.bib}

\usepackage{array}
\usepackage{enumitem}

\setlength{\parskip}{\baselineskip}

\newcommand{\bolditalic}[1]{\textbf{\textit{#1}}}

\DeclareSIUnit{\COP}{COP}
\newcommand{\cop}[1]{\$\SI{#1}{\COP}}

\DeclareSIUnit{\DOLLAR}{USD}
\newcommand{\dollar}[1]{\$\SI{#1}{\DOLLAR}}

\renewcommand{\comment}[1]{{\small $\ll$#1$\gg$}}

% chktex-file 24

\begin{document}

% ! --------------------------------------------------------|>
\section{Introducción}

\textsuperscript{1} WhatsApp y Telegram son aplicaciones de mensajería
instantánea para dispositivos móviles y de escritorio.
Permiten a los usuarios enviar mensajes de texto, realizar
llamadas y videollamadas, compartir archivos y fotos, y
crear grupos de chat. WhatsApp, propiedad de Facebook, es
conocido por su amplia adopción global y su interfaz
sencilla. Por otro lado, Telegram se destaca por su enfoque
en la privacidad, la capacidad de compartir archivos
grandes y la posibilidad de crear canales públicos. Ambas
aplicaciones han ganado popularidad por ofrecer formas
eficientes de comunicarse en línea.

\footnotetext[1]{
    \begin{itemize}
        \item Explicar que es cada cosa
        \item Dar contexto de las aplicaciones, para que se utilizan
    \end{itemize}
}

% ! --------------------------------------------------------|>
\section{Planteamiento del problema}

\textsuperscript{2} A pesar de las funcionalidades esenciales que ofrecen
aplicaciones como WhatsApp y Telegram para facilitar la
comunicación, es lamentable que algunas personas opten por
utilizar estas plataformas con fines maliciosos. Entre las
amenazas más comunes se encuentran el phishing, el malware
y el ransomware, que aprovechan la interconexión y la
popularidad de estas aplicaciones para llevar a cabo
actividades ilícitas.

\footnotetext[2]{
    Introducción a: Phishing, Malware y Ransomware
}

% ! --------------------------------------------------------|>
\section{Enfoque en el Phishing}

El phishing se basa en tácticas de error humano y presión,
con el atacante haciéndose pasar por alguien en quien la
víctima confía. Estos ataques a menudo crean una sensación
de urgencia que induce a la víctima a actuar rápidamente.
Dada su simplicidad y menor costo en comparación con el
ataque directo a sistemas, los hackers prefieren explotar
la vulnerabilidad humana en lugar de atacar redes o
sistemas directamente.

% + -----------------------------------------------|>
\subsection{Prevención de Ataques de Phishing}

DMARC (Domain-based Message Authentication, Reporting, and
Conformance) se destaca como una forma eficaz de combatir
el phishing. Puede evitar que los atacantes se apropien de
su nombre de dominio, lo que les permitiría suplantar su
sitio o servicio y acceder a los datos de sus clientes. Sin
embargo, es esencial aplicar una política DMARC de p=reject
para maximizar la efectividad contra estos ataques.

% + -----------------------------------------------|>
\subsection{Mitigación de Ataques de Phishing}

Cuando los clientes reciben correos electrónicos de
phishing aparentemente provenientes de su dominio, es
crucial rastrear las IP maliciosas. Los informes DMARC
ofrecen una excelente manera de supervisar las fuentes de
envío y rastrear estas IP, permitiendo incluirlas
rápidamente en una lista negra.

% ! --------------------------------------------------------|>
\section{Enfoque en el Malware}

El malware, o ``software malicioso'', es un término amplio
que engloba cualquier programa o código perjudicial para
los sistemas. Este tipo de software intrusivo busca
invadir, dañar o deshabilitar dispositivos, desde
ordenadores hasta dispositivos móviles, asumiendo a menudo
el control parcial de sus operaciones y afectando su
funcionamiento normal, de manera similar a una enfermedad
como la gripe.

El malware tiene como objetivo principal obtener dinero de
manera ilícita. Aunque generalmente no daña el hardware,
puede robar, cifrar o borrar datos, alterar o secuestrar
funciones básicas de la computadora y espiar la actividad
en el ordenador sin conocimiento o permiso del usuario.

% + -----------------------------------------------|>
\subsection{Prevención de ataques de malware}

\begin{itemize}[label=$\diamond$]
    \item Instalación de Software Antivirus: El primer paso es
          instalar software antivirus, capaz de detectar y eliminar
          virus y otros tipos de software malicioso de la
          computadora. Es crucial realizar esta instalación tan
          pronto como sea posible después de la infección para evitar
          daños significativos.

    \item Mantenimiento del Sistema Operativo: Mantener el sistema
          operativo actualizado es esencial. Las actualizaciones
          automáticas proporcionan defensas contra nuevos virus y
          malware. No instalar nada si no hay actualizaciones
          disponibles para la versión del sistema operativo.

    \item Contraseñas Seguras: Utilizar contraseñas seguras en lugar
          de simples (como ``12345'') contribuye significativamente a
          prevenir ataques de malware.
\end{itemize}

% + -----------------------------------------------|>
\subsection{Mitigación de Ataques de Malware}

Si el ordenador está infectado, se deben tomar medidas de
mitigación de inmediato. Realizar un análisis completo con
un programa antivirus antes de intentar cualquier otro paso
es crucial. La propagación rápida del malware puede causar
problemas graves, por lo que es esencial abordar la
infección de manera rápida y efectiva.

% ! --------------------------------------------------------|>
\section{Enfoque en el Ransomware}

El ransomware es una forma insidiosa de malware que
restringe el acceso a sistemas o archivos personales y
exige un rescate para restaurar dicho acceso. Originado a
finales de los años 80, inicialmente se exigía el pago por
correo postal, pero en la actualidad, los atacantes
prefieren criptomonedas o tarjetas de crédito.

% + -----------------------------------------------|>
\subsection{Prevención y medidas de seguridad}

La mejor defensa contra el ransomware incluye prácticas
sólidas de seguridad:

\begin{itemize}[label=$\diamond$]
    \item Contraseñas Seguras: Utilizar contraseñas fuertes fortalece
          la protección del sistema.

    \item Software Antivirus: La instalación de un software antivirus
          confiable es esencial para detectar y eliminar amenazas.

    \item Protocolos de Autenticación de Correo Electrónico: \\
          Implementar medidas como DMARC ayuda a asegurar la
          legitimidad de los correos electrónicos.
\end{itemize}

% + -----------------------------------------------|>
\subsection{Respuesta ante un ataque}

En caso de verse afectado por ransomware, tome medidas
inmediatas:

\begin{itemize}[label=$\diamond$]
    \item Cautela con Correos Electrónicos: Evite abrir correos
          electrónicos sospechosos que soliciten dinero y no haga
          clic en enlaces.

    \item Eliminación de Software Sospechoso: \\ Elimine cualquier
          software no confiable y absténgase de instalar nuevos
          programas hasta que la infección se haya erradicado.

    \item Respaldo de Archivos: \\ Asegúrese de que todos los
          archivos estén respaldados en un lugar seguro.
\end{itemize}

% ! --------------------------------------------------------|>
\section{Conclusiones}

En resumen, las aplicaciones de mensajería instantánea,
como WhatsApp y Telegram, brindan funcionalidades
esenciales para la comunicación, pero lamentablemente son
susceptibles a amenazas maliciosas como el phishing, el
malware y el ransomware. Estas actividades ilícitas
comprometen la seguridad de los usuarios. Al centrarse en
medidas preventivas, como la implementación de DMARC para
combatir el phishing y prácticas sólidas de seguridad para
prevenir el malware y el ransomware, se puede fortalecer la
defensa contra estas amenazas. Es imperativo abordar de
manera inmediata cualquier ataque, respaldar archivos y
promover la conciencia de seguridad entre los usuarios. En
última instancia, la seguridad en estas plataformas depende
de una combinación de tecnologías avanzadas y buenas
prácticas por parte de los usuarios.

% \newpage

% \printbibliography

\end{document}