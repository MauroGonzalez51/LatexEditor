\documentclass[letterpaper, 12pt]{article}

\usepackage[utf8]{inputenc}
\usepackage[english, spanish]{babel}
\usepackage{fullpage}
\usepackage{graphicx}
\usepackage{amsmath}
\usepackage{enumitem}
\usepackage{chngcntr}
\usepackage{setspace}
\usepackage{url}
\usepackage{csquotes}
\usepackage{float}
\usepackage{verbatim}
\usepackage{tabularx}
\usepackage{amsmath}
\usepackage{caption}
\usepackage{bm}
\usepackage{wrapfig}
\usepackage{siunitx}

\counterwithin{figure}{section}
\renewcommand{\thesection}{\arabic{section}}
\renewcommand{\thesubsection}{\thesection.\arabic{subsection}}
\renewcommand{\baselinestretch}{2}
\renewcommand{\thefigure}{\arabic{figure}}

\usepackage[style=numeric, maxnames=6, minnames=3, backend=biber, parentracker=true, sorting=none]{biblatex}
\DefineBibliographyStrings{english}{%chktex-file 1 chktex-file 6
      andothers = {\em et\addabbrvspace al\adddot}
}

\addbibresource{./Bibliography/bibliography.bib}

\usepackage{array}
\usepackage{enumitem}

\setlength{\parskip}{\baselineskip}

\newcommand{\bolditalic}[1]{\textbf{\textit{#1}}}

\DeclareSIUnit{\COP}{COP}
\newcommand{\cop}[1]{\$\SI{#1}{\COP}}

\DeclareSIUnit{\DOLLAR}{USD}
\newcommand{\dollar}[1]{\$\SI{#1}{\DOLLAR}}

\renewcommand{\comment}[1]{{\small $\ll$#1$\gg$}}

% chktex-file 24

\newcommand{\liderValentina}{Valentina Del Rio Jimenez~}
\newcommand{\memberMaria}{María Idárraga Meléndez~}
\newcommand{\memberKeys}{Keylin Olea Gamarra~}
\newcommand{\memberNicolas}{Nicolás Caraballo Fruto~}

\begin{document}

\section*{Explorando Horizontes educativos: Visiones en el foro}

\noindent\makebox[\linewidth]{\rule{\textwidth}{0.4pt}}

\begin{itemize}[label=$\diamond$]
      \item \liderValentina\textit{(T00081360)}
      \item \memberMaria\textit{(T00081727)}
      \item \memberKeys\textit{(T00079396)}
      \item \memberNicolas\textit{(T00080279)}
\end{itemize}

\noindent\makebox[\linewidth]{\rule{\textwidth}{0.4pt}}

\nocite{reference_book}
\nocite{capacitar_trabajadores}
\nocite{Soledad}
\nocite{Parra_2022}
\nocite{Unir_2023}

% ! Introducción [Escrito]

El futuro de la educación y la automatización nos invitan a
reflexionar sobre el impacto de los robots educativos, la
presencia de androides en nuestros hogares, la inmersión en
la realidad virtual y los cambios radicales en los métodos
de enseñanza. A través de las voces de nuestra comunidad,
exploraremos perspectivas diversas sobre estos temas que
moldearán la forma en que abordamos la educación en los
años venideros.

% chktex-file 38

El capítulo 7, titulado ``¡Edúquese quien pueda!'' del
libro ``¡Sálvese quien pueda!'' de Andrés
Oppenheimer~\cite{reference_book}, aborda el futuro de la
educación y el papel de los docentes en la era de la
automatización. Algunos puntos clave en dicho tema son:

\begin{itemize}[label=$\triangleright$]
      \item Robots Educativos: Los robots educativos son herramientas
            que se utilizan en el aula para que los alumnos aprendan de
            manera lúdica y adquieran conocimientos relacionados con
            las matemáticas, tecnologías, ciencias o ingeniería. Aunque
            los robots pueden enseñar conocimientos complejos, no
            pueden igualar a los maestros humanos en la formación de
            personas con principios morales y el sentido de propósito
            para mejorar el mundo.

      \item Robots en casa: Se espera que en menos de cinco años todos
            tengamos un robot en casa. Estos robots domésticos pueden
            realizar diversas tareas, desde limpieza hasta asistencia
            personal. Sin embargo, su implementación plantea desafíos y
            preguntas sobre la interacción humano-robot y el impacto en
            nuestra vida diaria.

      \item Realidad Virtual: La realidad virtual puede revolucionar la
            enseñanza al permitir experiencias inmersivas que
            transportan a los estudiantes a diferentes lugares y
            tiempos. Sin embargo, su uso también puede implicar
            desafíos como desórdenes psicológicos y neurológicos, la
            adicción, el aislamiento social, las memorias falsas, los
            problemas de visión y la baja en el rendimiento.

      \item Cambios en la Educación: Las ``clases al revés'' son una
            alternativa a las clases tradicionales que permiten que
            cada estudiante desarrolle lo mejor de sí mismo. En este
            modelo, los estudiantes preparan el material antes de la
            clase y luego usan el tiempo de clase para discutir,
            resolver problemas y profundizar en los conceptos con la
            guía del profesor.

      \item Prioridades de los Educadores: En la era de la
            automatización, los educadores deben ayudar a los niños a
            encontrar su pasión, fomentar la curiosidad y enseñar la
            perseverancia. Esto implica un cambio en el enfoque
            educativo hacia habilidades que no pueden ser fácilmente
            automatizadas, como la creatividad, el pensamiento crítico
            y las habilidades sociales.
\end{itemize}

% ! Opiniones [Foro]

\section*{Opiniones en el foro}

A continuación, se presentan las opiniones expresadas por
los miembros de nuestra foro en respuesta a los temas
abordados en el capítulo 7.

\begin{itemize}[label=$\triangleright$]
      \item \memberNicolas (Aporte):

            Adaptación y Transformación: Profesiones y Educación en la
            Cuarta Revolución Industrial.

            Buenas tardes a todos,

            Es un honor estar aquí para discutir dos temas críticos y
            estrechamente interconectados: el futuro de las profesiones
            en la cuarta revolución industrial y el futuro de la
            educación en la era de la virtualidad. La cuarta revolución
            industrial, impulsada por avances en tecnologías como la
            inteligencia artificial, la automatización, la realidad
            virtual y aumentada, está remodelando radicalmente la forma
            en que trabajamos y aprendemos. Es fundamental que
            entendamos y nos adaptemos a estos cambios para prosperar
            en un mundo en constante evolución.

            Comencemos por hablar sobre el futuro de las profesiones.
            La automatización y la inteligencia artificial están
            transformando la fuerza laboral en muchas industrias.
            Algunos trabajos se volverán obsoletos, pero surgirán
            nuevos roles que requieren habilidades diferentes. La
            adaptación será clave para garantizar la relevancia en el
            mercado laboral. Esto significa que debemos aprender a
            aprender a lo largo de toda la vida. La educación continua
            y la flexibilidad serán esenciales. Además, la colaboración
            entre humanos y máquinas se convertirá en la norma, lo que
            nos lleva a una reevaluación de las habilidades
            interpersonales y creativas.

            Ahora, hablemos del futuro de la educación. La virtualidad
            y la tecnología han permitido un acceso sin precedentes a
            la educación. La educación en línea, los cursos masivos
            abiertos en línea (MOOC) y las aulas virtuales están
            cambiando la forma en que aprendemos. Sin embargo, debemos
            considerar cómo adaptar la pedagogía y la evaluación para
            satisfacer las necesidades de los estudiantes en este
            entorno virtual. La personalización y la flexibilidad son
            esenciales para asegurarnos de que cada individuo tenga la
            oportunidad de desarrollar sus habilidades y conocimientos.

            La combinación de estos dos temas, el futuro de las
            profesiones y la educación en la era de la virtualidad, es
            un recordatorio de que debemos ser proactivos en la
            creación de un futuro sostenible para nosotros y las
            generaciones futuras. La educación es la clave para
            capacitar a las personas para las profesiones del futuro, y
            la tecnología es una herramienta poderosa en este proceso.

            En resumen, la cuarta revolución industrial nos desafía a
            ser ágiles, a aprender constantemente y a abrazar la
            tecnología como un habilitador de cambio. La educación y la
            formación deberán adaptarse para mantenerse al día con las
            demandas del mercado laboral en constante evolución.
            Juntos, podemos dar forma a un futuro en el que las
            profesiones y la educación evolucionen de manera sinérgica
            para el beneficio de todos.

            Gracias por su atención. Espero con interés sus preguntas y
            comentarios sobre estos temas tan importantes.

      \item \memberKeys (Aporte):
            ``Edúquese quien pueda'' es un análisis acertado de las tendencias actuales en la educación. El mundo está cambiando a un ritmo acelerado, y la educación debe adaptarse a estos cambios. El sistema educativo tradicional está basado en un modelo de transmisión de conocimientos que ya no es adecuado para el mundo actual.

            El futuro de la educación debe centrarse en el desarrollo
            de las habilidades de los estudiantes, no en la transmisión
            de conocimientos. Los estudiantes deben aprender a aprender
            por sí mismos, a pensar críticamente y a resolver
            problemas. Los profesores deben ser facilitadores que
            ayuden a los estudiantes a desarrollar estas habilidades.

            Es importante que todos los estudiantes, independientemente
            de su origen social o económico, tengan acceso a una
            educación de calidad. El sistema educativo debe ser
            equitativo y accesible para todos.

            El capítulo ``Edúquese quien pueda'' es un análisis
            importante que ayuda a comprender los desafíos y
            oportunidades que enfrenta la educación en el mundo actual.

            \begin{itemize}[label=$\diamond$]
                  \item \liderValentina (Comentario de un aporte):

                        Independientemente del estrato social de un individuo, se
                        postula la premisa de que todas las personas deberían
                        contar con igualdad de capacidades para su desarrollo. En
                        este contexto, es imperativo analizar los requisitos
                        inherentes a la persona que, a su vez, plantean desafíos
                        significativos para el individuo. ¿Cuáles son estos
                        requisitos fundamentales y, más aún, cuáles podrían ser los
                        desafíos que se derivan de los mismos?

                        \begin{itemize}
                              \item \memberNicolas (Respuesta):
                                    Los requisitos fundamentales para el desarrollo
                                    de todas las personas incluyen acceso a una
                                    educación de calidad, atención médica adecuada,
                                    oportunidades de empleo, y un entorno social que
                                    fomente la igualdad. Los desafíos asociados a
                                    estos requisitos pueden ser la falta de recursos
                                    económicos, discriminación, barreras educativas
                                    o de salud, y desigualdades estructurales que
                                    dificulten el acceso a oportunidades. Abordar
                                    estos desafíos es esencial para lograr una
                                    verdadera igualdad de capacidades y
                                    oportunidades para todos.

                              \item \memberKeys (Respuesta):
                                    Los requisitos fundamentales para el desarrollo
                                    de la persona son la satisfacción de sus
                                    necesidades básicas materiales y sociales.
                                    La falta de satisfacción de estos requisitos
                                    puede generar diversos desafíos individuales
                                    y sociales, como la pobreza, la discriminación,
                                    la violencia, la desigualdad, la injusticia y
                                    la falta de solidaridad.
                        \end{itemize}
            \end{itemize}

      \item \memberMaria (Aporte):
            En el capítulo 7 del libro ``¡Sálvese quien pueda!'' de Andrés
            Oppenheimer, el autor habla sobre la educación y cómo la
            tecnología está cambiando la forma en que aprendemos.

            Oppenheimer menciona que los robots pueden ser una
            herramienta útil para la educación, pero también señala que
            no pueden reemplazar completamente a los maestros humanos.

            Además, el autor discute cómo la realidad virtual puede ser
            una herramienta poderosa para el aprendizaje, pero también
            puede tener algunos efectos secundarios negativos.

            Oppenheimer también menciona que la educación debe ser más
            personalizada y adaptada a las necesidades individuales de
            los estudiantes.

            El autor argumenta que la educación debe centrarse en
            enseñar habilidades prácticas y relevantes para el mundo
            real, en lugar de simplemente enseñar teoría. En general,
            creo que el capítulo es muy informativo y ofrece una visión
            interesante sobre el futuro de la educación.

      \item \memberMaria (Aporte):

            Del mismo modo se puede ver cómo en este capítulo se aborda
            el impacto de la automatización y la inteligencia
            artificial en el sector educativo. El autor plantea que, a
            medida que las máquinas se vuelven más capaces de realizar
            tareas que antes eran realizadas por humanos, la educación
            se volverá cada vez más importante para el éxito en el
            mundo laboral.

            Oppenheimer comienza el capítulo señalando que la educación
            tradicional, basada en la transmisión de conocimientos,
            está cada vez más obsoleta. Los robots y la inteligencia
            artificial pueden acceder y procesar información mucho más
            rápido que los humanos, y pueden hacerlo con mayor
            precisión. Esto significa que la educación del futuro se
            centrará menos en la memorización de datos y más en el
            desarrollo de habilidades cognitivas, como el pensamiento
            crítico, la resolución de problemas y la creatividad.

            El autor sostiene que los maestros, en el futuro, deberán
            desempeñar un papel más de mentores y facilitadores del
            aprendizaje. Su función será ayudar a los estudiantes a
            desarrollar las habilidades necesarias para tener éxito en
            el mundo laboral automatizado.

            Oppenheimer también explora el impacto de la tecnología en
            la educación. Señala que la realidad virtual, la realidad
            aumentada y los cursos en línea están revolucionando la
            forma en que los estudiantes aprenden. Estas tecnologías
            permiten a los estudiantes aprender de una manera más
            interactiva y personalizada.

            El autor concluye el capítulo señalando que la educación es
            el arma más poderosa que las personas tienen para enfrentar
            los desafíos del futuro. A medida que el mundo se vuelve
            cada vez más automatizado, la educación será más importante
            que nunca para el éxito en el mundo laboral.

            En mi opinión, el capítulo 7 de ``Sálvese quien pueda'' es
            un análisis perspicaz del impacto de la automatización y la
            inteligencia artificial en la educación. El autor plantea
            argumentos convincentes sobre la importancia de la
            educación en el mundo laboral del futuro.

            Específicamente, me parece interesante la idea de que los
            maestros, en el futuro, deberán desempeñar un papel más de
            mentores y facilitadores del aprendizaje. Esta es una
            tendencia que ya se está observando en algunos países, y
            que creo que es la dirección correcta. Los maestros deben
            ayudar a los estudiantes a desarrollar las habilidades
            necesarias para tener éxito en el mundo laboral, y no
            simplemente transmitirles conocimientos.

            También creo que el autor tiene razón al señalar que la
            tecnología está revolucionando la educación. Las nuevas
            tecnologías ofrecen oportunidades de aprendizaje más
            interactivas y personalizadas, y creo que estas
            oportunidades serán cada vez más importantes en el futuro.

            En conclusión, creo que el capítulo 7 de ``Sálvese quien
            pueda'' es un capítulo importante para cualquiera que esté
            interesado en el futuro de la educación. El autor ofrece un
            análisis perspicaz de las tendencias que están
            transformando la educación, y ofrece sugerencias sobre cómo
            prepararnos para el futuro.
\end{itemize}

\section*{Desarrollo}

La incorporación de herramientas como la inteligencia
artificial en diversos ámbitos se consolida como una
realidad innegable. Paralelamente, es comprensible que un
considerable número de individuos manifieste inquietud
respecto al impacto que estas innovaciones tendrán en sus
empleos. Interrogantes como ``Quiero trabajar en este
campo, pero \dots'' o ``Me atrae esta carrera, sin embargo
\dots'' son expresiones palpables de una preocupación que
permea en la sociedad. Lamentablemente, este escenario
incierto confronta a muchas personas con el cuestionamiento
de la relevancia de invertir en educación y formación para
roles que podrían eventualmente ser reemplazados en el
futuro.

La falta de certeza acerca de los eventos futuros es un
conocimiento universalmente aceptado. No obstante, esta
noción conlleva, desde otra perspectiva, la necesidad
imperativa de prepararse para cualquier escenario que pueda
materializarse. En concordancia con la observación de
\memberNicolas en uno de sus aportes, donde destaca que
``Algunos trabajos se volverán obsoletos, pero surgirán
nuevos roles que requieren habilidades diferentes'', se
destaca la importancia de la adaptación como una habilidad
fundamental para asegurar la pertinencia en el dinámico
mercado laboral actual. La noción de ``adaptación'' emerge
como un elemento cardinal en este contexto.

Desde un enfoque histórico, el ser humano ha experimentado
transformaciones profundas en su entorno, desde cambios
socioculturales hasta revoluciones industriales. En cada
una de estas etapas, la imperativa necesidad de ajustarse y
evolucionar ha sido evidente como un componente esencial
para la supervivencia. La actualidad no escapa a esta
dinámica evolutiva, y es esencial reconocer que la historia
nos revela que la adaptabilidad ha sido una constante en
nuestra naturaleza.

Siguiendo la filosofía de Charles Darwin~\cite{Darwin} y su
teoría de la evolución, se postula que no es la especie más
fuerte ni la más inteligente la que prevalece, sino aquella
que mejor se adapta a los cambios. En este sentido, la
innata capacidad del ser humano para ajustarse a nuevas
circunstancias ha sido un motor impulsor de su propia
evolución. La inteligencia artificial y la automatización
se perfilan como la última frontera de los desafíos que
enfrenta la humanidad, y, una vez más, la adaptación surge
como el recurso central para asegurar la relevancia en un
mundo laboral en constante mutación.

En un futuro que se presenta no tan lejano como podría
parecer, es innegable que nuestra sociedad experimentará
cambios significativos. Un ejemplo ilustrativo de esta
transformación se encuentra en la introducción de los
``Robots repartidores'', una iniciativa que varias
compañías pusieron a prueba a partir del año 2022 o incluso
antes.

Tomemos, por ejemplo, a una de estas empresas pioneras en
la implementación de esta tecnología innovadora,
Amazon~\cite{Erard_2022}. Durante el pasado año, en plena
época de pandemia, la compañía lanzó al mercado pequeños
robots diseñados para permitir a los usuarios recibir
paquetes o compras del supermercado directamente en la
puerta de sus hogares. Inicialmente, esta propuesta parecía
ser la solución perfecta para ahorrar tiempo y aumentar la
eficiencia, promoviendo la no interrupción de la
productividad diaria. Sin embargo, la realidad demostró ser
diferente, ya que poco tiempo después, estos robos fueron
retirados, alegando que ``había aspectos del programa que
no satisfacían las necesidades de los clientes'' y
señalando que los robots no eran completamente autónomos.

Este episodio ilustra que, a pesar de los esfuerzos de las
empresas líderes por introducir cambios significativos en
nuestra rutina diaria, se evidencia que aún se necesita
tiempo para que estas innovaciones se integren plenamente.
La retirada de los robots repartidores de Amazon subraya la
importancia de considerar no solo la eficiencia
tecnológica, sino también las expectativas y necesidades de
los usuarios. En ese sentido, no se pueden forzar cambios
en el dia a dia de las personas por diminutos que sean sin
que tome tiempo.

Otro aspecto fundamental es el ámbito educativo. La
inteligencia artificial presenta herramientas que obsoletan
el sistema educativo tal como lo conocemos en la
actualidad. La educación~\cite{educacion_evolución} ha
experimentado cambios notorios desde sus inicios,
implementando metodologías de aprendizaje progresivas, como
la inversión de roles, tal como mencionaron \memberMaria y
\memberKeys en sus aportes. Sin embargo, llega un punto en
el que estas metodologías resultan insuficientes.

La noción actual de ``educación'' deberá evolucionar para
adaptarse al constante crecimiento de la inteligencia
artificial. Aspectos como el plagio, la capacidad de
responder preguntas complejas mediante un simple
\textit{Prompt} en un solucionador, o la resolución de
problemas matemáticos sin comprender su naturaleza, son
desafíos que deben considerarse al impartir clases.

A pesar de las preocupaciones, no todo es desfavorable en
este contexto. El uso de estas herramientas puede ser
aprovechado para beneficio educativo. La inteligencia
artificial ofrece la oportunidad de desarrollar nuevas
estrategias de aprendizaje, proporcionar trabajos en clase
dirigidos por la IA, generar exámenes de manera
automatizada y llevar a cabo un seguimiento detallado del
progreso de cada estudiante para crear planes de
aprendizaje personalizados, incluso para el curso en sí.
Las posibilidades en este ámbito son virtualmente
ilimitadas, siempre y cuando se utilicen de manera ética y
efectiva.

Siguiendo esta línea de pensamiento, es crucial tener en
cuenta que muchas de las innovaciones actuales están
vinculadas a las inteligencias artificiales. Por lo tanto,
se deben considerar diversas estrategias y desafíos
correspondientes, tales como:

\begin{itemize}[label=$\triangleright$]
      \item Desafíos del Cambio Tecnológico en la Educación: Reconocer
            la necesidad de adaptar el sistema educativo a las rápidas
            evoluciones de la inteligencia artificial es esencial. La
            obsolescencia de métodos educativos tradicionales podría
            generar resistencia al cambio, pero también es una
            oportunidad para mejorar la efectividad del aprendizaje.

      \item Aprovechamiento Ético de las Herramientas de
            IA:\@{}Destacar la importancia de utilizar la inteligencia
            artificial de manera ética, asegurando que su
            implementación no conduzca a la simplificación excesiva o
            la pérdida de calidad en la educación. Promover la
            integridad académica y la responsabilidad ética entre los
            estudiantes al utilizar herramientas de IA\@{}.

      \item Fomentar la Colaboración Humano-IA:\@{}Resaltar la
            importancia de una colaboración armoniosa entre humanos y
            la inteligencia artificial en el proceso educativo.
            Destacar que la tecnología debe ser una herramienta
            complementaria para potenciar la enseñanza y no reemplazar
            por completo la interacción humana.

      \item Desarrollo de Habilidades para el Futuro: Considerar cómo
            la educación puede enfocarse en el desarrollo de
            habilidades que las máquinas no pueden replicar fácilmente,
            como el pensamiento crítico, la creatividad y las
            habilidades sociales. Fomentar la formación en habilidades
            digitales y la comprensión profunda de la tecnología para
            que los estudiantes estén preparados para el futuro.

      \item Necesidad de Actualización Continua: Reconocer que la
            rápida evolución de la inteligencia artificial requerirá
            una constante actualización de los educadores y del sistema
            educativo en general. Destacar la importancia de programas
            de desarrollo profesional y la integración de expertos en
            tecnología en la planificación curricular.
\end{itemize}

En resumen, el impacto de la inteligencia artificial en la
educación es significativo y, aunque presenta desafíos,
también ofrece oportunidades para mejorar y personalizar el
proceso educativo. La clave está en abordar estos cambios
con una perspectiva ética y centrada en el desarrollo
integral de los estudiantes.

\section*{Conclusiones}

El foro proporcionó una perspectiva rica y diversa sobre el
impacto de la inteligencia artificial (IA) en la educación.
Los participantes reflexionaron sobre la necesidad de
adaptación en la cuarta revolución industrial, destacando
la importancia de aprender continuamente y desarrollar
habilidades interpersonales y creativas en un entorno donde
la automatización transforma la fuerza laboral. Además, se
enfatizó la obsolescencia del sistema educativo
tradicional, sugiriendo que el foco debe desplazarse hacia
el desarrollo de habilidades en lugar de la simple
adquisición de conocimientos.

Las opiniones críticas sobre el capítulo 7 del libro
``¡Sálvese quien pueda!'' de Andrés Oppenheimer aportaron a
la discusión, subrayando la utilidad de los robots
educativos, la presencia de la IA en la educación y los
desafíos de la realidad virtual. Se resaltó la necesidad de
una educación personalizada, adaptada a las necesidades
individuales, y se hizo hincapié en la relevancia de
enseñar habilidades prácticas y pertinentes para el mundo
real.

En el análisis de la evolución de la educación, inspirado
por los participantes, se reconoció la necesidad de ajustar
las metodologías de enseñanza para hacer frente a los
desafíos planteados por la IA.\@{}La adaptación se presentó
como un recurso esencial para mantener la pertinencia en el
mercado laboral y garantizar la efectividad del proceso
educativo.

La ética también emergió como un tema central en el foro.
Se destacó la importancia de un uso ético de las
herramientas de IA, asegurando la integridad académica y la
responsabilidad ética entre los estudiantes. Además, se
subrayó la necesidad de un desarrollo integral de los
estudiantes, incluyendo habilidades digitales y
competencias sociales.

En resumen, el foro delineó un panorama completo sobre la
intersección entre la inteligencia artificial y la
educación. La adaptación constante, la ética y el enfoque
en el desarrollo integral fueron identificados como los
pilares fundamentales para abordar los desafíos y
aprovechar las oportunidades que presenta la inteligencia
artificial en el ámbito educativo.

\newpage

\printbibliography

\end{document}