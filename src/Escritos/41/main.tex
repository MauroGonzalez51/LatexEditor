\documentclass[letterpaper, 11pt]{report}

\usepackage[utf8]{inputenc}
\usepackage[english, spanish]{babel}

\usepackage{fullpage}
\usepackage{graphicx}
\usepackage{amsmath}
\usepackage{enumitem}
\usepackage{chngcntr}
\usepackage{setspace}
\usepackage{xurl}
\usepackage{csquotes}
\usepackage{float}
\usepackage{verbatim}
\usepackage{tabularx}
\usepackage{amsmath}
\usepackage{caption}
\usepackage{bm}
\usepackage{wrapfig}
\usepackage{siunitx}
\usepackage{array}
\usepackage{longtable}



% \usepackage[margin=0.5in]{geometry}

\counterwithin{figure}{section}
\renewcommand{\thesection}{\arabic{section}}
\renewcommand{\thesubsection}{\thesection.\arabic{subsection}}
\renewcommand{\baselinestretch}{1.5}
\renewcommand{\thefigure}{\arabic{figure}}

\usepackage[style=apa, maxnames=6, minnames=3]{biblatex}
\DefineBibliographyStrings{english}{%chktex-file 1 chktex-file 6
      andothers = {\em et\addabbrvspace al\adddot}
}

\addbibresource{./Bibliography/bibliography.bib}


\setlength{\parskip}{\baselineskip}

\newcommand{\bolditalic}[1]{\textbf{\textit{#1}}}

\renewcommand{\comment}[1]{{\small $\ll$#1$\gg$}}

\renewcommand{\arraystretch}{1.3}
\setlength{\LTpre}{0pt}
\setlength{\LTpost}{0pt}

% chktex-file 8
% chktex-file 24
% chktex-file 44

\begin{document}

\begin{titlepage}
      \centering
      \includegraphics[width=0.3\textwidth]{../../../images/logo_utb.png}\par\vspace{1cm}
      {\scshape\LARGE Universidad Tecnológica de Bolívar \par}
      \vspace{1cm}

      {\scshape\Large Seminario de investigación \par}
      \vspace{2cm}

      \slshape {\Large \bfseries{} Análisis de una investigación cuantitativa: objetivos, preguntas y justificación  \\}
      \vspace{3cm}

      \slshape {\itshape{} Valeria Gómez \\}
      \slshape {\itshape{} Samuel Cárdenas \\}

      \vfill
      Revisado Por \\
      Katty Johanna Gómez \\
      {\large \today\par}
\end{titlepage}

\nocite{*}

\section*{Análisis de las tendencias de los perfiles de egresados de los programas de Administración de Empresas de la ciudad de Cartagena}

El objetivo de este articulo consiste en analizar e interpretar las tendencias
de los egresados de los programas de Administración de Empresas, los cuales se
han sometido bajo vigilancia tecnológica para establecer si cuentan con las
capacidades para afrontar el entorno de la ciudad de Cartagena y comprender sus
necesidades para la apuesta a la productividad de la ciudad.

Así mismo, busca distinguir los programas de Administración de Empresas a
través de variables, como el número de créditos, numero de semestres, modalidad
de formación y valor académico; mientras identifica las tendencias de
profundización en los perfiles de egresados y cuestionar las tendencias
estandarizadas de las apuestas de producción en la región junto a las
necesidades de competitividad.

\subsection*{Preguntas deducidas para la realización de la investigación}

\begin{itemize}
      \item ¿Cómo se describen los programas de Administración de Empresas que se ofrecen en Cartagena en cuanto a créditos, duración, modalidades, costos y ciclos propedéuticos?

      \item ¿Qué tendencias se pueden observar en los perfiles de los graduados de estos programas académicos?

      \item ¿Están estos programas académicos en sintonía con las necesidades y oportunidades productivas de Cartagena y Bolívar?

\end{itemize}

De esta misma manera, el artículo se justifica como una investigación esencial
porque los programas de Administración de Empresas en Cartagena necesitan
adaptarse a las dinámicas del entorno social, económico y productivo de la
región. Esto es especialmente importante en un contexto donde la competitividad
está en aumento y hay una transformación hacia la innovación y la economía
digital. Analizar los perfiles de los egresados es clave para ver si la
formación profesional que se ofrece realmente se alinea con las necesidades del
mercado laboral y las oportunidades productivas locales. Además, incluir la
vigilancia tecnológica como una herramienta metodológica ayuda a identificar
tendencias, anticipar cambios y fortalecer la relevancia de los programas
académicos. Esto no solo facilita la actualización de los planes de estudio,
sino que también mejora la empleabilidad de los egresados y potencia la
contribución de las universidades al desarrollo regional.

\printbibliography

\end{document}