\documentclass[letterpaper, 12pt]{article}

\usepackage[utf8]{inputenc}
\usepackage[english, spanish]{babel}

\usepackage{newtxtext}

\usepackage{fullpage}
\usepackage{graphicx}
\usepackage{amsmath}
\usepackage{enumitem}
\usepackage{chngcntr}
\usepackage{setspace}
\usepackage{xurl}
\usepackage{csquotes}
\usepackage{float}
\usepackage{verbatim}
\usepackage{tabularx}
\usepackage{amsmath}
\usepackage{caption}
\usepackage{bm}
\usepackage{wrapfig}
\usepackage{siunitx}

\counterwithin{figure}{section}
\renewcommand{\thesection}{\arabic{section}}
\renewcommand{\thesubsection}{\thesection.\arabic{subsection}}
\renewcommand{\baselinestretch}{2}
\renewcommand{\thefigure}{\arabic{figure}}

\usepackage[style=apa, maxnames=6, minnames=3]{biblatex}
\DefineBibliographyStrings{english}{%chktex-file 1 chktex-file 6
      andothers = {\em et\addabbrvspace al\adddot}
}

% \addbibresource{./Bibliography/bibliography.bib}

\usepackage{array}
\usepackage{enumitem}

\setlength{\parskip}{\baselineskip}

\newcommand{\bolditalic}[1]{\textbf{\textit{#1}}}

\DeclareSIUnit{\COP}{COP}
\newcommand{\cop}[1]{\$\SI{#1}{\COP}}

\DeclareSIUnit{\DOLLAR}{USD}
\newcommand{\dollar}[1]{\$\SI{#1}{\DOLLAR}}

\renewcommand{\comment}[1]{{\small $\ll$#1$\gg$}}

% chktex-file 24

\begin{document}

\section*{Discurso de grado 2023}

Estimados miembros de la facultad, distinguidos invitados,
padres, y, por supuesto, a los graduados de [Nombre],

Quisiera compartirles unas palabras y, a su vez, una
pequeña reflexión \dots

Desde tiempos inmemorables, el ser humano ha buscado
cumplir con el propósito fundamental de vivir para dejar
descendencia. Sin embargo, a medida que las eras han
evolucionado, hemos sido testigos de un cambio drástico en
la narrativa de la vida.

Hoy en día, el camino que trazamos desde el momento de
nuestro nacimiento hasta la madurez está marcado por una
serie de etapas predefinidas, y desafortunadamente, muchas
veces no somos nosotros mismos quienes las decidimos.

Esta lista de responsabilidades que el hombre contemporáneo
posee parece interminable y, en muchos sentidos,
inevitable. Nos cargamos de responsabilidades por el simple
hecho de existir en este mundo.

Este énfasis en las fases predefinidas de la vida puede, en
ocasiones, parecer abrumador. Sin embargo, en medio de esta
complejidad, emerge una verdad innegable: el poder de la
educación para proporcionar no solo conocimientos, sino
también las herramientas esenciales para navegar por este
intrincado laberinto de responsabilidades.

En los pasillos de [Nombre], cada uno de los estudiantes
aquí presente emprendió un viaje educativo hace muchos
años, sin saber qué le depararía el futuro. He aquí hoy,
desafiando las convenciones y superando las expectativas.
Han adquirido no solamente conocimientos académicos, sino
también habilidades críticas, la capacidad de adaptarse y
la fortaleza para enfrentar los desafíos con valentía.

Cada uno de ustedes puede optar por tomar un camino
diferente. Algunos pueden estar pensando en continuar sus
estudios en alguna universidad, quizás incluso con la
compañía de amigos. Otros, por su parte, podrían sentir la
tristeza que acompaña a la distancia que inevitablemente
comenzará a emerger en ciertas amistades. Esta transición
marca el paso a una nueva etapa, la vida adulta.

La vida adulta, con sus responsabilidades y desafíos, a
menudo trae consigo cambios en nuestras relaciones. Puede
que algunos vínculos se fortalezcan con la prueba del
tiempo, mientras que otros se transformen en formas que no
anticipamos. Este proceso de cambio, aunque a veces
desafiante, es inherentemente parte de la experiencia
humana.

En este punto, al mirar hacia el futuro, es fundamental
reconocer que cada desafío que se avecina es una
oportunidad de crecimiento. A medida que ingresan en esta
nueva fase, no solo están cerrando un capítulo, sino
abriendo las páginas de un libro lleno de posibilidades.
Cada elección, cada obstáculo superado, contribuirá a la
construcción de la narrativa única de sus vidas.

No podemos prever todos los giros y vueltas que el destino
les tiene reservados, pero sí podemos estar seguros de que
la educación que han recibido en [Nombre] les ha
proporcionado las herramientas necesarias para enfrentar lo
desconocido con valentía y resiliencia.

Algunos de ustedes pueden sentir la nostalgia del presente,
mientras otros están ansiosos por el futuro. Ambas
emociones son válidas y, de hecho, complementarias.
Recordemos que la travesía que emprendemos está marcada por
la dualidad de la añoranza y la anticipación.

En esta encrucijada de transición, les insto a abrazar cada
cambio, a aprender de cada experiencia y a apreciar la
belleza que reside en la incertidumbre. La vida adulta no
es solo un destino; es un viaje en constante evolución.

Ahora, mientras nos acercamos al cierre de este capítulo,
\comment{Aqui vendria la parte del homenaje}

\end{document}