\section{Conclusiones}

El análisis realizado permite concluir que el cambio climático representa una amenaza creciente para la estabilidad financiera de Cartagena, en particular en sectores como el turismo y las energías renovables. Los riesgos físicos generan pérdidas directas sobre la infraestructura y elevan los costos operativos, mientras que los riesgos de transición asociados a regulaciones ambientales incrementan la incertidumbre en la rentabilidad de las inversiones. 

Esta combinación ha limitado la atracción de capital privado, elevando las barreras de financiamiento en la ciudad.

En este escenario, las finanzas verdes ofrecen una vía para canalizar recursos hacia proyectos sostenibles; sin embargo, su impacto resulta insuficiente cuando la percepción de riesgo es elevada. El blended finance se configura, entonces, como una herramienta estratégica capaz de redistribuir riesgos entre actores públicos y privados, disminuir el costo promedio ponderado de capital (WACC) y mejorar la competitividad de los proyectos turísticos y energéticos en contextos de vulnerabilidad climática.

Los casos documentados en África, Asia y América Latina evidencian que este mecanismo puede movilizar inversiones significativas hacia sectores donde el riesgo percibido inhibe la participación del mercado. Aplicado a Cartagena, el blended finance tendría un alto potencial para viabilizar proyectos como la instalación de paneles solares en hoteles, la consolidación de la planta de hidrógeno verde de Ecopetrol y la construcción de infraestructuras de protección costera.

No obstante, también se identifican limitaciones relevantes: la dependencia de subsidios o capital concesional, la complejidad en la estructuración contractual y la falta de marcos regulatorios específicos en Colombia. Estos retos deben ser abordados mediante políticas públicas coherentes, incentivos fiscales y el fortalecimiento de capacidades técnicas locales.

En conclusión, el blended finance no solo ofrece una alternativa innovadora para enfrentar los riesgos financieros derivados del cambio climático, sino que también representa una oportunidad para posicionar a Cartagena como un referente regional en turismo sostenible y transición energética. Su adecuada implementación permitiría atraer capital privado, generar empleo verde y construir una economía más resiliente e inclusiva frente a los desafíos climáticos del siglo XXI.

