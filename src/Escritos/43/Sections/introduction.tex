\section{Introducción}

El cambio climático se ha convertido en uno de los principales factores de riesgo para la estabilidad financiera global, afectando especialmente a las economías locales que dependen de recursos naturales y actividades sensibles al clima. En este contexto, nuestra ciudad Cartagena de Indias enfrenta crecientes desafíos económicos y ambientales debido a su vulnerabilidad frente al aumento del nivel del mar, la erosión costera y la variabilidad climática, que amenazan tanto al sector turístico como al energético, dos pilares fundamentales de su desarrollo económico.

Diversos estudios y organismos internacionales, como la OCDE (2021) y el Banco Mundial (2021), han advertido que la exposición al riesgo climático incrementa los costos de inversión y reduce la capacidad de financiamiento de proyectos sostenibles. En Colombia, aunque se han implementado políticas de adaptación y mitigación, los instrumentos financieros tradicionales resultan insuficientes para cubrir la magnitud del problema, lo que exige la adopción de modelos innovadores de financiamiento sostenible.

En ese sentido, el enfoque del blended finance surge como una alternativa teórica y práctica relevante. Este modelo combina recursos públicos, privados y filantrópicos con el propósito de reducir los riesgos percibidos por los inversionistas y promover la financiación de proyectos con impacto ambiental positivo (OECD, 2018; Convergence, 2024). Desde la perspectiva de la economía verde, el blended finance permite canalizar inversiones hacia energías renovables, infraestructura resiliente y turismo sostenible, promoviendo así la competitividad y el desarrollo local en territorios altamente vulnerables al cambio climático.

La tesis central de esta investigación sostiene que la implementación de esquemas de financiamiento mixto en Cartagena puede contribuir a mitigar los riesgos financieros derivados del cambio climático, facilitando el acceso al capital y fortaleciendo la resiliencia económica de los sectores productivos locales.

Este estudio se justifica por la urgencia de encontrar mecanismos financieros innovadores que respondan a los retos ambientales actuales y a las exigencias internacionales de descarbonización. Además, busca generar evidencia académica que respalde la adopción de instrumentos híbridos en el contexto colombiano, con potencial para replicarse en otras ciudades costeras de América Latina.

En consecuencia, la discusión inicial plantea que el éxito de la transición hacia una economía sostenible en Cartagena dependerá de la capacidad de integrar las finanzas verdes, la cooperación internacional y el capital privado. Así, esta investigación contribuye al análisis académico y práctico del blended finance como herramienta clave para enfrentar los efectos económicos del cambio climático y avanzar hacia un modelo de desarrollo más resiliente e inclusivo.