\subsection{Metodología}

La presente investigación se desarrolló bajo un enfoque cualitativo, de tipo documental y exploratorio, orientado a analizar los riesgos financieros derivados del cambio climático en Cartagena y a evaluar el papel del blended finance como estrategia de financiamiento sostenible. La búsqueda de información se delimitó temporalmente al periodo comprendido entre 2014 y 2025, con el propósito de incluir los avances más recientes en materia de transición energética y financiamiento climático en Colombia, así como los estudios publicados a partir de la promulgación de la Ley 1715 de 2014, que regula el uso de energías renovables no convencionales.

El proceso metodológico se fundamentó en una revisión sistemática de literatura académica y técnica disponible en bases de datos de alto impacto como Scopus y Web of Science (WoS), complementada con informes de organismos multilaterales y financieros, entre ellos el Banco Mundial, la OCDE, el Banco Interamericano de Desarrollo (BID), Convergence, y la Climate Bonds Initiative. Para la selección de documentos se aplicaron tres criterios principales: la actualidad, considerando fuentes publicadas durante la última década (2014–2025); la pertinencia temática, centrada en finanzas verdes, blended finance, transición energética y turismo sostenible; y la rigurosidad metodológica, priorizando documentos respaldados por instituciones reconocidas o revistas científicas indexadas.

El análisis de la información se llevó a cabo mediante una interpretación cualitativa de los contenidos, con el fin de identificar patrones, tendencias y brechas en la literatura sobre la aplicación del blended finance en contextos urbanos y turísticos vulnerables al cambio climático. Los datos se sistematizaron en matrices temáticas que permitieron organizar la evidencia en torno a los principales riesgos financieros, las oportunidades de inversión sostenible y el potencial del blended finance para promover la resiliencia económica en Cartagena.