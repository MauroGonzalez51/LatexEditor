\section{Marco teórico}

\subsection{Cambio Climático y Riesgos Financieros}

El cambio climático representa una amenaza estructural para la estabilidad económica global, al generar impactos físicos, de transición y de responsabilidad que afectan tanto a los mercados financieros como a la planificación pública y privada (Banco Mundial, 2021). Los riesgos físicos incluyen eventos extremos como inundaciones, sequías y tormentas, que incrementan las pérdidas de activos y las interrupciones operativas; mientras que los riesgos de transición surgen de los cambios regulatorios y tecnológicos asociados a la descarbonización (Moody’s ESG Solutions, 2023).

En Colombia, las regiones costeras como Cartagena de Indias se encuentran particularmente expuestas a estos riesgos debido a su dependencia del turismo, la infraestructura portuaria y la energía. Según la CEPAL (2023), los costos económicos derivados del cambio climático podrían representar entre el 1,5\% y el 4\% del PIB nacional hacia 2030, si no se adoptan medidas de adaptación y mitigación efectivas. De acuerdo con la OCDE (2025), los riesgos financieros asociados al cambio climático pueden alterar la percepción de los inversionistas sobre la rentabilidad de los activos y aumentar el costo del capital, especialmente en proyectos sostenibles que requieren altos desembolsos iniciales. En este contexto, las políticas de financiamiento verde y los mecanismos de mitigación de riesgos resultan esenciales para garantizar la estabilidad financiera a largo plazo.

\subsection{Sector Turístico y Vulnerabilidad Climática}

El sector turístico constituye uno de los principales motores económicos de Cartagena, representando más del 20\% de su producto interno bruto local (Cámara de Comercio de Cartagena, 2023). Sin embargo, este sector es altamente vulnerable a los impactos del cambio climático, dado que depende de factores ambientales como la calidad de las playas, los ecosistemas costeros y la estabilidad de la infraestructura urbana (WTTC, s.f.). % TODO

Estudios de la Red de Conocimiento Climático (2023) evidencian que los incrementos en el nivel del mar, las temperaturas extremas y la erosión costera afectan la oferta hotelera, reducen la afluencia turística y generan pérdidas financieras significativas. Además, las variaciones climáticas obligan a las empresas turísticas a invertir en medidas de adaptación, incrementando los costos operativos y reduciendo la rentabilidad.

Desde una perspectiva financiera, la vulnerabilidad climática del turismo implica mayores exigencias de aseguramiento, costos de mantenimiento e incertidumbre sobre la demanda futura (Banco Mundial, 2021). Por ello, el acceso a financiamiento sostenible se convierte en un factor determinante para garantizar la continuidad del sector y su contribución al desarrollo local.

\subsection{Transición Energética y Energías Renovables}

La transición energética busca sustituir progresivamente las fuentes fósiles por energías limpias, con el fin de reducir las emisiones de gases de efecto invernadero (GEI) y cumplir los compromisos internacionales de descarbonización. En Colombia, la Ley 1715 de 2014 promueve la incorporación de energías renovables no convencionales, lo que ha permitido un incremento sostenido en la inversión en proyectos solares y eólicos (Departamento Administrativo de la Función Pública, 2014).

En el caso de Cartagena, la transición energética se ha materializado a través de proyectos de electrificación renovable liderados por empresas como Enel y Celsia, orientados a fortalecer la seguridad energética y la sostenibilidad ambiental (Reuters, 2025). Según Ecopetrol (2023), la inversión en hidrógeno verde y energía solar ha superado los USD 28,5 millones, consolidando a Colombia como un referente regional en innovación energética.

No obstante, el financiamiento de proyectos de energía limpia enfrenta obstáculos vinculados al alto costo del capital y a la percepción de riesgo tecnológico. En consecuencia, los instrumentos de finanzas verdes y esquemas mixtos de inversión han adquirido un papel estratégico para facilitar la transición energética y atraer capital privado (OECD, 2021; Climate Bonds Initiative, 2024).

\subsection{Blended Finance}

El blended finance (financiamiento mixto) es un enfoque innovador que combina recursos públicos, privados y filantrópicos con el fin de movilizar capital hacia proyectos de desarrollo sostenible. Según la OCDE (2018), este instrumento busca ``utilizar de forma estratégica los fondos públicos para catalizar inversión privada que, de otro modo, no se movilizaría''.

En América Latina, el blended finance se ha consolidado como una herramienta eficaz para financiar infraestructura sostenible, energías renovables y programas de desarrollo urbano (Convergence, 2024). Su estructura permite distribuir los riesgos financieros mediante instrumentos como el first-loss capital, las garantías parciales y los fondos de cobertura climática (OECD, 2025).

De acuerdo con Convergence (2023), Colombia ha emergido como un centro andino de innovación financiera en el uso del blended finance, impulsando proyectos de energía solar, eficiencia energética y desarrollo sostenible. Estas iniciativas no solo atraen capital extranjero, sino que también fortalecen la gobernanza financiera al integrar objetivos ambientales y sociales en la planificación económica.

En síntesis, el blended finance ofrece una vía práctica para reducir el riesgo percibido por los inversionistas, facilitar la participación del sector privado y acelerar la transición hacia una economía baja en carbono.