\section{Discusión}

Los hallazgos obtenidos confirman la estrecha relación entre el cambio climático y la estabilidad financiera de Cartagena. Tal como lo sugieren estudios del Banco Mundial (2022), la vulnerabilidad de las ciudades costeras incrementa el riesgo percibido por los inversionistas, lo cual eleva el costo de financiamiento y reduce la viabilidad de proyectos turísticos y energéticos. En el caso analizado, la evidencia local (pérdidas millonarias por inundaciones y presión regulatoria sobre el sector turístico) coincide con esta tendencia internacional.

En términos financieros, el blended finance se consolida como una herramienta para corregir la brecha entre riesgo percibido y rentabilidad esperada. El marco teórico mostró que este mecanismo actúa principalmente sobre el costo promedio ponderado de capital (WACC), al redistribuir los riesgos a través de instrumentos como garantías y first-loss capital. Los resultados evidenciaron que experiencias en África y América Latina lograron reducciones de entre 10\% y 30\% en el costo de capital, lo cual mejora significativamente la relación riesgo-retorno de proyectos verdes.

Aplicado a Cartagena, esta reducción tendría efectos directos en la competitividad del sector turístico y en la transición energética. Por ejemplo, un hotel en Bocagrande que enfrente una tasa de interés del 14\% para financiar paneles solares podría reducir este costo al 11\% con un esquema de blended finance (anexo 4). Aunque esta diferencia parece marginal, en un proyecto de USD 2 millones representa un ahorro financiero de aproximadamente USD 280.000 en intereses durante el periodo de repago, lo cual refuerza la viabilidad del negocio y aumenta el atractivo para inversionistas. % TODO: anexos

\noindent Caso estadístico hipotético.

\begin{itemize}
    \item WACC antes del blended finance: 14\%
    
    \item Reducción esperada (Popovic, 2024; convergence, 2023): 20\%
\end{itemize}

\noindent Nuevo WACC con blended finance

\begin{equation*}
    WACC_{\text{nuevo}} = 14\% \times (1 - 0.20) = 11.2\%
\end{equation*}

Por lo tanto, en el caso hipotético de un proyecto de USD 2 millones con un plazo de 5 años el ahorro se vería de la siguiente manera:

\begin{itemize}
    \item \textbf{Sin blended finance:}
    \[
        \text{Intereses} = \$2{,}000{,}000 \times 14\% \times 5 = \$1{,}400{,}000
    \]
    \item \textbf{Con blended finance:}
    \[
        \text{Intereses} = \$2{,}000{,}000 \times 11.2\% \times 5 = \$1{,}120{,}000
    \]
    \item \textbf{Ahorro financiero:}
    \[
        \text{Ahorro} = \$1{,}400{,}000 - \$1{,}120{,}000 = \$280{,}000
    \]
\end{itemize}

En conclusión, con el blended finance se logra un ahorra en intereses de USD $280.000$ un impacto bastante importante.

No obstante, la discusión también revela limitaciones relevantes. En primer lugar, la dependencia de subsidios o capital concesional plantea dudas sobre la sostenibilidad de los esquemas de blended finance en el largo plazo. En segundo lugar, la complejidad legal y contractual puede retrasar la ejecución de proyectos, generando costos adicionales. 

Finalmente, la falta de marcos regulatorios claros en Colombia para estructurar blended finance dificulta la confianza de los inversionistas internacionales.

A pesar de estas limitaciones, los beneficios potenciales superan los desafíos. Cartagena cuenta con condiciones estratégicas para convertirse en un laboratorio regional de blended finance, debido a su doble condición de destino turístico internacional y polo energético emergente. La implementación de proyectos piloto en energías renovables y protección costera permitiría validar el impacto de este mecanismo, generar aprendizajes institucionales y sentar las bases para un mercado financiero climático más robusto en el país.