\section{Objetivos de la investigación}

\subsection{Objetivo general}

Analizar los riesgos financieros derivados del cambio climático que afectan al sector turístico en Cartagena, Colombia, y evaluar el papel del blended finance como mecanismo estratégico para impulsar la transición hacia energías renovables en dicho sector. 

\subsection{Objetivos específicos}

\begin{itemize}
    \item Identificar los principales riesgos financieros asociados al cambio climático que afectan al sector turístico y a los proyectos de energías renovables en Cartagena.

    \item Examinar los mecanismos e instrumentos de las finanzas verdes, con especial énfasis en el blended finance, como herramientas para reducir la percepción de riesgo.
    
    \item Revisar experiencias internacionales y regionales de implementación de blended finance en proyectos de sostenibilidad y transición energética.
    
    \item Evaluar la aplicabilidad del blended finance al contexto de Cartagena, destacando su potencial para atraer capital privado e impulsar inversiones resilientes.
    
    \item Formular recomendaciones para inversionistas, instituciones financieras y entidades públicas orientadas a fortalecer el turismo y las energías limpias mediante esquemas de financiamiento innovadores.
\end{itemize}