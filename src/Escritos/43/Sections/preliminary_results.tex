\section{Resultados/ avances}

El análisis realizado permitió identificar que Cartagena enfrenta riesgos financieros significativos derivados del cambio climático, principalmente asociados a la vulnerabilidad del sector turístico y a los altos costos iniciales de las inversiones en energías renovables. Los hallazgos muestran que los eventos climáticos extremos, el aumento del nivel del mar y la degradación ambiental generan pérdidas económicas estimadas en más de USD 100 millones anuales para la economía local (Banco Mundial, 2021; Red de Conocimiento Climático, 2023).

En cuanto al acceso a financiamiento verde, se observó que la ciudad presenta una baja capacidad para atraer inversión privada, debido a la percepción de riesgo y a la limitada disponibilidad de mecanismos financieros especializados. Las entrevistas documentales y reportes analizados indican que las tasas de interés aplicadas a proyectos de infraestructura sostenible pueden superar el 11,2\% anual, lo que incrementa los costos de endeudamiento y reduce la rentabilidad esperada (OECD, 2021).

El análisis propio se centró en estimar el efecto de la aplicación del blended finance sobre el costo de capital de un proyecto solar tipo implementado en Cartagena. Suponiendo una inversión de USD 2 millones, la reducción de la tasa de interés del $11,2\%$ al $8\%$ gracias al financiamiento mixto representaría un ahorro de USD 64.000 anuales en intereses. En términos acumulados, esto equivaldría a una mejora del 15\% en la rentabilidad neta del proyecto durante los primeros cinco años de operación.

Asimismo, la revisión de experiencias regionales (Convergence, 2023; BID, 2022) permitió identificar que los proyectos apoyados por blended finance logran movilizar hasta tres veces más capital privado que los esquemas tradicionales. Este efecto multiplicador se explica por el uso de instrumentos de mitigación de riesgo, como el first-loss capital, las garantías parciales de crédito y los fondos de coinversión climática, los cuales aumentan la confianza de los inversionistas y facilitan la estructuración de proyectos sostenibles.

En el caso del sector turístico, los resultados evidencian una creciente necesidad de financiamiento para infraestructura resiliente, certificaciones ambientales y reconversión energética de los establecimientos hoteleros. Las proyecciones del WTTC (2024) estiman que una inversión promedio de COP 15.000 millones en adaptación climática podría incrementar los ingresos turísticos de la ciudad en un $12\%$ anual, gracias a la mejora de la competitividad y sostenibilidad del destino.

Finalmente, el análisis de datos comparativos demuestra que la adopción del blended finance en Cartagena podría reducir el costo del capital verde entre un $10\%$ y $30\%$, promoviendo la participación privada y aumentando la resiliencia económica de los sectores productivos más afectados por el cambio climático.

% chktex-file 44
\begin{table}[H]
    \small
    \setlength{\tabcolsep}{6pt}
    \renewcommand{\arraystretch}{1.2}
    \begin{tabularx}{\textwidth}{|p{0.32\textwidth}|p{0.36\textwidth}|p{0.25\textwidth}|}
        \hline
        \textbf{Dato} & \textbf{Fuente} & \textbf{Aporte} \\
        \hline
        Inversión en energía renovable en Colombia: más de COP \$9 billones en 2024, con al menos el 97\% destinados a energía solar fotovoltaica. &
        \href{https://www.portafolio.co/sostenibilidad/colombia-alcanzo-record-de-inversion-en-energia-renovable-con-mas-de-9-billones-de-pesos-en-2024-633474}{Portafolio, 2024} &
        Muestra escala del mercado renovable nacional; utilidad para estimar montos de proyectos grandes o interés del sector privado. \\
        \hline
        Costo de sistema solar residencial en Colombia: para un sistema pequeño (2-3 kWp) coste entre COP \$7-10 millones, para uno mediano (4-6 kWp) entre COP \$12-18 millones. Incluye componentes y puesta en marcha. &
        \href{https://www.opscolombia.com/blog/cuanto-cuesta-realmente-un-sistema-solar-en-colombia}{OPS Colombia, 2024} &
        Puedes usar esto para estimar inversión inicial en un hotel pequeño o medio en Cartagena, para luego proyectar ahorro y coste de capital. \\
        \hline
        Proyecto solar de DIAN Cartagena: instalación de 271 paneles solares (160 kW AC) + planta solar de Aduanas (250 kW AC), generando ahorro anual cercano a COP \$720 millones. &
        \href{https://www.portafolio.co/energia/de-impuestos-a-sostenibilidad-asi-avanza-la-dian-con-solar-en-cartagena-639503}{Portafolio, 2024} &
        Buen caso local que ya tiene magnitud (160 kW, 250 kW) y ahorro estimado; sirve para comparar retorno o impacto estimado. \\
        \hline
    \end{tabularx}
    \caption{Datos relevantes sobre inversión y proyectos solares en Colombia}
\end{table}