\section{Resumen}

Esta investigación tuvo como propósito analizar los riesgos financieros derivados del cambio climático en la ciudad de Cartagena, con especial atención en los sectores de turismo sostenible y energías renovables. El problema central radica en la dificultad para movilizar inversión privada hacia proyectos sostenibles debido al alto costo del capital y la percepción de riesgo. Con el objetivo de evaluar cómo el blended finance (financiamiento mixto) puede reducir estos riesgos y fortalecer la resiliencia económica local, se desarrolló un estudio cualitativo de tipo documental y exploratorio, fundamentado en fuentes académicas indexadas en Scopus y reportes de organismos internacionales como la OCDE, el BID y el Banco Mundial. Los resultados evidencian que la aplicación de esquemas de financiamiento mixto puede reducir los costos de capital hasta en un 30\%, además de incrementar la rentabilidad y sostenibilidad de los proyectos verdes. Se concluye que el blended finance constituye una herramienta clave para promover la transición energética y el desarrollo sostenible frente a los efectos económicos del cambio climático en Cartagena.