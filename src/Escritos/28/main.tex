\documentclass[letterpaper, 12pt]{article}

\usepackage[utf8]{inputenc}
\usepackage[english, spanish]{babel}
\usepackage{fullpage}
\usepackage{graphicx}
\usepackage{amsmath}
\usepackage{enumitem}
\usepackage{chngcntr}
\usepackage{setspace}
\usepackage{url}
\usepackage{csquotes}
\usepackage{float}
\usepackage{verbatim}
\usepackage{tabularx}
\usepackage{amsmath}
\usepackage{caption}
\usepackage{bm}
\usepackage{colortbl}
\usepackage{xcolor}
\usepackage{multicol}
\usepackage{wrapfig}
\usepackage{multirow}
\usepackage{mathptmx}

% \usepackage{hyperref}

\counterwithin{figure}{section}
\renewcommand{\thesection}{\arabic{section}}
\renewcommand{\thesubsection}{\thesection.\arabic{subsection}}
\renewcommand{\baselinestretch}{2}

\usepackage[style=apa, maxnames=6, minnames=3, backend=biber, parentracker=true, sorting=none]{biblatex}
\DefineBibliographyStrings{english}{%chktex-file 1 chktex-file 6
      andothers = {\em et\addabbrvspace al\adddot}
}
\addbibresource{./Bibliography/bibliography.bib}

\usepackage{array}

\setlength{\parskip}{0pt}

\raggedbottom{}

\newcommand{\bolditalic}[1]{\textbf{\textit{#1}}}

\begin{document}

\begin{titlepage}
      \centering
      \includegraphics[width=0.3\textwidth]{Images/logo_utb.png}\par\vspace{1cm}
      {\scshape\LARGE Universidad Tecnológica de Bolívar \par}
      \vspace{.5cm}

      {\scshape\Large Estrategias de Negociación $-$ 2476 \par}
      \vspace{.2cm}

      \vspace{.5cm}
      % chktex-file 8
      \slshape {\Large \bfseries{}Informe: Culturas de Negociación\\}
      \slshape {\small \bfseries{}Cultura Inglesa}
      \vspace{1cm}

      \slshape {\itshape{} Bley E. Vásquez \\}
      \slshape {\itshape{} Valentina D. Del Rio \\}
      \slshape {\itshape{} María I. Grau \\}
      \slshape {\itshape{} Alexa Cardona  \\}
      \slshape {\itshape{} Ana K. Hernández \\}
      \vfill
      Revisado Por \\
      Juan Francisco Silva\\
      {\large \today\par}
\end{titlepage}

% chktex-file 44
% chktex-file 24

\section*{Introduccion}

La cultura inglesa es un fascinante crisol de historia, tradición y modernidad,
que ha ejercido una influencia profunda y duradera en el mundo. Desde su rica
herencia literaria, con autores emblemáticos como William Shakespeare y Jane
Austen, hasta sus icónicos símbolos como la familia real, el té de la tarde y
los pubs, la cultura inglesa ha logrado un delicado equilibrio entre conservar
sus raíces históricas y abrazar la innovación. Londres, su capital, es un claro
reflejo de esta dualidad, siendo un centro cosmopolita que combina arte, moda y
tecnología, al tiempo que preserva monumentos históricos como el Big Ben y la
Torre de Londres. Además, la cultura inglesa ha sido un pilar del pensamiento
político y filosófico global, con figuras como John Locke y Adam Smith, cuyas
ideas han moldeado el liberalismo y el capitalismo moderno. Así, explorar la
cultura inglesa es sumergirse en una rica narrativa de influencias locales y
globales, donde la tradición y el cambio coexisten armoniosamente.

\section*{Datos Generales}

La cultura inglesa es una de las más influyentes del mundo, con una rica
historia marcada por su monarquía y el legado del Imperio Británico. La familia
real sigue siendo un símbolo importante, y su papel en la Revolución Industrial
y la expansión del idioma inglés ha dejado una huella global. El inglés ha sido
vehículo de una vasta producción literaria, destacando autores como Shakespeare
y Jane Austen, y ha impulsado la música con bandas icónicas como The Beatles.

Costumbres como el té de la tarde y festividades como Guy Fawkes Night son
parte de la identidad cultural, mientras que el deporte, especialmente el
fútbol, tiene gran relevancia. La religión oficial es la Iglesia Anglicana,
pero el país es notablemente diverso. La gastronomía refleja esta diversidad,
combinando platos tradicionales como el fish and chips con influencias
internacionales.

En educación y ciencia, Inglaterra es un referente global, con universidades
como Oxford y Cambridge, y científicos como Newton y Darwin que han dejado un
legado duradero. Inglaterra combina una sólida tradición democrática con una
sociedad que valora tanto la tradición como la innovación.

\section*{Fechas festivas}

En Inglaterra estas fechas se conocen como bank holidays, se conocen así debido
a que inicialmente eran los días en los que los bancos no abrían al público.
Este sigue siendo uno de los países con menos festivos oficiales en el año
(ocho en total), lo que hace que los británicos los esperen con gran
expectativa

\begin{itemize}[label=$\bullet$]
      \item 1 de enero: Año Nuevo (New Year's day)
      \item 17 de marzo: Dia de San Patricio (Saint Patrick's Day)
      \item 1 de abril: Dia de los santos inocentes (April Fool's day)
      \item Marzo- Abril: Viernes Santo (Good Friday, el viernes anterior al domingo de
            pascua)
      \item 23 de abril: Dia de San Jorge (St.~George's Day)
      \item Abril- Mayo: Lunes de Pascua (Easter Monday, el siguiente al domingo de pascua)
      \item Mayo (primera semana): Fiestas de principios de Mayo (Early May Bank Holiday)
      \item Finales de mayo / principios de junio: Fiesta de primavera (Spring Bank
            Holiday)
      \item 12 de julio: Batalla de Boyne (día de la Orden de Orange)
      \item Primer lunes de agosto: (Bank Holiday de verano)
      \item Finales de agosto: Fiesta de Verano (Summer Bank Holiday)
      \item 31 de octubre: Víspera de todos los Santos (o Halloween)
      \item 5 de noviembre: Noche de Guy Fawkes
      \item 24 de diciembre: Nochebuena
      \item 25 de diciembre: Dia de navidad (christmas day)
      \item 26 de diciembre: Boxing day (Dia después de navidad)
      \item 31 de diciembre: Nochevieja
\end{itemize}

\section*{Cultura Y Estilo De La Negociación Inglesa}

El estilo de negociación británico, conocido como “British negotiating style,”
se caracteriza por la diplomacia, formalidad, comunicación directa y un fuerte
respeto por las normas y procedimientos. Los empresarios británicos son leales
y transparentes en sus transacciones, y se aseguran de la calidad de los
productos que ofrecen, generando confianza con sus socios comerciales.

\subsection*{Características del Empresario Británico}

\begin{itemize}[label=$\bullet$]
      \item Seriedad: Mantiene un enfoque serio en las negociaciones.
      \item Franqueza: Responde de manera directa si está interesado en el producto.
      \item Transparencia: Proporciona información oportuna sobre los pagos a sus
            proveedores.
      \item Certificaciones: Valora que las empresas con las que desea colaborar cuenten
            con certificaciones, como el análisis y control de riesgos críticos (HACCP) y
            buenas prácticas agrícolas (BPA).
      \item Eurepgap: Exige esta certificación, que garantiza la trazabilidad del producto
            y la identificación de riesgos en la seguridad alimentaria.
\end{itemize}

\subsection*{Negocios en Inglaterra}

El Reino Unido es un país multilingüe y multiétnico, con marcadas diferencias
culturales entre Londres y otras regiones. Comprender estas variaciones es
crucial para el éxito en los negocios. El Reino Unido abarca un área de 243,610
km² y tiene una población de 65.64 millones (2016), compuesto por Inglaterra,
Escocia, Gales e Irlanda del Norte. Londres, la capital de Inglaterra, es la
ciudad más influyente en finanzas, política y cultura.

\subsection*{Personalidad del Reino Unido}

Cada país del Reino Unido es percibido como una entidad única, y sus habitantes
se sienten orgullosos de su identidad. Esto se refleja en un espíritu
competitivo, especialmente en eventos deportivos como el torneo de rugby de las
Seis Naciones. Es importante no referirse a los británicos como "ingleses" a
menos que se esté seguro de su procedencia, ya que puede resultar ofensivo.

Los británicos suelen considerarse educados, con buenos modales y una gran
conciencia de las diferencias de clase. A veces pueden parecer fríos o
distantes, y su humor, que tiende a ser seco y sutil, puede ser difícil de
captar al principio. Existen ciertos rasgos comunes que identifican a las
personas de reino unido, los cuales son:

\begin{itemize}[label=$\bullet$]
      \item Cortesía y buenos modales
      \item Reservados y moderados
      \item Consciencia de clase
      \item Individualismo
      \item Diplomacia y discreción
      \item Multiculturalidad y diversidad
\end{itemize}

\subsection*{Cultura De Negocio En Reino Unido: Consideraciones}

A pesar de las similitudes con otros países de habla inglesa, la cultura y el
estilo empresarial del Reino unido. Demanda una atención especial. Los aspectos
más importantes para tener en cuenta son:

\subsubsection*{Puntualidad}

La puntualidad es un valor clave para los británicos, esta, no solo es una
costumbre, si no una muestra de respeto hacia la persona con la que se está
llegando a un acuerdo. Llegar a tiempo ayuda crear relaciones profesionales más
fuerte y evidencia seriedad y compromiso.

\subsubsection*{Código de Vestimenta}

El protocolo de vestimenta en el mundo empresarial británico destaca la
importancia de proyectar una imagen profesional y pulcra. Aunque los hombres
suelen llevar traje y corbata, y las mujeres optan por una vestimenta más
formal, es esencial, adaptarse a las normas específicas de cada industria y
empresa.

\subsubsection*{Etiqueta de Negocios}

La etiqueta de negocios en Reino Unido se caracteriza por la formalidad y la
cortesía. Es importante mantener un tono adecuado para todas las interacciones
comerciales. Aunque las reuniones extensas con bebidas se han vuelto menos
comunes, el respeto y un comportamiento decente, siguen siendo claves para las
relaciones empresariales exitosas.

\subsubsection*{Lenguaje Corporal}

El lenguaje corporal de los británicos usualmente suele ser, reservado y
moderado, en el ámbito empresarial, es mejor evitar el contacto visual excesivo
y los gestos demasiado expresivos, ya que, son considerados inapropiados.

\subsubsection*{Presentaciones}

En el Reino Unido, es común usar el primer nombre desde el principio, aunque al
conocer a personas mayores, es mejor comenzar con un saludo formal usando su
apellido y título, esperando que inviten a usar otro trato. Los títulos
académicos rara vez se utilizan, ya que usarlos podría parecer arrogante.

\subsubsection*{Importancia de las Tarjetas de Visita}

En la cultura empresarial británica, las tarjetas de visita tienen un papel
importante para facilitar conexiones profesionales. Es habitual llevar un buen
suministro de tarjetas y entregarlas en reuniones de negocios, ya que es
considerado un aspecto esencial de las interacciones comerciales. Estas
tarjetas no solo proporcionan los datos de contacto, sino que también actúan
como una forma tangible de reforzar y recordar las conexiones clave hechas
durante los encuentros profesionales.

\section*{Innovación}

Inglaterra ha sido históricamente un líder en innovación, desde la Revolución
Industrial hasta el auge de la tecnología moderna. Hoy en día, el país sigue
siendo un centro global de investigación y desarrollo, con énfasis en áreas
como las finanzas, la biotecnología, la inteligencia artificial y la tecnología
verde. Algunos aspectos clave de la innovación en Inglaterra son:

\subsection*{Centros Tecnológicos y Startups}

Londres es uno de los principales hubs tecnológicos de Europa, con una próspera
escena de startups y un fuerte ecosistema de financiamiento de riesgo. Las
empresas tecnológicas, especialmente en el sector fintech, han crecido
exponencialmente en las últimas décadas.

\subsection*{Investigación y Desarrollo (I+D)}

El gobierno británico invierte considerablemente en I+D, con políticas de apoyo
a las empresas innovadoras y un enfoque en sectores estratégicos como la
inteligencia artificial y la ciberseguridad. Universidades de renombre como
Oxford, Cambridge y el Imperial College de Londres son líderes en investigación
científica.

\subsection*{Innovación Verde}

Inglaterra ha puesto un fuerte énfasis en la innovación relacionada con la
sostenibilidad y las tecnologías limpias. La transición hacia una economía baja
en carbono es una prioridad, y el país es líder en energía eólica marina y
tecnología de vehículos eléctricos.

\subsection*{Cultura Empresarial}

La cultura empresarial en Inglaterra promueve la mejora continua y la
adaptabilidad al cambio. Se fomenta la experimentación controlada y el
aprendizaje de los errores, lo que impulsa a las empresas a seguir innovando y
manteniéndose competitivas en un mercado global.

\section*{Tiempo de Negociación}

Las negociaciones en Inglaterra suelen ser detalladas y pueden alargarse debido
al enfoque meticuloso que los ingleses adoptan al analizar cada aspecto del
acuerdo. Sin embargo, la duración varía dependiendo del tipo de negociación:

\subsection*{Negociaciones Empresariales}

Este tipo de negociaciones puede extenderse durante varios meses, ya que los
ingleses prefieren revisar cada detalle y evitar cualquier ambigüedad. La
paciencia es fundamental, y no se espera que las decisiones se tomen
apresuradamente.

\subsection*{Contratos}

En las negociaciones contractuales, se presta una atención meticulosa a los
términos del acuerdo. Esto puede alargar el proceso, pero asegura que ambas
partes estén completamente alineadas antes de proceder.

\subsection*{Compras}

Las negociaciones de compras pueden ser más ágiles, pero siempre se hace
hincapié en la precisión y claridad de los términos. Es común que los
compradores ingleses realicen investigaciones exhaustivas antes de
comprometerse con un proveedor.

\subsection*{Negociaciones Cotidianas}

En las interacciones cotidianas, como fijar plazos o ajustar términos menores,
las negociaciones suelen ser breves y directas, aunque el tono sigue siendo
educado y respetuoso.

\section*{Lugar y Hora de Negociación}

\subsection*{Horarios Adecuados}

En Inglaterra, el horario común en el que se tratan las negociaciones es a
partir de la media mañana (11:00 am), ya que no es muy común pactar reuniones
durante el desayuno. Así mismo, la jornada laboral de los ingleses termina
entre las 16:00 o 17:00 horas. No se acostumbra a concertar citas fuera de este
horario ya que trabajar horas extras se percibe como una falta de organización.
Además, se excluyen los fines de semana y las vacaciones.

Por norma general, los británicos son personas que agendan citas con mucha
anticipación, siempre tomando en cuenta los criterios anteriores y, una vez que
se concierte dicha cita, no es necesaria hacer una confirmación.

Para casos de retrasos, Reino Unido es por excelencia el país de la
puntualidad, siempre estarán a tiempo, por ello así su retraso sea de unos
cinco minutos, llame para hacer saber de su dilación.

\subsection*{Lugares Adecuados}

Es posible que el lugar de la negociación escogido incluya desarrollarla
durante una comida: los almuerzos de negocios tienen lugar en Pubs (bares o
restaurantes lujosos).

En ocasiones, pueda que lo inviten a casas particulares, a pesar de que esto no
es muy común ya que los ingleses valoran mucho la intimidad familiar. De ser el
caso, tenga en cuenta que los ingleses son muy golosos, por lo que es bien
visto llevar bombones o alguna bebida alcohólica (vino de calidad).

\section*{Idioma y Marketing}

El inglés es el idioma que más se habla en el país y es el que impera en la
esfera de los negocios, pero dependiendo de la región en la que te encuentres
existen variantes del idioma como el gaélico o el irlandés.

Debido a la gran diversidad de idiomas, los ingleses suelen tratar el idioma
oficial con el que se comunicarán en la negociación de forma anticipada. Si
ambas partes manejan distintos idiomas, es común que se recurra al uso del
inglés.

\subsection*{Estrategias de Marketing en la Negociación}

\begin{itemize}
      \item Investigación Previa: Conocer a fondo la empresa y el mercado británico es
            esencial. Esto incluye entender las tendencias del mercado, la competencia y
            las necesidades específicas del cliente.
      \item Comunicación Clara: Utilizar un lenguaje claro y directo, pero siempre con
            cortesía. Frases como “What I propose is…” (Lo que propongo es…) pueden ser muy
            efectivas.
      \item Presentación de Propuestas: Al presentar una propuesta, es útil ser detallado y
            proporcionar datos concretos que respalden tus argumentos. Esto demuestra
            profesionalismo y preparación.
      \item Negociación de Condiciones: Estar preparado para negociar términos y
            condiciones. ¿Preguntas como “Is there any flexibility in your offer?” (¿Hay
            alguna flexibilidad en tu oferta?) pueden abrir la puerta a ajustes
            beneficiosos.
      \item Cierre del Trato: Resumir y confirmar los acuerdos es fundamental. Frases como
            “Let’s confirm the details to ensure we’re on the same page” (Confirmemos los
            detalles para asegurarnos de que estamos en la misma línea) ayudan a evitar
            malentendidos.
\end{itemize}

\section*{Vocabulario Clave}

\begin{itemize}
      \item Iniciar la negociación: “I’m looking forward to discussing our potential
            collaboration” (Estoy deseando discutir nuestra posible colaboración).
      \item Hacer una propuesta: “I’d like to suggest that…” (Me gustaría sugerir que…).
            Rechazar una propuesta: “I’m not sure that will work for us” (No estoy seguro
            de que eso funcione para nosotros).
      \item Comprometerse: “We could agree to that if…” (Podríamos estar de acuerdo con eso
            si…). Estrategias de Lenguaje y Marketing Combinadas en la Negociación
            Storytelling: El uso del storytelling, una técnica del marketing también es muy
            eficaz en las negociaciones en inglés. Al contar una historia sobre cómo un
            producto o servicio ha beneficiado a otros, se crea una narrativa persuasiva y
            emocionalmente atractiva para el interlocutor.
      \item Análisis Competitivo: En marketing, el análisis de la competencia es vital, y
            este conocimiento se puede utilizar en la negociación para destacar las
            ventajas competitivas de la oferta. Frases como: ``Compared to our competitors,
            we provide\ldots at a lower cost'' son comunes.
      \item Propuestas de Valor: En marketing, las propuestas de valor son fundamentales y
            estas se deben reflejar claramente en la negociación. En lugar de solo discutir
            el precio, se habla del valor agregado que ofrece tu producto o servicio.
\end{itemize}

\section*{Comunicación y Tecnología.}

La cultura inglesa se caracteriza por una comunicación formal y precisa, tanto
verbal como escrita. En el ámbito profesional, se valora el respeto y la
diplomacia, manteniendo un tono educado en correos electrónicos y reuniones.
Este estilo también se refleja en las tecnologías de comunicación, donde la
claridad y el respeto son fundamentales. Las plataformas digitales, como
Microsoft Teams y Slack, son comunes en el entorno laboral, facilitando la
colaboración sin perder la formalidad característica de la cultura británica.

La privacidad y la ética son pilares clave en el uso de la tecnología en el
Reino Unido. La normativa de protección de datos, como el GDPR, es
estrictamente aplicada, reflejando la fuerte preocupación por la seguridad de
los datos personales. El Reino Unido es conocido por equilibrar la tradición y
la innovación, siendo líder en áreas como inteligencia artificial y fintech.
Londres es un centro global para startups tecnológicas, respaldado por
políticas que fomentan el crecimiento empresarial y la investigación.

En cuanto al uso de redes sociales, los británicos mantienen un enfoque más
reservado y profesional en comparación con otras culturas. Aunque se utilizan
plataformas populares como LinkedIn y Twitter, el tono se mantiene moderado y
diplomático. Además, el gobierno británico ha impulsado iniciativas para
reducir la brecha digital, asegurando que la tecnología sea accesible para
todos, independientemente de su nivel socioeconómico.

\section*{Trato y Negociador}

El estilo de trato y negociación inglés suele ser formal, discreto y educado.
Aquí algunos aspectos clave:

En el Reino Unido se prefiere tratar con ejecutivos senior; se asume que la
edad significa autoridad y un comportamiento más formal y reservado en las
relaciones. El trato es bastante frío, distante y muy profesional. No está bien
visto utilizar argumentos emocionales ni gesticular en exceso.

Los ejecutivos ingleses no se caracterizan por preparar las reuniones con mucho
detalle. No es necesario establecer una agenda de temas a tratar. Normalmente
están más interesados en los resultados a corto plazo que en las relaciones a
largo plazo.

Aunque formales, los británicos suelen tener un agudo sentido del humor, que
puede surgir incluso en negociaciones serias. Esto no significa que no estén
tomando la situación en serio, sino que el humor es una herramienta para
relajar el ambiente.

Las reuniones de negocios comienzan y terminan con una breve charla informal
(small talk). Hay que hablar de temas banales: el viaje, el tiempo, el tráfico,
etc. No se deben hacer preguntas personales, ni siquiera del tipo de: “¿De
dónde es Usted?”.

Los ingleses suelen hacer pocos gestos y no opinar acerca de las ofertas que se
les presentan, es más común realizar simples comentarios indirectos o
insinuaciones. Las tácticas agresivas no son muy bien vistas por ellos.

En las decisiones, suelen guiarse por normas establecidas o por precedentes
similares, más que por sentimientos o ideas personales. Una vez que han tomado
la decisión de hacer el negocio, el trato pasa a ser muy directo y franco. No
tienen problemas en decir claramente lo que piensan. Cuando el negocio se
formaliza en un contrato insistirán en someterlo a la legislación británica. Lo
mejor es mantener una reunión previa (preliminary discussions) de las dos
partes con sus abogados respectivos.

En el Reino Unido prácticamente no existe ningún tipo de contacto físico en
público (besos, palmadas, abrazos). Además, se mantiene una amplia distancia
física con la persona con la que se habla. Generalmente, los británicos no
miran directamente a la persona a la que se dirigen y hacen muy pocos gestos
cuando hablan.

Se llama a la gente por el apellido precedido por los títulos de Mr. o Ms.
(pronunciado Mes). El título de Mrs. (pronunciado Missis) sólo se utiliza para
presentar a mujeres casadas. Los títulos de Sir y Madam sólo se usan con
personas acreditadas a ellos (estamentos militares, nobleza, etc.).

\section*{Imagen y Códigos de Etiqueta}

En cuanto a la imagen y códigos de etiqueta, debemos tener en cuenta ciertas
consideraciones al tratarse de negociaciones con el Reino Unido.

\subsection*{Forma de Vestir}

Es importante vestir bien. Los ingleses aprecian la ropa de buena calidad,
aunque no necesariamente nueva; evite las corbatas de rayas y escudos ya que
podrían parecer una imitación de sus corbatas regimentales o universitarias.
Llevan zapatos de cordones y camisas sin bolsillos $–$si los tienen, los dejan
vacíos$‐$. En el Reino Unido es importante vestir bien, sobre todo en Londres.

Los ingleses, además, odian el color marrón. De hecho, por poner un ejemplo, en
los años 80, si vestías zapatos marrones en el ``parquet'' te abucheaban.

\subsection*{Normas de Protocolo}

Un apretón de manos ligero y breve es la forma de saludo más usual. Las mujeres
no siempre estrechan la mano. Es mejor esperar a que sean ellas las que la
extiendan. Los ingleses están continuamente pidiendo disculpas, incluso por
pequeñas inconveniencias. Es aconsejable hacer lo mismo.

En el trato son muy formales y educados (british manners), especialmente con
las mujeres: les abren las puertas para que pasen, se levantan cuando entran en
una habitación y les acercan la silla en los restaurantes.

En la conversación no es adecuado hablar de la familia real (nunca se debe
criticar su estatus, riqueza o papel en la sociedad) ni del problema de Irlanda
del Norte. Tampoco es bueno establecer comparaciones con Estados Unidos. Les
gusta hablar de su historia (están muy orgullosos del Imperio Británico), de
animales (sobre todo de perros) y de deportes (fútbol, rugby y atletismo).

El tiempo climatológico es una conversación obligada. En las reuniones se suele
ofrecer café o té. No es necesario aceptarlo: el Reino Unido es uno de los
pocos países en el mundo en el que declinar una invitación a tomar algo no está
mal visto.

Muchas de los almuerzos de negocios tienen lugar en pubs, en los que se sirve
comida típicamente inglesa, bastante más ligera que la de la Europa
continental. En la mesa, los objetos tienen que pasarse siempre hacia la
izquierda y no se considera correcto probar comida del plato de otra persona.

\section*{Personalidad del Negocio}

La personalidad del negocio en la cultura inglesa se caracteriza por una
combinación de formalidad, diplomacia, ética y un enfoque en relaciones
comerciales a largo plazo. Las interacciones en el entorno laboral suelen ser
formales y estructuradas, con gran valor por la puntualidad y el
profesionalismo. En la comunicación, se evita el enfrentamiento directo,
prefiriendo un estilo diplomático y sutil, lo que permite mantener una
atmósfera respetuosa y evitar tensiones innecesarias.

Uno de los pilares clave es la ética en los negocios. La transparencia y la
integridad son fundamentales, y las empresas británicas prefieren cumplir con
las normativas y hacer promesas realistas. Las propuestas infladas o exageradas
tienden a generar desconfianza, ya que la reputación es esencial para construir
relaciones comerciales duraderas. Los británicos suelen adoptar una actitud
cautelosa y calculada hacia la toma de decisiones, prefiriendo evitar riesgos
innecesarios. Las decisiones importantes se toman tras un análisis cuidadoso de
los datos, buscando la estabilidad y la previsibilidad. Esta cautela también se
refleja en las relaciones de negocios, que se desarrollan lentamente, basándose
en la confianza y el respeto mutuo.

Aunque el ambiente de trabajo es formal, el humor tiene un lugar especial en
las interacciones cotidianas. Los británicos emplean un humor sutil y seco para
suavizar situaciones o generar camaradería, aunque siempre dentro de un marco
de respeto. Finalmente, aunque las empresas inglesas no son extremadamente
jerárquicas, sí respetan la estructura organizacional, involucrando a los
niveles superiores en las decisiones estratégicas importantes.

En conjunto, los negocios en el Reino Unido se conducen con una combinación de
respeto por las normas, diplomacia y un enfoque firme en la ética y la
construcción de relaciones a largo plazo.

\section*{Detalle de la Conversación}

Al momento de negociar con británicos, la conversación es importante. La
comunicación con ellos en un entorno empresarial y la manera en la que se
pactan encuentros destinados a la negociación refleja de forma oportuna su
cultura de negocios.

\subsection*{Comunicación}

Aunque los británicos pueden ser amables y educados, prefieren ir al punto de
manera directa pero respetuosa. Aprecian la precisión y la brevedad en las
comunicaciones empresariales.

En muchas ocasiones, se requiere que la solicitud se haga por escrito,
detallando el propósito de la reunión, los temas a tratar y el tiempo estimado.
Es importante establecer las fechas de las reuniones con anticipación, al menos
una semana antes, para evitar inconvenientes. Los británicos no aprecian
cambios repentinos de planes o retrasos.

\subsection*{Medios Para Pactar Reuniones}

Email: Es el medio preferido para iniciar las conversaciones y confirmar
reuniones. Ofrece claridad y un registro de lo acordado.

Llamadas Telefónicas: Aunque el correo electrónico es el medio preferido,
también es posible que utilicen llamadas telefónicas, especialmente para
cuestiones urgentes o de seguimiento.

Videoconferencias: Con el avance tecnológico, muchas empresas británicas están
abiertas a realizar reuniones virtuales, lo cual también facilita la
comunicación si las partes están en diferentes lugares.

\subsection*{A Quién Dirigirse}

No es fácil conseguir una entrevista con empresas inglesas. Si no se tiene
ningún contacto, es mejor dirigirse a la empresa, en general, más que a una
persona o departamento en concreto.

Las empresas británicas sólo conceden entrevistas a empresas con las que
realmente están interesadas en realizar negocios ya que el tiempo se valora
mucho. Es posible que antes de conceder la entrevista soliciten catálogos de
productos y listas de precios para analizar si realmente les interesa.

\section*{Datos Curiosos}

\begin{itemize}[label=$\bullet$]
      \item Los ingleses valoran mucho la puntualidad, llegar tarde a una reunión de
            negocios se percibe cómo una falta de respeto o un signo de desorganización.

      \item Las reuniones de negociones en Inglaterra tienden a ser bien estructuradas y
            organizadas, desviarse del tema o improvisar demasiado puede ser mal visto.

      \item El té es una parte fundamental de la vida diaria en Inglaterra, y las pausas
            para tomar te son comunes incluso en entornos de negocios, no es raro que se
            ofrezca una pausa para tomar té en reuniones largas.

      \item Prefieren un lenguaje corporal controlado y discreto, los gestos exagerados o
            demasiado entusiastas pueden interpretarse cómo falta de profesionalismo.

\end{itemize}

\newpage

\nocite{*}

\printbibliography

\end{document}