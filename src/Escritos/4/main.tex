\documentclass[letterpaper, 12pt]{report}

\usepackage[utf8]{inputenc}
\usepackage[english, spanish]{babel}
\usepackage{fullpage} % changes the margin
\usepackage{graphicx}
\usepackage{amsmath}
\usepackage{enumitem}
\usepackage{chngcntr}
\usepackage{setspace}
\usepackage{url}
\usepackage{csquotes}
\usepackage{float}
\usepackage{verbatim}
\usepackage{tabularx}
\usepackage{amsmath}

\counterwithin{figure}{section}
\renewcommand{\thesection}{\arabic{section}}
\renewcommand{\thesubsection}{\thesection.\arabic{subsection}}
\renewcommand{\baselinestretch}{1.5}

\usepackage[style=apa, maxnames=6, minnames=3, backend=biber, parentracker=true]{biblatex}
\DefineBibliographyStrings{english}{%chktex-file 1 chktex-file 6
    andothers = {\em et\addabbrvspace al\adddot}
}
\addbibresource{./Bibliography/bibliography.bib}

\usepackage{array}
\usepackage{enumitem}

\setlength{\parskip}{\baselineskip}

\usepackage{setspace}

\begin{document}

\setstretch{1}

\chapter*{Reseña critica: El oso que no lo era}

 % \noindent\hrulefill

 (Tashlin, 1967,El oso que no lo era)

\nocite{FrankTashlin}

\noindent\makebox[\linewidth]{\rule{\textwidth}{0.4pt}}

Por: Mauro González

\noindent\makebox[\linewidth]{\rule{\textwidth}{0.4pt}}

\setstretch{2}
\vspace{1cm}

El cuento <<El oso que no lo era>>, escrito por el
estadounidense, Frank Tashlin (1913--1972), es un relato
que transmite un mensaje muy importante, hablando de la
autoaceptación y la autoconfianza.

La historia sigue a un oso que vive en una cueva, quien
después de hibernar, despierta para encontrar que su hogar
ha sido destruido. Mientras el oso dormía, los humanos
construyeron una fábrica justo encima de su casa. Después
de caminar por un tiempo, el oso se encuentra con un
trabajador de la fábrica, quien le reclama por no estar
trabajando. Este incidente desencadena una serie de
confusiones y malentendidos, en los que el oso es visto y
citado como ``un hombre tonto, sin afeitar y con un abrigo
de pieles''. Desconcertado, el oso comienza a cuestionar su
propia identidad. El relato sigue al oso mientras intenta
comprender quién es realmente y cuál es su lugar en el
mundo. Luego llega el desenlace de la historia, que revela
el verdadero destino del oso.

% chktex-file 38
Llegando al dilema del relato, este plantea un fuerte
conflicto existencial por parte del oso, el cual pierde
totalmente su identidad debido a lo que los demás le
decían. ``Soy un oso, pero entonces ¿por qué siempre me
dicen lo mismo?'', ``¿Entonces no soy un oso?'', son
preguntas que el oso potencialmente podría estar haciéndose
a si mismo llevando a un punto de inflexión bastante fuerte
ya al final de la obra, en el cual el oso se rehúsa su
naturaleza ya que supuestamente ``él no era un oso''.

Estas son temáticas que, para un público infantil o joven,
son entendibles pero no llegan a captar del todo el
mensaje. En etapas como la infancia o pre-adolescencia, no
se le da mucha importancia a la pregunta de cómo nuestras
acciones pueden influenciar a los demás. Sin embargo, al
leer el relato como un lector más experimentado en la
temática, o como alguien que ha vivido en carne propia el
dilema de la historia --no literalmente--, el mensaje es
recibido de una mejor manera, dejando una bonita reflexión.

Siguiendo con la idea anterior, a medida que crecemos,
comprendemos que nuestras acciones tienen un impacto en los
demás. Expresar una opinión, por ejemplo, no es simplemente
decir lo que pensamos, ya que no podemos prever cómo la
otra persona reaccionará ante ella. Puede ser bien recibida
o puede hacer daño. Esto es común en la vida diaria, en el
trabajo, en la escuela o en la universidad, donde estamos
expuestos a las opiniones de los demás, que pueden
influenciarnos de diversas maneras.

Sin embargo, en algunos casos, la opinion de los demás
puede adquirir un peso tan importante en nuestra vida que
puede llegar a influir en nuestra manera de pensar y
actuar, lo que puede ser perjudicial para nuestra salud
mental y emocional. Llega un punto en el que la persona se
acostumbra a funcionar a traves de las opiniones de las
personas que conforman su entorno, poniendo en duda su
propio criterio, quitándole esa ``identidad'' a la persona
(\cite{Psicología}).

En cuanto a las consecuencias, realmente varían en la
medida en que la persona se ha visto afectada. Sin ir mas
lejos los casos de Bullying por todo el mundo, que
lamentablemente terminan en suicidios o dejando daños
irreparables. Las victimas de acoso escolar tienen hasta
tres veces mas riesgo de suicidio (\cite{AcosoEscolar}).

Y precisamente, es importante darse el trabajo de conocerse
a si mismo, conocer cuales son mis fortalezas y
debilidades, aprender a manejar las criticas negativas para
que no nos afecten, conocer cuales son mis gustos, en lo
que soy bueno y lo que no. Esta es una reflexión muy bonita
que se puede sacar de la obra, ademas de que no se necesita
ser un experto en textos para deducirla. Ahi, es donde
recae una de las razones por las que es categorizado como
cuento infantil.

Por otro lado, introduciendo la forma en que esta escrito
el texto. Es importante destacar la coherencia y
organización de los temas tratados. Para lograr una
comprensión efectiva de los mismos, es esencial que la
estructura sea coherente y esté bien organizada. La obra en
cuestión cumple con creces esta función, utilizando
imágenes y descripciones detalladas que permiten seguir con
la historia de manera mas fácil, ademas evitando el uso de
tecnicismos innecesarios y añadiendo pequeños toques de
humor que resultan atractivos para el lector.

En general, considero que el texto cumple con su cometido
al transmitir una enseñanza atemporal que no deja nada que
desear. Aunque el cuento está dirigido principalmente a un
público infantil, creo que cualquier persona,
independientemente de su edad, puede disfrutar de la
historia y reflexionar sobre su mensaje. En resumen,
recomendaría este cuento sin dudarlo, ya que es una obra
que perdura en el tiempo y deja una enseñanza valiosa para
todos los lectores.

\newpage

\printbibliography

\end{document}