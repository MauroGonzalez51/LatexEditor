\documentclass[letterpaper, 12pt]{article}

\usepackage[utf8]{inputenc}
\usepackage[english]{babel}
\usepackage{fullpage}
\usepackage{graphicx}
\usepackage{amsmath}
\usepackage{enumitem}
\usepackage{chngcntr}
\usepackage{setspace}
\usepackage[hyphens]{url}
\usepackage{csquotes}
\usepackage{float}
\usepackage{verbatim}
\usepackage{tabularx}
\usepackage{amsmath}
\usepackage{microtype}
\usepackage{ragged2e}

\counterwithin{figure}{section}
\renewcommand{\thesection}{\arabic{section}}
\renewcommand{\thesubsection}{\thesection.\arabic{subsection}}
\renewcommand{\baselinestretch}{1.5}

\usepackage[maxbibnames=3,style=authoryear,backend=biber]{biblatex}
\DeclareNameAlias{sortname}{family-given}
\DeclareNameAlias{default}{family-given}
\renewcommand*{\multinamedelim}{\addcomma\space}
\renewcommand*{\finalnamedelim}{\addspace\&\space}
\DefineBibliographyStrings{english}{%chktex-file 1 chktex-file 6
    andothers = {et\addspace al\adddot}
}

\DeclareFieldFormat{cite}{\mkbibemph{#1}}
\renewbibmacro*{cite}{%
    \printtext[bibhyperref]{%
        \printfield{title}%
        \setunit{\addcomma\space}%
        \printdate}}

\addbibresource{./Bibliography/bibliography.bib}

\usepackage{array}
\usepackage{enumitem}
\usepackage{breqn}

\raggedbottom{}

\usepackage{array}
\usepackage{enumitem}

\usepackage{setspace}
\setlength{\parskip}{\baselineskip}

\begin{document}

\nocite{*}

\section*{International Marketing Task}

Yesid Muñoz~\textit{(T00069159)}

\noindent\makebox[\linewidth]{\rule{\textwidth}{0.4pt}}

\bigskip

\section{Do you think that greater world trade implies more risk? Explain the benefits and risks involved.}

I would argue that greater world trade does not necessarily imply more risk,
since a more interconnected world allows countries to establish routes that
minimize vulnerabilities and make international commerce faster and more
efficient. Among the benefits, world trade generates significant economic
advantages because many countries strengthen their commercial relations, which
in turn fosters growth, market diversification, and broader opportunities for
development.

However, the increase in world trade also carries certain risks. One of them is
dependence on global supply chains, which makes countries and firms more
vulnerable to disruptions caused by geopolitical conflicts, pandemics, or
natural disasters. Likewise, excessive openness can affect local industries,
leading to job losses and social inequality in some sectors. Another important
risk is the environmental impact, since international transportation increases
carbon emissions and sometimes encourages the relocation of polluting
industries to countries with weaker regulations. Finally, persistent trade
imbalances may result in economic volatility and excessive debt.

\section{What effect do global linkages have on companies and customers?}

Global linkages have a profound impact on both companies and customers. For
companies, globalization allows them to take advantage of economies of scale,
improve operational efficiency, and access cheaper resources in different
countries. In addition, international presence helps diversify risk by
operating across multiple markets, which increases resilience. However,
globalization also brings greater complexity, as companies must adapt to
diverse regulations and face geopolitical tensions, such as economic
competition between the United States and China, which can disrupt supply
chains.

For customers, global linkages create clear benefits: greater product variety,
lower prices, and access to innovative goods that were previously inaccessible.
On the other hand, consumers can also be affected by global
disruptions—\textit{such as the shortage of semiconductors or medical
    supplies}—which highlight the fragility of depending on international markets
for essential goods.

\section{What challenges does international marketing currently have, considering new trends?}

International marketing faces significant challenges today, many of which are
linked to technological integration, especially artificial intelligence. AI
enables companies to process massive amounts of customer data faster and to
deliver more personalized experiences. However, it also brings challenges
related to data privacy, potential algorithmic biases, and the need for ethical
regulation.

Other key trends that pose challenges include the rise of AI-driven
storytelling, which must balance automation with authenticity to maintain
credibility; digital disinformation, which can distort market research and
damage consumer trust; and cultural and linguistic adaptation, since marketing
messages must be tailored to resonate within each regional context. In
addition, regulatory frameworks such as the GDPR in Europe and other
data-protection laws worldwide restrict the indiscriminate use of personal
information in global campaigns.

\newpage

\printbibliography

\end{document}