\documentclass[letterpaper, 12pt]{article}

\usepackage[utf8]{inputenc}
\usepackage[english, spanish]{babel}
\usepackage{fullpage}
\usepackage{graphicx}
\usepackage{amsmath}
\usepackage{enumitem}
\usepackage{chngcntr}
\usepackage{setspace}
\usepackage{url}
\usepackage{csquotes}
\usepackage{float}
\usepackage{verbatim}
\usepackage{tabularx}
\usepackage{amsmath}
\usepackage{caption}
\usepackage{bm}
\usepackage{wrapfig}
\usepackage{siunitx}

\counterwithin{figure}{section}
\renewcommand{\thesection}{\arabic{section}}
\renewcommand{\thesubsection}{\thesection.\arabic{subsection}}
\renewcommand{\baselinestretch}{2}
\renewcommand{\thefigure}{\arabic{figure}}

\usepackage[style=numeric, maxnames=6, minnames=3, backend=biber, parentracker=true, sorting=none]{biblatex}
\DefineBibliographyStrings{english}{%chktex-file 1 chktex-file 6
    andothers = {\em et\addabbrvspace al\adddot}
}

\addbibresource{./Bibliography/bibliography.bib}

\usepackage{array}
\usepackage{enumitem}

\setlength{\parskip}{\baselineskip}

\newcommand{\bolditalic}[1]{\textbf{\textit{#1}}}

\DeclareSIUnit{\COP}{COP}
\newcommand{\cop}[1]{\$\SI{#1}{\COP}}

\DeclareSIUnit{\DOLLAR}{USD}
\newcommand{\dollar}[1]{\$\SI{#1}{\DOLLAR}}

\renewcommand{\comment}[1]{{\small $\ll$#1$\gg$}}

% chktex-file 24

\begin{document}

\section*{Dignidad y Libertad en el contexto de la película ``El patrón, radiografía de un crimen''}

\noindent\makebox[\linewidth]{\rule{\textwidth}{0.4pt}}

\begin{itemize}[label=$\diamond$]
    \item Mauro González~\textit{(T00067622)}
\end{itemize}

\noindent\makebox[\linewidth]{\rule{\textwidth}{0.4pt}}

\nocite{patron2014}
\nocite{39657713009}

% * -------------------------------------------------------|>

La película ``El patrón: radiografía de un crimen'' es un
drama argentino dirigido por Sebastián Schindel y lanzado
en el año 2014. El filme, protagonizado por Joaquín Furriel
como Hermógenes y Luis Ziembrowski como el patrón, explora
las complejidades de la relación laboral entre un peón
rural y su empleador. La trama se centra en Hermógenes,
quien, bajo la apariencia de un trabajo común, se ve
envuelto en una espiral de endeudamiento y servidumbre a
medida que trabaja en una carnicería, bajo el mando de su
patrón.

Para empezar, quisiera hablar de una frase fundamental, que
a su vez describe en una u otra medida, las situaciones por
las que ha pasado la persona que la menciona. La frase en
cuestión es \textit{``La vida es un destino a cumplir''}
dicha por Hermógenes en una de sus conversaciones con
\textit{Marcelo Di Giovanni} (su abogado).

Esta frase nos presenta en si una inquietante duda sobre la
libertad del individuo, ¿se puede decir que uno es libre,
si, por mas cosas que hagamos llegaremos a ese punto final
que es nuestro ``destino''? Todo esto empieza a adquirir
relevancia en el momento en que Hermógenes comienza a verse
atrapado en un destino predefinido, no puede dejar su
trabajo (lo necesita como sustento para su familia), ademas
de basar todas sus acciones en el dia, en lo que le diga
alguien mas que tiene que hacer.

La cruel situación laboral y personal, limitan la capacidad
de elección y autonomía, Hermógenes en algún punto de la
película decide regresarse a su antiguo lugar de residencia
(debido también a la insistencia por parte de su esposa),
pero simplemente no puede, debido a toda la manipulación
que recibía por parte de su patrón.

Siguiendo con el dia a dia de Hermógenes, asimismo como el
mostraba su lealtad hacia su patrón, el debía de
recompensarlo de vuelta, no necesariamente brindándole mas
beneficios, pero al menos ser un poco mas flexible con sus
tareas. La realidad es que todo esto es todo lo contrario,
parece ser que mientras mas Hermógenes le demostraba sus
ganas de trabajar, mas su patrón las pisoteaba, obligándolo
a hacer una y otra vez, tareas simplemente inhumanas.

Tener días de descanso para salir con su familia, verse
recompensado todo el trabajo que hacia, era simplemente un
sueño para Hermógenes. En escenas concretas de la película,
en las que se muestra como su patrón lo ordenaba a pintar
el techo, ademas de utilizar veneno para las ratas, etc
\dots{}~\comment{Todo esto para hacerlo un domingo}. Esto
puede parecer sensato en un primer momento, ya que el
patrón les brindaba un lugar para quedarse
\comment{Bastante precarias por cierto}, pero esto no viene
al caso, ¿podia él salir un domingo luego de trabajar al
parque con su esposa?, ¿podia tener un dia de descanso
luego de jornadas de trabajo tan intensivas como las tenia?

Independientemente de todo lo que trabajaba Hermógenes,
nunca seria capaz de disfrutar su sueldo, ya que su patrón
se lo descontaba, a su vez que lo engatusaba con sus
palabras, dándole a su vez una idea muy lejana, en la que
tenia una casa propia.

En este cruel ciclo de manipulación, se desdibuja la
dignidad humana de Hermógenes, la imposición constante de
tareas inhumanas, la negación de días de descanso y la
confiscación de su salario no solo son violaciones a sus
derechos laborales, sino también ataques a su integridad
como ser humano.

Por ultimo, quisiera hablar de una situación muy en
especifico. Luego de que Hermógenes tomara junto con su
esposa, la decision de volver, acto seguido el patrón se da
cuenta de esto y decide presentarle unas ``sinceras y
agradables'' disculpas. Quisiera a su vez, dar bastante
contexto de la situación ya que en esto se basa el
comentario de esta parte.

Escenas atrás Hermógenes, discutía con su esposa, y cito
``no nos queda un peso, no alcanza para nada'', escena en
la que se puede apreciar también las dificultades para
conseguir alimento por las que pasaba la pareja. Luego de
esto el patrón lo invita para disculparse en un restaurante
de la ciudad \comment{Aunque, humanamente imposible de
acceder para Hermógenes}, en la que el pide su comida
\comment{Cosa de todos los dias} y empieza a comer en su
cara, sin escrúpulo alguno.

Esta impactante escena en el restaurante encapsula las
profundas desigualdades económicas y la falta de
consideración por parte del patrón hacia Hermógenes. La
elección de un lugar inaccesible para el trabajador subraya
la brecha entre ambos mundos, donde las necesidades básicas
de Hermógenes, como la alimentación diaria, son ignoradas y
menospreciadas. Al comer de manera ostentosa frente a su
empleado, el patrón no solo exhibe su indiferencia hacia
las dificultades económicas de Hermógenes, sino que también
demuestra una clara falta de respeto por su dignidad como
ser humano.

Este acto no es simplemente un reflejo de la disparidad
financiera, sino una manifestación directa de cómo el poder
y el estatus del patrón son utilizados para socavar la
integridad de Hermógenes. Al forzarlo a presenciar su
propia impotencia y humillación, el patrón demuestra una
crueldad despiadada. La dignidad de Hermógenes es pisoteada
en un acto de menosprecio flagrante, dejando al descubierto
las consecuencias devastadoras de la explotación y la falta
de consideración en el ámbito laboral.

En conclusión, las situaciones analizadas en este trabajo
son solo una muestra de las numerosas injusticias y
desafíos que Hermógenes enfrenta en ``El patrón: radiografía
de un crimen''. La película revela un panorama desolador de
la explotación laboral y la pérdida de dignidad en un
entorno donde el poder y la vulnerabilidad chocan. Aunque
estas escenas son destacadas, son solo la punta del
iceberg, reflejando la complejidad de la lucha de
Hermógenes por preservar su libertad y dignidad en un
sistema implacable.

\newpage

\printbibliography

\end{document}