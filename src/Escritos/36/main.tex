\documentclass[letterpaper, 12pt]{report}

\usepackage[utf8]{inputenc}
\usepackage[english]{babel}
\usepackage{fullpage} % changes the margin
\usepackage{graphicx}
\usepackage{amsmath}
\usepackage{enumitem}
\usepackage{chngcntr}
\usepackage{setspace}
\usepackage{url}
\usepackage{csquotes}
\usepackage{float}
\usepackage{verbatim}
\usepackage{tabularx}
\usepackage{amsmath}

\counterwithin{figure}{section}
\renewcommand{\thesection}{\arabic{section}}
\renewcommand{\thesubsection}{\thesection.\arabic{subsection}}
\renewcommand{\baselinestretch}{2}

\usepackage[style=apa, maxnames=6, minnames=3, backend=biber, parentracker=true, sorting=none]{biblatex}
\DefineBibliographyStrings{english}{%chktex-file 1 chktex-file 6
    andothers = {\em et\addabbrvspace al\adddot}
}
\addbibresource{./Bibliography/bibliography.bib}

\usepackage{array}
\usepackage{enumitem}
\usepackage{setspace}
\setlength{\parskip}{\baselineskip}

\begin{document}

\chapter*{International Marketing Task}

\noindent\makebox[\linewidth]{\rule{\textwidth}{0.4pt}}

\noindent
Valentina Del Rio Jimenez, T00081360 \\
Programa: Finanzas y negocios internacionales

\noindent\makebox[\linewidth]{\rule{\textwidth}{0.4pt}}

\nocite{*}

\section{Do you think that the greater world trade implies more risk? Explain the benefits and risks involved.}

Yes, greater world trade does imply more risk, but it also brings significant benefits.

\textbf{One of the benefits could be}: Expanding into international markets gives companies the chance to grow beyond their own country, opening new opportunities for increased sales and revenue that might not be possible locally. Exporting helps businesses reach more customers, spread out risks, and even take advantage of favorable exchange rates. Plus, it pushes companies to stay competitive, leading to better products, more innovation, and improved efficiency. It’s also a smart way to avoid hitting a wall in saturated local markets by tapping into foreign markets where competition may be lower. 

And on the other hand, \textbf{one of the risks could be}: trading in global markets isn’t without its challenges. Political instability, changing regulations, or sudden cancellations of trade deals can seriously disrupt a company’s plans. Fluctuations in exchange rates, especially in emerging markets, can make it hard to predict profits and plans. Understanding local cultures is also important, because misreading customer expectations can harm a company's reputation. There are also financial risks like customers in other countries who do not pay on time and legal risks such as theft of intellectual property, including proprietary ideas or branding.

\section{What effect do global linkages have on companies and customers?}

Global connections have really changed how companies and customers interact today. For businesses, globalization means easier access to new markets, supply options, and worldwide talent. This helps them grow and come up with new ideas. Companies can buy materials from around the world at good prices, hire experts no matter where they live, and reach many more customers outside their own country. At the same time, these global links create more competition, which pushes companies to keep improving their products and services to stay ahead. For customers, globalization means more choices, better quality, and often lower prices. For example, someone in Europe might buy skincare products from the U.S. or electronics assembled in Asia. It also brings cultural variety into everyday products like Coca-Cola and McDonald's offering region-specific flavors that make shopping a richer experience. But there are some downsides, too. Customers might feel that products lose some of their authentic cultural feel, or they could be hurt by supply chain issues that affect availability and prices. All in all, global connections help both companies and consumers by encouraging growth, new ideas, and diversity in options. However, they also bring challenges, making it important for businesses to stay flexible and act responsibly.

\section{What challenges does international marketing currently have? Considering new trends?}

In today's connected world, international marketing faces a bunch of challenges brought about by globalization and fast technological changes. One big obstacle is fierce competition because online platforms let companies from all over the globe enter the same markets, which means they need to find clever ways to stand out. Another tricky part is reaching diverse audiences trying to use one-size-fits-all campaigns often don’t work because cultures and preferences differ so much. Plus, many companies don’t have enough resources whether it’s money or staff to run multiple campaigns across different countries effectively. Language barriers add another layer of complexity, as content and customer support must be fully personalized to local languages. Regulations like GDPR also come into play, influencing how companies handle customer data and communications. On top of that, businesses need to understand that popular channels differ from one place to another what works in China or Russia might not be the same as global platforms like Facebook or Google. Overall, international marketing these days heavily relies on digital tools and the global reach, so companies need to be flexible, culturally aware, and smart with how they use their resources.

\printbibliography

\end{document}