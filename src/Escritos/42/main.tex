\documentclass[letterpaper, 12pt]{article}

\usepackage[utf8]{inputenc}
\usepackage[english]{babel}
\usepackage{fullpage}
\usepackage{graphicx}
\usepackage{amsmath}
\usepackage{enumitem}
\usepackage{chngcntr}
\usepackage{setspace}
\usepackage{url}
\usepackage{csquotes}
\usepackage{float}
\usepackage{verbatim}
\usepackage{tabularx}
\usepackage{amsmath}

\counterwithin{figure}{section}
\renewcommand{\thesection}{\arabic{section}}
\renewcommand{\thesubsection}{\thesection.\arabic{subsection}}
\renewcommand{\baselinestretch}{2}

\usepackage[style=apa, maxnames=2, minnames=1, backend=biber, parentracker=true, sorting=none]{biblatex}
\DefineBibliographyStrings{english}{%chktex-file 1 chktex-file 6
    andothers = {\em et\addabbrvspace al\adddot}
}
\addbibresource{./Bibliography/bibliography.bib}

\usepackage{array}
\usepackage{enumitem}
\usepackage{setspace}
\setlength{\parskip}{\baselineskip}

\begin{document}
\nocite{*}

\begin{titlepage}
    \centering
    \includegraphics[width=0.3\textwidth]{../../../images/logo_utb.png}
    \par\vspace{.2cm}

    {\scshape\LARGE Universidad Tecnológica de Bolívar \par}
    \vspace{0.5cm}

    {\scshape\Large International Marketing \par}
    \vspace{1.0cm}

    \slshape {\Large \bfseries{}STP NIKE\\}
    \vspace{1cm}

    \slshape {\itshape{} Valentina Daniela Del Rio Jimenez, T00067622 \\}
    \slshape {\itshape{} María Isabel Grau Solipa, T00064593 \\}
    \slshape {\itshape{} Yesid Alejandro Muñoz Saavedra, T00069159 \\}
    \slshape {\itshape{} Karoline Shekinna Herrera Álvarez, T00069005 \\}
    \slshape {\itshape{} Katryn Iglesia Rivas, T00068485 \\}
    \vfill

    Reviewed by \\
    Dayana Maria Sanchez Monterrosa\\
    {\large \today\par}
\end{titlepage}

\section{Segmentation}

Nike applies a multi-level segmentation strategy, combining demographic, geographic, psychographic, and behavioral criteria to capture different consumer profiles.

\subsection{Demographic Segmentation}

\subsubsection{Age}
Nike targets distinct age groups with tailored approaches:
\begin{itemize}
    \item \textbf{Youth (15--25 years):} This segment seeks identity, athletic fashion, and urban trends, representing a significant portion of Nike's target market focused on lifestyle and self-expression.
    \item \textbf{Adults (26--40 years):} Consumers with greater purchasing power, interested in performance and quality. This demographic prioritizes functionality and durability in athletic wear.
    \item \textbf{Children and Adolescents:} Nike Kids line focuses on school and sports footwear and apparel, targeting parents and young consumers with age-appropriate designs and functionality.
\end{itemize}

\subsubsection{Gender}
Nike develops differentiated products for men and women, including footwear, apparel, and accessories. In recent years, Nike has strengthened its women's offering through specific collections and empowerment campaigns, recognizing the growing female athletic market.

\subsubsection{Income}
The brand primarily targets middle and upper-class consumers who are willing to pay premium prices for prestige, quality, and innovation. This positioning allows Nike to maintain higher profit margins while reinforcing its premium brand image.

\subsection{Geographic Segmentation}

Nike maintains a global presence across more than 190 countries, with key markets including the United States, Europe, China, and Latin America. The company employs local adaptation strategies, adjusting campaigns according to cultural contexts and regional preferences.

Examples of geographic adaptation include:
\begin{itemize}
    \item \textbf{United States:} Sponsorship of major leagues such as the NBA and NFL
    \item \textbf{Latin America:} Support for national soccer teams and football culture
    \item \textbf{Asia:} Emphasis on athletic fashion and lifestyle trends
\end{itemize}

\subsection{Psychographic Segmentation}

Nike targets consumers with active lifestyles who are interested in sports, fitness, and wellness. The brand appeals to individuals whose personality traits include seeking personal improvement, motivation, performance, and authenticity.

Key psychographic segments include:
\begin{itemize}
    \item \textbf{Professional Athletes:} Seeking cutting-edge technology and innovation
    \item \textbf{Fitness Enthusiasts:} Prioritizing comfort, performance, and style
    \item \textbf{Urban Youth:} Pursuing fashion, social belonging, and aspirational lifestyle
\end{itemize}

\subsection{Behavioral Segmentation}

\subsubsection{Brand Loyalty}
Nike has cultivated a highly loyal consumer base, with many customers demonstrating strong brand affinity and repeat purchase behavior. This loyalty is reinforced through consistent brand messaging and quality products.

\subsubsection{Benefits Sought}
Consumers choose Nike for various benefits:
\begin{itemize}
    \item \textbf{Innovation:} Advanced footwear technologies such as Air, Zoom, and React cushioning systems
    \item \textbf{Prestige and Social Status:} Brand recognition and perceived quality
    \item \textbf{Inspiration and Motivation:} Embodied in the iconic ``Just Do It'' campaign
\end{itemize}

\subsubsection{Usage Frequency}
Nike serves different usage patterns:
\begin{itemize}
    \item \textbf{Intensive Users:} Athletes and individuals who train daily, requiring \\ high-performance products
    \item \textbf{Occasional Users:} Consumers who purchase Nike products primarily for fashion rather than athletic performance
\end{itemize}

\section{Targeting}

\subsection{Importance of Segmentation}

For Nike, market segmentation is fundamental rather than superfluous, as it enables the company to identify distinct consumer profiles and tailor its products, messaging, and marketing campaigns to each specific group. Nike transcends traditional sports retail by offering a comprehensive lifestyle brand, which necessitates a deep understanding of the unique needs and motivations across diverse demographics including youth, adults, women, children, and professional athletes.

\subsection{Multi-Segment Strategy}

Nike employs a comprehensive multi-segment approach rather than focusing on a single market. Through its extensive portfolio of footwear, apparel, and accessories, the company reaches consumers with varied interests and characteristics. This strategy encompasses urban youth seeking fashion and social belonging, high-performance athletes prioritizing technological innovation, and fitness enthusiasts valuing comfort and style. Nike's approach maintains a unified central message of motivation and inspiration embodied in the iconic ``Just Do It'' campaign while addressing segment-specific needs.

\subsection{Global vs. Local Segmentation}

Nike manages global segments including athletes, fitness enthusiasts, and fashion-conscious youth while adapting communication strategies and sponsorships to regional specificities. The company maintains consistent global brand identity while implementing localized strategies to enhance market connection. For instance, Nike emphasizes basketball in the United States, soccer in Latin America, and sports fashion in Asia, demonstrating cultural adaptation within a global framework.

\subsection{Cross-Country Segment Consistency}

While Nike's core segments remain fundamentally similar across markets—active individuals, athletes, identity-seeking youth, and aspirational consumers—priorities shift according to cultural and sporting contexts within each region. This approach allows Nike to maintain global brand coherence while optimizing local market effectiveness through targeted regional strategies.

\subsection{Segmentation Analysis}

\subsubsection{Market Size}

Nike's largest segmentation opportunities lie within demographic and geographic bases, encompassing diverse age groups (youth, adults, children), gender categories, and income levels across 190+ countries. This broad coverage provides substantial market scale. Psychographic segmentation, while strong in brand resonance, targets more specific lifestyle-oriented consumers. Behavioral segmentation addresses medium-large markets, focusing on brand loyalists and occasional users, though requiring significant investment in customer relationship management.

\subsubsection{Growth Potential}

Geographic segmentation presents the highest growth potential, particularly in emerging markets such as China and other developing economies. Psychographic segmentation also demonstrates strong growth prospects, driven by global wellness trends, fitness culture, and evolving lifestyle preferences. Demographic and behavioral segments show steady but moderate growth rates, dependent on broader social changes and technological innovation adoption.

\subsubsection{Competitive Landscape}

Nike faces varying competitive intensity across segmentation bases. Psychographic segmentation provides the strongest differentiation opportunity through Nike's distinctive brand identity and lifestyle positioning. Behavioral competition is moderately high, as competitors like Adidas, Puma, and Under Armour also emphasize loyalty, innovation, and modernity. Demographic competition remains intense across all major athletic brands, while geographic competition includes both global competitors and strong regional players in local markets.

\subsubsection{Profitability Analysis}

Profitability peaks where Nike captures high-spending demographic groups through premium lifestyle positioning. Demographic segmentation yields high margins from adult consumers and the growing women's market. Psychographic segmentation enables premium pricing through lifestyle branding and influencer partnerships. Behavioral segmentation generates high profitability from loyal athletes but moderate returns from casual buyers. Geographic profitability remains strong in developed markets but faces margin pressure in emerging economies due to price sensitivity and local competition.

\subsubsection{Market Accessibility}

Demographic and psychographic segments offer the highest accessibility through Nike's established marketing infrastructure. The company maintains strong market presence with clear product differentiation and universal brand recognition anchored by the ``Just Do It'' message. Geographic and behavioral targeting require more sophisticated strategies, including culturally adapted localized campaigns and personalized customer retention programs, demanding greater resource investment and strategic complexity.

\section{Brand Positioning}

Nike defines its mission as ``to bring inspiration and innovation to every athlete in the world''operating under the philosophy that ``if you have a body, you are an athlete''. This comprehensive vision integrates functional components, such as technological innovation, with symbolic elements that embody inspiring values. The iconic slogan ``Just Do It'' has evolved into a global symbol associated with performance, innovation, and empowerment.

Since its inception, Nike has cultivated the concept of personal achievement. When launching ``Just Do It'' in 1988, the campaign featured an 80-year-old runner with the message ``Sport is for everyone'', transforming the phrase into a motivational mantra encouraging individuals to ``move, dream, and dare''. Through this approach, Nike has built a positioning strategy that transcends the product itself, appealing to self-improvement and inclusion.

\subsection{Functional Positioning}

Nike invests substantially in research and development, focusing on sports technology innovation. The company integrates advanced technologies such as Nike Air, Flyknit, and Dri-FIT, which enhance performance, cushioning, and product durability. Additionally, Nike designs sport-specific footwear and apparel, providing users with specialized tools to optimize their athletic performance.

This commitment to functionality supports the brand's perceived quality, justifying premium pricing and attracting athletes seeking high-performance equipment. The technological advancement serves as a cornerstone of Nike's competitive advantage in the athletic wear market.

\subsection{Symbolic Positioning}

Simultaneously, Nike positions itself as a symbol of status, triumph, and personal empowerment. The ``Just Do It'' campaign inspires both professional athletes and everyday individuals to transcend their limitations. The 1995 campaign ``If You Let Me Play'' demonstrated how sports participation can transform girls' lives, delivering a powerful message of empowerment and social change.

Strategic partnerships with sports icons including Michael Jordan, LeBron James, and other elite athletes reinforce Nike's image of excellence. These figures embody the success values that the brand represents, creating aspirational connections with consumers. Nike has successfully positioned its products as symbols of exclusivity, with ownership often perceived as ``a symbol of status and achievement''.

Nike's communication strategy emphasizes aspirational narratives through storytelling that connects consumers with a ``champion's mindset'' rather than focusing solely on specific product attributes.

\subsection{Experiential Positioning}

Nike strategically creates emotional connections and memorable experiences through immersive brand activations. The company executes pop-up stores, exclusive product launches, and sporting events that generate direct consumer interaction with the brand. Community-building initiatives such as organized races and training sessions through Nike Run Club and Nike Training Club allow users to exercise collectively while fostering a sense of belonging to an active community.

In the digital realm, applications like Nike Training Club offer personalized content and challenges, creating enriching user experiences. Nike also produces emotionally resonant advertisements, such as ``Dream Crazy'' featuring Colin Kaepernick, which strengthen emotional connections with consumers by addressing social causes and motivational narratives.

The comprehensive experiential marketing strategy, combined with high-quality content and strategic hashtag campaigns (\#JustDoIt, \#Breaking2), fosters loyal communities and emotional brand identification.

\subsection{Campaign Examples}

\subsubsection{``Just Do It'' (1988--present)}
This iconic slogan motivates individuals to initiate and persevere in their athletic endeavors. The inaugural commercial featured an elderly man running and declared that ``sport is for everyone'', establishing an inclusive and inspiring brand message that continues to resonate globally.

\subsubsection{``If You Let Me Play'' (1995)}
A pioneering campaign for female empowerment that demonstrated how sports access ``can change girls' lives''. This campaign reinforced Nike's commitment to social values and gender equality in athletics.

\subsubsection{``Dream Crazy'' (2018)}
Featuring Colin Kaepernick, this campaign associated Nike with social justice causes. \\ Through bold and emotional messaging, it addressed themes of perseverance in the face of adversity, strengthening audience emotional connection with the Nike brand while taking a stand on important social issues.

\subsubsection{Nike Run/Training Club}
These ongoing digital and physical initiatives offer guided training programs and shared challenges. The platforms and in-person events consolidate the brand experience by encouraging healthy lifestyle habits within an active, engaged community.


\newpage
\printbibliography

\end{document}