\documentclass[letterpaper, 11pt]{report}

\usepackage[utf8]{inputenc}
\usepackage[english, spanish]{babel}

\usepackage{fullpage}
\usepackage{graphicx}
\usepackage{amsmath}
\usepackage{enumitem}
\usepackage{chngcntr}
\usepackage{setspace}
\usepackage{xurl}
\usepackage{csquotes}
\usepackage{float}
\usepackage{verbatim}
\usepackage{tabularx}
\usepackage{amsmath}
\usepackage{caption}
\usepackage{bm}
\usepackage{wrapfig}
\usepackage{siunitx}
\usepackage{array}
\usepackage{enumitem}

\counterwithin{figure}{section}
\renewcommand{\thesection}{\arabic{section}}
\renewcommand{\thesubsection}{\thesection.\arabic{subsection}}
\renewcommand{\baselinestretch}{1.5}
\renewcommand{\thefigure}{\arabic{figure}}

\usepackage[style=apa, maxnames=6, minnames=3]{biblatex}
\DefineBibliographyStrings{english}{%chktex-file 1 chktex-file 6
      andothers = {\em et\addabbrvspace al\adddot}
}

\addbibresource{./Bibliography/bibliography.bib}


\setlength{\parskip}{\baselineskip}

\newcommand{\bolditalic}[1]{\textbf{\textit{#1}}}

\renewcommand{\comment}[1]{{\small $\ll$#1$\gg$}}

% chktex-file 24

\begin{document}

\begin{titlepage}
      \centering
      \includegraphics[width=0.3\textwidth]{Images/logo_utb.png}\par\vspace{1cm}
      {\scshape\LARGE Universidad Tecnológica de Bolívar \par}
      \vspace{1cm}

      {\scshape\Large Seminario de Investigación \par}
      \vspace{1cm}

      \slshape {\Large \bfseries{} Actividad IV\\}
      \vspace{2cm}

      \slshape {\itshape{} Valentina Daniela Del Rio Jimenez \\}
      \slshape {\itshape{} Alanys Marin Frias \\}
      \vfill
      Revisado Por \\
      Katty Gómez Acevedo\\
      {\large \today\par}
\end{titlepage}

\nocite{*}

\section*{Introducción}

En esta actividad, se analizará el marco teórico del artículo
\textit{``Explorando el vínculo entre capacidades financieras y aprendizaje
      organizacional en pymes: un análisis del estado del arte''}, publicado en la
revista científica \textit{CEA} del Instituto Tecnológico Metropolitano (ITM).
A partir de este estudio, se evaluará la estructura del marco teórico, su
pertinencia en relación con el problema de investigación y su contribución al
desarrollo del estudio.

\section*{Índice \textit{(explícito o implícito)} del marco teórico}

El artículo no presenta un índice explícito del marco teórico, pero a partir de
su estructura se pueden identificar los siguientes componentes implícitos:

\begin{enumerate}
      \item Capacidades financieras en pymes: Se definen y contextualizan como el conjunto
            de recursos y conocimientos financieros que permiten una gestión eficiente en
            pequeñas y medianas empresas.

      \item Aprendizaje organizacional en pymes: Se explica cómo las empresas adquieren,
            procesan y aplican conocimientos para mejorar su desempeño.

      \item Relación entre capacidades financieras y aprendizaje organizacional: Se
            justifica la importancia de integrar ambas áreas para mejorar la toma de
            decisiones estratégicas.

      \item Estado del arte y vacíos en la literatura: Se identifican estudios previos
            sobre la temática y se destacan oportunidades de investigación en contextos
            específicos, como América Latina.

\end{enumerate}

\section*{¿El marco teórico está completo?}

El marco teórico del artículo es sólido y bien estructurado, ya que proporciona
definiciones claras sobre las capacidades financieras y el aprendizaje
organizacional en pymes, permitiendo una comprensión precisa de ambos
conceptos. Además, incluye una revisión exhaustiva de estudios previos, lo que
respalda teóricamente la investigación y facilita la identificación de patrones
y tendencias en el tema. Un aspecto relevante es que establece una relación
directa entre las capacidades financieras y el aprendizaje organizacional,
demostrando su impacto en la sostenibilidad y competitividad de las pymes.
Asimismo, el artículo reconoce vacíos en la literatura, destacando la necesidad
de profundizar en esta relación, especialmente en contextos específicos como
América Latina. Sin embargo, aunque el marco teórico es sólido, sería más
completo si incorporara estudios de caso o evidencia empírica que ejemplifiquen
cómo las pymes aplican sus capacidades financieras para mejorar el aprendizaje
organizacional en distintos sectores, permitiendo así un análisis más concreto
y aplicable a la realidad empresarial.

\section*{¿Está relacionado con el problema de investigación?}

El marco teórico está completamente alineado con el problema de investigación,
ya que explica de manera clara y fundamentada la importancia de analizar la
relación entre las capacidades financieras y el aprendizaje organizacional en
las pymes. Además, justifica cómo esta conexión influye en su sostenibilidad y
competitividad, permitiéndoles adaptarse a entornos dinámicos y mejorar su
gestión estratégica. Asimismo, proporciona referencias teóricas sólidas que
respaldan la investigación, garantizando un enfoque estructurado y basado en
estudios previos que refuerzan la relevancia del tema abordado, explora la
conexión o vínculo entre las CF y el AO en las PYMES, y analiza cómo impactan
la gestión y la sostenibilidad, importante para evolucionar y enfrentar los
desafíos.

Por igual, integra referencias a estudios previos sobre el problema de
investigación, vinculando las teorías con aplicaciones específicas.

\section*{¿Ayudó a los investigadores en su estudio? ¿De qué manera?}

El marco teórico fue fundamental en el desarrollo del estudio, ya que sirvió
como guía para la revisión bibliográfica, permitiendo identificar estudios
previos relevantes y estructurar la investigación de manera coherente. Además,
presentó teorías y conceptos acerca de CF y AO, permitiéndole a los
investigadores identificar, buscar y seleccionar literatura relevante para su
revisión sistemática sobre estudios previos de estas temáticas. Brindó un marco
referencial de CF y AO, permitiéndole a los investigadores interpretar mejor
los resultados del estudio.

\printbibliography

\end{document}