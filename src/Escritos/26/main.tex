\documentclass[letterpaper, 12pt]{report}

\usepackage[utf8]{inputenc}
\usepackage[english, spanish]{babel}
\usepackage{fullpage} % changes the margin
\usepackage{graphicx}
\usepackage{amsmath}
\usepackage{enumitem}
\usepackage{chngcntr}
\usepackage{setspace}
\usepackage{url}
\usepackage{csquotes}
\usepackage{float}
\usepackage{verbatim}
\usepackage{tabularx}
\usepackage{amsmath}

\counterwithin{figure}{section}
\renewcommand{\thesection}{\arabic{section}}
\renewcommand{\thesubsection}{\thesection.\arabic{subsection}}
\renewcommand{\baselinestretch}{2}

\usepackage[style=apa, maxnames=6, minnames=3, backend=biber, parentracker=true, sorting=none]{biblatex}
\DefineBibliographyStrings{english}{%chktex-file 1 chktex-file 6
    andothers = {\em et\addabbrvspace al\adddot}
}
\addbibresource{./Bibliography/bibliography.bib}

\usepackage{array}
\usepackage{enumitem}
\usepackage{setspace}
\setlength{\parskip}{\baselineskip}

\begin{document}

\chapter*{Testeo en animales por parte de investigaciones médicas y de fármacos}

\noindent\makebox[\linewidth]{\rule{\textwidth}{0.4pt}}

\noindent
Valentina Del Rio Jimenez, T00081360 \\
Programa: Finanzas y negocios internacionales

\noindent\makebox[\linewidth]{\rule{\textwidth}{0.4pt}}

\nocite{*}

La comprensión a los demás en una sociedad acelerada y sin suficiente reflexión
se convierte en todo un reto, a pesar de la innegable naturaleza comunitaria
del ser humano, a menudo se relega la empatía, es decir, la capacidad de
ponerse en el lugar del otro. La dificultad que enfrentamos a nuestras propias
falencias para comprender y relacionarnos armoniosamente como sociedad refleja
este desequilibrio.

La historia no miente. Hemos presenciado conflictos y escasez que han sido,
en gran medida, resueltos por avances tecnológicos que nos facilitan la vida
y se presentan como revolucionarios. Irónicamente, la tecnología no ha
solucionado nuestras diferencias internas. Nuestra tendencia a priorizar
nuestra propia supervivencia sobre la de otros seres vivos ha generado
consecuencias devastadoras para la naturaleza evidenciando esta desconexión
con nuestro entorno y con otras especies.

En este sentido, surge un interrogante crucial para desafiar nuestros
principios: ¿Es ético y necesario seguir utilizando pruebas en animales
en el campo de la investigación médica y farmacéutica? Aunque anteriormente
esta práctica se justificaba por la necesidad de avances científicos, hoy en
día, con los avances tecnológicos, se plantea la posibilidad de alternativas
más justas para esos seres vivos que no tienen voz, y, por supuesto, más
éticas.

Es necesario reconsiderar nuestras acciones y elegir métodos que respeten
la vida humana, así como la de otras especies. Es claro que el testeo en
animales plantea dilemas éticos que requieren revisión. Aunque es vital
asegurar la seguridad de los productos, existen alternativas éticas, con
avances tecnológicos, las pruebas podrían realizarse en simulaciones basadas
en IA.~Encontrar alternativas éticas y confiables es una obligación moral y
legal, ya que la experimentación animal implica sufrimiento
innecesario~(\cite{Spielmann2001}).

La Ley 84 de 1989 indica un significativo avance en la protección de
los animales en el campo de la investigación científica en Colombia.
Al reconocer el bienestar animal como un valor fundamental, establece
principios rectores basados en las ``3 Rs'': Reducir, reemplazar y refinar
el uso de animales en experimentación. Además, fomenta la creación de
métodos alternativos al uso de animales en pruebas~\textcite{Ley84de1989}.

Aunque han pasado más de tres décadas desde la promulgación de esta ley,
todavía existe una brecha entre la normativa legal y las prácticas de
ciertas empresas colombianas de la industria cosmética. Múltiples empresas
siguen realizando pruebas en animales, lo que ha generado preocupaciones
sobre el cumplimiento de las leyes y el bienestar de los animales implicados.
La discrepancia resulta desalentadora, ya que la intención de la ley es
resolver el problema en lugar de perpetuarlo.

Hay que destacar que ciertas marcas presentes en el territorio nacional, como
\textit{Esika} y Natura, han optado por una postura proactiva al eliminar las pruebas
en animales de sus procesos y productos es fundamental. Estas empresas
muestran que se puede lograr el éxito sin poner en peligro el bienestar de
los animales, y sirven como un ejemplo a seguir para las demás empresas del
sector~(\cite{FundacionAmigosdelPlaneta2023}).

La eliminación de la dependencia del testeo animal representa una vía
prometedora gracias a los avances tecnológicos, no solo en Colombia, sino
a nivel global. La inteligencia artificial (IA) basada en simulaciones tiene
un gran potencial para sustituir los métodos convencionales de experimentación.

Es posible plantear una objeción posible respecto a la incertidumbre sobre
si el uso exclusivo de simulaciones es eficaz. ¿De qué manera podemos
garantizar que las simulaciones sean 100\% efectivas? Es posible plantear
que, la clave está en invertir suficiente tiempo y recursos. Aumentar la
confiabilidad de los resultados se logra al permitir realizar un mayor
número de simulaciones y reducir el margen de error.

Es válido, desde una perspectiva más conservadora, considerar si la
dependencia de la tecnología nos vuelve más vulnerables. No obstante, es
fundamental tener en cuenta que, desde el principio, la tecnología ha tenido
como objetivo proporcionar herramientas que mejoren la calidad de vida
humana. El uso de simulaciones basadas en IA para reducir el sufrimiento
animal no solo se ajusta a este objetivo, sino que también representa un
paso adelante hacia prácticas más éticas y sostenibles en la investigación
científica y el desarrollo de productos~(\cite{HerrmannJayne2019}).

La decisión de eliminar las pruebas en animales no solo se basa en
consideraciones éticas, sino que también potencia la conexión entre las
marcas y sus clientes. Un ejemplo de esto es The Body Shop. La cual se ha
destacado por su compromiso con la ética en la fabricación de productos
cosméticos desde que comenzó. Al decidir no realizar pruebas en animales,
la marca no solo sigue sus principios fundamentales, sino que también se
destaca en el mercado como la elección preferida para los consumidores
conscientes. La calidad de los productos, así como la integridad detrás de
su elaboración, son valoradas por una sólida base de clientes generada a
través de este enfoque. La asociación con la causa libre de crueldad animal
ha probado ser una estrategia efectiva para fomentar la lealtad de los
clientes y destacarse en un mercado cada vez más competido.

Por otro lado, Lush ha basado su marca en la frescura y la ética. La cual
ha capturado la atención y el aprecio de los consumidores conscientes al
adoptar prácticas de fabricación que evitan el testeo en animales y se
centran en ingredientes frescos y naturales. El compromiso con la
sostenibilidad y la transparencia en sus procesos de producción también
han ayudado a generar una percepción positiva de la marca. Dicha mentalidad
no solo ha llevado a una base fiel de clientes que eligen apoyar activamente
la marca por sus valores compartidos, sino también por la calidad
de sus productos.

Por último, KVD Vegan Beauty ha mostrado que dejar de realizar pruebas en
animales puede ser una estrategia efectiva para mejorar la reputación de una
empresa entre sus clientes. La marca ha atraído a consumidores que valoran
tanto la calidad del maquillaje como el compromiso con el bienestar animal
al ofrecer una línea completa de productos veganos y libres de crueldad
animal. La construcción de una imagen sólida y diferenciada en un mercado
competitivo ha sido contribuida por este enfoque ético. La transparencia y
autenticidad sobre sus valores éticos han aumentado la confianza del
consumidor y promovido relaciones duraderas con sus clientes.

Ha habido un largo debate ético sobre la experimentación animal en la
investigación científica. A pesar de que ha facilitado avances médicos
significativos, su uso plantea serias preocupaciones sobre el bienestar
animal. Es necesario reconsiderar esta práctica y encontrar opciones éticas
que respeten la vida y dignidad de los animales.

La tecnología, en particular las simulaciones basadas en inteligencia
artificial, proporciona un camino prometedor para disminuir
considerablemente la dependencia de la experimentación animal. La transición
hacia métodos más humanitarios es tanto un imperativo moral como una
oportunidad para avanzar hacia un futuro en el que el progreso científico
no entre en conflicto con el respeto por la vida animal.

Es tiempo de aplicar un enfoque ético y responsable en la investigación
científica, que permita generar conocimiento sin ocasionar sufrimiento
innecesario. Es esencial este cambio de paradigma para construir un mundo
más justo y compasivo, en el que la ciencia y la ética convivan en armonía
para el beneficio de todas las formas de vida en nuestro planeta.

\printbibliography

\end{document}