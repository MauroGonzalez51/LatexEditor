\documentclass[letterpaper, 12pt]{report}

\usepackage[utf8]{inputenc}
\usepackage[english]{babel}
\usepackage{fullpage}
\usepackage{graphicx}
\usepackage{amsmath}
\usepackage{enumitem}
\usepackage{chngcntr}
\usepackage{setspace}
\usepackage{url}
\usepackage{csquotes}
\usepackage{float}
\usepackage{verbatim}
\usepackage{tabularx}
\usepackage{amsmath}

\counterwithin{figure}{section}
\renewcommand{\thesection}{\arabic{section}}
\renewcommand{\thesubsection}{\thesection.\arabic{subsection}}
\renewcommand{\baselinestretch}{2}

\usepackage[style=apa, maxnames=6, minnames=3, backend=biber, parentracker=true, sorting=none]{biblatex}
\DefineBibliographyStrings{english}{%chktex-file 1 chktex-file 6
    andothers = {\em et\addabbrvspace al\adddot}
}
\addbibresource{./Bibliography/bibliography.bib}

\usepackage{array}
\usepackage{enumitem}
\usepackage{setspace}
\setlength{\parskip}{\baselineskip}

\begin{document}
\nocite{*}

\chapter*{Netflix Case Analysis}

\noindent
Valentina Del Rio Jimenez, T00081360 \\
María Isabel Grau Solipa, T00064593 \\
Yesid Alejandro Muñoz Saavedra, T00069159 \\

\section{What are the strengths and weaknesses that Netflix had when implementing its strategy? }

One of Netflix’s greatest strengths during its international expansion was its
gradual approach, often called a phased expansion model. The company began in
culturally and geographically similar markets, such as Canada in 2010, where
the challenges of “foreignness” were relatively low (Brennan, 2018). This
step-by-step strategy allowed Netflix to build its internationalization
capabilities and learn how to adapt its services before entering more diverse
and distant markets. In addition, Netflix’s technological foundation
(especially its data-driven algorithms and personalization system) enabled it
to offer content tailored to different audiences, increasing customer
engagement. Another significant advantage was its partnerships with local
companies, such as Vodafone in Ireland and Telefónica in Latin America, which
facilitated trust-building and easier market entry.

Nonetheless, Netflix also encountered weaknesses within its strategy. The
company was constrained by its dependence on negotiating content licenses
region by region, which delayed its ability to launch globally. Furthermore,
linguistic and cultural barriers presented difficulties, since many audiences
demanded local-language programming. Its subscription model also faced
resistance in markets where free or pirated media was common. These weaknesses
highlight that Netflix’s strategy, though innovative, required constant
adaptation to local realities and consumer behaviors.

\section{What challenges did Netflix face at the time of its international expansion?}

Netflix’s global growth was accompanied by numerous challenges. Regulatory
restrictions were one of the main barriers, as rules about media distribution
varied from country to country (Brennan, 2018). Cultural and linguistic
differences added another layer of complexity, with many viewers showing a
preference for localized content rather than English-dominated programming.
Strong local competition also posed difficulties, as companies such as Hotstar
in India and Canal Play in France had already secured market share through
local-language offerings. In addition, global competitors like Amazon Prime
Video were often already established in strategic markets, further limiting
Netflix’s first-mover advantage. Finally, in regions where free content was
prevalent, consumers were hesitant to pay for subscriptions, forcing Netflix to
innovate in pricing and accessibility. Collectively, these challenges required
Netflix to remain highly flexible and customer centric.

\section{Conduct an industry analysis that includes an analysis of the company, the customer/consumer, and the competition.}

\begin{itemize}
    \item \textbf{Company}: Netflix’s international expansion can be analyzed through three dimensions. At the company level, Netflix positioned itself as a global leader in online streaming through innovative technology, data-driven personalization, and strong partnerships. Its ability to rapidly expand to 190 countries within seven years reflects its organizational adaptability. Moreover, its strategic investment in global hits like \textit{Money Heist and Narcos}, combined with local productions in various regions, allowed it to appeal to both international and local audiences.

    \item \textbf{Customer/Consumer}: Viewers from different countries showed very diverse
          preferences. While some enjoyed big international hits, many others looked for
          shows in their own language and content that matched their culture. In regions
          where most people used mobile devices, Netflix had to improve its mobile
          streaming quality and pricing options to meet these needs.

    \item \textbf{Competition}: When it comes to competition, Netflix has faced tough challenges
          from giants like Amazon Prime Video and Disney+, along with many local
          favorites across different regions. These competitors often had the upper hand
          thanks to their personalized local content and well-established distribution
          networks. Still, Netflix managed to stand out by mixing international stories
          with original local productions, helping it connects with a wide audience and
          niche viewers alike~(\cite{brennan2018netflix}).

\end{itemize}

In conclusion, Netflix’s global expansion demonstrates how a balance between
global standardization and local adaptation is essential in international
strategy. Its strengths in technology, partnerships, and phased learning
allowed it to overcome weaknesses such as licensing barriers and cultural gaps.
At the same time, it faced formidable challenges from regulations, competitors,
and diverse consumer behaviors. Netflix’s ability to remain customer-centric,
flexible, and innovative ultimately positioned it as a leading force in the
streaming industry.

\printbibliography

\end{document}