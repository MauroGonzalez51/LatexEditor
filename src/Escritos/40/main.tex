\documentclass[letterpaper, 11pt]{report}

\usepackage[utf8]{inputenc}
\usepackage[english, spanish]{babel}

\usepackage{fullpage}
\usepackage{graphicx}
\usepackage{amsmath}
\usepackage{enumitem}
\usepackage{chngcntr}
\usepackage{setspace}
\usepackage{xurl}
\usepackage{csquotes}
\usepackage{float}
\usepackage{verbatim}
\usepackage{tabularx}
\usepackage{amsmath}
\usepackage{caption}
\usepackage{bm}
\usepackage{wrapfig}
\usepackage{siunitx}
\usepackage{array}
\usepackage{enumitem}
\usepackage{longtable}



% \usepackage[margin=0.5in]{geometry}

\counterwithin{figure}{section}
\renewcommand{\thesection}{\arabic{section}}
\renewcommand{\thesubsection}{\thesection.\arabic{subsection}}
\renewcommand{\baselinestretch}{1.5}
\renewcommand{\thefigure}{\arabic{figure}}

\usepackage[style=apa, maxnames=6, minnames=3]{biblatex}
\DefineBibliographyStrings{english}{%chktex-file 1 chktex-file 6
      andothers = {\em et\addabbrvspace al\adddot}
}

\addbibresource{./Bibliography/bibliography.bib}


\setlength{\parskip}{\baselineskip}

\newcommand{\bolditalic}[1]{\textbf{\textit{#1}}}

\renewcommand{\comment}[1]{{\small $\ll$#1$\gg$}}

\renewcommand{\arraystretch}{1.3}
\setlength{\LTpre}{0pt}
\setlength{\LTpost}{0pt}

% chktex-file 8
% chktex-file 24
% chktex-file 44

\begin{document}

\begin{titlepage}
      \centering
      \includegraphics[width=0.3\textwidth]{../../../images/logo_utb.png}\par\vspace{1cm}
      {\scshape\LARGE Universidad Tecnológica de Bolívar \par}
      \vspace{1cm}

      {\scshape\Large Seminario de investigación, universidad tecnológica de bolívar  \par}
      \vspace{2cm}

      \slshape {\Large \bfseries{} Deducción de ideas de investigación desde artículos científicos  \\}
      \vspace{3cm}

      \slshape {\itshape{} Valeria Gómez \\}
      \slshape {\itshape{} Samuel Cárdenas \\}

      \vfill
      Revisado Por \\
      Katty Johanna Gómez \\
      {\large \today\par}
\end{titlepage}

\nocite{*}

El artículo \textit{``La imagen testigo en la investigación social. Lecturas de
      una secuencia compleja''}, de Adolfo Baltar-Moreno y Cielo Patricia
Puello-Sarabia, presenta un material de investigación cualitativa resultado de
un trabajo de campo en El Carmen de Bolívar. En él se plantea la importancia
del audiovisual como herramienta metodológica para comprender fenómenos
sociales, resaltando cómo los datos visuales y la etnografía audiovisual
permiten captar experiencias, significados y dinámicas comunitarias desde una
perspectiva cercana y participativa.

Ideas de investigación que se desprenden del artículo:

\begin{itemize}
      \item El audiovisual como medio para producir conocimiento social y no solo como
            registro.
      \item El valor de los datos visuales en la investigación cualitativa.
      \item La etnografía audiovisual como alternativa metodológica para estudiar
            comunidades.
      \item La co-construcción del conocimiento entre investigador y participantes.
      \item El uso de muestreos flexibles (teórico, de oportunidad y en cadena) adaptados
            al contexto.
      \item La democratización del conocimiento a través de formatos accesibles que
            trasciendan lo escrito.
\end{itemize}

En el siguiente articulo: \textit{``Análisis de las tendencias de los perfiles
      de egresados de los programas de administración de empresas de la ciudad de
      Cartagena''}. Podemos notar como tiene un carácter cuantitativo, a través del
análisis de tendencias, expresado a través del análisis de los perfiles de los
egresados en la ciudad de Cartagena; determinando si están en la capacidad de
atender las exigencias y necesidades del entorno, las cuales están enfocadas en
la producción de la ciudad.

Ideas de investigación que se perciben en el artículo:

\begin{itemize}
      \item Identificar las tendencias de los egresados para atender, afrontar y superar
            los retos del mundo empresarial
      \item Analizar como el programa de Administración de Empresas en Cartagena está
            diseñado para responder a las necesidades del contexto económico y productivo
            del entorno.
      \item El uso de la vigilancia tecnológica sirve como herramienta para identificar
            tendencias y necesidades en los programas académicos de Administración de
            empresas.
      \item La medida en que los perfiles de egresados se alinean con las apuestas
            productivas y las necesidades de competitividad de Cartagena y Bolívar.
      \item Identificar características curriculares, como la gestión de procesos, gestión
            estratégica, innovación y proyectos predominan en el perfil de los egresados de
            estos programas.
\end{itemize}

\printbibliography

\end{document}