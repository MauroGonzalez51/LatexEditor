\documentclass[twocolumn, 12pt]{article}

\usepackage[utf8]{inputenc}
\usepackage[english, spanish]{babel}
\usepackage{fullpage}
\usepackage{graphicx}
\usepackage{amsmath}
\usepackage{enumitem}
\usepackage{chngcntr}
\usepackage{setspace}
\usepackage{url}
\usepackage{csquotes}
\usepackage{float}
\usepackage{verbatim}
\usepackage{tabularx}
\usepackage{amsmath}
\usepackage{caption}
\usepackage{bm}

\counterwithin{figure}{section}
\renewcommand{\thesection}{\arabic{section}}
\renewcommand{\thesubsection}{\thesection.\arabic{subsection}}
\renewcommand{\baselinestretch}{1.5}

\usepackage[style=apa, maxnames=6, minnames=3, backend=biber]{biblatex}
\DefineBibliographyStrings{english}{%chktex-file 1 chktex-file 6
    andothers = {\em et\addabbrvspace al\adddot}
}
\addbibresource{./Bibliography/bibliography.bib}

\usepackage{array}
\usepackage{enumitem}

\setlength{\parskip}{0pt}

\raggedbottom{}

\newcommand{\bolditalic}[1]{\textbf{\textit{#1}}}

\begin{document}

\begin{titlepage}
    \centering
    \includegraphics[width=0.3\textwidth]{Images/logo_utb.png}\par\vspace{1cm}
    {\scshape\LARGE Universidad Tecnológica de Bolívar \par}
    \vspace{1cm}

    {\scshape\Large FÍSICA ELÉCTRICA \par}
    \vspace{.2cm}

    % chktex-file 8
    {\scshape\Large INSERTE GRUPO \par}
    \vspace{1cm}
    % chktex-file 8
    \slshape {\Large \bfseries{}LAB 1 - OSCILACIONES MECÁNICAS:\@{}~MOVIMIENTO ARMÓNICO SIMPLE.\\}
    \vspace{4cm}

    \slshape {\itshape{} Mauro González, T00067622 \\}
    % \slshape {\itshape{} German De Armas Castaño, T00068765 \\}
    % \slshape {\itshape{} Angel Vega Rodriguez, T00068186 \\}
    % \slshape {\itshape{} Juan Jose Osorio Ariza, T00067316 \\}
    % \slshape {\itshape{} Jorge Alberto Rueda Salgado, T00068722 \\}
    \vfill
    Revisado Por \\
    Gabriel Hoyos Gomez Casseres\\
    {\large \today\par}
\end{titlepage}

% ! ----------------------------------------------------------------------|>
\section{Introducción}

% ! ----------------------------------------------------------------------|>
\section{Objetivos}

\subsection{Objetivo general}

\subsection{Objetivos específicos}

% ! ----------------------------------------------------------------------|>
\section{Preparación de la practica}

% * ----------------------------------------------------|>
\subsection{¿Que es una oscilación?}

Una oscilación permite representar a los movimientos de
tipo vaivén a la manera de un péndulo o, dicho de
determinados d fenómenos, a la intensidad que se acrecienta
y disminuye de forma alternativa con más o menos
regularidad.

En diversos campos vinculados a la ciencia, la oscilación
consiste en la transformación, alteración, perturbación o
fluctuación de un sistema a lo largo del tiempo.

\nocite{definicion-oscilacion}

% * ----------------------------------------------------|>
\subsection{¿Qué es una oscilación armónica?}

Decimos que un objeto experimenta oscilación o vibración
cuando se mueve de manera repetitiva en relación a una
posición de equilibrio, bajo la influencia de fuerzas que
tienden a restaurar su posición original.

Cuando estas fuerzas restauradoras son proporcionales a la
distancia desde el punto de equilibrio, se produce un
fenómeno llamado movimiento armónico simple
(\textit{m.a.s}), que también es conocido como movimiento
vibratorio armónico simple (\textit{m.v.a.s.}). En general,
estas fuerzas restauradoras obedecen la ley de Hooke:

{\large
\begin{equation}
    \vec{F} = -\vec{\textit{k}} \cdot \vec{x}
    \label{eq:ley-de-hooke}
\end{equation}
}

\subsubsection{Caracteristicas del movimiento armonico simple}

\begin{enumerate}
    \item Vibratorio: El cuerpo oscila en torno a una posición de
          equilibrio siempre en el mismo plano

    \item Periódico: El movimiento se repite cada cierto tiempo
          denominado periodo (T). Es decir, el cuerpo vuelve a tener
          las mismas magnitudes cinemáticas y dinámicas cada T
          segundos

    \item Se describe mediante una función sinusoidal (seno o coseno
          indistintamente) {\large
                  \begin{equation}
                      \begin{gathered}
                          x = A \cdot \cos(\omega \cdot t + \phi_0) \\
                          x = A \cdot \sen(\omega \cdot t + \phi_0) \\
                      \end{gathered}
                  \end{equation}
              }
\end{enumerate}

A la partícula o sistema que se mueve según un movimiento
armónico simple se les denomina \textit{oscilador armónico}

\nocite{movimiento-armonico-simple}

% * ----------------------------------------------------|>
\subsection{¿Cuáles son las características de una oscilación armónica (amplitud, periodo, frecuencia,
    frecuencia cíclica (o angular), fase inicial (o constante de fase))?}

\begin{itemize}[label=$\triangleright$]
    \item Amplitud [{\large $A$}]: Elongación máxima. Su unidad de
          medidas en el Sistema Internacional es el metro
          \bolditalic{(m)}.

    \item Periodo [{\large $T$}]: El tiempo que tarda en cumplirse
          una oscilación completa. Es la inversa de la frecuencia T =
          1/f. Su unidad de medida en el Sistema Internacional es el
          segundo \bolditalic{(s)}.

    \item Frecuencia [{\large $f$}]: El número de oscilaciones o
          vibraciones que se producen en un segundo. Su unidad de
          medida en el Sistema Internacional es el Hercio (Hz). 1 Hz
          = 1 oscilación / segundo = $1 s^{-1}$.

    \item Frecuencia angular [{\large $\omega$}]: Representa la
          velocidad de cambio de la fase del movimiento. Se trata del
          número de periodos comprendidos en $2\pi$ segundos. Su
          unidad de medida en el sistema internacional es el radián
          por segundo (\@~rad/s\@~). Su relación con el período y la
          frecuencia es: {\large
          \begin{equation}
              \omega = \frac{2 \cdot \pi}{T} = 2 \cdot \pi f
          \end{equation}
          }

    \item Fase [{\large $\phi$}]: Se trata del ángulo que representa
          el estado inicial de vibración, es decir, la elongación x
          del cuerpo en el instante t = 0. Su unidad de medida en el
          Sistema Internacional es el radián (rad).
\end{itemize}

\nocite{movimiento-armonico-simple}

% * ----------------------------------------------------|>
\subsection{¿Cómo se calcula el periodo de oscilación de los péndulos simple, de resorte y uno compuesto
    o físico?}

% + -----------------------------|>
\subsubsection{Pendulo simple}

El periodo de un péndulo simple depende de su longitud y de
la aceleración debido a la gravedad. El periodo es
completamente independiente de otros factores, como masa y
desplazamiento máximo.

{\large
\begin{equation}
    T = 2\pi \cdot \sqrt{\frac{L}{g}}
\end{equation}
}

Donde, \bolditalic{L}, es la longitud del péndulo y
\bolditalic{g}, es la gravedad.

\nocite{tipos-pendulos-1}

% + -----------------------------|>
\subsubsection{Pendulo de resorte}

El período de un péndulo de resorte se refiere al tiempo
que le toma a una masa conectada a un resorte completar un
ciclo completo de oscilación, yendo desde una posición
extrema a la otra y regresando.

    {\large
        \begin{equation}
            T = 2\pi \cdot \sqrt{\frac{m}{k}}
        \end{equation}
    }

Donde, \bolditalic{k}, es la constante de elasticidad del
resorte.

\nocite{tipos-pendulos-2}

% + -----------------------------|>
\subsubsection{Pendulo compuesto o fisico}

En el caso del péndulo físico, la fuerza de gravedad actúa
sobre el centro de masa (CM) de un objeto. Cuando un
péndulo físico está colgado de un punto pero es libre de
girar, lo hace debido al torque aplicado en el CM,
producido por el componente del peso del objeto que actúa
tangente al movimiento del CM\@.

{\large
\begin{equation}
    T = 2\pi \cdot \sqrt{\frac{I}{m g L}}
    \label{eq:T-pendulo-fisico}
\end{equation}
}

Donde, \bolditalic{I}, es el momento de inercia.

\nocite{tipos-pendulos-1}

% * ----------------------------------------------------|>
\subsection{Calcula el periodo de oscilación para una barra de longitud L y masa M, suspendida de uno de
    sus extremos dentro de un campo gravitacional g.}

El periodo de oscilación para una barra de longitud y masa
M, suspendida de uno de sus extremos dentro de un campo
gravitacional g, se puede calcular utilizando la fórmula
para el periodo de un péndulo físico. Un péndulo físico es
cualquier objeto que puede oscilar libremente alrededor de
un eje fijo que no pasa por su centro de masa. La formula
para calcular el periodo de un péndulo físico fue descrita
en~\ref{eq:T-pendulo-fisico},

Para una barra delgada y uniforme suspendida a uno de sus
extremos, el momento de inercia alrededor del eje de
rotación es

    {\large
        \begin{equation}
            I = \frac{1}{3}\cdot M L^{2}
            \label{eq:momento-inercia-barra-delgada}
        \end{equation}
    }

La distancia desde el eje de rotación hasta el centro de
masa de la barra es de $d = L / 2$. Sustituyendo estos
valores en~\ref{eq:T-pendulo-fisico}, obtenemos:

{\large
\begin{equation}
    \begin{gathered}
        T = 2\pi \cdot \sqrt{\frac{\frac{1}{3}\cdot M L^{2}}{M g (L / 2) }} \\
        T = 2\pi \cdot \sqrt{\frac{2L}{3g}}
    \end{gathered}
\end{equation}
}

% ! ----------------------------------------------------------------------|>
\section{Resumen del procedimiento}

\printbibliography

\end{document}