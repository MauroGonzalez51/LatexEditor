\documentclass[twocolumn, 12pt]{article}

\usepackage[utf8]{inputenc}
\usepackage[english, spanish]{babel}
\usepackage{fullpage}
\usepackage{graphicx}
\usepackage{amsmath}
\usepackage{enumitem}
\usepackage{chngcntr}
\usepackage{setspace}
\usepackage{url}
\usepackage{csquotes}
\usepackage{float}
\usepackage{verbatim}
\usepackage{tabularx}
\usepackage{amsmath}
\usepackage{caption}
\usepackage{bm}
\usepackage{colortbl}
\usepackage{xcolor}

\usepackage{multirow}

% \usepackage{hyperref}

\definecolor{LigthGray}{rgb}{0.7098, 0.7294, 0.7215}
\definecolor{LigthGrayPlus}{rgb}{0.6862, 0.7254, 0.7294}
\definecolor{White}{rgb}{1, 1, 1}
\definecolor{Red}{rgb}{1, 0, 0}

\counterwithin{figure}{section}
\renewcommand{\thesection}{\arabic{section}}
\renewcommand{\thesubsection}{\thesection.\arabic{subsection}}
\renewcommand{\baselinestretch}{1.5}

\usepackage[style=numeric, maxnames=6, minnames=3, backend=biber, parentracker=true, sorting=none]{biblatex}
\DefineBibliographyStrings{english}{%chktex-file 1 chktex-file 6
    andothers = {\em et\addabbrvspace al\adddot}
}
\addbibresource{./Bibliography/bibliography.bib}

\usepackage{array}

\setlength{\parskip}{0pt}

\raggedbottom{}

\newcommand{\bolditalic}[1]{\textbf{\textit{#1}}}

\newcommand{\Celsius}[0]{°$\mathcal{C}$}
\newcommand{\Kelvin}[0]{°$\mathcal{K}$}
\newcommand{\Fahrenheit}[0]{°$\mathcal{F}$}

\begin{document}

\begin{titlepage}
    \centering
    \includegraphics[width=0.3\textwidth]{Images/logo_utb.png}\par\vspace{1cm}
    {\scshape\LARGE Universidad Tecnológica de Bolívar \par}
    \vspace{1cm}

    {\scshape\Large FÍSICA CALOR Y ONDAS \par}
    \vspace{.2cm}

    % chktex-file 8
    {\scshape\Large Grupo 1 \par}
    \vspace{1cm}
    % chktex-file 8
    \slshape {\Large \bfseries{}Informe de Laboratorio No. V\\}
    \slshape {\small \bfseries{}ONDAS MECÁNICAS:~Velocidad del sonido}
    \vspace{2cm}

    \slshape {\itshape{} Mauro González, T00067622 \\}
    \slshape {\itshape{} German De Armas Castaño, T00068765 \\}
    \slshape {\itshape{} Angel Vega Rodriguez, T00068186 \\}
    \slshape {\itshape{} Juan Jose Osorio Ariza, T00067316 \\}
    \slshape {\itshape{} Jorge Alberto Rueda Salgado, T00068722 \\}
    \vfill
    Revisado Por \\
    Duban Andres Paternina Verona\\
    {\large \today\par}
\end{titlepage}

% chktex-file 44
% chktex-file 24

% ! ----------------------------------------------------------------------|>
\section{Introducción}

La determinación del calor específico de los sólidos es una
parte fundamental de la termodinámica y juega un papel
esencial en la comprensión de cómo los materiales almacenan
y liberan energía térmica. En esta experiencia de
laboratorio, se lleva a cabo un estudio detallado de la
transferencia de calor entre sólidos y líquidos para
determinar el calor específico de varios materiales. Esto
se logra mediante la medición de cambios de temperatura y
la aplicación de principios termodinámicos fundamentales.
La experimentación práctica en esta área es crucial para la
aplicación de conceptos teóricos en situaciones del mundo
real y es esencial para una amplia gama de campos, desde la
física hasta la ingeniería.

% ! ----------------------------------------------------------------------|>
\section{Objetivos}

% + ----------------------------------------|>
\subsection{Objetivo general}

El objetivo principal de esta práctica de laboratorio es
determinar el calor específico de sólidos utilizando un
enfoque experimental basado en la transferencia de calor y
principios termodinámicos.

% + ----------------------------------------|>
\subsection{Objetivos específicos}

\begin{itemize}[label=$\triangleright$]
    \item Medir la temperatura inicial y final de una mezcla de agua
          y sólidos después de una transferencia de calor controlada.

    \item Determinar la masa equivalente del calorímetro utilizado en
          el experimento.

    \item Calcular el calor específico de los sólidos utilizando los
          datos recopilados y las ecuaciones pertinentes.

    \item Comparar y analizar los resultados obtenidos utilizando dos
          métodos diferentes para calcular el calor específico.
\end{itemize}

% ! ----------------------------------------------------------------------|>
\section{Marco Teórico}

% + -------------------------------------------------------------|>
\subsection{Ley cero de la termodinámica~\cite{Fernández}}

Se dice que dos cuerpos están en equilibrio térmico cuando,
al ponerse en contacto, sus variables de estado no cambian.
En torno a esta simple idea se establece la ley cero.

La ley cero de la termodinámica establece que, cuando dos
cuerpos están en equilibrio térmico con un tercero, estos
están a su vez en equilibrio térmico entre sí.

% ! ----------------------------------------------------------------------|>
\section{Montaje Experimental}

% ! ----------------------------------------------------------------------|>
\section{Datos Experimentales}

\noindent\makebox[\linewidth]{\rule{.9\linewidth}{0.4pt}}

% ! ----------------------------------------------------------------------|>
\section{Análisis de datos}

% + -------------------------------------------------------------|>
\subsection{Análisis}

% ! ----------------------------------------------------------------------|>
\section{Conclusiones}

En esta experiencia de laboratorio, se llevaron a cabo
mediciones precisas de la transferencia de calor entre
sólidos y agua. Se determinaron los calores específicos de
los sólidos utilizando dos métodos diferentes, uno que
considera el calor absorbido por el agua y otro que tiene
en cuenta la masa equivalente del calorímetro. Se
encontraron diferencias en los resultados obtenidos por
estos métodos, lo que demuestra la importancia de
considerar la contribución del calorímetro en el proceso.

Estos experimentos proporcionaron una comprensión práctica
de los conceptos fundamentales de la termodinámica y
demostraron la relación entre la masa, la temperatura y el
calor específico de los sólidos. Además, destacaron la
necesidad de la precisión en la medición y la importancia
de la calibración adecuada de los instrumentos utilizados
en experimentos de transferencia de calor.

\printbibliography

\end{document}