\documentclass[twocolumn, 12pt]{article}

\usepackage[utf8]{inputenc}
\usepackage[english, spanish]{babel}
\usepackage{fullpage}
\usepackage{graphicx}
\usepackage{amsmath}
\usepackage{enumitem}
\usepackage{chngcntr}
\usepackage{setspace}
\usepackage{url}
\usepackage{csquotes}
\usepackage{float}
\usepackage{verbatim}
\usepackage{tabularx}
\usepackage{amsmath}
\usepackage{caption}
\usepackage{bm}
\usepackage{colortbl}
\usepackage{xcolor}

\usepackage{multirow}

% \usepackage{hyperref}

\definecolor{LigthGray}{rgb}{0.7098, 0.7294, 0.7215}
\definecolor{LigthGrayPlus}{rgb}{0.6862, 0.7254, 0.7294}
\definecolor{White}{rgb}{1, 1, 1}
\definecolor{Red}{rgb}{1, 0, 0}

\counterwithin{figure}{section}
\renewcommand{\thesection}{\arabic{section}}
\renewcommand{\thesubsection}{\thesection.\arabic{subsection}}
\renewcommand{\baselinestretch}{1.5}

\usepackage[style=apa, maxnames=6, minnames=3, backend=biber]{biblatex}
\DefineBibliographyStrings{english}{%chktex-file 1 chktex-file 6
    andothers = {\em et\addabbrvspace al\adddot}
}
\addbibresource{./Bibliography/bibliography.bib}

\usepackage{array}
\usepackage{enumitem}

\setlength{\parskip}{0pt}

\raggedbottom{}

\newcommand{\bolditalic}[1]{\textbf{\textit{#1}}}

\begin{document}

\begin{titlepage}
    \centering
    \includegraphics[width=0.3\textwidth]{Images/logo_utb.png}\par\vspace{1cm}
    {\scshape\LARGE Universidad Tecnológica de Bolívar \par}
    \vspace{1cm}

    {\scshape\Large FÍSICA CALOR Y ONDAS \par}
    \vspace{.2cm}

    % chktex-file 8
    {\scshape\Large Grupo 1 \par}
    \vspace{1cm}
    % chktex-file 8
    \slshape {\Large \bfseries{}Informe de Laboratorio No. I\\}
    \vspace{2cm}

    \slshape {\itshape{} Mauro González, T00067622 \\}
    \slshape {\itshape{} German De Armas Castaño, T00068765 \\}
    \slshape {\itshape{} Angel Vega Rodriguez, T00068186 \\}
    \slshape {\itshape{} Juan Jose Osorio Ariza, T00067316 \\}
    \slshape {\itshape{} Jorge Alberto Rueda Salgado, T00068722 \\}
    \vfill
    Revisado Por \\
    Duban Andres Paternina Verona\\
    {\large \today\par}
\end{titlepage}

% ! ----------------------------------------------------------------------|>
\section{Introducción}

Las oscilaciones mecánicas constituyen un fenómeno
fundamental en el estudio de la física, abarcando una
amplia variedad de sistemas, desde partículas microscópicas
hasta estructuras macroscópicas. Un ejemplo destacado de
estas oscilaciones es el movimiento armónico simple, en el
que un sistema realiza un vaivén periódico alrededor de una
posición de equilibrio. Este tipo de movimiento presenta
características intrínsecas que permiten su análisis y
comprensión a través de la aplicación de leyes y fórmulas
específicas.

\vspace{.5cm}

En esta experiencia de laboratorio, nos centraremos en la
exploración y comprobación de los conceptos relacionados
con el movimiento armónico simple y su aplicación en la
determinación experimental del período de oscilación de
diferentes tipos de péndulos. Para ello, utilizaremos una
variedad de equipos y simuladores que nos permitirán
observar y analizar las propiedades fundamentales de los
sistemas oscilatorios y su relación con las leyes físicas
que los rigen.

% ! ----------------------------------------------------------------------|>
\section{Objetivos}

% + ----------------------------------------|>
\subsection{Objetivo general}

\begin{itemize}[label=$\triangleright$]
    \item Comprobar experimentalmente la validez de las fórmulas
          utilizadas para calcular el período de oscilación de
          péndulos simples y compuestos, a través de la aplicación de
          principios del movimiento armónico simple.
\end{itemize}

% + ----------------------------------------|>
\subsection{Objetivos específicos}

\begin{itemize}[label=$\triangleright$]
    \item Familiarizarse con los conceptos de oscilación y movimiento
          armónico simple mediante el estudio y análisis previo de
          las propiedades de los sistemas oscilatorios.

    \item Identificar las características clave de un movimiento
          armónico simple, tales como amplitud, período, frecuencia,
          frecuencia angular y fase inicial, y comprender su
          significado físico.

    \item Comprender y aplicar las fórmulas que permiten calcular el
          período de oscilación para distintos tipos de péndulos,
          incluyendo el péndulo simple, el péndulo de resorte y el
          péndulo compuesto.
\end{itemize}

% ! ----------------------------------------------------------------------|>
\section{Marco Teórico}

% ! ----------------------------------------------------------------------|>
\section{Montaje Experimental}

% ! ----------------------------------------------------------------------|>
\section{Datos Experimentales}

% + ----------------------------------------|>
\subsection{Péndulo Simple}

% chktex-file 44

\begin{table}[H]
    \begin{tabularx}{\linewidth}{|>{\centering\arraybackslash}X|>{\centering\arraybackslash}X|>{\centering\arraybackslash}X|}
        \hline
        \rowcolor{LigthGray} No.   & Longitud \bolditalic{(M)} & Angulo \bolditalic{(°)} \\ \hline
        1                          & 0.3700                    & 15.0000                 \\\hline
        \rowcolor{LigthGrayPlus} 2 & 0.3050                    & 15.0000                 \\\hline
        3                          & 0.4450                    & 15.0000                 \\\hline
    \end{tabularx}

\end{table}

\vspace{-.5cm}

\begin{table}[H]
    \begin{tabularx}{\linewidth}{|>{\centering\arraybackslash}X|>{\centering\arraybackslash}X|>{\centering\arraybackslash}X|}
        \hline
        \rowcolor{LigthGray} \multicolumn{3}{|c|}{Tiempo \bolditalic{(s)}} \\ \hline
        12.4100                          & 12.5100 & 12.1100               \\ \hline
        \rowcolor{LigthGrayPlus} 11.7000 & 11.1500 & 11.0300               \\\hline
        13.5000                          & 13.2800 & 13.3800               \\\hline
    \end{tabularx}

\end{table}

\vspace{-.5cm}

\begin{table}[H]
    \begin{tabularx}{\linewidth}{|>{\centering\arraybackslash}X|>{\centering\arraybackslash}X|}
        \hline
        \rowcolor{LigthGray} Promedio \bolditalic{(s)} & Periodo \bolditalic{(Hz)} \\ \hline
        12.3433                                        & 1.2343                    \\\hline
        \rowcolor{LigthGrayPlus} 11.2933               & 1.1293                    \\\hline
        13.3867                                        & 1.3387                    \\\hline
    \end{tabularx}

\end{table}

\vspace{-.5cm}

\begin{table}[H]
    \begin{tabularx}{\linewidth}{|>{\centering\arraybackslash}X|>{\centering\arraybackslash}X|}
        \hline
        \rowcolor{LigthGrayPlus} \textbf{Oscilaciones} & 10 \\\hline
    \end{tabularx}
\end{table}

% + ----------------------------------------|>
\subsection{Péndulo Compuesto}

\begin{table}[H]
    \begin{tabularx}{\linewidth}{|>{\centering\arraybackslash}X|>{\centering\arraybackslash}X|>{\centering\arraybackslash}X|>{\centering\arraybackslash}X|}
        \hline
        \rowcolor{LigthGray} No. & Masa \bolditalic{(Kg)} & Longitud \bolditalic{(M)} & Distancia \bolditalic{(M)} \\ \hline
        1                        & 0,0490                 & \multirow{3}{*}{0.2470}   & 0,0500                     \\
        2                        & 0,0490                 &                           & 0,0840                     \\
        3                        & 0,0490                 &                           & 0,1020                     \\\hline
    \end{tabularx}
\end{table}

\vspace{-.5cm}

\begin{table}[H]
    \begin{tabularx}{\linewidth}{|>{\centering\arraybackslash}X|>{\centering\arraybackslash}X|>{\centering\arraybackslash}X|}
        \hline
        \rowcolor{LigthGray} \multicolumn{3}{|c|}{Tiempo \bolditalic{(s)}} \\ \hline
        4,0000                          & 3,7100 & 3,7400                  \\\hline
        \rowcolor{LigthGrayPlus} 3,8400 & 3,7000 & 4,1000                  \\\hline
        3,9000                          & 3,8100 & 3,9800                  \\\hline

    \end{tabularx}
\end{table}

\vspace{-.5cm}

\begin{table}[H]
    \begin{tabularx}{\linewidth}{|>{\centering\arraybackslash}X|>{\centering\arraybackslash}X|}
        \hline
        \rowcolor{LigthGray} Promedio \bolditalic{(s)} & Periodo \bolditalic{(Hz)} \\ \hline
        3,8167                                         & 0,7633                    \\\hline
        \rowcolor{LigthGrayPlus} 3,8800                & 0,7760                    \\\hline
        3,8967                                         & 0,7793                    \\\hline
    \end{tabularx}
\end{table}

\vspace{-.5cm}

\begin{table}[H]
    \begin{tabularx}{\linewidth}{|>{\centering\arraybackslash}X|>{\centering\arraybackslash}X|}
        \hline
        \rowcolor{LigthGrayPlus} \textbf{Oscilaciones} & 5 \\\hline
    \end{tabularx}
\end{table}

% + ----------------------------------------|>
\subsection{Péndulo de Resorte}

\begin{table}[H]
    \begin{tabularx}{\linewidth}{|>{\centering\arraybackslash}X|>{\centering\arraybackslash}X|}
        \hline
        \rowcolor{LigthGray} No.   & Masa \bolditalic{(Kg)} \\\hline
        1                          & 0,0100                 \\\hline
        \rowcolor{LigthGrayPlus} 2 & 0,0150                 \\\hline
        3                          & 0,0200                 \\\hline
    \end{tabularx}
\end{table}

\vspace{-.5cm}

\begin{table}[H]
    \begin{tabularx}{\linewidth}{|>{\centering\arraybackslash}X|>{\centering\arraybackslash}X|}
        \hline
        \rowcolor{LigthGray} Longitud Inicial \bolditalic{(M)} & Longitud Final \bolditalic{(M)} \\\hline
        1                                                      & 0,0100                          \\\hline
        \rowcolor{LigthGrayPlus} 2                             & 0,0150                          \\\hline
        3                                                      & 0,0200                          \\\hline
    \end{tabularx}
\end{table}

\vspace{-.5cm}

\begin{table}[H]
    \begin{tabularx}{\linewidth}{|>{\centering\arraybackslash}X|>{\centering\arraybackslash}X|>{\centering\arraybackslash}X|}
        \hline
        \rowcolor{LigthGray} $\Delta X$ \bolditalic{(M)} & K \bolditalic{(N/m)} & Periodo [Calculado] \bolditalic{(Hz)} \\\hline
        0,0750                                           & 1,3080               & 0,5494                                \\\hline
        0,1100                                           & 1,3377               & 0,6653                                \\\hline
        0,1500                                           & 1,3080               & 0,7769                                \\\hline
    \end{tabularx}
\end{table}

\vspace{-.2cm}

\begin{itemize}[label=$\triangleright$]
    \item K Promedio:~\textcolor{Red}{$1.3179$}
\end{itemize}

\vspace{-.2cm}

\begin{table}[H]
    \begin{tabularx}{\linewidth}{|>{\centering\arraybackslash}X|>{\centering\arraybackslash}X|>{\centering\arraybackslash}X|}
        \hline
        \rowcolor{LigthGray} \multicolumn{3}{|c|}{Tiempo \bolditalic{(s)}} \\ \hline
        5,5400                          & 5,5300 & 5,1700                  \\\hline
        \rowcolor{LigthGrayPlus} 6,1300 & 6,2000 & 6,3600                  \\\hline
        8,1300                          & 7,2600 & 7,9600                  \\\hline
    \end{tabularx}
\end{table}

\vspace{-.5cm}

\begin{table}[H]
    \begin{tabularx}{\linewidth}{|>{\centering\arraybackslash}X|}
        \hline
        \rowcolor{LigthGray} Promedio \bolditalic{(s)} \\ \hline
        5,4133                                         \\\hline
        \rowcolor{LigthGrayPlus} 6,2300                \\\hline
        7,7833                                         \\\hline

    \end{tabularx}
\end{table}

\vspace{-.5cm}

\begin{table}[H]
    \begin{tabularx}{\linewidth}{|>{\centering\arraybackslash}X|>{\centering\arraybackslash}X|}
        \hline
        \rowcolor{LigthGrayPlus} \textbf{Oscilaciones} & 10 \\\hline
    \end{tabularx}
\end{table}

% ! ----------------------------------------------------------------------|>
\section{Análisis de datos}

% ! ----------------------------------------------------------------------|>
\section{Conclusiones}

\end{document}