\documentclass[twocolumn, 12pt]{article}

\usepackage[utf8]{inputenc}
\usepackage[english, spanish]{babel}
\usepackage{fullpage}
\usepackage{graphicx}
\usepackage{amsmath}
\usepackage{enumitem}
\usepackage{chngcntr}
\usepackage{setspace}
\usepackage{url}
\usepackage{csquotes}
\usepackage{float}
\usepackage{verbatim}
\usepackage{tabularx}
\usepackage{amsmath}
\usepackage{caption}
\usepackage{bm}
% \usepackage{hyperref}

\counterwithin{figure}{section}
\renewcommand{\thesection}{\arabic{section}}
\renewcommand{\thesubsection}{\thesection.\arabic{subsection}}
\renewcommand{\baselinestretch}{1.5}

\usepackage[style=apa, maxnames=6, minnames=3, backend=biber]{biblatex}
\DefineBibliographyStrings{english}{%chktex-file 1 chktex-file 6
    andothers = {\em et\addabbrvspace al\adddot}
}
\addbibresource{./Bibliography/bibliography.bib}

\usepackage{array}
\usepackage{enumitem}

\setlength{\parskip}{0pt}

\newcommand{\bolditalic}[1]{\textbf{\textit{#1}}}

\begin{document}

\begin{titlepage}
    \centering
    \includegraphics[width=0.3\textwidth]{Images/logo_utb.png}\par\vspace{1cm}
    {\scshape\LARGE Universidad Tecnológica de Bolívar \par}
    \vspace{1cm}

    {\scshape\Large FÍSICA CALOR Y ONDAS \par}
    \vspace{.2cm}

    % chktex-file 8
    {\scshape\Large Grupo 1 \par}
    \vspace{1cm}
    % chktex-file 8
    \slshape {\Large \bfseries{}LAB 2 - ONDAS ESTACIONARIAS EN UNA CUERDA.~RESONANCIA:\@{}~MOVIMIENTO ARMÓNICO SIMPLE.\\}
    \vspace{4cm}

    \slshape {\itshape{} Mauro González, T00067622 \\}
    % \slshape {\itshape{} German De Armas Castaño, T00068765 \\}
    % \slshape {\itshape{} Angel Vega Rodriguez, T00068186 \\}
    % \slshape {\itshape{} Juan Jose Osorio Ariza, T00067316 \\}
    % \slshape {\itshape{} Jorge Alberto Rueda Salgado, T00068722 \\}
    \vfill
    Revisado Por \\
    Duban Andres Paternina Verona\\
    {\large \today\par}
\end{titlepage}

% ! ----------------------------------------------------------------------|>
\section{Introducción}

Las ondas, fenómenos omnipresentes en la naturaleza, han
capturado la atención de científicos y curiosos durante
siglos. Estas oscilaciones que se propagan a través de un
medio transportan energía y generan una amplia gama de
efectos observables, desde el suave susurro del viento
hasta la música que nos llega a través de los altavoces.
Una comprensión sólida de las propiedades y el
comportamiento de las ondas es esencial para desentrañar
los misterios detrás de su funcionamiento y aplicaciones en
diferentes campos.

A través de esta actividad, tendremos la oportunidad de
investigar los conceptos clave relacionados con la
naturaleza y el comportamiento de las ondas estacionarias,
y en particular, abordaremos el fenómeno de resonancia.
Estudiaremos cómo se forman las ondas estacionarias en una
cuerda y cómo su frecuencia de resonancia está
intrínsecamente relacionada con sus propiedades físicas,
como la longitud de la cuerda, la tensión y la densidad
lineal de masa.

% ! ----------------------------------------------------------------------|>
\section{Objetivos}

\subsection{Objetivo general}

\begin{itemize}[label=$\triangleright$]
    \item Estudiar las ondas estacionarias generadas en una cuerda
          fija en sus extremos y analizar el fenómeno de resonancia
          en este contexto.
\end{itemize}

\subsection{Objetivos específicos}

\begin{itemize}[label=$\triangleright$]
    \item Familiarizarse con los conceptos básicos de las ondas,
          incluyendo la naturaleza de las ondas sinusoidales, su
          amplitud, frecuencia, longitud de onda y rapidez de
          propagación.

    \item Comprender los principios detrás de las ondas estacionarias
          y cómo se forman en una cuerda fija en ambos extremos, así
          como identificar los modos normales de vibración.

    \item Calcular y relacionar la longitud de onda, la frecuencia y
          la rapidez de la onda en función de las propiedades físicas
          de la cuerda, como su longitud, tensión y densidad lineal
          de masa.

    \item Explorar el fenómeno de resonancia, entendiendo cómo
          ciertas frecuencias pueden causar una transferencia de
          energía más eficiente a través de la cuerda y cómo esto se
          relaciona con los modos normales de vibración.
\end{itemize}

% ! ----------------------------------------------------------------------|>
\section{Preparación de la practica}

% * -------------------------------------------------------|>
\subsection{¿Que es una onda?}

En el ámbito de la física, el término ``onda'' se emplea
para describir la propagación de energía sin la necesidad
de un desplazamiento de materia. Esta propagación implica
una alteración o perturbación que se mueve en un medio
específico y que, una vez que ha pasado, no produce cambios
permanentes en dicho medio. Este fenómeno abarca una amplia
variedad de situaciones, que van desde las ondas en la
superficie de un líquido hasta la luz, que en sí misma es
un tipo de onda.~(\cite{ondas-definicion})

% * -------------------------------------------------------|>
\subsection{¿Qué es una función de onda?}

En mecánica cuántica, una función de onda ($\Psi$) es una
forma de describir el estado físico de un sistema de
partículas. Las propiedades mencionadas de la función de
onda permiten interpretarla como una función de cuadrado
integrable. La ecuación de Schrödinger proporciona una
ecuación determinista para explicar la evolución temporal
de la función de onda y, por tanto, del estado físico del
sistema en el intervalo comprendido entre dos
medidas.~(\cite{funcion-ondas})

% * -------------------------------------------------------|>
\subsection{¿Qué es una onda senoidal y defina sus características: Amplitud, frecuencia, periodo, constante
    de fase, fase inicial, longitud de onda, numero de onda angular, frecuencia angular, rapidez de
    propagación?}

% + --------------------------------------|>
\subsubsection{Onda senoidal}

Una onda senoidal, o sinusoide es la gráfica de una función
matemática seno de la trigonometría. Consiste en una
frecuencia única con una amplitud constante. En su forma
mas simple, una ecuación de voltaje senoidal es:

\begin{equation*}
    V = V_{\max} \sen q
\end{equation*}

Tomado de:~\cite{onda-senoidal}

% + --------------------------------------|>
\subsubsection{Caracteristicas}

\begin{itemize}[label=$\triangleright$]
    \item Amplitud [A] \bolditalic{(m)}: La amplitud de las funciones
          seno y coseno es la distancia vertical entre el eje
          sinusoidal y el valor máximo o mínimo de la función. En
          relación con las ondas sonoras, la amplitud es una medida
          de lo fuerte que es algo.~(\cite{onda-amplitud})

    \item Frecuencia [f] \bolditalic{(Hz)}: La frecuencia es el
          número de ciclos por segundo de una onda sinusoidal.
          Cuantos más ciclos ocurren por segundo, mayor será la
          frecuencia.~(\cite{fluke})

    \item Periodo [T] \bolditalic{(s)}: Es el tiempo requerido para
          producir un ciclo completo de una forma de
          onda.~(\cite{fluke})

    \item Constante de fase [$\phi$]: la fase indica la situación
          instantánea en el ciclo, de una magnitud que varía
          cíclicamente, siendo la fracción del periodo transcurrido
          desde el instante correspondiente al estado tomado como
          referencia.~(\cite{fase})

    \item Fase inicial [$\phi$]: Podemos representar un ciclo en un
          círculo de 360°, diciendo que «fase» es la diferencia en
          grados entre un punto sobre este círculo y un punto de
          referencia, una rotación de 360° es equivalente a un ciclo.
          Gráficamente, el valor que esta fase toma en un instante
          cero se denomina: fase inicial.~(\cite{fase})

    \item Longitud de onda [$\lambda$] \bolditalic{(m)}: En una onda
          periódica la longitud de onda es la distancia física entre
          dos puntos a partir de los cuales la onda se
          repite.~(\cite{ESERO})

    \item Numero de onda [$\kappa$] \bolditalic{(Hz/m)}: El número de
          onda es una magnitud que representa el número de ciclos
          completados por la onda por unidad de
          distancia.~(\cite{Ingenierizando_2023})

    \item Frecuencia angular [$\omega$] \bolditalic{($Hz \cdot
                  rad$)}: La frecuencia angular, también llamada pulsación,
          es la velocidad a la que una onda realiza las oscilaciones.
          Es decir, la frecuencia angular indica cuánto de rápido
          oscila un movimiento armónico
          simple.~(\cite{Ingenierizando_2023b})

    \item Rapidez de propagación [$v$] \bolditalic{(m/s)}: La
          velocidad de propagación en términos simples, es la
          velocidad a la que se propaga una onda, es decir, la
          velocidad de propagación es la velocidad a la que avanza
          una onda. Así pues, la velocidad de propagación de una onda
          es la relación entre el espacio que avanza y el tiempo que
          necesita para recorrerlo.~(\cite{Ingenierizando_2023c})
\end{itemize}

% * -------------------------------------------------------|>
\subsection{¿Cuál es la expresión general de una onda senoidal viajera? Identifique cada una de sus
    características en la expresión.}

La expresión general es

\begin{equation*}
    x = A \cos (\omega t + \kappa z + \phi)
\end{equation*}

Tomado de:~\cite{ONDAS}

% * -------------------------------------------------------|>
\subsection{¿Qué es una onda mecánica?}

Las ondas mecánicas avanzan a través de un medio elástico,
cuyas partículas oscilan en torno a un punto fijo. El medio
en cuestión puede ser gaseoso, líquido o sólido.

Para que exista una onda mecánica es necesario que haya una
fuente que genere la perturbación y un medio por el cual
dicha perturbación pueda propagarse. Además se necesita un
medio físico que permita a los elementos influirse entre
sí.~(\cite{J_Merino_2022})

% * -------------------------------------------------------|>
\subsection{¿De qué depende la rapidez de una onda mecánica?}

Por ser una onda mecánica, la rapidez de su propagación
depende del medio de propagación elástico. La velocidad de
propagación de la perturbación, dependerá de la proximidad
de las partículas del medio y de sus fuerzas de
cohesión.~(\cite{Lasondas})

% * -------------------------------------------------------|>
\subsection{¿De qué depende la rapidez de una onda mecánica?}

La velocidad de un pulso u onda en una cuerda bajo tensión
se puede calcular con la ecuación:

\begin{equation*}
    v = \sqrt{\frac{F_T}{\mu}}
\end{equation*}

donde $F_T$ es la tensión de la cuerda y $\mu$ es la masa
por longitud de la cuerda.~(\cite{Moebs_2021})

% * -------------------------------------------------------|>
\subsection{¿Cómo se calcula la rapidez de transferencia de energía (potencia) por ondas sinusoidales en
    una cuerda?}

La potencia promediada en el tiempo de una onda mecánica
sinusoidal, que es la tasa media de transferencia de
energía asociada a una onda cuando pasa por un punto, se
puede hallar al tomar la energía total asociada a la onda
dividida entre el tiempo que tarda en transferirse la
energía.

\begin{equation*}
    P_{ave} = \frac{1}{2} \mu A^{2} \omega^{2}v
\end{equation*}

Tomado de:~\cite{Moebs_2021b}

% * -------------------------------------------------------|>
\subsection{¿En qué consiste el principio de superposición de ondas?}

Cuando varias ondas se combinan en un punto, el
desplazamiento de cualquier elemento de medio en un tiempo
dado es la suma vectorial de los desplazamientos que
produciría cada onda individual que actúe por sí sola, esto
se denomina Principio de
Superposición.~(\cite{superposicion})

% * -------------------------------------------------------|>
\subsection{¿Cuál es la resultante de la superposición de dos ondas senoidales viajeras? ¿En qué casos la
    interferencia es constructiva y en qué casos destructiva?}

La resultante de la superposición de dos ondas senoidales
es la suma de las ondas en el mismo sentido con la misma
frecuencia y amplitud, esto resulta en una nueva onda
viajera con un desfase que es la media de los desfases
respectivos y cuya amplitud depende del
desfase.~(\cite{Superposicióndeondas})

\begin{itemize}[label=$\triangleright$]
    \item Interferencia constructiva: Decimos que se produce una
          interferencia constructiva en un punto P cuando la amplitud
          con la que vibra dicho punto es máxima. Esto ocurren en
          aquellos puntos del medio en los que las ondas están en
          fase, que son los mismos en los que la diferencia entre las
          distancias a los focos de cada onda es un número entero de
          longitudes de onda.

          \begin{equation*}
              \Delta x = n \cdot \lambda
          \end{equation*}

    \item Interferencia destructiva: Decimos que se produce una
          interferencia destructiva en un punto P cuando la amplitud
          con la que vibra dicho punto es mínima. Esto ocurren en
          aquellos puntos del medio en los que las ondas están en
          oposición de fase, que son los mismos en los que la
          diferencia entre las distancias a los focos de cada onda es
          un número impar de semilongitudes de onda. Denominamos a
          estos puntos nodos.

          \begin{equation*}
              \Delta x = (2 \cdot n + 1) \cdot \frac{\lambda}{2}
          \end{equation*}
\end{itemize}

Tomado de:~\cite{Fernández}

% * -------------------------------------------------------|>
\subsection{¿Qué es una onda estacionaria y cómo resulta?}

Llamamos onda estacionaria a un caso particular de
interferencia que se produce cuando se superponen dos ondas
de la misma dirección, amplitud y frecuencia, pero sentido
contrario. En una onda estacionaria los distintos puntos
que la conforman oscilan en torno a su posición de
equilibrio a medida que transcurre el tiempo pero el patrón
de la onda no se mueve, de ahí su
nombre.~(\cite{Fernándezb})

% * -------------------------------------------------------|>
\subsection{¿Qué son los modos normales de oscilación?}

Un modo normal de un sistema oscilatorio es la frecuencia a
la cual la estructura deformable oscilará al ser
perturbada. Los modos normales son también llamados
frecuencias naturales o frecuencias resonantes. Para cada
estructura existe un conjunto de estas frecuencias que es
único.~(\cite{modososcilacion})

% * -------------------------------------------------------|>
\subsection{Para una onda estacionaria fija en ambos extremos encuentre: (\dots)}

Para una cuerda fija en ambos extremos que forma una onda
estacionaria, los modos normales de vibración se
caracterizan por diferentes longitudes de onda y
frecuencias.

\begin{itemize}[label=$\triangleright$]
    \item Longitud de onda de modos normales: La longitud de onda
          para el n-ésimo modo normal de vibración en una cuerda fija
          en ambos extremos está dada por:

          \begin{equation*}
              \lambda_{n} = \frac{2 \cdot L}{n}
          \end{equation*}

    \item Frecuencia de modos normales en función de la rapidez de la
          onda y longitud de la cuerda: La frecuencia $f_n$ del
          n-ésimo modo normal está relacionada con la rapidez de la
          onda $v$ y la longitud de onda ($\lambda_{n}$).

          \begin{equation*}
              f_n = \frac{v}{\lambda_{n}} = \frac{n v}{2L}
          \end{equation*}

    \item Frecuencia de modos normales en función de la tensión de la
          cuerda y la densidad lineal de masa: La frecuencia del
          n-ésimo modo normal también está relacionada con la tensión
          en la cuerda Y la densidad lineal de masa ($\mu$).

          \begin{equation*}
              f = \frac{n}{2 L} \sqrt{\frac{F_T}{\mu}}
          \end{equation*}

    \item Frecuencia fundamental: La frecuencia fundamental ($f_{1}$)
          s la frecuencia del primer modo normal (modo fundamental),
          que es cuando la cuerda vibra en una única mitad de ciclo.
          Para el modo fundamental $(n=1)$:

          \begin{equation*}
              f_{1} = \frac{v}{2L} = \frac{v}{2L}
          \end{equation*}

          Esta es la frecuencia más baja en la cual la cuerda puede
          vibrar.
\end{itemize}

Tomado de:~\cite{Fernándezc}

% * -------------------------------------------------------|>
\subsection{¿En qué consiste el fenómeno de resonancia?}

La Resonancia es un fenómeno que amplifica una vibración.
Se produce cuando una vibración se transmite a otro objeto
cuya frecuencia natural es igual o muy cercana a la de la
fuente.~(\cite{Erbessd_2022})

% * -------------------------------------------------------|>
\subsection{¿Qué son las frecuencias de resonancia?}

La frecuencia en la que la amplitud alcanza su máximo se
conoce como frecuencia de resonancia. Todos los objetos y/o
sistemas tienen una o más frecuencias de resonancia.

La frecuencia de resonancia se denota como ($f_0$) En el
caso de una onda continua, podemos calcular la frecuencia
de resonancia utilizando la siguiente fórmula:

\begin{equation*}
    v = \lambda f
\end{equation*}

Tomado de:~\cite{Resonancia}

% ! ----------------------------------------------------------------------|>
\section{Resumen del procedimiento}

\bolditalic{Determinacion de los primeros armonicos en una cuerda resonante}

\begin{enumerate}
    \item \textbf{Medición de la Masa de la Cuerda:} Se mide la masa de un
          tramo de la cuerda a utilizar para la experiencia, con el
          propósito de calcular la densidad lineal de masa de la
          cuerda.

    \item \textbf{Configuración del Montaje:} Se arma un montaje que
          representa una cavidad resonante con la cuerda fijada en
          ambos extremos. Se solicita al profesor una longitud
          específica $(L)$ para la cuerda y dos valores de masa
          $(M_1)$ Y $(M_2)$ que tensionarán la cuerda.

    \item \textbf{Cálculo de la Densidad Lineal de Masa:} Utilizando los datos
          de la masa de la cuerda y la longitud
          $(L)$, se calcula la densidad lineal de masa $(\mu)$ de la cuerda.

    \item \textbf{Cálculo de la Rapidez de Propagación:} Se calcula la rapidez
          de propagación de las ondas en la cuerda utilizando la relación entre la tensión $(T)$ y
          la densidad lineal de masa $(\mu)$, para cada uno de los valores de
          masa $(M_1)$ y $(M_2)$

    \item \textbf{Cálculo de Frecuencias de Resonancia:} Se determinan
          teóricamente la frecuencia fundamental de resonancia y tres
          modos de oscilación adicionales para las tensiones
          correspondientes a las masas $(M_1)$ y $(M_2)$, utilizando
          la fórmula relacionada con la longitud de la cuerda y la
          rapidez de propagación.

    \item \textbf{Registro de Datos y Resultados:} Todos los cálculos, valores
          obtenidos y resultados se registran en una tabla para su
          posterior análisis y comparación con los datos
          experimentales.
\end{enumerate}

\printbibliography

\end{document}