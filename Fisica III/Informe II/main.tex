\documentclass[twocolumn, 12pt]{article}

\usepackage[utf8]{inputenc}
\usepackage[english, spanish]{babel}
\usepackage{fullpage}
\usepackage{graphicx}
\usepackage{amsmath}
\usepackage{enumitem}
\usepackage{chngcntr}
\usepackage{setspace}
\usepackage{url}
\usepackage{csquotes}
\usepackage{float}
\usepackage{verbatim}
\usepackage{tabularx}
\usepackage{amsmath}
\usepackage{caption}
\usepackage{bm}
\usepackage{colortbl}
\usepackage{xcolor}

\usepackage{multirow}

% \usepackage{hyperref}

\definecolor{LigthGray}{rgb}{0.7098, 0.7294, 0.7215}
\definecolor{LigthGrayPlus}{rgb}{0.6862, 0.7254, 0.7294}
\definecolor{White}{rgb}{1, 1, 1}
\definecolor{Red}{rgb}{1, 0, 0}

\counterwithin{figure}{section}
\renewcommand{\thesection}{\arabic{section}}
\renewcommand{\thesubsection}{\thesection.\arabic{subsection}}
\renewcommand{\baselinestretch}{1.5}

\usepackage[style=apa, maxnames=6, minnames=3, backend=biber]{biblatex}
\DefineBibliographyStrings{english}{%chktex-file 1 chktex-file 6
    andothers = {\em et\addabbrvspace al\adddot}
}
\addbibresource{./Bibliography/bibliography.bib}

\usepackage{array}

\setlength{\parskip}{0pt}

\raggedbottom{}

\newcommand{\bolditalic}[1]{\textbf{\textit{#1}}}

\begin{document}

\begin{titlepage}
    \centering
    \includegraphics[width=0.3\textwidth]{Images/logo_utb.png}\par\vspace{1cm}
    {\scshape\LARGE Universidad Tecnológica de Bolívar \par}
    \vspace{1cm}

    {\scshape\Large FÍSICA CALOR Y ONDAS \par}
    \vspace{.2cm}

    % chktex-file 8
    {\scshape\Large Grupo 1 \par}
    \vspace{1cm}
    % chktex-file 8
    \slshape {\Large \bfseries{}Informe de Laboratorio No. II\\}
    \slshape {\small \bfseries{} ONDAS ESTACIONARIAS EN UNA CUERDA.~RESONANCIA:\@{}~MOVIMIENTO ARMÓNICO SIMPLE.}
    \vspace{2cm}

    \slshape {\itshape{} Mauro González, T00067622 \\}
    \slshape {\itshape{} German De Armas Castaño, T00068765 \\}
    \slshape {\itshape{} Angel Vega Rodriguez, T00068186 \\}
    \slshape {\itshape{} Juan Jose Osorio Ariza, T00067316 \\}
    \slshape {\itshape{} Jorge Alberto Rueda Salgado, T00068722 \\}
    \vfill
    Revisado Por \\
    Duban Andres Paternina Verona\\
    {\large \today\par}
\end{titlepage}

% chktex-file 44

% ! ----------------------------------------------------------------------|>
\section{Introducción}

Las ondas estacionarias son un tipo especial de ondas que se forman cuando dos ondas 
viajeras de igual frecuencia y amplitud se superponen y se mueven en direcciones opuestas. 
En una onda estacionaria, los puntos individuales que la componen experimentan oscilaciones 
alrededor de sus posiciones de equilibrio a medida que pasa el tiempo. Sin embargo, lo que 
distingue a una onda estacionaria es que su patrón general no se desplaza a lo largo del medio, 
de ahí su denominación.

En esta experiencia de laboratorio, nos centraremos en la exploración y comprobación de 
cada uno de los conceptos relacionados con la propagación de las ondas estacionarias en 
una cuerda. Para ello, se utilizará una cuerda flexible y una fuente de 
vibración controlable que nos permitirán observar y analizar cómo cambian 
los patrones de ondas estacionarias.


% ! ----------------------------------------------------------------------|>
\section{Objetivos}

% + ----------------------------------------|>
\subsection{Objetivo general}

\begin{itemize}[label=$\triangleright$]
    \item Explicar las propiedades de las ondas estacionarias en un contexto 
        experimental específico, a partir del análisis de factores como la 
        frecuencia, la longitud de onda y la tensión en el medio.    
\end{itemize}

% + ----------------------------------------|>
\subsection{Objetivos específicos}

\begin{itemize}[label=$\triangleright$]
    \item Determinar las propiedades de una onda estacionaria tales como la 
        longitud de onda, la amplitud, la frecuencia y la velocidad de una onda 
        estacionaria en un medio específico.

    \item Identificar los puntos de formación de nodos (puntos de amplitud mínima) 
        y antinodos (puntos de amplitud máxima).

    \item Estudiar la relación entre la frecuencia y la longitud de onda y el como 
        afectan la formación y el patrón de las ondas estacionarias en una cuerda
\end{itemize}

% ! ----------------------------------------------------------------------|>
\section{Marco Teórico}

% * -------------------------------------------------------|>
\subsection{Onda mecánica}

Las ondas mecánicas avanzan a través de un medio elástico,
cuyas partículas oscilan en torno a un punto fijo. El medio
en cuestión puede ser gaseoso, líquido o sólido.

Para que exista una onda mecánica es necesario que haya una
fuente que genere la perturbación y un medio por el cual
dicha perturbación pueda propagarse. Además se necesita un
medio físico que permita a los elementos influirse entre
sí.~(\cite{J_Merino_2022})

% * -------------------------------------------------------|>
\subsection{Expresión para calcular la velocidad de una onda en una cuerda}

La velocidad de un pulso u onda en una cuerda bajo tensión
se puede calcular con la ecuación:

\begin{equation}
    v = \sqrt{\frac{F_T}{\mu}}
    \label{eq:velocidad-onda-cuerda}
\end{equation}

donde $F_T$ es la tensión de la cuerda y $\mu$ es la masa
por longitud de la cuerda.~(\cite{Moebs_2021})

% * -------------------------------------------------------|>
\subsection{Formulas para ondas estacionarias en una cuerda}

Para una cuerda fija en ambos extremos que forma una onda
estacionaria, los modos normales de vibración se
caracterizan por diferentes longitudes de onda y
frecuencias.

\begin{itemize}[label=$\triangleright$]
    \item Longitud de onda de modos normales: La longitud de onda
          para el n-ésimo modo normal de vibración en una cuerda fija
          en ambos extremos está dada por:

          \begin{equation}
              \lambda_{n} = \frac{2 \cdot L}{n}
              \label{eq:n_lambda}
          \end{equation}

    \item Frecuencia de modos normales en función de la rapidez de la
          onda y longitud de la cuerda: La frecuencia $f_n$ del
          n-ésimo modo normal está relacionada con la rapidez de la
          onda $v$ y la longitud de onda ($\lambda_{n}$).

          \begin{equation*}
              f_n = \frac{v}{\lambda_{n}} = \frac{n v}{2L}
          \end{equation*}

    \item Frecuencia de modos normales en función de la tensión de la
          cuerda y la densidad lineal de masa: La frecuencia del
          n-ésimo modo normal también está relacionada con la tensión
          en la cuerda Y la densidad lineal de masa ($\mu$).

          \begin{equation}
              f = \frac{n}{2 L} \sqrt{\frac{F_T}{\mu}}
              \label{eq:n_frecuencia}
          \end{equation}

    \item Frecuencia fundamental: La frecuencia fundamental ($f_{1}$)
          s la frecuencia del primer modo normal (modo fundamental),
          que es cuando la cuerda vibra en una única mitad de ciclo.
          Para el modo fundamental $(n=1)$:

          \begin{equation*}
              f_{1} = \frac{v}{2L} = \frac{v}{2L}
          \end{equation*}

          Esta es la frecuencia más baja en la cual la cuerda puede
          vibrar.
\end{itemize}

% * -------------------------------------------------------|>
\subsection{Calculo de errores}

{\large
    \begin{equation}
        \begin{gathered}
            E_{absoluto} = \left\lvert V_E - V_A\right\rvert \\
            E_{relativo} = \frac{E_A}{V_E} \\
        \end{gathered}
        \label{eq:calculo-errores}
    \end{equation}
}

% ! ----------------------------------------------------------------------|>
\section{Montaje Experimental}

\begin{figure}[H]
    \begin{center}
        \includegraphic[width=0.9\linewidth]{./Images/Imagen1.png}
    \end{center}
    \caption{}
\end{figure}

Equipo utilizado:

\begin{itemize}[label=$\triangleright$]
    \item Generador de señales. 

    \item Vibrador mecánico. 

    \item Soportes universales.

    \item Porta pesas. 

    \item Balanza.

    \item Cinta métrica. 

    \item Polea.

    \item 86cm de cuerda.
\end{itemize}

En esta práctica comprobamos experimentalmente los cuatro primeros armónicos de 
una cavidad resonante \textit{(la frecuencia fundamental de resonancia y los tres siguientes armónicos)}.

Dado el montaje visto en la imagen conformado por una cuerda, un oscilador, 
una polea y un soporte de poleas, tomamos la medida desde el vibrador mecánico 
hasta el extremo de la polea, lo que será nuestra longitud $L$ \bolditalic{(m)}.

Luego, con las distintas masas, en este caso 3 masas de peso conocido cada una, 
las vamos variando en el extremo de la cuerda, para posteriormente con el 
generador de señales hacer vibrar la cuerda hasta alcanzar una frecuencia en 
la que se capte un armónico y así con los tres siguientes armónicos. Hicimos este paso para 
cada una de las masas $M_{1}$, $M_{2}$ y $M_{3}$.

Finalmente registramos en sus respectivas tablas cada uno de los datos
registrados que serán comparados con los datos teóricos más adelante.

% ! ----------------------------------------------------------------------|>
\section{Datos Experimentales}

\begin{table}[H]
    \begin{center}
        \begin{tabularx}{0.9\linewidth}{|>{\centering\arraybackslash}X|>{\centering\arraybackslash}X|}
            \hline
            \multicolumn{2}{|c|}{\textbf{Constantes}}   \\\hline
            $M_{cuerda}$ & $1 \times 10^{-3} Kg$        \\\hline
            $L_{cuerda}$ & $1,425 m$                    \\\hline
            $\mu$        & $7,0175 \times 10^{-4} Kg/m$ \\\hline
            $L$          & $8,6 \times 10^{-1} m$       \\\hline
        \end{tabularx}
    \end{center}
\end{table}

\begin{table}[H]
    \begin{center}
        \begin{tabularx}{0.9\linewidth}{|>{\centering\arraybackslash}X|>{\centering\arraybackslash}X|}
            \hline
            \multicolumn{2}{|c|}{$M_{1} = 0,0349 Kg$}       \\\hline
            No. & Frecuencia experimental \bolditalic{(Hz)} \\\hline
            1   & $15,43$                                   \\\hline
            2   & $30,39$                                   \\\hline
            3   & $45,73$                                   \\\hline
            4   & $60,00$                                   \\\hline
        \end{tabularx}
    \end{center}
\end{table}

\vspace{-.5cm}

\begin{table}[H]
    \begin{center}
        \begin{tabularx}{0.9\linewidth}{|>{\centering\arraybackslash}X|>{\centering\arraybackslash}X|}
            \hline
            \multicolumn{2}{|c|}{$M_{2} = 0,0399 Kg$}       \\\hline
            No. & Frecuencia experimental \bolditalic{(Hz)} \\\hline
            1   & $12,98$                                   \\\hline
            2   & $26,00$                                   \\\hline
            3   & $39,24$                                   \\\hline
            4   & $52,20$                                   \\\hline
        \end{tabularx}
    \end{center}
\end{table}

\vspace{-.5cm}

\begin{table}[H]
    \begin{center}
        \begin{tabularx}{0.9\linewidth}{|>{\centering\arraybackslash}X|>{\centering\arraybackslash}X|}
            \hline
            \multicolumn{2}{|c|}{$M_{3} = 0,0449 Kg$}       \\\hline
            No. & Frecuencia experimental \bolditalic{(Hz)} \\\hline
            1   & $13,56$                                   \\\hline
            2   & $28,50$                                   \\\hline
            3   & $43,04$                                   \\\hline
            4   & $58,38$                                   \\\hline
        \end{tabularx}
    \end{center}
\end{table}

% ! ----------------------------------------------------------------------|>
\section{Análisis de datos}

% + ----------------------------------------|>
\subsection{Análisis}

% * -----------------------------|>
\subsubsection{Calculo de errores}

Usando la formula~(\ref{eq:n_frecuencia})
y~(\ref{eq:calculo-errores}),

\begin{table}[H]
    \begin{tabularx}{.9\linewidth}{|>{\centering\arraybackslash}X|>{\centering\arraybackslash}X|>{\centering\arraybackslash}X|}
        \hline
        \multicolumn{3}{|c|}{$M_{1}$}                                        \\\hline
        No. & Frecuencia Teórica \bolditalic{(Hz)} & Error \bolditalic{(\%)} \\\hline
        1   & $12,8418$                            & $20,1544$               \\\hline
        2   & $25,6836$                            & $18,3245$               \\\hline
        3   & $38,5254$                            & $18,7008$               \\\hline
        4   & $51,3672$                            & $16,8060$               \\\hline

    \end{tabularx}
\end{table}

\vspace{-.5cm}

\begin{table}[H]
    \begin{tabularx}{.9\linewidth}{|>{\centering\arraybackslash}X|>{\centering\arraybackslash}X|>{\centering\arraybackslash}X|}
        \hline
        \multicolumn{3}{|c|}{$M_{2}$}                                        \\\hline
        No. & Frecuencia Teórica \bolditalic{(Hz)} & Error \bolditalic{(\%)} \\\hline
        1   & $13,7309$                            & $5,4689$                \\\hline
        2   & $27,4619$                            & $5,3232$                \\\hline
        3   & $41,1928$                            & $4,7406$                \\\hline
        4   & $54,9237$                            & $4,9591$                \\\hline
    \end{tabularx}
\end{table}

\vspace{-.5cm}

\begin{table}[H]
    \begin{tabularx}{.9\linewidth}{|>{\centering\arraybackslash}X|>{\centering\arraybackslash}X|>{\centering\arraybackslash}X|}
        \hline
        \multicolumn{3}{|c|}{$M_{3}$}                                        \\\hline
        No. & Frecuencia Teórica \bolditalic{(Hz)} & Error \bolditalic{(\%)} \\\hline
        1   & $14,5659$                            & $6,9057$                \\\hline
        2   & $29,1318$                            & $2,1686$                \\\hline
        3   & $43,6976$                            & $1,5050$                \\\hline
        4   & $58,2635$                            & $0,1999$                \\\hline

    \end{tabularx}
\end{table}

% * -----------------------------|>
\subsubsection{Rapidez de propagacion en funcion de la frecuencia}

Usando la formula~(\ref{eq:n_lambda}) y

\begin{equation*}
    \lambda = \frac{v}{f}
\end{equation*}

\begin{table}[H]
    \begin{tabularx}{.9\linewidth}{|>{\centering\arraybackslash}X|}
        \hline
        $\lambda_{n}$ \bolditalic{(m)} \\\hline
        $1: 1,7200$                    \\\hline
        $2: 0,8600$                    \\\hline
        $3: 0,5733$                    \\\hline
        $4: 0,4300$                    \\\hline
    \end{tabularx}
\end{table}

\begin{table}[H]
    \begin{tabularx}{0.9\linewidth}{|>{\centering\arraybackslash}X|>{\centering\arraybackslash}X|>{\centering\arraybackslash}X|}
        \multicolumn{3}{c}{\large $M_{1}$}                                                                         \\\hline
        Velocidad Experimental \bolditalic{(m/s)} & Velocidad Teórica \bolditalic{(m/s)} & Error \bolditalic{(\%)} \\\hline
        $26,5396$                                 & $22,0879$                            & $20,1544$               \\\hline
        $26,1354$                                 & \dots                                & $18,3245$               \\\hline
        $26,2185$                                 & \dots                                & $18,7008$               \\\hline
        $25,8000$                                 & \dots                                & $16,8060$               \\\hline
    \end{tabularx}
\end{table}

\vspace{-.5cm}

\begin{table}[H]
    \begin{tabularx}{0.9\linewidth}{|>{\centering\arraybackslash}X|>{\centering\arraybackslash}X|>{\centering\arraybackslash}X|}
        \multicolumn{3}{c}{\large $M_{2}$}                                                                         \\\hline
        Velocidad Experimental \bolditalic{(m/s)} & Velocidad Teórica \bolditalic{(m/s)} & Error \bolditalic{(\%)} \\\hline
        $22,3256$                                 & $23,6172$                            & $5,4689$                \\\hline
        $22,3600$                                 & \dots                                & $5,3232$                \\\hline
        $22,4976$                                 & \dots                                & $4,7406$                \\\hline
        $22,4460$                                 & \dots                                & $4,9591$                \\\hline
    \end{tabularx}
\end{table}

\vspace{-.5cm}

\begin{table}[H]
    \begin{tabularx}{0.9\linewidth}{|>{\centering\arraybackslash}X|>{\centering\arraybackslash}X|>{\centering\arraybackslash}X|}
        \multicolumn{3}{c}{\large $M_{3}$}                                                                         \\\hline
        Velocidad Experimental \bolditalic{(m/s)} & Velocidad Teórica \bolditalic{(m/s)} & Error \bolditalic{(\%)} \\\hline
        $23,3232$                                 & $25,0533$                            & $6,9057$                \\\hline
        $24,5100$                                 & \dots                                & $2,1686$                \\\hline
        $24,6763$                                 & \dots                                & $1,5050$                \\\hline
        $25,1034$                                 & \dots                                & $0,1999$                \\\hline
    \end{tabularx}
\end{table}

% * -----------------------------|>
\subsubsection{}

Al obtener una frecuencia de resonancia mayor, dada la
formula

\begin{equation*}
    \lambda = \frac{v}{f}
\end{equation*}

Esto quiere decir que $f$ y $\lambda$ son inversamente
proporcionales.

% * -----------------------------|>
\subsubsection{}

No depende la velocidad de propagación de una onda en una
cuerda de la frecuencia de la misma. Como
establece~(\ref{eq:velocidad-onda-cuerda}), la velocidad
dependerá solamente de la tension de la cuerda, ademas de
su densidad lineal.

% * -----------------------------|>
\subsubsection{}

Al aumentar la tension, también aumentara la velocidad de
propagación, asi como lo establece la
ecuación~(\ref{eq:velocidad-onda-cuerda})

% * -----------------------------|>
\subsubsection{}

Al aumentar la densidad lineal de la cuerda, la velocidad
de propagación disminuirá, debido a que son inversamente
proporcionales como lo
establece~(\ref{eq:velocidad-onda-cuerda})

% * -----------------------------|>
\subsubsection{}

Si se mantiene la misma tension en la cuerda, pero se
disminuye la distancia entre los dos extremos, la
frecuencia fundamental aumentará.

Precisamente en la ecuación~(\ref{eq:n_frecuencia}), se
establece que la longitud y la frecuencia resultante son
inversamente proporcionales.

% ! ----------------------------------------------------------------------|>
\section{Conclusiones}

\printbibliography

\end{document}