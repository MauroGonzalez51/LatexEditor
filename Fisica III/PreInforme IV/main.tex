\documentclass[twocolumn, 12pt]{article}

\usepackage[utf8]{inputenc}
\usepackage[english, spanish]{babel}
\usepackage{fullpage}
\usepackage{graphicx}
\usepackage{amsmath}
\usepackage{enumitem}
\usepackage{chngcntr}
\usepackage{setspace}
\usepackage{url}
\usepackage{csquotes}
\usepackage{float}
\usepackage{verbatim}
\usepackage{tabularx}
\usepackage{amsmath}
\usepackage{caption}
\usepackage{bm}
% \usepackage{hyperref}

\counterwithin{figure}{section}
\renewcommand{\thesection}{\arabic{section}}
\renewcommand{\thesubsection}{\thesection.\arabic{subsection}}
\renewcommand{\baselinestretch}{1.5}

\usepackage[style=numeric, maxnames=6, minnames=3, backend=biber, parentracker=true, sorting=none]{biblatex}
\DefineBibliographyStrings{english}{%chktex-file 1 chktex-file 6
    andothers = {\em et\addabbrvspace al\adddot}
}

\addbibresource{./Bibliography/bibliography.bib}

\usepackage{array}
\usepackage{enumitem}

\setlength{\parskip}{0pt}

\newcommand{\bolditalic}[1]{\textbf{\textit{#1}}}

\begin{document}

\begin{titlepage}
    \centering
    \includegraphics[width=0.3\textwidth]{Images/logo_utb.png}\par\vspace{1cm}
    {\scshape\LARGE Universidad Tecnológica de Bolívar \par}
    \vspace{1cm}

    {\scshape\Large FÍSICA CALOR Y ONDAS \par}
    \vspace{.2cm}

    % chktex-file 8
    {\scshape\Large Grupo 1 \par}
    \vspace{1cm}
    % chktex-file 8
    \slshape {\Large \bfseries{}LAB 4 - ONDAS MECÁNICAS:~Velocidad del sonido.\\}
    \vspace{4cm}

    \slshape {\itshape{} Mauro González, T00067622 \\}
    % \slshape {\itshape{} German De Armas Castaño, T00068765 \\}
    % \slshape {\itshape{} Angel Vega Rodriguez, T00068186 \\}
    % \slshape {\itshape{} Juan Jose Osorio Ariza, T00067316 \\}
    % \slshape {\itshape{} Jorge Alberto Rueda Salgado, T00068722 \\}
    \vfill
    Revisado Por \\
    Duban Andres Paternina Verona\\
    {\large \today\par}
\end{titlepage}

% ! ----------------------------------------------------------------------|>
\section{Introducción}

El estudio de las ondas mecánicas, en particular las ondas
de sonido, es esencial para comprender cómo se propagan y
se comportan en diferentes condiciones. En esta
experiencia, nos centraremos en la medición de la velocidad
del sonido en el aire y cómo esta velocidad varía con la
temperatura. Para lograr esto, utilizaremos un montaje
experimental que involucra la generación de pulsos de
sonido y su registro con un micrófono. Este experimento nos
permitirá explorar cómo la temperatura influye en la
velocidad del sonido en el aire y cómo se relaciona con
otras propiedades del gas.

% ! ----------------------------------------------------------------------|>
\section{Objetivos}

\subsection{Objetivo general}

\begin{itemize}[label=$\rightharpoonup$]
    \item Medir la velocidad del sonido en el aire como función de la
          temperatura en un tubo confinado y comprender la relación
          entre la velocidad del sonido, la temperatura y otras
          propiedades del gas.
\end{itemize}

\subsection{Objetivos específicos}

\begin{itemize}[label=$\rightharpoonup$]
    \item Generar pulsos de sonido y registrarlos con un micrófono
          para determinar la velocidad del sonido en el aire.

    \item Variar la temperatura del aire en el tubo y observar cómo
          afecta la velocidad del sonido.

    \item Analizar los datos recopilados para comprender la relación
          entre la velocidad del sonido, la temperatura y las
          propiedades del gas.
\end{itemize}

% ! ----------------------------------------------------------------------|>
\section{Preparación de la practica}

% + ----------------------------------------------------------|>
\subsection{Parámetros del movimiento ondulatorio.~\cite{Entradas_2016}}

Una onda es una perturbación que se propaga a través de un
determinado medio o en el vacío, con transporte de energía
pero sin transporte de materia.

La descripción de estos movimientos se realiza mediante una
ecuación de onda que determina cuál es el estado de
perturbación de cada uno de los puntos situados en la
dirección de propagación $(x)$ en un instante cualquiera
$(t)$. Aunque la onda no propaga materia, es decir, las
partículas no se desplazan en la dirección de propagación,
éstas sí efectúan un movimiento vibratorio que las sitúa en
cada momento a una determinada distancia del punto de
equilibrio, que se denomina elongación $(y)$. La ecuación
de onda es una función $y = f(x,t)$ que suele expresarse
mediante una serie de magnitudes o parámetros
característicos del movimiento ondulatorio:

\begin{itemize}[label=$\rightharpoonup$]
    \item Amplitud \textbf{[A]}: es la elongación máxima o, lo que es
          lo mismo, la máxima distancia de cualquier punto de la onda
          medida respecto a su posición de equilibrio. Se expresa en
          unidades de longitud.

    \item Longitud de onda \textbf{[$\lambda$]}: s la distancia que
          existe entre dos puntos sucesivos que se encuentran en el
          mismo estado de vibración (misma elongación, velocidad,
          aceleración…). Se expresa en unidades de longitud.

    \item Periodo \textbf{[T]}: es el tiempo necesario para describir
          una oscilación completa o, también, el tiempo que emplea la
          onda en recorrer una longitud de onda. Se expresa en
          unidades de tiempo.

    \item Frecuencia \textbf{[f]}: es el número de oscilaciones por
          unidad de tiempo. Se expresa en Hz ($s^{-1}$).
\end{itemize}

Todo esto nos lleva a las siguientes ecuaciones de
onda~\cite{Tomé_2018}

\begin{equation*}
    \begin{gathered}
        r(t) = A \sen(\omega t + \psi_{0}) \\
        y(x, t) = A \sen(\omega t + \kappa x + \psi_{0}) \\
    \end{gathered}
\end{equation*}

% + ----------------------------------------------------------|>
\subsection{Velocidad de onda y velocidad de fase}

\subsubsection*{Velocidad de onda~\cite{Velocidaddeonda}}

La velocidad de las ondas es la velocidad de una onda
progresiva: una perturbación, en forma de oscilación, que
viaja de un lugar a otro y transporta energía

La velocidad de la onda depende de su frecuencia $f$ y de
su longitud de onda $\lambda$.

\begin{equation*}
    \begin{gathered}
        \lambda = \frac{v}{f}\\
        v = \lambda f \\
    \end{gathered}
\end{equation*}

\subsubsection*{Velocidad de fase~\cite{Velocidaddefase}}

La velocidad de fase de una onda es la tasa a la cual la
fase de la misma se propaga en el espacio. Esta es la
velocidad a la cual la fase de cualquier componente en
frecuencia de una onda se propaga (que puede ser diferente
para cada frecuencia).

    {\large
        \begin{equation*}
            v_{\psi} = \frac{\omega}{\kappa}
        \end{equation*}
    }

% + ----------------------------------------------------------|>
\subsection{Ondas de presión en los gases.}

Las ondas de presión son movimientos de vibración en un
sistema mecánico que se propagan a través del medio a alta
velocidad~\cite{ondasdepresion}. En los gases, estas ondas
se propagan como ondas de presión a distintas velocidades,
por ejemplo, al mover la mano desplazamos aire a la
velocidad de la mano, al hablar producimos una onda que se
mueve aproximadamente a la velocidad del sonido y un pistón
de coche produce una onda de choque que se mueve a
velocidad del pistón, por lo general a una velocidad
superior a la del sonido~\cite{Presiónengases}.

En los gases, el principio de Pascal es válido al igual que
en los líquidos. La tendencia a expandirse es la que hace
que los recipientes que los contienen tiendan a tomar
formas redondeadas, como al inflar pelotas o globos, las
burbujas. Los contenedores de gases a presión se hacen de
forma redondeada para soportar mejor la presión
interior~\cite{Libretexts_2020}.

Además, el comportamiento de las ondas de presión en los
gases es fundamental para entender fenómenos como la
propagación del sonido y las explosiones. Por ejemplo,
cuando un objeto explota, las ondas de presión generadas
por la explosión se propagan en todas las direcciones y
pueden causar daños significativos en su
camino~\cite{ondasdepresion}

% ! ----------------------------------------------------------------------|>
\section{Resumen del procedimiento}

En esta experiencia, se utilizará un montaje experimental
que implica generar pulsos de sonido mediante un parlante y
registrar estos pulsos con un micrófono colocado a
diferentes distancias. Se realizarán dos mediciones, una
con el micrófono en una posición inicial y otra en una
posición ligeramente diferente. La diferencia en el tiempo
de llegada de los pulsos de sonido al micrófono se
utilizará para calcular la velocidad del sonido. Además, se
podrá variar la temperatura del aire en el tubo para
observar su influencia en la velocidad del sonido y
analizar los resultados obtenidos. Este experimento nos
proporcionará una comprensión más profunda de las ondas
mecánicas y su comportamiento en función de la temperatura.

\printbibliography

\end{document}