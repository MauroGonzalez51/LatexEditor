\documentclass[letterpaper, 12pt]{report}
\usepackage[utf8]{inputenc}
\usepackage[english, spanish]{babel}
\usepackage{fullpage} % changes the margin
\usepackage{graphicx} 
\usepackage{enumitem} 
\usepackage{chngcntr}
\usepackage{multirow}
\counterwithin{figure}{section}
\renewcommand{\thesection}{\arabic{section}} 
\renewcommand{\thesubsection}{\thesection.\arabic{subsection}}
\renewcommand{\baselinestretch}{1.5}
\usepackage{float}
\bibliographystyle{apalike}
\setlength\belowcaptionskip{10pt}
\linespread{1.5}

\raggedbottom{}

\begin{document}

\begin{titlepage}
	\centering
	\includegraphics[width=0.3\textwidth]{Images/logo_utb.png}\par\vspace{1cm}
	{\scshape\LARGE Universidad Tecnológica de Bolívar \par}
	\vspace{1cm}

	{\scshape\Large FÍSICA ELÉCTRICA \par}
	\vspace{.2cm}

	% chktex-file 8
	{\scshape\Large H1 - C \par}
	\vspace{1cm}
	% chktex-file 8
	\slshape {\Large \bfseries{}Informe de Laboratorio No. 2 \\}
	\vspace{1cm}

	\slshape {\itshape{} Mauro González, T00067622 \\}
	\slshape {\itshape{} German De Armas Castaño, T00068765 \\}
	\slshape {\itshape{} Angel Vega Rodriguez, T00068186 \\}
	\slshape {\itshape{} Juan Jose Osorio Ariza, T00067316 \\}
	\slshape {\itshape{} Juan Eduardo barón, T00065901 \\}
	\vfill
	Revisado Por \\
	Gabriel Hoyos Gomez Casseres\\
	{\large \today\par}
\end{titlepage}

% ----------------------------------------------------------------------|>
\section{Introducción}

\newpage

% ----------------------------------------------------------------------|>
\section{Objetivos}

\subsection{Objetivo General}

\subsection{Objetivos específicos}

\newpage

% ----------------------------------------------------------------------|>
\section{Marco Teórico}

% ------------------------------------------------ |>
\subsection{Fórmula de la desviación  estándar}

\includegraphics[scale = .6]{./Images/FormulaDesviacionEstandar.png} \hfill \break{}~\cite{ImagenDesviacionEstandar}
% chktex-file 24
\label{img:desviacionEstandar}


% ----------------------------------------------------------------------|>
\section{Montaje Experimental}

% ----------------------------------------------------------------------|>
\section{Datos Experimentales}

\begin{table}[h]
	\begin{center}
		% chktex-file 44
		\begin{tabular}{|c|c|c|}
			\multicolumn{3}{c}{Resistencias}              \\ \hline
			Medición & Código        & Multímetro digital \\ \hline
			1        & 220 $\pm$ 5\% & 217,9 $\Omega$     \\ \hline
			2        & 150 $\pm$ 5\% & 149,1 $\Omega$     \\ \hline
			3        & 470 $\pm$ 5\% & 465,0 $\Omega$     \\ \hline
			4        & 330 $\pm$ 5\% & 324,3 $\Omega$     \\ \hline
		\end{tabular}
		\caption{Tabla de resistencias}
		% chktex-file 24
		\label{tab:resistencias}
	\end{center}
\end{table}

% ---------- TABLAS DE VOLTAJE EN AC ---------- |>

\begin{table}[H]
	\begin{center}
		% chktex-file 44
		\begin{tabular}{|c|c|}
			\multicolumn{2}{c}{Voltaje en AC}                     \\ \hline
			Multímetro digital & Multímetro analógico (1/Persona) \\ \hline
			5,44 V             & 6,80 V                           \\
			                   & 6,70 V                           \\
			                   & 6,70 V                           \\
			                   & 6,80 V                           \\
			                   & 6,60 V                           \\ \hline
			Promedio           & 6,72 V                           \\ \hline
		\end{tabular}
		\caption{Voltaje en AC [Primera Medición]}
		% chktex-file 24
		\label{tab:voltajeAC_1}
	\end{center}
\end{table}

\begin{table}[H]
	\begin{center}
		% chktex-file 44
		\begin{tabular}{|c|c|}
			\multicolumn{2}{c}{Voltaje en AC}                     \\ \hline
			Multímetro digital & Multímetro analógico (1/Persona) \\ \hline
			17,17 V            & 21,08 V                          \\
			                   & 20,75 V                          \\
			                   & 21,75 V                          \\
			                   & 20,05 V                          \\
			                   & 21,10 V                          \\ \hline
			Promedio           & 20,95 V                          \\ \hline
		\end{tabular}
		\caption{Voltaje en AC [Segunda Medición]}
		% chktex-file 24
		\label{tab:voltajeAC_2}
	\end{center}
\end{table}

% ---------- TABLAS DE VOLTAJE EN DC ---------- |>

\begin{table}[H]
	\begin{center}
		% chktex-file 44
		\begin{tabular}{|c|c|}
			\multicolumn{2}{c}{Voltaje en DC}                     \\ \hline
			Multímetro digital & Multímetro analógico (1/Persona) \\ \hline
			6,88 V             & 8,20 V                           \\
			                   & 8,20 V                           \\
			                   & 8,20 V                           \\
			                   & 8,20 V                           \\
			                   & 8,15 V                           \\ \hline
			Promedio           & 8.19 V                           \\ \hline
		\end{tabular}
		\caption{Voltaje en DC [Primera Medición]}
		% chktex-file 24
		\label{tab:voltajeDC_1}
	\end{center}
\end{table}

\begin{table}[H]
	\begin{center}
		% chktex-file 44
		\begin{tabular}{|c|c|}
			\multicolumn{2}{c}{Voltaje en DC}                     \\ \hline
			Multímetro digital & Multímetro analógico (1/Persona) \\ \hline
			17,47 V            & 18,85 V                          \\
			                   & 18,50 V                          \\
			                   & 18,50 V                          \\
			                   & 19,00 V                          \\
			                   & 18,50 V                          \\ \hline
			Promedio           & 18,67 V                          \\ \hline
		\end{tabular}
		\caption{Voltaje en DC [Segunda Medición]}
		% chktex-file 24
		\label{tab:voltajeDC_2}
	\end{center}
\end{table}

% ----------------------------------------------------------------------|>
\section{Análisis de datos}

\subsection{Error en las mediciones de voltaje AC y DC}

Usando la formula de la desviación estándar,~\ref{img:desviacionEstandar},
se obtiene:

\begin{table}[H]
	\begin{center}
		% chktex-file 44
		\begin{tabular}{|c|c|}
			\multicolumn{2}{c}{Voltaje en AC}               \\ \hline
			Promedio                  & 6,72                \\ \hline
			Desviación estándar       & 0,083666            \\ \hline
			Valor $\pm$ Incertidumbre & 6,72 $\pm$ 0,083666 \\ \hline
		\end{tabular}
		\caption{Desviación estándar - Voltaje en AC [ Primera Medición ]}
	\end{center}
\end{table}

\begin{table}[H]
	\begin{center}
		% chktex-file 44
		\begin{tabular}{|c|c|}
			\multicolumn{2}{c}{Voltaje en AC}                 \\ \hline
			Promedio                  & 20,95                 \\ \hline
			Desviación estándar       & 0,618328              \\ \hline
			Valor $\pm$ Incertidumbre & 20,95  $\pm$ 0,618328 \\ \hline
		\end{tabular}
		\caption{Desviación estándar - Voltaje en AC [ Segunda Medición ]}
	\end{center}
\end{table}

\begin{table}[H]
	\begin{center}
		% chktex-file 44
		\begin{tabular}{|c|c|}
			\multicolumn{2}{c}{Voltaje en DC}               \\ \hline
			Promedio                  & 8,19                \\ \hline
			Desviación estándar       & 0,022361            \\ \hline
			Valor $\pm$ Incertidumbre & 8,19 $\pm$ 0,022361 \\ \hline
		\end{tabular}
		\caption{Desviación estándar - Voltaje en DC [ Primera Medición ]}
	\end{center}
\end{table}

\begin{table}[H]
	\begin{center}
		% chktex-file 44
		\begin{tabular}{|c|c|}
			\multicolumn{2}{c}{Voltaje en DC}                \\ \hline
			Promedio                  & 18,67                \\ \hline
			Desviación estándar       & 0,238747             \\ \hline
			Valor $\pm$ Incertidumbre & 18,67 $\pm$ 0,238747 \\ \hline
		\end{tabular}
		\caption{Desviación estándar - Voltaje en DC [ Segunda Medición ]}
	\end{center}
\end{table}

% ------------------------------------------------ |>
\subsection{Error en las mediciones de resistencias}

\begin{table}[H]
	\begin{center}
		% chktex-file 44
		\begin{tabular}{|c|c|c|}
			\multicolumn{3}{c}{Resistencias}                         \\ \hline
			Código        & Multímetro digital & Porcentaje de Error \\ \hline
			220 $\pm$ 5\% & 217,9 $\Omega$     & 0.9664 \%           \\ \hline
			150 $\pm$ 5\% & 149,1 $\Omega$     & 0.604  \%           \\ \hline
			470 $\pm$ 5\% & 465,0 $\Omega$     & 1.075  \%           \\ \hline
			330 $\pm$ 5\% & 324,3 $\Omega$     & 1.757  \%           \\ \hline
		\end{tabular}
	\end{center}
\end{table}

\subsection{Cálculos con la Ley de Ohm}

\begin{table}[H]
	\begin{center}
		% chktex-file 44
		\begin{tabular}{|c|c|c|c|c|c|c|c|c|c|}
			% Table Header
			\hline
			\multicolumn{3}{c}{Datos [Multimetro]}                 &
			\multicolumn{2}{c}{Valor [Voltios]}                    &
			\multicolumn{4}{c}{Resistencias [ Multimetro ]}          \\ \hline

			% 1ra fila
			\multicolumn{3}{c}{Voltaje en AC [ Primera Medicion ]} &
			\multicolumn{2}{c}{5,44}                               &
			217,90                                                 &
			149,10                                                 &
			465,00                                                 &
			324,30                                                   \\

			% 2da fila
			\multicolumn{3}{c}{Voltaje en AC [ Segunda Medición ]} &
			\multicolumn{2}{c}{17,17}                              &
			                                                       &
			                                                       &
			                                                       &
			\\

			% 3ra fila
			\multicolumn{3}{c}{Voltaje en DC [ Primera Medicion ]} &
			\multicolumn{2}{c}{6,88}                               &
			                                                       &
			                                                       &
			                                                       &
			\\

			% 4ta fila
			\multicolumn{3}{c}{Voltaje en DC [ Segunda Medicion ]} &
			\multicolumn{2}{c}{17,47}                              &
			                                                       &
			                                                       &
			                                                       &
			\\ \hline
		\end{tabular}
		\caption{Calculo de la corriente con la Ley de Ohm - Primera Parte}
		\label{tab:LeyDeOhmPrimeraParte}
	\end{center}
\end{table}

\begin{table}[H]
	\begin{center}
		% chktex-file 44
		\begin{tabular}{| p{2cm} | p{2cm} | p{2cm} | p{2cm} |}
			% Table Header
			\hline
			\multicolumn{4}{c}{Resultados [ I ]}      \\ \hline


			% 1ra fila
			1        & 2        & 3        & 4        \\ \hline

			% 2da fila
			0,024966 & 0,036486 & 0,011699 & 0,016775 \\ \hline

			% 3ra fila
			0,078798 & 0,115158 & 0,036925 & 0,052945 \\ \hline

			% 4ta fila
			0,031574 & 0,046144 & 0,014796 & 0,021215 \\ \hline

			% 5ta fila
			0,080174 & 0,117170 & 0,037570 & 0,053870 \\ \hline
		\end{tabular}
		\caption{Calculo de la corriente con la Ley de Ohm - Segunda Parte}
		\label{tab:LeyDeOhmSegundaParte}
	\end{center}
\end{table}

% ------------------------------------------------ |>
\subsection{Respuesta a preguntas de la guía de laboratorio}

% -----------------------|>
\subsubsection{¿Cuáles son los pasos que debo hacer para realizar una medida
	con un multímetro?}

A pesar de ser el instrumento básico que cualquier persona relacionada con
la electrónica o la electricidad utiliza, la forma de utilizarlo cambia un poco
dependiendo de lo que queramos medir. Por ejemplo, para medir \textit{Voltaje},
la medición se realiza en paralelo, mientras que para medir \textit{Corriente},
la medición es en serie.

% -----------------------|>
\subsubsection{¿Cómo se conecta un multímetro para medir voltaje, corriente
	y resistencia?}

\begin{itemize}
	\item Voltaje: La medición de voltaje se realiza en paralelo, así que
	      sólo es necesario colocar la punta positiva del multímetro (roja) con
	      el punto positivo a medir, también hay que colocar la punta negativa
	      (negro) con el punto negativo a medir.~\cite{AcMaxMexico}

	\item Corriente: Lo primero a tener en cuenta es que la corriente no se
	      mide de la misma manera que el voltaje. La corriente se mide en serie.
	      Para hacer la medición en serie, hay que abrir nuestro circuito y
	      conectar el multímetro como si fuera un resistor más. La resistencia
	      del multímetro es muy pequeña y rara vez altera la exactitud de
	      nuestras mediciones.~\cite{AcMaxMexico}

	\item Resistencia: Para medir con un multímetro, sólo se colocan las puntas
	      en cada terminal de nuestra resistencia que queremos medir, sin importar
	      el color. Las puntas se colocan en el multímetro en el mismo lugar que
	      para medir voltaje y se selecciona la función con la letra griega omega
	      mayúscula $\Omega$.~\cite{AcMaxMexico}
\end{itemize}

% -----------------------|>
\subsubsection{¿Qué precauciones debo tener en cuenta con los instrumentos a
	la hora de realizar una medición?}

Según, (\cite{flukeCorporation}, Guía de Seguridad para Multímetros Digitales),
Antes de realizar una medida con el multímetro se debe someter a una inspección
visual. Compruebe que el multímetro, las sondas de prueba y los accesorios no
presentan daños físicos. Asegúrese de que todas las conexiones encajan
firmemente y de que no se aprecie metal al descubierto ni grietas en la carcasa.
No utilice nunca un multímetro ni sondas de prueba que presenten daños.

\vspace{.5cm}

Una vez finalizada la inspección visual, compruebe que el multímetro
funciona correctamente. Nunca se debe de dar por hecho.

% -----------------------|>
\subsubsection{¿Cuáles son las ventajas y desventajas del multímetro digital
	respecto al analógico?}

\begin{center}
	Ventajas
\end{center}

\begin{itemize}
	\item No muestra errores de lectura en la pantalla.
	\item En la mayoría de los multímetros, detección automática de
	      la polaridad.
	\item Detección automática del rango de medición.
	\item Mayor precisión del valor medido mostrado en la pantalla.
	\item Menos sensible a interferencias, ruidoso flujos magnéticos.
	\item Más económico de producir porque tiene menos componentes mecánicos.
\end{itemize}

\vspace{.5cm}

\begin{center}
	Desventajas
\end{center}

\begin{itemize}
	\item Con frecuencias altas existen mediciones incorrectas en corriente
	      alterna.
	\item Los picos de tensión pueden dañar el multímetro.
\end{itemize}

~\cite{CesarCinjordiz}

% ----------------------------------------------------------------------|>
\section{Conclusiones}

\newpage
\bibliography{./Bibliography/bibliography.bib}

\end{document}