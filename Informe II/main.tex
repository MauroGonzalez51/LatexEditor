\documentclass[letterpaper, 12pt]{report}
\usepackage[utf8]{inputenc}
\usepackage[english, spanish]{babel}
\usepackage{fullpage} % changes the margin
\usepackage{graphicx} 
\usepackage{enumitem} 
\usepackage{chngcntr}
\counterwithin{figure}{section}
\renewcommand{\thesection}{\arabic{section}} 
\renewcommand{\thesubsection}{\thesection.\arabic{subsection}}
\renewcommand{\baselinestretch}{1.5}
\usepackage{float}
\bibliographystyle{apalike}
\setlength\belowcaptionskip{10pt}
\linespread{1.5}

\raggedbottom{}

\begin{document}

\begin{titlepage}
	\centering
	\includegraphics[width=0.3\textwidth]{Images/logo_utb.png}\par\vspace{1cm}
	{\scshape\LARGE Universidad Tecnológica de Bolívar \par}
	\vspace{1cm}

	{\scshape\Large FÍSICA ELÉCTRICA \par}
	\vspace{.2cm}

	% chktex-file 8
	{\scshape\Large H1 - C \par}
	\vspace{1cm}
	% chktex-file 8
	\slshape {\Large \bfseries{}Informe de Laboratorio No. 2 \\}
	\vspace{1cm}

	\slshape {\itshape{} Mauro González, T00067622 \\}
	\slshape {\itshape{} German De Armas Castaño, T00068765 \\}
	\slshape {\itshape{} Angel Vega Rodriguez, T00068186 \\}
	\slshape {\itshape{} Juan Jose Osorio Ariza, T00067316 \\}
	\slshape {\itshape{} Juan Eduardo barón, T00065901 \\}
	\vfill
	Revisado Por \\
	Gabriel Hoyos Gomez Casseres\\
	{\large \today\par}
\end{titlepage}

% ----------------------------------------------------------------------|>
\section{Introducción}

\newpage

% ----------------------------------------------------------------------|>
\section{Objetivos}

\subsection{Objetivo General}

\subsection{Objetivos específicos}

\newpage

% ----------------------------------------------------------------------|>
\section{Marco Teórico}

% ----------------------------------------------------------------------|>
\section{Montaje Experimental}

% ----------------------------------------------------------------------|>
\section{Datos Experimentales}

\begin{table}[h]
	\begin{center}
		% chktex-file 44
		\begin{tabular}{|c|c|c|}
			\multicolumn{3}{c}{Resistencias}              \\ \hline
			Medición & Código        & Multímetro digital \\ \hline
			1        & 220 $\pm$ 5\% & 217,9 $\Omega$     \\ \hline
			2        & 150 $\pm$ 5\% & 149,1 $\Omega$     \\ \hline
			3        & 470 $\pm$ 5\% & 465,0 $\Omega$     \\ \hline
			4        & 330 $\pm$ 5\% & 324,3 $\Omega$     \\ \hline
		\end{tabular}
		\caption{Tabla de resistencias}
		% chktex-file 24
		\label{tab:resistencias}
	\end{center}
\end{table}

% ---------- TABLAS DE VOLTAJE EN AC ---------- |>

\begin{table}[H]
	\begin{center}
		% chktex-file 44
		\begin{tabular}{|c|c|}
			\multicolumn{2}{c}{Voltaje en AC}                     \\ \hline
			Multímetro digital & Multímetro analógico (1/Persona) \\ \hline
			5,44 V             & 6,80 V                           \\
			                   & 6,70 V                           \\
			                   & 6,70 V                           \\
			                   & 6,80 V                           \\
			                   & 6,60 V                           \\ \hline
			Promedio           & 6,72 V                           \\ \hline
		\end{tabular}
		\caption{Voltaje en AC [Primera Medición]}
		% chktex-file 24
		\label{tab:voltajeAC_1}
	\end{center}
\end{table}

\begin{table}[H]
	\begin{center}
		% chktex-file 44
		\begin{tabular}{|c|c|}
			\multicolumn{2}{c}{Voltaje en AC}                     \\ \hline
			Multímetro digital & Multímetro analógico (1/Persona) \\ \hline
			17,17 V            & 21,08 V                          \\
			                   & 20,75 V                          \\
			                   & 21,75 V                          \\
			                   & 20,05 V                          \\
			                   & 21,10 V                          \\ \hline
			Promedio           & 20,95 V                          \\ \hline
		\end{tabular}
		\caption{Voltaje en AC [Segunda Medición]}
		% chktex-file 24
		\label{tab:voltajeAC_2}
	\end{center}
\end{table}

% ---------- TABLAS DE VOLTAJE EN DC ---------- |>

\begin{table}[H]
	\begin{center}
		% chktex-file 44
		\begin{tabular}{|c|c|}
			\multicolumn{2}{c}{Voltaje en DC}                     \\ \hline
			Multímetro digital & Multímetro analógico (1/Persona) \\ \hline
			6,88 V             & 8,20 V                           \\
			                   & 8,20 V                           \\
			                   & 8,20 V                           \\
			                   & 8,20 V                           \\
			                   & 8,15 V                           \\ \hline
			Promedio           & 8.19 V                           \\ \hline
		\end{tabular}
		\caption{Voltaje en DC [Primera Medición]}
		% chktex-file 24
		\label{tab:voltajeDC_1}
	\end{center}
\end{table}

\begin{table}[H]
	\begin{center}
		% chktex-file 44
		\begin{tabular}{|c|c|}
			\multicolumn{2}{c}{Voltaje en DC}                     \\ \hline
			Multímetro digital & Multímetro analógico (1/Persona) \\ \hline
			17,47 V            & 18,85 V                          \\
			                   & 18,50 V                          \\
			                   & 18,50 V                          \\
			                   & 19,00 V                          \\
			                   & 18,50 V                          \\ \hline
			Promedio           & 18,67 V                          \\ \hline
		\end{tabular}
		\caption{Voltaje en DC [Segunda Medición]}
		% chktex-file 24
		\label{tab:voltajeDC_2}
	\end{center}
\end{table}

% ----------------------------------------------------------------------|>
\section{Análisis de datos}

\subsection{Respuesta a preguntas de la guía de laboratorio}

% ----------------------------------------------------------------------|>
\section{Conclusiones}

\newpage
\bibliography{./Bibliography/bibliography.bib}

\end{document}