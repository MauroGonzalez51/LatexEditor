\documentclass[twocolumn, 12pt]{article}

\usepackage[utf8]{inputenc}
\usepackage[english, spanish]{babel}
\usepackage{fullpage} % changes the margin
\usepackage{graphicx}
\usepackage{amsmath}
\usepackage{enumitem}
\usepackage{chngcntr}
\usepackage{setspace}
\usepackage{url}
\usepackage{csquotes}
\usepackage{float}
\usepackage{verbatim}
\usepackage{tabularx}
\usepackage{amsmath}

\counterwithin{figure}{section}
\renewcommand{\thesection}{\arabic{section}}
\renewcommand{\thesubsection}{\thesection.\arabic{subsection}}
\renewcommand{\baselinestretch}{1.5}

\usepackage[style=apa, maxnames=6, minnames=3, backend=biber]{biblatex}
\DefineBibliographyStrings{english}{%chktex-file 1 chktex-file 6
	andothers = {\em et\addabbrvspace al\adddot}
}
\addbibresource{./Bibliography/bibliography.bib}

\usepackage{array}
\usepackage{enumitem}
\usepackage{breqn}

\raggedbottom{}

\begin{document}

% 									|		PRACTICA 6                 |
%  									|		GUÍA No. 8                 |

\begin{titlepage}
	\centering
	\includegraphics[width=0.3\textwidth]{Images/logo_utb.png}\par\vspace{1cm}
	{\scshape\LARGE Universidad Tecnológica de Bolívar \par}
	\vspace{1cm}

	{\scshape\Large FÍSICA ELÉCTRICA \par}
	\vspace{.2cm}

	% chktex-file 8
	{\scshape\Large H1 - C \par}
	\vspace{1cm}
	% chktex-file 8 chktex-file 13
	\slshape {\Large \bfseries{} LAB 6 - FENÓMENOS ELECTROMAGNÉTICOS \\}
	\slshape {\small \bfseries{} Guía de laboratorio No. 8}
	\vspace{1cm}

	\slshape {\itshape{} Mauro González, T00067622 \\}
	\slshape {\itshape{} German De Armas Castaño, T00068765 \\}
	\slshape {\itshape{} Angel Vega Rodriguez, T00068186 \\}
	\slshape {\itshape{} Juan Jose Osorio Ariza, T00067316 \\}
	\slshape {\itshape{} Juan Eduardo barón, T00065901 \\}
	\vfill
	Revisado Por \\
	Gabriel Hoyos Gomez Casseres\\
	{\large \today\par}
\end{titlepage}

% ----------------------------------------------------------------------|>

\section{Introducción}

Los fenómenos electromagnéticos surgen como resultado de la
interacción entre campos eléctricos y magnéticos, los
cuales se relacionan con la presencia de cargas eléctricas
en movimiento. Cuando una carga eléctrica se desplaza,
genera un campo eléctrico y magnético en su entorno.

La comprensión de la relación entre el campo eléctrico y el
campo magnético se ha desarrollado gracias a las leyes de
Faraday y la ley de Lenz. Estas leyes permiten conocer cómo
se pueden modificar los campos eléctrico y magnético y qué
factores influyen en ellos, como la tensión inducida en un
circuito cerrado o la polaridad de la fuerza electromotriz.

En este contexto, el transformador surge como un
dispositivo que representa esta dualidad, y que se utiliza
para transferir energía eléctrica de un circuito a otro
mediante el principio de inducción electromagnética. El
transformador está compuesto por dos o más bobinas de
alambre enrolladas alrededor de un núcleo de hierro, las
cuales pueden acoplarse magnéticamente para transferir
energía entre ellas.

En cuanto a la descripción del campo electromagnético, las
ecuaciones de Maxwell son las expresiones fundamentales que
permiten describirlo. Dichas ecuaciones pueden expresarse
en forma integral y constituyen la base de la teoría
electromagnética.

{\normalsize
\begin{equation}
	\begin{gathered}
		\oint_r \vec{E} \cdot d \vec{l} = - \frac{d}{dt} \int_{A}^{} \vec{B} \cdot \,d\vec{A} \\
		\oint_r \vec{H} \cdot d \vec{l} = \frac{d}{dt} \int_{A}^{} \vec{D} \cdot d \,d\vec{A} + \int_{A}^{} \vec{J} \cdot \,d\vec{A} \\
		\oint_S \vec{D} \cdot d \vec{S} = \int_{V}^{} q \,d{v} \\
		\oint_S \vec{B} \cdot d \vec{S} = 0
	\end{gathered}
\end{equation}
}

En este pre-informe vamos a analizar fenómenos
electromagnéticos, el funcionamiento de un trasformador
teniendo en cuenta diferentes leyes, cálculos y conceptos
que serán tratados.

% ----------------------------------------------------------------------|>
\section{Objetivos}

% -----------------------------------|>
\subsection*{Objetivo General}

\begin{itemize}[label=$\triangleright$]
	\item Observar y explicar algunos fenómenos donde se evidencia la
	      relación entre el campo eléctrico y campo magnético.
\end{itemize}

% -----------------------------------|>
\subsection*{Objetivos específicos}

\begin{itemize}[label=$\triangleright$]
	\item Evidenciar los fenómenos que ocurren con la implementación
	      del imán siguiendo las leyes de electromagnetismo.
	\item Comprender la inducción de un campo eléctrico según la ley
	      de Faraday, Ley de Ampere y Ley de Ampere-Maxwell.
	\item Entender la importancia que tiene un transformador.
\end{itemize}

% ----------------------------------------------------------------------|>
\section{Preparación de la practica}

% -----------------------------------------------------------|>
\subsection*{Ley de Faraday}

Establece que un cambio en el flujo magnético a través de
una superficie cerrada induce una fuerza electromotriz
\textit{(FEM)} en un circuito eléctrico que rodea esa
superficie. De ahí que se conozca que la tensión inducida
en un circuito cerrado es directamente proporcional a la
rapidez del cambio de flujo magnético que pasa a través de
una espira (o lazo). Matemáticamente, la ley de Faraday se
expresa como:

{\Large
\begin{equation}
	\varepsilon = - \frac{d\phi_B}{dt} \centering
\end{equation}
}

El signo menos es una indicación del sentido de la fem
inducida. Si la bobina tiene N vueltas, aparece una fem en
cada vuelta que se pueden sumar, es el caso de los tiroides
y solenoides, en estos casos la fem inducida será:

{\Large
\begin{equation}
	\varepsilon = - N \frac{d\phi_B}{dt} = - \frac{d(N\phi_B)}{dt}
\end{equation}
}

Tomado de:\hfill \break{}
(\cite{LeyFaraday}),~(\cite{FaradayLawKhanAcademy})

% -----------------------------------------------------------|>
\subsection*{Ley de Lenz}

Es una consecuencia del principio de conservación de la
energía aplicado a la inducción electromagnética la cual
nos dice en qué dirección fluye la corriente, y establece
la dirección de la corriente inducida debe ser tal que su
propio campo magnético se dirija de una manera que se
oponga al flujo cambiante que causa el campo del imán que
se aproxima, es decir, la dirección siempre es tal que se
opone al cambio de flujo que la produce. Esto significa que
cada campo magnético generado por una corriente inducida va
en la dirección opuesta al cambio en el campo original. Por
lo tanto, la corriente inducida circula de manera que sus
líneas de campo magnético a través del bucle se dirigen
desde la parte trasera a la delantera del bucle.

	{\Large
		\begin{equation}
			\varepsilon = - \frac{d\phi_m}{dt}
		\end{equation}
	}

Tomado de:~(\cite{FaradayLawKhanAcademy})

% ----------------------------------------------------------------------|>
\subsection*{Let de Ampere y Ley de Ampere-Maxwell}

La ley de Ampere establece que un campo magnético que pasa
por una trayectoria cerrada por el que fluye la corriente,
provoca que este campo magnético sea igual a la
permeabilidad constante del espacio, multiplicada por la
fuerza total de la corriente. En otras palabras, la ley
establece que la corriente que fluye en un conductor
produce un campo magnético. La ley de los amperios de
Maxwell establece que un campo magnético que pasa por un
camino cerrado que contiene una corriente hace que este
campo magnético sea igual a la permeabilidad constante del
espacio igual a la suma de los dos tipos de corrientes;
corriente total y corriente de desplazamiento. En otras
palabras, la ley establece que los campos magnéticos son
producidos tanto por corrientes de conducción como de
desplazamiento.

Esta ley determina que la circulación del campo magnético a
lo largo de una línea cerrada es equivalente a la suma
algebraica de las intensidades de las corrientes que
atraviesan la superficie delimitada por la línea cerrada,
multiplicada por la permitividad del medio. En concreto
para el vacío:

{\Large
\begin{equation}
	\oint \vec{B} d \vec{l} = \mu_{\scriptscriptstyle 0} I_{\scriptscriptstyle T}
\end{equation}
}

\begin{itemize}[label=$\triangleright$]
	\item {\Large $\mu_0$}: Permeabilidad del espacio
	\item {\Large $I_T$}: Intensidad de la corriente neta
\end{itemize}

\nocite{FenomenosElectromagneticos}

% ----------------------------------------------------------------------|>
\subsection*{Corrientes parásitas}

Ocurren cuando el conductor pasa a través de un campo
magnético alterno o viceversa. El movimiento relativo causa
circulación electrónica o corriente de inducción dentro de
los conductores. Estas líneas circulares producen campos
magnéticos, que son resistentes a los efectos del campo
magnético montado. El campo magnético se usa o mayor que la
conductividad del conductor, o la velocidad de tráfico
relativamente mayor, se generan más parásitos y los campos
opuestos~(\cite{FenomenosElectromagneticos}).

% ----------------------------------------------------------------------|>
\subsection*{¿Qué es un transformador y cuál es su uso?}

Un transformador de potencia es una máquina de CA estática
que le permite cambiar ciertas funciones actuales, como el
voltaje o la corriente, mientras mantiene la frecuencia y
la potencia.

Estas máquinas ayudan a mejorar la seguridad y eficiencia
de los sistemas de energía durante la distribución y
regulación a través de las largas
distancias~(\cite{FenomenosElectromagneticos})

\newpage

\printbibliography

\end{document}