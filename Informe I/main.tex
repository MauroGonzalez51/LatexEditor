\documentclass[letterpaper, 12pt]{report}
\usepackage[utf8]{inputenc}
\usepackage[english, spanish]{babel}
\usepackage{fullpage} % changes the margin
\usepackage{graphicx} 
\usepackage{enumitem} 
\usepackage{chngcntr}
\usepackage{setspace}
\counterwithin{figure}{section}
\renewcommand{\thesection}{\arabic{section}} 
\renewcommand{\thesubsection}{\thesection.\arabic{subsection}}
\renewcommand{\baselinestretch}{1.5}
\usepackage{float}
\bibliographystyle{apalike}
\setlength\belowcaptionskip{10pt}
\linespread{1.5}

\begin{document}

\begin{titlepage}
	\centering
	\includegraphics[width=0.3\textwidth]{Images/logo_utb.png}\par\vspace{1cm}
	{\scshape\LARGE Universidad Tecnológica de Bolívar \par}
	\vspace{1cm}

	{\scshape\Large FÍSICA ELÉCTRICA \par}
	\vspace{.2cm}

	% chktex-file 8
	{\scshape\Large H1 - C \par}
	\vspace{1cm}
	% chktex-file 8
	\slshape {\Large \bfseries{}Informe de Laboratorio No. 1 \\}
	\vspace{1cm}

	\slshape {\itshape{} Mauro González, T00067622 \\}
	\slshape {\itshape{} German De Armas Castaño, T00068765 \\}
	\slshape {\itshape{} Angel Vega Rodriguez, T00068186 \\}
	\slshape {\itshape{} Juan Jose Osorio Ariza, T00067316 \\}
	\slshape {\itshape{} Juan Eduardo barón, T00065901 \\}
	\vfill
	Revisado Por \\
	Gabriel Hoyos Gomez Casseres\\
	{\large \today\par}
\end{titlepage}

% ----------------------------------------------------------------------|>
\section{Introducción}

En el presente informe con los resultados obtenidos en el laboratorio junto
con unos de los ejercicios más básicos de la física experimental se busca crear
conclusiones puntuales acerca de la energía eléctrica resultante debido a la
acción de fuerzas externas como la fricción; A su vez estudiando como
interactúan con diversos objetos, entregando nuevos conceptos tales como el
campo eléctrico, el cual permite describir un fenómeno especifico a pie de
detalles como sus características y propiedades en un área y objeto determinado.

\vspace{.5cm}

Analizando a fondo y de manera empírica podemos percibir el campo electrostático
como una distribución espacial que ejerce fuerza sobre otros cuerpos siendo
bastante similar al  funcionamiento de fuerzas fundamentales como la
gravedad, estas conjeturas a partir de similitudes pueden ser bases
fundamentales para la comprensión de próximos laboratorios y conceptos.

\vspace{.5cm}

Estos experimentos son una base fundamental para dar una
idea del potencial que tienen algunos fenómenos físicos, tal como lo es la
electricidad, ademas de conocer varias de sus aplicaciones.

\newpage

% ----------------------------------------------------------------------|>
\section{Objetivos}

\subsection{Objetivo General}

Analizar  y comprender mediante los conceptos y/o experiencia adquirida en la
practica el comportamiento que tienen ciertos materiales cuando son sometidos
a distintas fuerzas electrostáticas.

\subsection{Objetivos Específicos}

\begin{itemize}
	\item Comprender  en el laboratorio las temáticas y conceptos descritos
	      en la guía.
	\item Analizar mediante la experimentación los fenómenos
	      electrostáticos (contacto, frotación, inducción).
	\item Relacionar los fenómenos descritos en la práctica con aquellos
	      fenómenos evidenciados en la cotidianidad.
\end{itemize}

\newpage

% ----------------------------------------------------------------------|>
\section{Marco Teórico}

% ----------------------------------------------------------------------|>
\section{Montaje, datos experimentales e análisis de datos}

% ----------------------------------------|>
\subsection{Video 1. Electricidad por fricción}

% -----------------------------|>
\subsubsection{Experimento 1}

\begin{itemize}
	\item Se frota la barra de plástico con una tela y luego procede a
	      acercarla a las  virutas de madera, aluminio, papel, una lata de
	      metal y agua.

	\item Observado: Al acercar la barra a los materiales estos se atraerán
	      sin importar si son aislantes o conductores
\end{itemize}

Explicación: Esto se debe a que cuando la varilla de plástico roza contra la
tela, se carga, por lo que la varilla tiende a eliminar estos electrones en
exceso para volver al equilibrio y lo hace tratando de transferir los
electrones  a otro material.~\cite{GenerarCargaElectrica}

% -----------------------------|>
\subsubsection{Experimento 2}

\begin{itemize}
	\item Se acerca la barra previamente cargada a un dispositivo el cual
	      emula el comportamiento de un electroscopio.

	\item Observado: Al acercar la barra a el dispositivo, se separan las
	      láminas de aluminio.
\end{itemize}

Explicación: Esto se debe a que la varilla de plástico se carga con
electrones que pasan a través del alambre de cobre y hacen que las placas
de aluminio se carguen con la misma polaridad
eléctrica.~\cite{GenerarCargaElectrica}

% -----------------------------|>
\subsubsection{Experimento 3}

\begin{itemize}
	\item Se acerca la barra de plástico previamente cargada a una esfera de
	      icopor suspendida en un péndulo.

	\item Observado: Al acercar la barra a la bola de icopor esta queda adherida durante
	      un momento y luego se separa.
\end{itemize}

Explicación: Esto se debe a que los tres objetos tienen la misma carga,
porque el tubo de plástico cargado negativamente se transfiere a las
esferas, y los tres objetos quedan cargados negativamente, y por lo tanto,
al ser cargas iguales, se repelen entre sí.~\cite{LeyCargasElectricas}

% ----------------------------------------|>
\subsection{Video 2. Electricidad por fricción. Péndulos eléctricos}

% -----------------------------|>
\subsubsection{Experimento 1}

\begin{itemize}
	\item Se frota un  trozo de acrílico con una bolsa de plástico,
	      luego esta se acerca a una bola de icopor recubierta de aluminio,
	      luego la bola de plástico se acerca a la bola.

	\item Observado: Al acercar el pedazo de acrílico a la bola estos se
	      atraen para posteriormente repelerse, luego cuando se acerca la bolsa
	      a la bola estas se atraen
\end{itemize}

Explicación: Esto se debe a que el acrílico queda cargado negativamente,
quiere decir, que ha ganado electrones, los cuales transfiere a la esfera
y como ambas quedan con la misma carga, se repelen, luego como la bolsa
perdió electrones y quedo con carga positiva, la esfera y la bolsa se
atraen por sus cargas diferentes.
~\cite{GenerarCargaElectrica}~\cite{LeyCargasElectricas}

% -----------------------------|>
\subsubsection{Experimento 2}

\begin{itemize}
	\item Se frota un  trozo de acrílico con una bolsa de plástico y se
	      acerca a la bola  para luego proceder a frotar el acrílico en tela de
	      poliéster y se vuelve a acercar a la bola.

	\item Observado: La bola se carga negativamente gracias a el acrílico
	      y la bolsa de plástico, luego el acrílico es frotado con la tela de
	      poliéster y se acerca a la bola y estas se atraen.
\end{itemize}

Explicación: La bola se carga negativamente cuando, al frotar el
acrílico con tela de poliéster, el acrílico perdió los electrones
que ingresan a la tela, por lo que cuando se acerca la bola  se
sienten atraídos por sus diferentes cargas, pero cuando el acrílico
se sigue frotando en la  tela, eliminará los electrones que le dio la bola,
y cuando se acerca el acrílico, la pelota comienza a llevar electrones al
acrílico para restaurar el equilibrio y, por lo tanto, el acrílico
comienza a restaurar los electrones.~\cite{LeyCargasElectricas}

% -----------------------------|>
\subsubsection{Experimento 3}

\begin{itemize}
	\item Se frota el acrílico con la bolsa de plástico y luego se frota
	      el acrílico con la tela de poliéster, ambas se acercaran a las bolas
	      de icopor.

	\item Observado: Se acerca el acrílico a las bolas de icopor que ya
	      fueron previamente cargadas y estas se repelen, luego al frotar el
	      acrílico a la tela de poliéster y acercarlo a las bolas estas se
	      atraen.
\end{itemize}

Explicación: Al frotar el acrílico con la bolsa, esta queda cargado
negativamente, ya que la bolsa le sede sus electrones a el acrílico luego el
acrílico se acerca a cada bola para cargarla negativamente y al juntar las
tres bolas estas se repelen porque tienen la misma carga negativa; Luego al
frotar el acrílico con la tela este queda cargado positivamente ya que le
cedió electrones y al acercarse a la bolas estas al inicio se atraen pero
al haber un intercambio de cargas se repelen.~\cite{LeyCargasElectricas}

% ----------------------------------------|>
\subsection{Video 3. Electroscopio de William Gilbert}

% -----------------------------|>
\subsubsection{Experimento 1}

\begin{itemize}
	\item Se tiene una bola de icopor recubierta de aluminio luego se
	      frota con una tela una barra de acrílico y otra de vidrio y se
	      acercan la bola.

	\item Observado: Luego de ser frotada con la tela se acerca la barra de
	      acrílico a la bola de icopor y estas se repelen, luego se frota la
	      barra de vidrio y se hace el mismo procedimiento y estas se atraen.
\end{itemize}

Explicación: La varilla de acrílico roza la tela y se carga, al acercarse
a la esfera se repelen, es decir ambas llevan la misma carga, luego la
varilla de vidrio roza la tela y se acerca a la esfera se atraen, entonces,
tienen cargas diferentes, porque la pelota que le queda lleva la carga que
le imparte la varilla de acrílico.~\cite{LeyCargasElectricas}

% -----------------------------|>
\subsubsection{Experimento 2}

\begin{itemize}
	\item Se frota una barra de acrílico con la tela y se acerca a un
	      electroscopio.

	\item Observado: Las láminas de aluminio se separan.
\end{itemize}

Explicación: Frotando la varilla de acrílico con un paño se realiza una
transferencia de electrones, luego se acerca a la esfera de aluminio del
electroscopio, y por inducción se observa que las placas se separan por
la transferencia de electrones a través del cable de cobre, actuando como
conductor, el resultado es que ambas placas tienen la misma carga y como
tienen la misma carga se repelen.~\cite{GenerarCargaElectrica}

% ----------------------------------------|>
\subsection{Video 4. Electroscopio, el versorium de Gilbert}

% -----------------------------|>
\subsubsection{Experimento 1}

\begin{itemize}
	\item Se frota la barra de acrílico con la tela de jean y luego procede
	      a acercarse al versorium.

	\item Observado: Al acercase la barra de acrílico previamente frotada
	      con la tela  al versorium y la aguja de este se mueve.
\end{itemize}

Experimento: Cuando se frota una varilla de acrílico con un paño,
se carga, y cuando se acerca al versorium, se crea un campo de fuerza
por las cargas que actúan sobre él, haciendo que la aguja se mueva,
aunque tenga la misma carga o diferente carga eléctrica.
~\cite{GenerarCargaElectrica}

% -----------------------------|>
\subsubsection{Experimento 2}

\begin{itemize}
	\item Se frota una barra de acrílico con un pedazo de tela elástica
	      para posteriormente acercarla a otra barra de acrílico que
	      está colgando.

	\item Observado: Al acercase las barras de acrílico al una ser frotada con
	      tela la que está colgando se mueve.
\end{itemize}

Explicación: Cuando la varilla de acrílico se frota con la tela flexible,
hay una interacción de cargas, luego cuando la varilla se acerca a la otra
varilla que está suspendida, la varilla cargada transferirá la carga a la
otra varilla, debido a esto, primero dos las varillas se juntan, pero después,
dado que ambas todavía están cargadas con la misma polaridad, se repelen
entre sí.~\cite{TransferenciaContacto}

% ----------------------------------------|>
\subsection{Video 5. Van de Graaf y jaula de Faraday}

% -----------------------------|>
\subsubsection{Experimento 1}

\begin{itemize}
	\item Se acerca la esfera de descarga al Van de Graaf, luego se
	      coloca encima los recipientes de aluminio, los vasos y la
	      tela metálica.

	\item Observado: Cuando la esfera se acerca se puede observar una chispa
	      luego cuando se acercan los recipientes y la tela metálica con tiras
	      de papel, estas se levantan.
\end{itemize}

Explicación: Las chispas se generan por la gran diferencia de potencial que
hace la maquina, al colocar los recipientes de aluminio parecen salir volando
porque todos están cargados con la misma polaridad entonces se repelen entre
si porque el generador tiene carga negativa. Cuando se colocan los vasos y
la tela, los papelitos se levantan, debido a que las cargas se acumulan en
la superficie exterior de los recipientes y no en la interna y además poseen
la misma polaridad eléctrica.~\cite{LeyCargasElectricas}

% ----------------------------------------|>
\subsection{Respuesta a preguntas de la guía de laboratorio}

% -----------------------------|>
\subsubsection{¿Cuando decimos que un cuerpo esta cargado electricamente?}

R// Cuando un átomo, o un cuerpo, tiene la misma cantidad de cargas
positivas (protones) y negativas (electrones) se dice que está eléctricamente
neutro.

\vspace{.5cm}

Si se produce un desequilibrio entre la cantidad de electrones y
protones, se dice que está electrizado. El cuerpo que pierde electrones
queda con carga positiva y el que recibe electrones queda con carga negativa.
Se llama carga eléctrica (q) al exceso o déficit de electrones que posee un
cuerpo respecto al estado neutro.~\cite{CargaElectrica}

% -----------------------------|>
\subsubsection{¿Qué es lo que se transfiere de un cuerpo a otro en
	el proceso de cargar eléctricamente un cuerpo?}

R// Al frotar dos cuerpos eléctricamente neutros (número de electrones igual
al número de protones), ambos se cargan, uno con carga positiva y el otro con
carga negativa, es decir, se transfieren electrones de un cuerpo a otro.
~\cite{ElectrizacionWikipedia}

% -----------------------------|>
\subsubsection{¿Cuál es el requisito para que dos cuerpos interactúen
	eléctricamente?}

R// Para que dos cuerpos interactúen eléctricamente solo se necesita que
estén cargados positivamente o negativamente ya que cuando dos cuerpos
interactúan se producen fuerzas eléctricas de repulsión y/o atracción.
~\cite{TransferenciaParticulas}

% ----------------------------------------------------------------------|>
\section{Conclusiones}

\newpage
\bibliography{./Bibliography/bibliography.bib}

\end{document}