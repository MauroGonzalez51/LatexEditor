\documentclass[letterpaper, 12pt]{report}

\usepackage[utf8]{inputenc}
\usepackage[english, spanish]{babel}

\usepackage{newtxtext}

\usepackage{fullpage}
\usepackage{graphicx}
\usepackage{amsmath}
\usepackage{enumitem}
\usepackage{chngcntr}
\usepackage{setspace}
\usepackage{xurl}
\usepackage{csquotes}
\usepackage{float}
\usepackage{verbatim}
\usepackage{tabularx}
\usepackage{amsmath}
\usepackage{caption}
\usepackage{bm}
\usepackage{wrapfig}
\usepackage{siunitx}

\counterwithin{figure}{section}
\renewcommand{\thesection}{\arabic{section}}
\renewcommand{\thesubsection}{\thesection.\arabic{subsection}}
\renewcommand{\baselinestretch}{2}
\renewcommand{\thefigure}{\arabic{figure}}

\usepackage[style=apa, maxnames=6, minnames=3]{biblatex}
\DefineBibliographyStrings{english}{%chktex-file 1 chktex-file 6
      andothers = {\em et\addabbrvspace al\adddot}
}

\addbibresource{./Bibliography/bibliography.bib}

\usepackage{array}
\usepackage{enumitem}

\setlength{\parskip}{\baselineskip}

\newcommand{\bolditalic}[1]{\textbf{\textit{#1}}}

\DeclareSIUnit{\COP}{COP}
\newcommand{\cop}[1]{\$\SI{#1}{\COP}}

\DeclareSIUnit{\DOLLAR}{USD}
\newcommand{\dollar}[1]{\$\SI{#1}{\DOLLAR}}

\renewcommand{\comment}[1]{{\small $\ll$#1$\gg$}}

% chktex-file 24

\begin{document}

\chapter*{Resiliencia en tiempos de conflicto}

\vspace*{-2cm}

\noindent Los mecanismos de resistencia de Mayerlis Angarita en los Montes de Maria.

\noindent\makebox[\linewidth]{\rule{\textwidth}{0.4pt}}

\begin{itemize}[label=$\triangleright$]
      \item Abner Gabriel Orozco Ballestas \textit{(T00065222)}
      \item Mauro Alonso Gonzalez Figueroa \textit{(T00067622)}
      \item Angelis Dayanis Ahumedo Salas \textit{(T00069889)}
      \item Sharon Karina Benitez Amaya \textit{(T00066278)}
\end{itemize}
Mayerlis Angarita es una líder social y defensora de los derechos humanos en
los Montes de María, una región profundamente afectada por el conflicto armado
en Colombia. Su historia es un testimonio de la resistencia y la capacidad
humana para transformar el sufrimiento en fuerza colectiva. Esta propuesta
tiene como objetivo explorar los mecanismos de resistencia que Mayerlis ha
utilizado para enfrentar y superar el impacto del conflicto armado, y cómo
estos han contribuido a la reconstrucción del tejido social en su comunidad.

El conflicto armado en Colombia ha dejado profundas heridas en las comunidades,
especialmente en aquellas que, como los Montes de María, fueron epicentros de
violencia. En medio de esta adversidad, la historia de Mayerlis Angarita
destaca como un ejemplo inspirador de resiliencia y liderazgo. Analizar sus
estrategias de resistencia no sólo arroja luz sobre su experiencia personal,
sino que también ofrece valiosas lecciones sobre la construcción de paz y la
recuperación en contextos post-conflicto.

\section*{Justificación}

La historia de Mayerlis Angarita en los Montes de María es un ejemplo de cómo
la resiliencia y el liderazgo pueden transformar comunidades afectadas por el
conflicto armado. En una región marcada por la violencia y la desintegración
social, Mayerlis ha demostrado que es posible reconstruir el tejido social y
promover la paz a través de estrategias de resistencia y empoderamiento
comunitario. Analizar sus mecanismos de resistencia no solo nos permite
entender mejor su experiencia personal, sino que también ofrece valiosas
lecciones para otras comunidades en contextos similares.

\section*{Objetivos}

\subsection*{Objetivo general}

\begin{itemize}
      \item Examinar los mecanismos de resistencia que Mayerlis Angarita ha empleado para
            sobrevivir y resistir frente al conflicto armado, y cómo estos han influido en
            la transformación de su comunidad y en la promoción de la paz en los Montes de
            María.
\end{itemize}

\subsection*{Objetivos específicos}

\begin{itemize}
      \item Identificar y describir los mecanismos de resistencia utilizados por Mayerlis
            Angarita
            \begin{itemize}
                  \item Analizar documentos y testimonios relacionados con su trabajo y su impacto en
                        la comunidad.
            \end{itemize}

      \item Documentar y difundir las estrategias de resistencia y empoderamiento
            comunitario

      \item Difundir las estrategias de resistencia y empoderamiento comunitario en otras
            comunidades afectadas por el conflicto armado

      \item Explorar el uso de la memoria histórica como herramienta de resistencia
            \begin{itemize}
                  \item Documentar las iniciativas de memoria histórica promovidas por Mayerlis y su
                        impacto en la comunidad.
                  \item Evaluar cómo estas iniciativas han contribuido a la reconciliación y a la
                        construcción de una identidad colectiva.
            \end{itemize}
\end{itemize}

% \printbibliography

\end{document}